\documentclass[a4paper,12pt,twoside]{book}
\usepackage[utf8]{inputenc}
\usepackage[magyar]{babel}
\usepackage{pstricks,pstricks-add,pst-math,pst-xkey}
\usepackage{amssymb}
\usepackage{amsmath}
%\usepackage{pxfonts}
\usepackage{mathpazo}
\usepackage[T1]{fontenc}
\usepackage{textcomp}
\usepackage[a4paper,top=2cm,outer=2cm,bottom=2cm,inner=2cm]{geometry}
\usepackage{multirow}
\usepackage{color}
\usepackage{titletoc}
\usepackage{titlesec}
\usepackage{subfig}
\usepackage{wrapfig}
\usepackage{graphicx}
\usepackage{ulsy}
\usepackage{verbatim}
\usepackage{mdwlist}
\usepackage[thmmarks]{ntheorem}

\DeclareMathOperator{\tg}{tg}
\DeclareMathOperator{\ctg}{ctg}
\DeclareMathOperator{\arctg}{arctg}
\DeclareMathOperator{\arcctg}{arcctg}
\DeclareMathOperator{\sh}{sh} %sinh
\DeclareMathOperator{\ch}{ch} %cosh
\renewcommand{\th}{\qopname\relax o{th}}
\DeclareMathOperator{\cth}{cth}
\DeclareMathOperator{\arsh}{arsh}
\DeclareMathOperator{\arch}{arch}
\DeclareMathOperator{\arth}{arth}
\DeclareMathOperator{\arcth}{arcth}

\DeclareMathOperator{\Inf}{Inf}
\DeclareMathOperator{\Sup}{Sup}

\DeclareMathOperator{\Int}{Int}
\DeclareMathOperator{\Front}{Front}

\newtheorem{tetel}{Tétel}[chapter]
\newtheorem{defi}{Definíció}[chapter]
\newtheorem{lemma}{Lemma}[chapter]

\newenvironment{tetelAbraval}{%
\par\medskip\noindent\refstepcounter{tetel}\hbox{\bf Tétel \arabic{chapter}.\arabic{tetel} }
\it\ %\ignorespaces
}{%
\par\bigskip %
}

\newenvironment{bizAbraval}{%
\par\medskip\noindent\refstepcounter{biz}\hbox{\bf Bizonyítás \arabic{chapter}.\arabic{biz} }
%\ignorespaces
}{%
\leavevmode\unskip\penalty9999 \hbox{}\nobreak\hfill
    \quad\hbox{\rule{1ex}{1ex}}%
\par\bigskip %
}

\theoremstyle{break}
\newtheorem{defiNL}[defi]{Definíció}

\theorembodyfont{\normalfont}
\theoremsymbol{\rule{1ex}{1ex}}
\newtheorem{biz}{Bizonyítás}[chapter]
\newtheorem*{bizUO}{Bizonyítás}
\newtheorem{bizNL}[biz]{Bizonyítás}

\theoremstyle{plain}
\newtheorem{bizNoNL}[biz]{Bizonyítás}

\newcommand{\integ}[1]{\ensuremath{\int #1\, dx}}
\newcommand{\integDT}[1]{\ensuremath{\int #1\, dt}}
\newcommand{\hatInteg}[3]{\ensuremath{\int^{#2}_{x=#1} #3\, dx}}
\newcommand{\hatIntegLimits}[4]{\ensuremath{\int\limits^{#2}_{#1} #3\, d#4}}
\newcommand{\intfv}[3]{\ensuremath{\int^{#2}_{t=#1} #3\, dt}}

\title{\textbf{Analízis I. informatikusoknak}\\\Large Jegyzet mérnök-informatikus hallgatók részére}
\author{Készítette: Kriván Bálint\\ \normalsize dr. Tasnádi Tamás előadásai alapján}
\date{2009. szeptember - 2009. december 19.}

\parindent 0pt

\begin{document}

\maketitle

\tableofcontents

\chapter{Valós számok}

\[\mathbb{N}\subset\mathbb{Z}\subset\mathbb{Q}\subset\boxed{\mathbb{R}}\subset\mathbb{C}\]

\section{Csoportok}

\begin{defi}
 Legyen $G$ halmaz és $\cdot$ egy kétváltozós művelet $G$-n: $G\times G \to G$.\\
 $(G,\cdot)$ \textbf{csoport}, ha
 \begin{enumerate*}
  \item $(a\cdot b)\cdot c = a\cdot (b\cdot c) \quad \forall a, b, c \in G$ (tehát asszociatív)
  \item $\exists e\in G$, hogy $e\cdot g=g\cdot e = g \quad \forall g\in G$ (létezik egység elem)
  \item $\forall g\in G$ esetén $\exists! g^{-1}$ melyre $g\cdot g^{-1} = g^{-1}\cdot g = e$ (létezik inverze)
 \end{enumerate*}
\end{defi}

\begin{defi}
 Egy csoportot \textbf{kommutatív csoport}nak (Abel-csoport) hívunk, ha a művelet kommutatív. Tehát $(G;\cdot)$ kommutatív csoport, ha
 \[a\cdot b = b\cdot a \quad \forall a, b \in G\]
\end{defi}

\begin{defi}
 A \textbf{test} egy olyan $F = (T, + , \cdot)$ kétműveletes algebrai struktúrát jelöl, ahol $T$ kommutatív csoportot alkot a + (``összeadás'') műveletre nézve, a $\cdot$ (``szorzás'') kommutatív, asszociatív, minden nem nulla elemnek van inverze a $\cdot$ műveletre nézve, továbbá a $\cdot$ művelet disztributív a + műveletre. Tehát:
 \begin{enumerate*}
  \item $(T,+)$ kommutatív csoport: $e=0, a^{-1}=-a$.
  \item $(T\setminus\{0\},\cdot)$ kommutatív csoport: $e=1, a^{-1}=\frac{1}{a}$.
  \item $a\cdot(b+c) = ab+ac \quad \forall a,b,c \in T$
 \end{enumerate*}

\end{defi}


\section{Valós számok axiómái}

\begin{description*}
 \item[1-9.] $(\mathbb{R}, +, \cdot)$ testet alkot.
\end{description*}

\subsection{Rendezési axiómák}

$\forall a,b,c \in \mathbb{R}$-re
\begin{description*}
 \item[10.] A következő 3 közül pontosan 1 teljesül: $a<b; a=b; a>b$.
 \item[11.] Tranzitivitás: $a<b \hbox{ és } b<c \quad \Rightarrow a<c$
 \item[12.] A rendezés monoton az összeadásra: ha $a<b$, akkor $a+c < b+c$
 \item[13.] A rendezés monoton a szorzásra:ha $a<b$, akkor $a\cdot c < b\cdot c$
\end{description*}

\subsection{Archimédesz-féle axióma}

\begin{description*}
 \item[14.] $\forall x\in \mathbb{R}$-hez $\exists n\in\mathbb{N}$, hogy $n>x$.
\end{description*}

\subsection{Cantor-féle axióma}

\begin{description*}\label{Cantor}
 \item[15.] $\mathbb{R}$ teljes:\\
  Ha $I_n = [a_n, b_n]\subset\mathbb{R} \neq \emptyset$ és $a_n\leqslant a_{n+1}, b_{n+1}\leqslant b_n \quad \forall n\in\mathbb{N}$:
 \[I_1\subset I_2\subset I_3\subset \ldots \subset I_n \quad\hbox{akkor}\quad \exists\xi\in\mathbb{R} \hbox{, hogy}\]
 \[\xi\in\bigcap_{n\in\mathbb{N}} I_n\]
 \emph{Megjegyzés}: $\mathbb{Q}$-ban nem teljesül!
\end{description*}

\chapter{Számsorozatok}

\section{Kör kerülete}

\begin{wrapfigure}{r}{0.43\textwidth}
   \vspace{-35pt}
  \begin{center}

\psset{xunit=1.0cm,yunit=1.0cm,algebraic=true,dotstyle=*,dotsize=3pt 0,linewidth=0.8pt,arrowsize=3pt 2,arrowinset=0.25}
\begin{pspicture*}(-3.53,-1.93)(3.05,4.71)
\pscircle(-0.16,1.26){2.87}
\psplot{-3.53}{3.05}{(--1.6--2.71*x)/0.93}
\psline(0.78,3.97)(2.14,2.98)
\psplot{-3.53}{3.05}{(--5.57--1.69*x)/-2.32}
\psline(-0.16,1.26)(-1.84,-1.06)
\psline(-0.16,1.26)(1.46,3.48)
\pscustom{\parametricplot{0.6432888984861224}{0.941601252332483}{0.57*cos(t)+-0.16|0.57*sin(t)+1.26}\lineto(-0.16,1.26)\closepath}
\pscustom{\parametricplot{-2.1999914012573103}{-1.9016790474109497}{0.57*cos(t)+-0.16|0.57*sin(t)+1.26}\lineto(-0.16,1.26)\closepath}
\psplot{-0.16}{3.05}{(--3.16--1.72*x)/2.3}
\psplot{-3.53}{-0.16}{(-3.3-1.8*x)/-2.4}
\psdots(-0.16,1.26)
\psdots(2.14,2.98)
\psdots(0.78,3.97)
\psdots(-1.84,-1.06)
\psdots(-2.56,-0.54)
\psdots(-1.13,-1.58)
\rput[bl](-1.20,-0.50){1}
\psdots(1.46,3.48)
\rput[bl](0.59,0.9){$\dfrac{\pi}{n}$}
\rput[bl](-0.25,0.02){$\dfrac{\pi}{n}$}
\rput[bl](1.39,1.79){1}
\end{pspicture*}
\end{center}
 \vspace{20pt}
\end{wrapfigure}

A kör ($K$) kerületét beírt ($k_n^{}$) és hozzáírt ($K_n$) sokszögek kerületével közelítjük.\\
\[k_n = n\cdot 2\cdot\sin \dfrac{\pi}{n}\]
\[K_n = n\cdot 2\cdot\tg \dfrac{\pi}{n}\]
\[k_n \leqslant K \leqslant K_n\]

\[
\begin{array}{l|cccccc}
n & 3 & 4 & 10 & 100 & \ldots & \to\infty \\\hline
k_m & 3\sqrt{3} & 4\sqrt{2} & 6,1803 & 6,2822 & \ldots & \to2\pi\\
K_n & 6\sqrt{3} & 8 & 6,4983 & 6,2853 & \ldots & \to2\pi
\end{array}
\]

\section{Bevezetés; definíciók}

\begin{defi}
Valós számsorozatok
\[f: \mathbb{N} \mapsto \mathbb{R} \qquad a_n = f(n)\]
Megjegyzés: Van olyan, hogy néha nem értelmezzük a függvényt az első néhány természetes számon, pl.: $a_n = \dfrac{1}{n} \quad (n=1,2,3,\ldots)$
\end{defi}

\begin{defi}Az $\{a_n\}$ sorozat felülről korlátos, ha $\exists K\in\mathbb{R}:a_n \leqslant K \quad \forall n\in\mathbb{N}$.\end{defi}
\begin{defi}Az $\{a_n\}$ sorozat alulról korlátos, ha $\exists k\in\mathbb{R}:a_n \geqslant k \quad \forall n\in\mathbb{N}$.\end{defi}
\begin{defi}Az $\{a_n\}$ sorozat korlátos, ha alulról \textbf{és} felülről is korlátos.\end{defi}

\begin{defi}\label{HatarertekDefi1}Az $\{a_n\}$ \textbf{határértéke} $A\in\mathbb{R}$, ha $\forall\varepsilon>0:\exists N(\varepsilon)$, hogy $|a_n-A|<\varepsilon$, ha $n>N(\varepsilon)$.\\
Jelölés: $\displaystyle{\lim_{n\to\infty}a_n = A}$, vagy $a_n \xrightarrow{n\to\infty} A$
\end{defi}

\begin{defi}Az $\{a_n\}$ sorozat konvergens, ha $\exists A\in\mathbb{R}$, hogy $\displaystyle{\lim_{n\to\infty}a_n = A}$.\end{defi}

Intervallumok:
\begin{itemize*}
 \item zárt: $[a;b]=\{x\in\mathbb{R}|a\leqslant x\leqslant b\}$
 \item nyílt: $]a;b[$ vagy $(a;b)=\{x\in\mathbb{R}|a < x < b\}$
 \item félig nyílt v. zárt: $(a;b]=\{x\in\mathbb{R}|a < x \leqslant b\}$ 
\end{itemize*}

\begin{defi}
$\displaystyle \lim_{n\to\infty}a_n=A\in\mathbb{R}$, ha $\forall\varepsilon > 0$ esetén a sorozat véges számú eleme van az $(A-\varepsilon; A+\varepsilon)$ intervallumon kívül. Ez ekvivalens \aref{HatarertekDefi1}. definícióval.
\end{defi}

\subsection{Példák}

(1) $\displaystyle a_n = \frac{1}{n}. \quad \lim_{n\to\infty} \frac{1}{n} = 0$. Bizonyítás \aref{HatarertekDefi1}. definíció felhasználásával:
\[\left|\frac{1}{n}-0\right|<\varepsilon \quad\rightarrow\quad \frac{1}{n}<\varepsilon\]
\[n>\frac{1}{\varepsilon} \quad\rightarrow\quad N(\varepsilon)=\left[\frac{1}{\varepsilon}\right]\]
Tehát küszöbindexnek $\forall\varepsilon>0$-hoz az $\left[\dfrac{1}{\varepsilon}\right]$-t választhatjuk.\\

(2) $\displaystyle b_n = \frac{(-1)^n}{n}. \quad \lim_{n\to\infty} \frac{(-1)^n}{n} = 0$.
\[\left|\frac{(-1)^n}{n}-0\right|<\varepsilon \quad\rightarrow\quad \frac{1}{n}<\varepsilon \quad\rightarrow\quad N(\varepsilon)=\left[\frac{1}{\varepsilon}\right]\]\\

(3) $\displaystyle c_n = \frac{3+n}{5-2n} \xrightarrow{n\to\infty} -\frac{1}{2}$
\[\left|\frac{3+n}{5-2n}+\frac{1}{2}\right|=\left|\frac{6+2n+5-2n}{2(5-2n)}\right|=\left|\frac{11}{10-4n}\right|<\varepsilon\]
Ha $n\geqslant 3$, akkor:
\[\left|\frac{11}{10-4n}\right|=\frac{11}{4n-10}<\varepsilon \quad\rightarrow\quad \frac{11+10\varepsilon}{4\varepsilon}<n \quad\rightarrow\quad N(\varepsilon)=\max\left\{3, \left[\frac{11+10\varepsilon}{4\varepsilon}\right]\right\}\]\\

(4) $\displaystyle d_n = \frac{n^2-2n}{3n^3+2} \xrightarrow{n\to\infty} 0$
\[\left|\frac{n^2-2n}{3n^3+2}-0\right|\overset{x\geqslant 2}{=}\frac{n^2-2n}{3n^3+2}<\varepsilon \qquad \hbox{nehéz egzaktul megoldani $\rightarrow$ becslés}\]
\[\frac{n^2-2n}{3n^3+2} < \frac{n^2}{3n^3} = \frac{1}{3n} < \varepsilon \quad\rightarrow\quad n > \frac{1}{3\varepsilon}  \quad\rightarrow\quad N(\varepsilon)=\max\left\{2, \left[\frac{1}{3\varepsilon}\right]\right\}\]\\

(5) $\displaystyle e_n = \frac{n^3-2n}{3n^3-2} \xrightarrow{n\to\infty} \frac{1}{3}$
\[\left|\frac{n^3-2n}{3n^3-2}-\frac{1}{3}\right|=\left|\frac{3n^3-6n-3n^3+2}{9n^3-6}\right|=\left|\frac{-6n+2}{9n^3-6}\right|\overset{x\geqslant 1}{=}\frac{6n-2}{9n^3-6}<\frac{6n}{9n^3-6n^3}=\frac{2}{n^2}<\varepsilon\]
\[\frac{2}{n^2}<\varepsilon \quad\rightarrow\quad n > \sqrt{\frac{2}{\varepsilon}} \quad\rightarrow\quad N(\varepsilon)=\max\left\{1, \left[\sqrt{\frac{2}{\varepsilon}}\right]\right\}\]

\subsection{Divergens sorozatok és $\pm$ végtelenbe tartás}

\begin{defi}Az $\{a_n\}$ sorozat \textbf{divergens} ha $\nexists A\in\mathbb{R}$, hogy $a_n\to A$.\end{defi}

\begin{defi}$a_n \xrightarrow{n\to\infty} \infty$, ha $\forall P\in\mathbb{R}>0:\exists N(P)$ küszöbindex, hogy $a_n>P$, ha $n>N(P)$.\end{defi}

\begin{defi}$a_n \xrightarrow{n\to\infty} -\infty$, ha $\forall M\in\mathbb{R}<0:\exists N(M)$ küszöbindex, hogy $a_n<M$, ha $n>N(M)$.\end{defi}

\begin{lemma}$\boxed{a_n \to \infty \Longleftrightarrow -a_n \to -\infty}$\end{lemma}

\subsection{Példák}

(1) $a_n=n^3+10n+2 \to \infty$\\
Kell, hogy adott $P$-re $a_n>P$:
\[a_n=n^3+10n+2\overset{*}{>}n^3>P \quad \Longrightarrow \quad n>\sqrt[3]{P}\]
(*) becslünk: ha egy az erdetinél kisebb sorozatot vizsgálunk, akkor az ehhez tartozó küszöbindex az eredetihez is jó lesz.\\
Tehát küszöbindexnek $\forall P>0$-hoz például az $N(P)=\left[\sqrt[3]{P}\right]$-t választhatjuk.\\

(2) $a_n=n^3-10n+2 \to \infty$
\[a_n = n^3-10n+2 > n^3-10n \overset{n\geqslant 5}{>} n^3-\frac{n^3}{2}=\frac{n^3}{2} > P \quad \Longrightarrow \quad n>\sqrt[3]{2P}\]
\[N(P)=\max\left\{5, \left[\sqrt[3]{2P}\right]\right\}\]
\\
\textit{Megjegyzés}: $a_n = (-1)^n$ divergens, nem tart se $+\infty$, se $-\infty$-be.

\begin{tetel}Ha egy sorozat konvergens, akkor a hatáértéke egyértelmű, tehát:
\[\boxed{\lim_{n\to\infty} a_n = A\in\mathbb{R} \hbox{ és } \lim_{n\to\infty} a_n = B\in\mathbb{R} \quad \Longrightarrow \quad A=B}\]\end{tetel}
\begin{bizAbraval}Indirekt. Tfh: $A\neq B$. Legyen például: $B<A$

\begin{wrapfigure}{r}{0.31\textwidth}
\vspace{-48pt}
  \begin{center}
\psset{xunit=1.0cm,yunit=1.0cm,algebraic=true,dotstyle=*,dotsize=3pt 0,linewidth=0.8pt,arrowsize=3pt 2,arrowinset=0.25}
\begin{pspicture*}(-5.18,-0.62)(0.2,0.78)
\psaxes[xAxis=true,yAxis=false,labels=y,Dx=1,Dy=1,ticksize=0 0,subticks=0]{->}(0,0)(-5.18,-0.62)(0.2,0.78)
\rput[b](-4.2,-0.21){(}
\psline{-}(-3.5,0.20)(-3.5,-0.20)
\rput[b](-3.55,-0.60){$B$}
\rput[b](-2.8,-0.21){)}

\rput[b](-2.1,-0.21){(}
\psline{-}(-1.4,0.20)(-1.4,-0.20)
\rput[b](-1.45,-0.60){$A$}
\rput[b](-0.7,-0.21){)}
\end{pspicture*}
\end{center}
\vspace{-20pt}
\end{wrapfigure}

Legyen $\varepsilon = \dfrac{|A-B|}{3}$.
\[\left.\begin{array}{l}
    a_n \to A \Rightarrow a_n\in(A-\varepsilon, A+\varepsilon) \hbox{ ha } n>N_a(\varepsilon)\\
    a_n \to B \Rightarrow a_n\in(B-\varepsilon, B+\varepsilon) \hbox{ ha } n>N_b(\varepsilon)\\
  \end{array}\right\} \begin{array}{c}
    \hbox{ha } n > \max\{N_a(\varepsilon), N_b(\varepsilon)\} \quad \Longrightarrow \\
    a_n\in(A-\varepsilon, A+\varepsilon)\cup(B-\varepsilon, B+\varepsilon) = \emptyset
  \end{array}
\]
Tehát az állításunk hamis, így igaz az eredeti állítás.
\end{bizAbraval}

\section{Ekvivalens megfogalmazások}

$P$ és $Q$ két állítás:
\begin{enumerate}
 \setlength{\itemsep}{1pt}
 \setlength{\parskip}{0pt}
 \setlength{\parsep}{0pt}

 \item $P \Leftrightarrow Q \qquad Q \Leftrightarrow P$
 \item $P$ ekvivalens $Q$-val.
 \item $P$ pontosan akkor teljesül, amikor $Q$.
 \item $P$ akkor és csak akkor teljesül, ha $Q$.
 \item $P$ szükséges és elégséges feltétele $Q$-nak.
 \item $Q$ szükséges és elégséges feltétele $P$-nek.
\end{enumerate}

\begin{enumerate}
 \setlength{\itemsep}{1pt}
 \setlength{\parskip}{0pt}
 \setlength{\parsep}{0pt}

 \item $P \Rightarrow Q$
 \item $P$-ből következik $Q$.
 \item $P$ maga után vonja $Q$-t.
 \item $P$ implikálja $Q$-t.
 \item $P$ elégséges feltétele $Q$-nak.
 \item $Q$ szükséges feltétele $P$-nek.
\end{enumerate}

\section{Korlátosság és konvergencia kapcsolata}

\begin{tetel}$\boxed{$Ha az $\{a_n\}$ sorozat konvergens, akkor korlátos$}$\end{tetel}
\textit{Megjegyzés}: Minden konvergens sorozat korlátos.\\
A korlátosság szükséges feltétele a konvergenciának, a konvergencia elégséges feltétele a korlátos\-ságnak.
\begin{bizNL}
$a_n\to A\in\mathbb{R}$, tehát $\forall\varepsilon>0: \exists N(\varepsilon)$, hogy $a_n\in(A-\varepsilon,A+\varepsilon) \quad n>N(\varepsilon)$. Tehát az $(A-\varepsilon,A+\varepsilon)$-on kívül csak véges sok elemek esnek kívül: $a_0, a_1, \ldots, a_{N(\varepsilon)}$
\[\Longrightarrow \left\{\begin{array}{llll}
  \exists k & : \forall n\hbox{-re} & k\leqslant a_n & k = \min\{a_0, a_1, \ldots, a_{N(\varepsilon)}, A-\varepsilon\}\\
  \exists K & : \forall n\hbox{-re} & K\geqslant a_n & K = \max\{a_0, a_1, \ldots, a_{N(\varepsilon)}, A+\varepsilon\}\\
\end{array}\right.\]
Mivel van alsó és felső korlátja, ezért korlátos.
\end{bizNL}

\section{Műveletek konvergens számsorozatokkal}

\begin{tetel}\label{OsszegHatarertek} $\boxed{a_n \to A$ és $b_n \to B \quad \Longrightarrow \quad (a_n+b_n) \to A+B}$\end{tetel}
\begin{bizNL}
  A $c_n = a_n + b_n$ sorozatról kell belátni, hogy $C=A+B$ a határértéke. Számsorozatok konvergenciája szerint: $\forall\varepsilon:\exists N(\varepsilon)$, hogy $|c_n-C| < \varepsilon$, ha $n>N_c(\varepsilon)$. Legyen $\varepsilon^* = \frac{\varepsilon}{2}$. Hasonlóan elmondható, hogy:
  \[
    \left.\begin{array}{l}
      |a_n-A|<\varepsilon^* \quad \forall n>N_a(\varepsilon^*) \\
      |b_n-B|<\varepsilon^* \quad \forall n>N_b(\varepsilon^*)
    \end{array}\right\} \hbox{Ha } n>\max\{N_a(\varepsilon^*),N_b(\varepsilon^*)\} \hbox{, akkor }
  \]
  \[|c_n-C| = |a_n+b_n-(A+B)| \underset{^*}{\leqslant} |a_n-A|+|b_n-B| < 2\varepsilon^* = \varepsilon\]
  Tehát $N_c(\varepsilon)$ küszöbindexnek $N_c(\varepsilon)=\max\left\{N_a\left(\dfrac{\varepsilon}{2}\right),N_b\left(\dfrac{\varepsilon}{2}\right)\right\}$-t választhatjuk.\\
  ($^*$) = háromszög egyenlőtlenség!
\end{bizNL}

\begin{tetel}\label{KonstansSzorSorozat} $\boxed{a_n \to A \quad \Longrightarrow \quad c\cdot a_n \to c\cdot A}$\end{tetel}
\begin{biz}
\[|c a_n - cA| = |c||a_n-A| < \varepsilon \quad \Longrightarrow \quad |a_n-A| < \frac{\varepsilon}{|c|} \quad \rightarrow \quad N_{ca}(\varepsilon) = N_a\left(\frac{\varepsilon}{|c|}\right)\]
Következmény:
\[a_n \to A \hbox{ és } b_n \to B \quad \Longrightarrow \quad (a_n-b_n) \to A-B\]
Könnyen bizonyítható az előző két tétel felhasználásával.
\end{biz}

\begin{tetel}\label{0szorozvaKorlatos} $\boxed{a_n \to 0 $ és $b_n$ korlátos $\quad \Longrightarrow \quad a_nb_n \to 0}$\end{tetel}
\begin{biz}
\[|a_nb_n - 0| = |a_nb_n| = |a_n||b_n| < \varepsilon\]
Mivel $b_n$ korlátos, ezért $b_n\leqslant K \in\mathbb{R}$:
\[|a_n| < \frac{\varepsilon}{K} \quad \hbox{teljesül, ha: } n>N_a\left(\frac{\varepsilon}{K}\right)=N_{ab}(\varepsilon)\]
\end{biz}

\begin{tetel}\label{SzorzatLimesz}$\boxed{a_n \to A $ és $b_n \to B \quad \Longrightarrow \quad a_nb_n \to AB}$\end{tetel}
\begin{biz}
\[a_nb_n = \overbrace{(\underbrace{a_n-A}_{\to 0})(\underbrace{b_n-B}_{\to 0})}^{\hbox{\ref{KonstansSzorSorozat}. tétel köv.}}+\overbrace{\underbrace{Ab_n}_{\to AB}+\underbrace{Ba_n}_{\to AB}}^{\hbox{\ref{KonstansSzorSorozat}. tétel}}-AB \quad \to \quad AB\]

\end{biz}

\begin{tetel}\label{AbszolutertekLimesz}$\boxed{a_n\to A \quad \Longrightarrow \quad |a_n|\to |A|}$\end{tetel}
\begin{biz}
\[||a_n|-|A|| \leqslant |a_n-A| < \varepsilon \quad \rightarrow \quad N_{|a|}(\varepsilon)=N_{a}(\varepsilon)\]
\end{biz}

\begin{tetel}$\boxed{b_n \to B\neq 0 \quad \Longrightarrow \quad \frac{1}{b_n} \to \frac{1}{B}}$\end{tetel}
\begin{biz}
\[\left|\frac{1}{b_n}-\frac{1}{B}\right| = \left|\frac{B-b_n}{b_nB}\right| = \frac{|b_n-B|}{|b_n||B|}\]
Felhasználva \aref{AbszolutertekLimesz}. tételt: $|b_n|\to |B|$, tehát $\exists N_1\left(\dfrac{|B|}{2}\right)$, hogy $n>N_1\left(\dfrac{|B|}{2}\right)$ esetén:
\[|b_n| \in \left(|B|-\frac{|B|}{2}, |B|+\frac{|B|}{2}\right) \quad \Longrightarrow \quad |b_n| > \frac{|B|}{2}\]
Másrészt viszont $\exists N_2\left(\dfrac{\varepsilon}{2}|B|^2\right)$, hogy $|b_n - B| < \dfrac{\varepsilon}{2}|B|^2$. Ezt a két dolgot felhasználva, ha $n>\max\{N_1, N_2\}=N(\varepsilon)$:
\[\frac{|b_n-B|}{|b_n||B|} < \frac{|b_n-B|}{\frac{|B|}{2}|B|} < \frac{\frac{\varepsilon}{2}|B|^2}{\frac{|B|}{2}|B|} = \varepsilon \]
Bizonyítottuk az állítást, hiszen $\forall \varepsilon: \exists N(\varepsilon)$, hogy $n>N(\varepsilon)$ esetén $\left|\frac{1}{b_n}-\frac{1}{B}\right| < \varepsilon$.
\end{biz}

\begin{tetel}$\boxed{a_n \to A $ és $b_n\to B\neq 0 \quad \Longrightarrow \quad \dfrac{a_n}{b_n} \to \dfrac{A}{B}}$\end{tetel}
\begin{biz}
\[\frac{a_n}{b_n} = \underbrace{a_n}_{\to A}\cdot\underbrace{\frac{1}{b_n}}_{\to 1/B} \quad \to \quad \frac{A}{B}\]
Felhasználva az előző, illetve \aref{SzorzatLimesz}. tételt.
\end{biz}

\subsection{Példák}

\[a_n = \underbrace{\frac{1}{n^2}}_{\to 0} + \underbrace{\frac{2}{n^2}}_{\to 0} + \ldots + \underbrace{\frac{500}{n^2}}_{\to 0} \xrightarrow{n\to\infty} 0\]
Azért használhatjuk \aref{OsszegHatarertek}. tételt, mert véges sok tagot adunk össze!
\[b_n = \underbrace{\frac{1}{n^2}}_{\to 0} + \underbrace{\frac{2}{n^2}}_{\to 0} + \ldots + \underbrace{\frac{n}{n^2}}_{\to 0} \xrightarrow{n\to\infty} ?\]
Itt már nem használhatjuk; át kell alakítani:
\[b_n = \frac{1+2+\ldots+n}{n^2} = \frac{\frac{n(n+1)}{2}}{n^2} = \frac{n+1}{2n} = \frac{1+\frac{1}{n}}{2} \xrightarrow{n\to\infty} \frac{1}{2}\]
\[c_n = \left(1+\frac{1}{n}\right)^3 = \left(1+\frac{1}{n}\right)\left(1+\frac{1}{n}\right)\left(1+\frac{1}{n}\right) \xrightarrow{n\to\infty} 1 \]
Viszont
\[d_n = \left(1+\frac{1}{n}\right)^n \xrightarrow{n\to\infty} e \quad \hbox{majd visszatérünk rá!}\]
\[e_n = \frac{3n^3-2n^2+5}{-n^3+3n-6} = \frac{3-\frac{2}{n}+\frac{5}{n^3}}{-1+\frac{3}{n^2}-\frac{6}{n^3}} \xrightarrow{n\to\infty} -3\]
\[f_n = \underbrace{\frac{n^2-8}{3n^3+n}}_{\to 0}\underbrace{\cos(2n+6)}_{\hbox{korlátos}} \xrightarrow{n\to\infty} 0 \qquad 0\cdot\hbox{korlátos} \to 0 \; \hbox{(lásd \ref{0szorozvaKorlatos}. tétel!)}\]

\section{Néhány jól használható tétel}

\begin{tetel}$\boxed{a_n\geqslant 0$ és $a_n\to A \quad \Longrightarrow \quad \sqrt{a_n} \to \sqrt{A}}$\end{tetel}
\begin{bizNL}
  \begin{enumerate}
    \item Ha $A=0$:
    \[|\sqrt{a_n}-\sqrt{A}|=\underbrace{\sqrt{a_n}\leq \varepsilon}_{\hbox{igaz, ha $a_n<\varepsilon^2$}}\]
    Tehát $N_{\sqrt{a}}(\varepsilon)=N_a(\varepsilon^2)$. $\checkmark$
    
    \item Ha $A>0$
    \[|\sqrt{a_n}-\sqrt{A}| = \left|\big(\sqrt{a_n}-\sqrt{A}\big)\cdot\frac{\sqrt{a_n}+\sqrt{A}}{\sqrt{a_n}+\sqrt{A}}\right| = \left|\frac{a_n-A}{\sqrt{a_n}+\sqrt{A}}\right| \leqslant \frac{|a_n-A|}{\sqrt{A}}\]
    \[\frac{|a_n-A|}{\sqrt{A}} < \varepsilon \quad \hbox{ha } \; |a_n-A|<\varepsilon\sqrt{A}\]
    Tehát $N_{\sqrt{a}}(\varepsilon)=N_a(\varepsilon\sqrt{A})$. $\checkmark$
    
  \end{enumerate}
\end{bizNL}

\begin{lemma}$\boxed{a_n\geqslant 0$ és $a_n\to A \quad \Longrightarrow \quad \sqrt[k]{a_n} \to \sqrt[k]{A} \quad \forall k\in\mathbb{N}}$\end{lemma}

\begin{tetel}$\boxed{$Ha $\lim_{n\to\infty}a_n=A $ és $f$ folytonos $A$-ban, akkor $}$
\[\boxed{\lim_{n\to\infty}f(a_n)=f(A)=f(\lim_{n\to\infty}a_n)}\]\end{tetel}
\begin{biz}Következik a folytonosság definíciójából
\end{biz}

\subsection{Példák}
\[a_n = \underbrace{\sqrt{2n^2+3n-1}}_{\to\infty}-\underbrace{\sqrt{2n^2+n}}_{\to\infty} \qquad \hbox{határozatlan alak ($\infty-\infty$)}\]
Konjugálttal bővítünk:
\[a_n = \frac{(2n^2+3n-1)-(2n^2+n)}{\sqrt{2n^2+3n-1}+\sqrt{2n^2+n}} = \frac{2n-1}{\sqrt{2n^2+3n-1}+\sqrt{2n^2+n}} = \]
\[ = \frac{2-\frac{1}{n}}{\sqrt{2+\frac{3}{n}-\frac{1}{n^2}}+\sqrt{2+\frac{1}{n}}} \to \frac{2}{\sqrt{2}+\sqrt{2}} = \frac{1}{\sqrt{2}}\]
\\
\[b_n = \sqrt[3]{n^3+3n^2+1}-\sqrt[3]{n^3+4}=\frac{n^3+3n^2+1-(n^3+4)}{\sqrt[3]{n^3+3n^2+1}^2+\sqrt[3]{(n^3+3n^2+1)(n^3+4)}+\sqrt[3]{n^3+4}^2} = \]
\[= \frac{3n^2-3}{\sqrt[3]{n^3+3n^2+1}^2+\sqrt[3]{(n^3+3n^2+1)(n^3+4)}+\sqrt[3]{n^3+4}^2} = \]
\[= \frac{3-\frac{3}{n^2}}{\sqrt[3]{1+\frac{3}{n}+\frac{1}{n^3}}^2+\sqrt[3]{(1+\frac{3}{n}+\frac{1}{n^3})(1+\frac{4}{n^3})}+\sqrt[3]{1+\frac{4}{n^3}}^2} \to \frac{3}{1+1+1} = 1\]
Megjegyzés: $a^3 - b^3 = (a-b)(a^2+ab+b^2)$\\
\\
\[c_n = \frac{\sqrt[4]{2n^4+n^3-2n^2+8}}{\sqrt[3]{n^6+5n^2+3}} = \frac{n\sqrt[4]{2+\frac{1}{n}-\frac{2}{n^2}+\frac{8}{n^4}}}{n^2\sqrt[3]{1+\frac{5}{n^4}+\frac{3}{n^6}}} = \frac{1}{n}\cdot c \to 0\]

\begin{tetel}$\boxed{$Ha $a_n\to\infty \quad \Longrightarrow \quad \frac{1}{a_n}\to 0}$\end{tetel}
\begin{bizNL}$a_n>P\geqslant 0$, ha $n>N_a(P) \quad \Longrightarrow \quad 0<\frac{1}{a_n}<\frac{1}{P}$, ha $n>N_a(P)$. Tehát $N_{\frac{1}{a}}(\varepsilon)=N_a(\frac{1}{P})$
\end{bizNL}

\begin{lemma}\label{Poz0RecpirokaTartVegtelen}$\boxed{0<a_n\to 0 \quad \Longrightarrow \quad \frac{1}{a_n}\to\infty}$\end{lemma}
\begin{lemma}$\boxed{0>a_n\to 0 \quad \Longrightarrow \quad \frac{1}{a_n}\to -\infty}$\end{lemma}
\begin{lemma}$\boxed{a_n\to\infty \quad \Longrightarrow \quad \frac{1}{|a_n|}\to \infty}$\end{lemma}

\begin{tetel}$\boxed{a_n\to 0 \quad \Longleftrightarrow \quad |a_n|\to 0}$\end{tetel}
\begin{biz}$|a_n-0|=|a_n|<\varepsilon \qquad ||a_n|-0|=|a_n|<\varepsilon$
\end{biz}

\subsection{Hasonló tételek}

$\dfrac{0}{\infty}=0; \quad \dfrac{\hbox{korlátos}}{\infty}=0; \quad \dfrac{\infty}{+0}=\infty; \quad \infty\cdot\infty=\infty; \quad \infty+\infty=\infty$

\subsection{Határozatlan alakok}
$\dfrac{0}{0}; \quad \dfrac{\infty}{\infty}; \quad 0\cdot\infty; \quad \infty - \infty; \quad \infty^0; \quad 1^\infty; \quad 0^0$\\
Megoldási lehetőségek: azonos átalakítás, becslés, (L'Hospital szabály)

\begin{tetel}$\boxed{$A limesz monoton: $a_n<b_n$ és $a_n\to A, b_n\to B \quad \Longrightarrow \quad A\leqslant B}$\\
\textit{Megjegyzés}: $a_n\leqslant b_n$-re is igaz az állítás.\\
\textit{Pl}: $a_n=1-\frac{1}{n}<b_n=1+\frac{1}{n} \quad \lim a_n = \lim b_n = 1$\end{tetel}
\begin{biz}Indirekt. Tfh: $B>A$\end{biz} % befejezzük a biz-t, mert akkor a float az nem lesz jó!

\begin{wrapfigure}{r}{0.35\textwidth}
\vspace{-40pt}
  \begin{center}
\psset{xunit=1.0cm,yunit=1.0cm,algebraic=true,dotstyle=*,dotsize=3pt 0,linewidth=0.8pt,arrowsize=3pt 2,arrowinset=0.25}
\begin{pspicture*}(-5.18,-0.62)(0.2,0.78)
\psaxes[xAxis=true,yAxis=false,labels=y,Dx=1,Dy=1,ticksize=0 0,subticks=0]{->}(0,0)(-5.18,-0.62)(0.2,0.78)
\rput[b](-4.2,-0.21){(}
\psline{-}(-3.5,0.20)(-3.5,-0.20)
\rput[b](-3.55,-0.60){$B$}
\rput[b](-2.8,-0.21){)}

\rput[b](-2.1,-0.21){(}
\psline{-}(-1.4,0.20)(-1.4,-0.20)
\rput[b](-1.45,-0.60){$A$}
\rput[b](-0.7,-0.21){)}
\end{pspicture*}
\end{center}
\vspace{-20pt}
\end{wrapfigure}

Legyen $\varepsilon = \dfrac{|A-B|}{3}$, ekkor a számsorozatok konvergenciája alapján:
\[A-\varepsilon < a_n < A+\varepsilon \qquad n>N_a(\varepsilon)\]
\[B-\varepsilon < b_n < B+\varepsilon \qquad n>N_b(\varepsilon)\]
Elsőt $-1$-el megszorozva, majd átrendezve a következőt kapjuk:
\[-A-\varepsilon < -a_n < -A+\varepsilon\]
Hozzáadva a $B$-s egyenlőtlenséghez:
\[B-A-2\varepsilon < b_n-a_n < \underbrace{\underbrace{B-A}_{\hbox{$-3\varepsilon$}}+2\varepsilon}_{\hbox{$-\varepsilon$}} \qquad n>\max\{N_a(\varepsilon),N_b(\varepsilon)\}\]
\[\hbox{Tehát} \quad b_n-a_n < -\varepsilon \;\hbox{\blitza}\]
Ellentmondás, hiszen feltétel szerint $a_n<b_n$ ($a_n\leqslant b_n$), tehát $b_n-a_n$ biztosan pozitív (nem negatív). $\hfill \rule{1ex}{1ex}$

\begin{tetel}$\boxed{$Rendőr-elv$}$\\
Ha $\lim a_n=A\in\mathbb{R}$ és $\lim b_n=A$, illetve $a_n\leqslant c_n\leqslant b_n$, akkor $\forall n>N_0\in\mathbb{N}$-re akkor $\exists\lim c_n = A$ \end{tetel}
\begin{biz}
\[\left.
\begin{array}{cc}
  A-\varepsilon<a_n<A+\varepsilon & \quad n > N_a(\varepsilon)\\
  A-\varepsilon<b_n<A+\varepsilon & \quad n > N_b(\varepsilon)
\end{array}\right\} \quad \begin{array}{ll}
  A-\varepsilon<a_n\leqslant c_n\leqslant b_n<A+\varepsilon\\
  \hbox{ha } n>\max\{N_a(\varepsilon),N_b(\varepsilon),N_0\}
\end{array}
\]
\[A-\varepsilon<c_n<A+\varepsilon\]
\end{biz}

\begin{tetel}$\boxed{$Speciális rendőr-elv$}$\\
Ha $a_n\to\infty$ és $a_n\leqslant b_n \; \forall n>N_0$, akkor $b_n\to\infty$.\\
Ha $a_n\to-\infty$ és $a_n\geqslant b_n \; \forall n>N_0$, akkor $b_n\to-\infty$.
\end{tetel}
\begin{biz}
Csak az elsőt bizonyítjuk:\\
$\forall P>0$, ha $n>N_a(P)$, akkor: $P<a_n\leqslant b_n$. Tehát:
\[\forall P>0 \quad n>N_b(P) = \max\{N_a(P),N_0\} \; \Rightarrow \; P<b_n\]
\end{biz}

\subsection{Példák a rendőr-elv használatára}

(1) $a_n=\dfrac{2n^6+n^3-n}{n^4+3} \to \infty$\\
A fenti sorozatot egy nálánál kisebb sorozattal közelítjük, melyről belátjuk, hogy a végtelenhez tart. Ekkor felhasználhatjuk a speciális rendőr-elvet és kész:
\[a_n=\frac{2n^6+n^3-n}{n^4+3}\geqslant\frac{2n^6+0-n^6}{n^4+3n^4}=\frac{n^6}{4n^4}=\frac{1}{4}n^2 \; \to \; \infty \qquad \hbox{Tehát } a_n\to\infty\]\\

(2) $b_n=\dfrac{1}{\sqrt{n^2+1}}+\dfrac{1}{\sqrt{n^2+2}}+\ldots+\dfrac{1}{\sqrt{n^2+n}}$\\
Ezt a ,,sima'' rendőr-elvvel oldjuk meg (kisebb sorozathoz minden elem helyére a legkisebbet írjuk, a nagyobbhoz pedig a legnagyobbat):
\[\frac{n}{\sqrt{n^2+n}} \leqslant b_n \leqslant \frac{n}{\sqrt{n^2+1}}\]
\[\underbrace{\frac{1}{\sqrt{1+\frac{1}{n}}}}_{\to 1} \leqslant b_n \leqslant \underbrace{\frac{1}{\sqrt{1+\frac{1}{n^2}}}}_{\to 1}\]
Tehát rendőr-elv alapján $b_n\to 1$.

\section{További gyakran használt határértékek}

\begin{tetel}\label{aAzNediken}$\displaystyle \boxed{\lim_{n\to\infty} a^n=\begin{cases}
  0, \hbox{ ha } |a|<1 \\
  1, \hbox{ ha } a=1 \\
  +\infty, \hbox{ ha } a>1 \qquad (\nexists) \\
  \nexists, \hbox{ ha } a\leq -1
\end{cases}}$\\
\end{tetel}
\begin{bizNL}
Itt az $0<a<1$ esetet bizonyítjuk:\\
$\forall \varepsilon > 0 \quad \rightarrow \quad |a^n-0|=|a^n|=a^n<\varepsilon$
\[a^n<\varepsilon \qquad \Longrightarrow \qquad n\cdot \underbrace{\ln a}_{<0} < \underbrace{\ln \varepsilon}_{<0 \hbox{ ha } 0<\varepsilon<1}\]
\[n>\frac{\ln\varepsilon}{\ln a}>0 \qquad \hbox{jó küszöbindexnek: } N(\varepsilon)=\left[\frac{\ln\varepsilon}{\ln a}\right]\]
\end{bizNL}

\begin{tetel}\label{nKaN_Tart0}
 $\displaystyle \boxed{ \lim_{n\to\infty} n^k\cdot a^n = 0 \qquad $ ha $|a|<1$ és $k\in\mathbb{N}^+$ $}$
\end{tetel}
\begin{bizNL}
Elfogadjuk, később l'Hospital szabállyal beláthatjuk. Konkrét példára esetleg monoton csök\-kenéssel és korlátossággal bizonyítható.
\end{bizNL}

\begin{tetel}$\displaystyle \boxed{\lim_{n\to\infty} \sqrt[n]{p}=1 \qquad $ ha $p>0}$\\
\end{tetel}
\begin{bizNL}
Ha $p=1$, akkor $\checkmark$.\\
Ha $p>1$, akkor $\sqrt[n]{p}>1$, tehát $\sqrt[n]{p}=1+x_n$. Tehát kell, hogy $\lim x_n = 0$.
\[p=(1+x_n)^n=1+n\cdot x_n+\binom{n}{2}x^2_n+\ldots\]
\[p=(1+x_n)^n > 1+n\cdot x_n \quad \Rightarrow \quad x_n < \frac{p-1}{n}\]
\[\underbrace{0}_{\to 0} < x_n < \underbrace{\frac{p-1}{n}}_{\to 0}\]
Tehát $x_n\to 0$ a rendőr-elv alapján. $\checkmark$\\
Ha $0<p<1$, akkor:\\
\[\sqrt[n]{p}=\frac{1}{\sqrt[n]{\frac{1}{p}} \to 1} \to \frac{1}{1} = 1 \quad \checkmark\]
\textit{Megjegyzés}: Bernoulli-egyenlőtlenség: ha $x>-1$, akkor $(1+x)^n>1+n\cdot x$.
\end{bizNL}

\begin{tetel}\label{Nagysagrendek} $n\to\infty, \quad a>1, \quad k>0$
\[\boxed{ n^n \overset{\alpha}{\gg} n! \overset{\beta}{\gg} a^n \overset{\gamma}{\gg} n^k \gg \log n }\]
Illetve $a^n \gg b^n$, ha $a>b>1$. Továbbá $n^k \gg n^l$, ha $k>l>0$.
\end{tetel}
\textit{Megjegyzés}: $x_n \gg y_n$, ha $\displaystyle \lim_{n\to\infty} \frac{x_n}{y_y} = \infty$.
\begin{bizNL}
($\alpha$):
\[\frac{n^n}{n!} = \frac{n\cdot n\cdot \ldots \cdot n}{n(n-1)(n-3)\cdot\ldots\cdot 1} = \frac{n}{n}\cdot\frac{n}{n-1}\cdot\frac{n}{n-2}\cdot\ldots\cdot n > \underbrace{1\cdot 1\cdot \ldots \cdot n}_{\to\infty}\]
Tehát spec. rendőr-elv alapján az eredeti sorozat is a végtelenhez tart! $\checkmark$\\
\\
($\beta$): $a>1$
\[\frac{n!}{a^n}=\frac{1\cdot 2\cdot 3\cdot\ldots\cdot n}{a\cdot a\cdot a\cdot \ldots \cdot a} = \underbrace{\frac{1\cdot 2\cdot 3\cdot\ldots\cdot [a]}{a\cdot a\cdot a\cdot \ldots \cdot a}}_{\hbox{konst! $K>0$}}\cdot\underbrace{\frac{[a]+1}{a}}_{>1}\cdot\underbrace{\frac{[a]+2}{a}}_{>1}\cdot\ldots\cdot\frac{n}{a} \geqslant K\cdot 1\cdot 1\cdot\ldots\cdot \frac{n}{a} \to \infty \; \checkmark\]\\
($\gamma$): $a>1$, $k>0$
\[\frac{n^k}{a^n}=n^k\cdot\left(\frac{1}{a}\right)^n\to 0 \qquad \hbox{\ref{nKaN_Tart0}. tétel alapján ($0<\frac{1}{a}<1$)}\]
\[\Longrightarrow \frac{a^n}{n^k}\to\infty \qquad \hbox{\ref{Poz0RecpirokaTartVegtelen}. lemma alapján}\]
\end{bizNL}

\begin{tetel}$\displaystyle \boxed{\lim_{n\to\infty} \sqrt[n]{n}=1}$\end{tetel}
\begin{biz}
\[\sqrt[n]{n}=n^{\frac{1}{n}}=(e^{\ln n})^\frac{1}{n}=e^{\frac{1}{n}\cdot\ln n}\]
\[\lim_{n\to\infty} \frac{1}{n}\cdot\ln n = 0 \qquad \hbox{felhasználva a fentit}\]
\[\sqrt[n]{n} \to e^0 = 1 \quad \checkmark\]
\end{biz}

\begin{tetel} \label{ReszsorozatHatarerteke} $\boxed{$ Ha $a_n\to A$ és $a_{n_k}$ az $a_n$ sorozat egy részsorozata, akkor $a_{n_k}\to A}$ \end{tetel}
\begin{bizNL}
$\varepsilon>0$ esetén $N_a(\varepsilon)$ jó küszöbindex a részsorozathoz is!
\end{bizNL}

\subsection{Példák a fentiek használatára}

(1) $a_n=\dfrac{3^{2n}}{4^n+3^{n+1}}=\dfrac{9^n}{4^n+3\cdot 3^n}=\dfrac{1}{\left(\frac{4}{9}\right)^n+3\cdot \left(\frac{1}{3}\right)^n}\to \dfrac{1}{0+3\cdot 0} \to \infty$\\
Ennél \aref{aAzNediken} tételt használtuk fel.\\

(2) $b_n=\dfrac{n^2+9^{n+1}}{2n^5+3^{2n-1}}=\dfrac{n^2+9\cdot 9^n}{2n^5+3^{-1}\cdot 9^n}=\dfrac{\frac{n^2}{9^n}+9}{\frac{2n^5}{9^n}+3^{-1}} \to \dfrac{0+9}{0+3^{-1}} = 27$\\
Itt \aref{Nagysagrendek} tételt használtuk fel, vagyis az exponenciális függvény nagyságrendje nagyobb a hatványfüggvénynél.\\

(3) $c_n=\sqrt[3n]{n}=\sqrt[3n]{\dfrac{3n}{3}} = \dfrac{\sqrt[3n]{3n}}{\sqrt[3n]{3}} \to \dfrac{1}{1} = 1$\\
Felhasználtuk \aref{ReszsorozatHatarerteke} tételt, hiszen mindkettő $\left(\sqrt[3n]{3n}, \sqrt[3n]{3}\right)$ részsorozat.\\

(4) $d_n = \sqrt[n]{\dfrac{3n^6+8n^2}{4n^3-2n+1}}$. Rendőr-elvet használunk:
\[\sqrt[n]{\dfrac{n^2}{4n^3+n^3}} \leqslant d_n \leqslant \sqrt[n]{\dfrac{3n^6+8n^6}{4n^3-2n^3}}\]
\[\sqrt[n]{\dfrac{1}{5n}} \leqslant d_n \leqslant \sqrt[n]{\dfrac{11n^3}{2}}\]
\[\underbrace{\underbrace{\sqrt[n]{\dfrac{1}{5}}}_{\to 1}\cdot\underbrace{\dfrac{1}{\sqrt[n]{n}}}_{\to \frac{1}{1}}}_{\to 1} \leqslant d_n \leqslant \underbrace{\underbrace{\sqrt[n]{\dfrac{11}{2}}}_{\to 1}\cdot\underbrace{\left(\sqrt[n]{n}\right)^3}_{\to 1^3}}_{\to 1}\]
Tehát $d_n\to 1$.\\

(5) $e_n = \sqrt[n]{\dfrac{3^n+5^n}{2^n+4^n}}$. Rendőr-elv használatával:
\[\sqrt[n]{\dfrac{5^n}{4^n+4^n}} \leqslant e_n \leqslant \sqrt[n]{\dfrac{5^n+5^n}{4^n}}\]
\[\underbrace{\sqrt[n]{\frac{1}{2}}\sqrt[n]{\left(\dfrac{5}{4}\right)^n}}_{\to \frac{5}{4}} \leqslant e_n \leqslant \underbrace{\sqrt[n]{2}\sqrt[n]{\left(\dfrac{5}{4}\right)^n}}_{\to \frac{5}{4}}\]
Tehát $e_n \to \dfrac{5}{4}$.

\section{Rekurzív sorozatok}

Adott a kezdőelem ($a_0$) és az eggyel előre lépés szabálya (vagyis a rekurzió).
\[a_{n+1} = f(a_n, a_{n-1}, \ldots)\]

Amit vizsgálhatunk:
\begin{itemize*}
 \item Monotonitás
 \item Korlátosság
 \item Konvergencia $\Rightarrow$ határérték
\end{itemize*}

Példa: $a_0 = 2$, $a_{n+1} = 1+\sqrt{a_n}$. Sejtés: mon. nő. Bizonyítsuk teljes indukcióval:
$a_1 < a_2, a_2 < a_3 \checkmark$. Tfh: $n$-re igaz, bizonyítsuk $n+1$-re:
\[a_n \leqslant a_{n+1}\]
\[\sqrt{a_n} \leqslant \sqrt{a_{n+1}}\]
\[a_{n+1}=1+\sqrt{a_n} \leqslant 1+\sqrt{a_{n+1}} = a_{n+2}\]
Tehát igaz $\forall n$-re, tehát monoton nő a sorozatunk.\\

Ha van határérték, akkor $\{a_n\}$ és az $\{a_{n+1}\}$ sorozat is oda tart, tehát:
\[\begin{array}{ccc}
   a_{n+1} & = & 1+\sqrt{a_n}\\
   \downarrow && \downarrow\\
   A & = & 1+ \sqrt{A}
  \end{array}
\]
Ebből $A=\dfrac{3+\sqrt{5}}{2}$ vagy $A=\dfrac{3-\sqrt{5}}{2}$. Mivel a sorozatunk mon. nő ezért a határértéke biztos nagyobb mint az első elem, tehát ha van határéték, akkor csak $A=\dfrac{3+\sqrt{5}}{2}$ lehet az. Bizonyítsuk be, hogy ez egy felső korlátja a sorozatnak; Sejtés: $a_n \leqslant A = \dfrac{3+\sqrt{5}}{2}$. Ezt is teljes indukcióval bizonyítjuk. Első pár elemre igaz, tfh: $n$-re igaz, bizonyítsuk $n+1$-re:
\[\rnode{All}{a_n \leqslant \dfrac{3+\sqrt{5}}{2}}\]
\[\hbox{Biz: }1+\sqrt{a_n} \leqslant \dfrac{3+\sqrt{5}}{2}\]
\[\sqrt{a_n} \leqslant \dfrac{1+\sqrt{5}}{2}\]
\[\rnode{Biz}{a_n \leqslant \dfrac{1+5+2\sqrt{5}}{4} = \frac{3+\sqrt{5}}{2}} \nccurve[angleA=0, angleB=0]{->}{Biz}{All} \ncput*{\checkmark}\]

Összefoglalva: $\{a_n\}$ monoton nő és felülről korlátos, tehát $\displaystyle \exists \lim_{n\to\infty} a_n = \frac{3+\sqrt{5}}{2}$.

\section{Egy kitüntetett számsorozat}

\[\boxed{e_n = \left(1+\frac{1}{n}\right)^n}\]

\begin{tetel}
 $\left(1+\dfrac{1}{n}\right)^n = \boxed{e_n \leqslant e_{n+1}} = \left(1+\dfrac{1}{n+1}\right)^{n+1}$
\end{tetel}
\begin{biz}
 \[\rnode{Eredeti}{\sqrt[n+1]{\left(1+\dfrac{1}{n}\right)^n} \overset{?}{\leqslant} 1+\dfrac{1}{n+1}}\]
 \[\rnode{SzamtaniMertani}{\sqrt[n+1]{1\cdot\left(1+\dfrac{1}{n}\right)^n} \leqslant \dfrac{1+n\left(1+\frac{1}{n}\right)}{n+1}} \qquad \begin{array}{c}
  \hbox{\small Lásd számtani-mértani}\\
  \hbox{\small közti összefüggés}
\end{array}
  \nccurve[angleA=180, angleB=180, nodesep=5pt]{->}{SzamtaniMertani}{Eredeti} \ncput*{\checkmark}\]
Tehát az $e_n$ sorozat \textbf{monoton nő}.
\end{biz}

\begin{tetel}
 $e_n = \left(1+\dfrac{1}{n}\right)^n \leqslant 4$
\end{tetel}
\begin{biz}
 \[\frac{1}{2}\cdot\frac{1}{2}\cdot\left(1+\dfrac{1}{n}\right)^n \overset{?}{\leqslant} 1\]
 \[\sqrt[n+2]{\frac{1}{2}\cdot\frac{1}{2}\cdot\left(1+\dfrac{1}{n}\right)^n} \leqslant \frac{\frac{1}{2}+\frac{1}{2}+n(1+\frac{1}{n})}{n+2} = 1 = \sqrt[n+2]{1}\]
\end{biz}

Mivel az $e_n$ sorozat felülről korlátos és monoton növő, ezért $\exists \lim e_n = e$.

\begin{tetel}
 \[\boxed{\left(1+\frac{x}{n}\right)^n \xrightarrow{n\to\infty} e^x \qquad x\in\mathbb{R}}\]
\end{tetel}
\begin{biz}
 Csak speciális $x$-ekre bizonyítjuk. Legyen $x=-1$:
 \[\left(1-\frac{1}{n}\right)^n \xrightarrow{n\to\infty} ?\]
 \[\left(1-\frac{1}{n}\right)^n = \frac{1}{\left(\frac{n}{n-1}\right)^n} =  \frac{1}{\left(\frac{n-1+1}{n-1}\right)^n} = \frac{1}{\left(\frac{n-1+1}{n-1}\right)^n} = \frac{1}{\left(1+\frac{1}{n-1}\right)^n} = \]
\[ = \underbrace{\frac{1}{\left(1+\frac{1}{n-1}\right)^{n-1}}}_{\to 1/e}\cdot\underbrace{\left(\frac{n-1}{n}\right)}_{\to 1} \to \frac{1}{e}\]
 Legyen $x=\frac{1}{p}$:
 \[\left(1+\frac{1}{pn}\right)^n = \sqrt[p]{\underbrace{\left(1-\frac{1}{pn}\right)^{pn}}_{\to e}} \to \sqrt[p]{e} = e^{\frac{1}{p}} \]
 \emph{Megjegyzés}: $p.$ gyökfüggvény folytonos.
\end{biz}

\subsection{Példák}

\emph{1. példa}:
\[a_n = \left(\frac{n+6}{n+4}\right)^{n-3} = \frac{n^{n-3}}{n^{n-3}}\cdot\frac{\big(1+\frac{6}{n}\big)^{n-3}}{\big(1+\frac{4}{n}\big)^{n-3}} = \frac{\big(1+\frac{6}{n}\big)^{n}}{\big(1+\frac{4}{n}\big)^{n}}\cdot \underbrace{\frac{\big(1+\frac{6}{n}\big)^{-3}}{\big(1+\frac{4}{n}\big)^{-3}}}_{\to 1} \to \frac{e^6}{e^4} = e^2 \]

\emph{2. példa}:
\[a_n = \left(\frac{2n^2+2}{2n^2-1}\right)^{2n^2} \qquad b_n = \underbrace{\left(\frac{2n^2+2}{2n^2-1}\right)^{4n^2}}_{=a_n^2} \qquad c_n = \underbrace{\left(\frac{2n^2+2}{2n^2-1}\right)^{2n^3}}_{a_n^n} \qquad d_n = \underbrace{\left(\frac{2n^2+2}{2n^2-1}\right)^{2n}}_{\sqrt[n]{a_n}}\]
\[a_n =  \left(\frac{2n^2}{2n^2}\right)^{2n^2}\cdot \frac{\Big(1+\frac{2}{2n^2}\Big)^{2n^2}}{\Big(1-\frac{1}{2n^2}\Big)^{2n^2}} \to \frac{e^2}{e^{-1}} = e^3\]
Azért, mert $\displaystyle \left(1+\frac{2}{2n^2}\right)^{2n^2}$ egy részszorozata $\displaystyle \left(1+\frac{2}{n}\right)^{n}$-nek (lásd \aref{ReszsorozatHatarerteke} tétel).\\
Mivel $b_n = (a_n)^2$, ezért $b_n \to e^6$.\\
Tekintve, hogy $a_n \to e^3$, $\exists N_0$, hogy ha $n>N_0$, akkor $a_n > 8$. Mivel $c_n = (a_n)^n$, ezért speciális rendőrelv alapján:
\[\underbrace{8^n}_{\to \infty} < (a_n)^n = \underbrace{c_n}_{\to\infty}\]
A fenti logikából kiindulva, $\exists M_0$, hogy ha $n>M_0$, akkor $8< a_n < 27$, tehát:
\[\underbrace{\sqrt[n]{8}}_{\to 1} < \underbrace{\sqrt[n]{a_n}}_{=d_n} < \underbrace{\sqrt[n]{27}}_{\to 1}\]
Tehát rendőrelv alapján: $d_n\to 1$.

\section{További fontosabb tételek}

\begin{defi}
 Az $\{a_n\}$ sorozatban az $a_k$ elem \textbf{csúcs}, ha $\forall n>k$-ra $a_n\leqslant a_k$.
\end{defi}

\begin{tetel}
 $\boxed{$Minden sorozatnak van monoton részsorozata$}$
\end{tetel}
\begin{biz}
 Két eset lehetséges: \begin{enumerate*}
  \item Ha véges sok csúcs van, akkor a csúcsok után $\exists$ monoton növekedő részsorozat, hiszen minden elemet követ nála nagyobb vagy egyenlő (különben lenne még csúcs)
  \item Ha végtelen sok csúcs van, akkor a csúcsok monoton csökkenő részsorozatot alkotnak
 \end{enumerate*}
 Tehát mindkét esetben létezik monoton részsorozat.
\end{biz}

\begin{tetel} $\boxed{$Bolzano-Weierstrass tétel$}$\\
Minden korlátos sorozatnak van konvergens részsoro\-zata.
\end{tetel}
\begin{bizNL}
 Előző tétel alapján van monoton részsorozata, ami korlátos, hiszen az eredeti is az volt, tehát $\exists$ határértéke, azaz konvergens.
\end{bizNL}

\begin{tetel}
 $\boxed{$Cauchy-féle konvergencia kritérium$}$\\
 Az $\{a_n\}$ sorozat pontosan akkor konvergens, ha $\forall\varepsilon>0$-hoz $\exists N(\varepsilon)$, hogy
 \[|a_n-a_m|<\varepsilon \qquad n,m>N(\varepsilon)\]
 \emph{Megjegyzés}: Nem hivatkozik a határértékre!
\end{tetel}
\addtocounter{biz}{1}

Másképp megfogalmazva:
\begin{defi}
 Az $\{a_n\}$ sorozat \textbf{Cauchy-sorozat}, ha $\forall\varepsilon>0$-hoz $\exists N(\varepsilon)$, hogy
 \[|a_n-a_m|<\varepsilon \qquad n,m>N(\varepsilon)\]
\end{defi}
\begin{tetel}
 $\boxed{$Cauchy-féle konvergencia kritérium$}$\\
 \[\exists \lim_{n\to\infty} a_n = A\in\mathbb{R} \quad \Leftrightarrow \quad \{a_n\} \hbox{ Cauchy-sorozat}\]
\end{tetel}
\begin{biz}
 \begin{description*}
  \item[$\Rightarrow$] 
    \[\exists \lim_{n\to\infty} a_n = A\in\mathbb{R} \Rightarrow \forall \varepsilon>0 \hbox{-hoz } \exists N(\varepsilon) \hbox{, hogy } |a_n-A|<\varepsilon \hbox{, ha } n>N(\varepsilon)\]
  Legyen $n,m>N(\varepsilon)$, ekkor:
  \[|a_n-a_m| = |a_n-A+A-a_m| = |(a_n-A)-(a_m-A)| \leqslant |a_n-A|+|a_m-A| < 2\varepsilon\]
  \item[$\Leftarrow$] Ezt nem bizonyítjuk.
 \end{description*}
\end{biz}

Példa ennek használatára:
\[s_1 = 1\]
\[s_2 = 1+\frac{1}{2}\]
\[s_3 = 1+\frac{1}{2}+\frac{1}{3}\]
\[\vdots\]
\[s_n = 1+\frac{1}{2}+\frac{1}{3}+\ldots+\frac{1}{n} = \sum^{n}_{k=1} \frac{1}{k}\]
Létezik-e $\displaystyle \lim_{n\to\infty} s_n$? Másképp fogalmazva: Cauchy-sorozat-e?
\[s_{2n}-s_n = \left(1+\frac{1}{2}+\ldots+\frac{1}{2n}\right)-\left(1+\frac{1}{2}+\ldots+\frac{1}{n}\right) = \]
\[ = \left(\frac{1}{n+1}+\frac{1}{n+2}+\ldots+\frac{1}{2n}\right) \geqslant n\cdot\frac{1}{2n} = \frac{1}{2}\]
Tehát $\varepsilon < \dfrac{1}{2}$-hez biztosan nem létezik $N(\varepsilon)$!

\section{Torlódási pont, $\varlimsup$, $\varliminf$}

\begin{defi}
 $A\in\mathbb{R}$ $\varepsilon>0$ sugarú környezete: $K_\varepsilon(A) = (A-\varepsilon, A+\varepsilon)$.
\end{defi}

\begin{defi}
 $\infty$ környezete: $K_P(\infty) = (P, \infty)$.
\end{defi}

\begin{defi}
 $-\infty$ környezete: $K_M(-\infty) = (-\infty, M)$.
\end{defi}

\begin{defi}
 $t\in\mathbb{R}\cup\{+\infty,-\infty\}$ az $\{a_n\}$ sorozat torlódási pontja, ha $t$ minden környezetében végtelen sok eleme esik a sorozatnak.
\end{defi}

Legyen $S\subset\mathbb{R}\cup\{\pm\infty\}$ az $\{a_n\}$ sorozat torlódási pontjainak halmaza.\\

\emph{Példák}:\\
$a_n = 1$, ekkor $S=\{1\}$\\
$a_n = (-1)^n$; $S=\{1, -1\}$\\
$a_n = \dfrac{1}{n}$; $S=\{0\}$\\
$a_n = n$; $S=\{\infty\}$\\

\begin{tetel}
 \[\boxed{\lim_{n\to\infty} a_n = A \; \Leftrightarrow \; S=\{A\} \qquad A\in\mathbb{R}\cup\{\pm\infty\}}\]
Tehát $\{a_n\}$ sorozat akkor és csak akkor konvergens ha pontosan 1 torlódási pontja van.
\end{tetel}
\addtocounter{biz}{1} % nincs bizonyítás

\begin{tetel}
 Ha $S$ felülről korlátos, akkor $\exists$ legnagyobb torlódási pont ($\Sup S\in S$). Ha $S$ alulról korlátos, akkor $\exists$ legkisebb torlódási pont ($\Inf S\in S$).
\end{tetel}
\addtocounter{biz}{1} % nincs bizonyítás

\begin{tetel}
 $t$ pontosan akkor torlódási pontja $\{a_n\}$-nek, ha létezik $\{a_n\}$-nek $t$-hez konvergáló részsorozata.
\end{tetel}
\addtocounter{biz}{1} % nincs bizonyítás

\begin{defi}
 Legyen $\{a_n\}$ sorozat torlódási pontjainak halmaza $S$. Ekkor:
 \[\left.\begin{array}{c}
    \displaystyle \varlimsup_{n\to\infty} a_n = \limsup_{n\to\infty} a_n=\Sup S\\\\[-2mm]
    \displaystyle \varliminf_{n\to\infty} a_n = \liminf_{n\to\infty} a_n=\Inf S
   \end{array}\right\} \in\mathbb{R}\cup\{\pm\infty\}
  \]
\end{defi}

\begin{tetel}
 \[\boxed{\lim_{n\to\infty} a_n = A\in\mathbb{R}\cup\{\pm\infty\} \quad \Leftrightarrow \quad \varlimsup_{n\to\infty} a_n=\varliminf_{n\to\infty} a_n=A}\]
\end{tetel}
\addtocounter{biz}{1} % nincs bizonyítás

\subsubsection{Példák}

\begin{enumerate}
 \item $a_n = 2^{(-1)^n\cdot n} = \left\{\begin{array}{l}2^n \hbox{, ha $n$ páros}\\ 2^{-n} \hbox{, ha $n$ páratlan}\end{array}\right.$\\
 $S = \{0; +\infty\} \quad \rightarrow \quad \varlimsup a_n = \infty, \varliminf a_n = 0$. Mivel $\varlimsup a_n \neq \varliminf a_n$, ezért $\nexists \lim a_n$!

 \item $b_n = (-1)^n = \left\{\begin{array}{l}1 \hbox{, ha $n$ páros}\\ -1 \hbox{, ha $n$ páratlan}\end{array}\right.$\\
 $S = \{-1; 1\}$, tehát $\nexists \lim a_n$!
\end{enumerate}


\chapter{Numerikus sorok}

Numerikus sor: $a_1 + a_2 + a_3 + \ldots$\\
Összeg: $\displaystyle S = \sum^\infty_{n=1} a_n$
\begin{defi} Részletösszeg: $S_n = a_1+a_2+\ldots+a_n=\displaystyle \sum^n_{k=1}a_k$ \end{defi}
\begin{defi} A \textbf{sor összege} a részletösszeg-sorozat határértéke:
\[S=\sum^\infty_{n=1} a_n = \lim_{n\to\infty} S_n = \lim_{n\to\infty} \sum^n_{k=1} a_k\]
\end{defi}

\begin{defi} A $\displaystyle \sum^{\infty}_{n=1} a_n$ sor \textbf{konvergens}, ha $S_n$ sorozat konvergens. \end{defi}

\begin{defi} A $\displaystyle \sum^{\infty}_{n=1} a_n$ sor \textbf{divergens}, ha $S_n$ sorozat divergens. \end{defi}

Két speciális eset: $\displaystyle \sum^{\infty}_{n=1} = \pm\infty$, ha $\displaystyle\lim_{n\to\infty} S_n = \pm\infty$\\

\begin{defi}
 \[S=\sum^\infty_{n=1} a_n = \underbrace{a_1+a_2+\ldots+a_n}_{\hbox{részletösszeg: $S_n$}} + \underbrace{a_{n+1}+a_{n+2}+\ldots}_{\hbox{maradékösszeg: $r_n$}}\]
 \[r_n = \sum^\infty_{k=n+1} a_k\]
 \[S = S_n + r_n \qquad \forall n\in\mathbb{N}\]
\end{defi}

\section{Példák}

(1) $\displaystyle a_n = 1 \; \rightarrow \; \sum^\infty_{n=1} 1 = \lim_{n\to\infty} S_n = \lim_{n\to\infty} n = \infty$\\
\\
(2) $\displaystyle \sum^\infty_{n=1} (-1)^n = -1+1-1+\ldots = ?$\\
$\displaystyle S_n = \begin{cases}
  0, \hbox{ ha páros}\\
  -1, \hbox{ ha $n$ páratlan}
\end{cases} \qquad \Rightarrow \qquad \lim_{n\to\infty} S_n = \nexists \quad \hbox{div.}$\\
\\
(3) $\displaystyle \sum^\infty_{n=1} \frac{1}{n(n+1)} = ?$
\[a_n = \frac{1}{n(n+1)} = \frac{1}{n} - \frac{1}{n+1}\]
\[S = \lim_{n\to\infty}S_n = \lim_{n\to\infty}\left(\rnode{B}{\underbrace{1-\dfrac{1}{n+1}}}\right)=1\]
\[\left.\begin{array}{l}
S_1 = a_1 = \displaystyle 1-\frac{1}{2}\\   
S_2 = a_1+a_2 = \displaystyle \left(1-\frac{1}{2}\right)+\left(\frac{1}{2}-\frac{1}{3}\right)\\
S_n = a_1+a_2+\ldots+a_n = \displaystyle  \left(1-\frac{1}{2}\right)+\left(\frac{1}{2}-\frac{1}{3}\right)+\ldots+\left(\frac{1}{n}-\frac{1}{n+1}\right)
  \end{array}\right\} S_n = \rnode{A}{\overbrace{1-\frac{1}{n+1}}} \nccurve[angleB=-90,angleA=90,linestyle=dashed,linewidth=0.5pt]{->}{A}{B}\]
\textbf{Teleszkópikus összeg}!\\
\\
(4) $\displaystyle \sum^\infty_{n=0} \left(\frac{1}{2}\right)^n = 1+\frac{1}{2}+\frac{1}{4}+\ldots$.\\
Ez egy \textbf{geometriai sor} (lásd később), melynek kvóciense $\frac{1}{2}$.\\
\[S_1 = 1\]
\[S_2 = 1+\frac{1}{2} = \frac{3}{2}\]
\[S_3 = \frac{3}{2} + \frac{1}{4} = \frac{7}{4}\]
\[S_n = \frac{2^n-1}{2^{n-1}} = 2-\frac{1}{2^{n-1}} \qquad \rightarrow \hbox{teljes ind.-val bizonyítható}\]
Tehát $\displaystyle \sum^\infty_{n=0} \left(\frac{1}{2}\right)^n = \lim_{n\to\infty} S_n = \lim_{n\to\infty} 2-\frac{1}{2^{n-1}} = 2$.\\
\\
(5) \textbf{Harmonikus sor}:
\[\sum^\infty_{n=1} \frac{1}{n} = 1 + \frac{1}{2} + \frac{1}{3} + \frac{1}{4} + \ldots\]
\[S_1 = 1\]
\[S_2 = 1+\frac{1}{2}\]
\[S_4 = 1+\frac{1}{2}+\underbrace{\frac{1}{3}}_{> \frac{1}{4}}+\frac{1}{4} > 1+2\cdot\frac{1}{2}\]
\[S_8 = 1+\frac{1}{2}+\underbrace{\frac{1}{3}}_{> \frac{1}{4}}+\frac{1}{4}+\underbrace{\frac{1}{5}}_{> \frac{1}{8}}+\underbrace{\frac{1}{6}}_{> \frac{1}{8}}+\underbrace{\frac{1}{7}}_{> \frac{1}{8}}+\frac{1}{8} > 1+3\cdot\frac{1}{2}\]
\[S_{16} = 1+\frac{1}{2}+\underbrace{\frac{1}{3}}_{> \frac{1}{4}}+\frac{1}{4}+\underbrace{\frac{1}{5}}_{> \frac{1}{8}}+\ldots+\frac{1}{8}+\underbrace{\frac{1}{9}}_{> \frac{1}{16}}+\ldots+\frac{1}{16} > 1+4\cdot\frac{1}{2}\]
\[S_{2^k} \geq 1+k\cdot\frac{1}{2} \; \xrightarrow{k\to\infty}\;  \infty \]
\[\Rightarrow \; \sum^\infty_{n=1} \frac{1}{n} = \infty\]

\section{Geometriai sor}

Egymást követő tagok hányadosai állandók: $q = \dfrac{a_{n+1}}{a_n} \quad \forall n\in\mathbb{N}$. $a_n = a_0\cdot q^n$. $q\neq 0$ (lásd később). Tehát a \textbf{geometria sor}:
\[\boxed{\sum^\infty_{n=0} a_0\cdot q^n}\]

\subsection{Véges geometriai sor összege}

\[\left.\begin{array}{rcl}
  \displaystyle S_n = \sum^{n-1}_{k=0} a_k & = & a_0 + a_0\cdot q + a_0\cdot q^2 + \ldots + a_0\cdot q^{n-1} \\
  \displaystyle q\cdot S_n & = & a_0\cdot q + a_0\cdot q^2 + \ldots + a_0\cdot q^{n-1}+ a_0\cdot q^{n}
\end{array}\right\}S_n(q-1)=a_0\cdot q^{n}-a_0\]
\[\boxed{S_n = a_0 \cdot \frac{1-q^n}{1-q}}\]
Itt látható, hogy $q\neq 1$, ha $q=1$, akkor azt nem szokás geometriai sornak nevezni! (Ezért zárjuk ki a defiben)

\subsection{Végtelen geometriai sor összege}

\[S=\sum^\infty_{n=0} a_0\cdot q^n = \lim_{n\to\infty} S_n = \lim_{n\to\infty} a_0\cdot\frac{1-q^n}{1-q} = \begin{cases}
  \boxed{\dfrac{a_0}{1-q}, \quad \hbox{ha $|q|<1$}}\\
  \nexists, \quad \hbox{ha $|q|\geqslant 1$}
\end{cases}\]
\[S=\sum^\infty_{n=0} a_0\cdot q^n=a_0\cdot\frac{1}{1-q} \qquad \hbox{ha $|q|<1$}\]

\subsection{Példák}

(1) $\displaystyle \sum^{\infty}_{k=3} \frac{(-5)^{k+1}}{2^{3k+4}} = \sum^{\infty}_{k=3} \frac{-5}{2^4}\cdot\underbrace{\left(\frac{-5}{8}\right)^k}_{|q|<1\checkmark}=\underbrace{\frac{-5}{16}\cdot\left(\frac{-5}{8}\right)^3}_{=a_0}\cdot\;\frac{1}{1-\frac{-5}{8}}=\frac{5^4}{2^{13}}\cdot\frac{8}{13}=\frac{5^4}{2^{10}\cdot 13}$\\
\\
\\
(2) $\displaystyle \sum^{\infty}_{k=1} \frac{(-2)^{2k+1}+(-3)^{k+3}}{5^k} = ?$\\
\[S_n = \sum^{n}_{k=1} \frac{(-2)^{2k+1}+(-3)^{k+3}}{5^k}= \sum^{n}_{k=1} \left(\frac{(-2)^{2k+1}}{5^k} + \frac{(-3)^{k+3}}{5^k}\right) = \quad \hbox{\textbf{véges} tagú összeg, tehát:}\]
\[= \sum^{n}_{k=1} \frac{(-2)^{2k+1}}{5^k} + \sum^{n}_{k=1} \frac{(-3)^{k+3}}{5^k} = -2\cdot\sum^{n}_{k=1} \left(\frac{4}{5}\right)^k + -27\cdot\sum^{n}_{k=1} \left(\frac{-3}{5}\right)^k=\]
\[=-2\cdot\frac{4}{5}\cdot\frac{1-\left(\frac{4}{5}\right)^n}{1-\frac{4}{5}}-27\cdot\frac{-3}{5}\cdot\frac{1-\left(\frac{-3}{5}\right)^n}{1-\frac{-3}{5}}\]
\[S = \lim_{n\to\infty} S_n = \frac{-8}{5}\cdot\frac{1}{\frac{1}{5}}+\frac{81}{5}\cdot\frac{1}{\frac{8}{5}}=-8+\frac{81}{8}\]
Tehát konvergens geometriai sorokat lehet külön-külön összeadni:
\[\boxed{\lim_{n\to\infty} S_n = \lim_{n\to\infty}\left(S_n^{(1)}+S_n^{(2)}\right)=\lim_{n\to\infty}S_n^{(1)}+\lim_{n\to\infty}S_n^{(2)} = S^{(1)}+S^{(2)}}\]

\section{Tételek}

\begin{tetel}$\boxed{$Cauchy-kritérium sorokra$}$\\
$\displaystyle \sum^\infty_{n=1} a_n$ sor konvergens $\Leftrightarrow \forall \varepsilon>0$ esetén $\exists N(\varepsilon)\in\mathbb{N}$, hogy $|a_m+a_{m+1}+\ldots+a_{m+k}|<\varepsilon$, ha $m>N(\varepsilon)$ és $k\in\mathbb{N}$.
\end{tetel}
\begin{biz}
Áttérünk részletösszeg-sorozatra:
\[|a_m+a_{m+1}+\ldots+a_{m+k}|=\rnode{C}{|S_{m+k}-S_{m-1}|<\varepsilon} \qquad \hbox{ha $m>N(\varepsilon)$}\]
\[\rnode{D}{\hbox{Cauchy-kritérium az $S_n$ részletösszeg sorozatra $\Leftrightarrow \; S_n$ \rnode{E}{Cauchy-sorozat} }}
\ncline[arrowsize=4pt 0.01,doubleline=true,doublesep=0.06,linewidth=0.5pt]{<->}{D}{C}\]
\[\rnode{F}{\displaystyle \exists\lim_{n\to\infty} S_n = S = \sum^\infty_{n=1} a_n \quad} \nccurve[arrowsize=4pt 0.01,doubleline=true,doublesep=0.06,linewidth=0.5pt,angleA=-90]{<->}{E}{F}\]
\end{biz}

\begin{tetel}$\boxed{$Sor konvergenciájának szükséges feltétele$}$\\
\[\exists\sum^\infty_{n=1} a_n = S\in\mathbb{R} \quad \Rightarrow \quad \lim_{n\to\infty} a_n = 0\]
\end{tetel}
\begin{biz}
\[\exists\sum^\infty_{n=1} a_n \quad\Leftrightarrow\quad \hbox{teljesül a Cauchy-krit. a sorra}\]
\[\overset{k=0}{\Longrightarrow} \forall\varepsilon > 0 \;:\; \exists N(\varepsilon)\in\mathbb{N}\hbox{, hogy } |a_n|<\varepsilon\hbox{, ha } n>N(\varepsilon) \quad \Leftrightarrow \lim_{n\to\infty} a_n = 0\]
\end{biz}

% eggyel csökkentjük, hiszen jön egy másik bizonyítás :)
\addtocounter{biz}{-1}
\begin{biz} Másik megközelízés:
\[\begin{array}{ccccc}
S_{n+1} &=& S_n & + & a_{n+1} \\
\downarrow && \downarrow&& \downarrow\\
\infty && \infty &\Rightarrow& 0
  \end{array}\]
\end{biz}

\subsection{Példák a sor konvergenciájának szükséges feltételére}

(1) $\displaystyle \sum_{n} (-1)^n = \nexists$ mert $\displaystyle \lim_{n\to\infty} (-1)^n = \nexists \quad (\boxed{\neq 0})$.\\
\\
(2) $\displaystyle \sum_{n} 1 = \infty \; (\nexists)$ mert $\displaystyle \lim_{n\to\infty} 1 = 1 \boxed{\neq 0}$.

\section{Váltakozó előjelű sorok}

\begin{defi}
 A $\displaystyle \sum^\infty_{n=1} a_n$ sor \textbf{váltakozó előjelű}, ha $a_n\cdot a_{n+1}<0 \quad \forall n\in\mathbb{N}$-re.
\end{defi}
\emph{Például}: $\displaystyle c_1-c_2+c_3-c_4+\ldots = \sum^\infty_{n=1} \underbrace{(-1)^{n+1}\cdot c_n}_{a_n}$, ahol $c_n>0 \quad \forall n\in\mathbb{N}$.

\begin{defi}
 A $\displaystyle \sum^\infty_{n=1} a_n=\sum^\infty_{n=1} (-1)^{n+1}\cdot c_n$ sor \textbf{Leibniz-sor}, ha:
 \begin{enumerate*}
  \item Válatakozú előjelű (lásd fent)
  \item $c_n = |a_n|$ monoton csökkenő sorozat; $c_{n+1}=|a_{n+1}| \leqslant c_n = |a_n| \quad \forall n\in\mathbb{N}$
  \item $c_n = |a_n| \xrightarrow{n\to\infty} 0$
 \end{enumerate*}
\end{defi}

\begin{tetel} $\boxed{$Leibniz-kritérium$}$\\
\[\hbox{Ha } \sum^\infty_{n=1} a_n=\sum^\infty_{n=1} (-1)^{n+1}\cdot c_n \hbox{ sor Leibniz-sor, akkor konvergens}\]
\end{tetel}

\begin{biz}Csak vázlat:\end{biz}

\begin{wrapfigure}{l}{0.45\textwidth}
   \vspace{-35pt}
  \begin{center}
\psset{xunit=1.0cm,yunit=1.0cm,algebraic=true,dotstyle=o,dotsize=3pt 0,linewidth=0.8pt,arrowsize=3pt 2,arrowinset=0.25}
\begin{pspicture*}(-0.4,-0.6)(7.08,6.32)
\psaxes[xAxis=true,yAxis=true,labels=x,Dx=1,Dy=1,ticksize=-2pt 0,subticks=2]{->}(0,0)(-0.4,-0.48)(7.08,6.32)
\psline[linestyle=dotted](1,6)(2,6)
\psline[linestyle=dotted](2,1.5)(3,1.5)
\psline[linestyle=dotted](3,4)(4,4)
\psline[linestyle=dotted](4,2.25)(5,2.25)
\psline[linestyle=dotted](5,3.6)(6,3.6)
\pcline{<->}(2,1.5)(2,6)
\ncput*{$c_2$}
\pcline{<->}(3,1.5)(3,4)
\ncput*{$c_3$}
\pcline{<->}(4,2.25)(4,4)
\ncput*{$c_4$}
\pcline{<->}(5,2.25)(5,3.6)
\ncput*{$c_5$}
\pcline{<->}(6,2.5)(6,3.6)
\ncput*{$c_6$}
\psplot[linestyle=dashed,dash=2pt 2pt,plotpoints=200]{1.3}{7.0}{-3/x+3}
\psplot[linestyle=dashed,dash=2pt 2pt,plotpoints=200]{0.9}{7.0}{3/x+3}
\psdots[dotstyle=*](1,6)
\rput[bl](0.6,5.68){$S_1$}
\psdots[dotstyle=*](2,1.5)
\rput[bl](1.94,1.1){$S_2$}
\psdots[dotstyle=*](3,4)
\rput[bl](2.94,4.22){$S_3$}
\psdots[dotstyle=*](4,2.25)
\rput[bl](3.9,1.8){$S_4$}
\psdots[dotstyle=*](5,3.6)
\rput[bl](4.92,3.8){$S_5$}
\psdots[dotstyle=*](6,2.5)
\rput[bl](6.04,2.08){$S_6$}
\psdots[dotstyle=*](2,6)
\psdots[dotstyle=*](3,1.5)
\psdots[dotstyle=*](4,4)
\psdots[dotstyle=*](5,2.25)
\psdots[dotstyle=*](6,3.6)
\end{pspicture*}
\end{center}
 \vspace{-80pt}
\end{wrapfigure}

$S_1 = a_1 = c_1$\\
$S_2 = a_2 = c_1-c_2=S_1-c_2$\\
$S_3 = S_2+c_3$\\
$S_4 = S_3-c_3$\\
$S_5 = S_4+c_4$\\
$\ldots$

\begin{enumerate}
\renewcommand{\theenumi}{\pscirclebox[boxsep=false,linewidth=0.3pt]{\arabic{enumi}}}

 \item $\left.\begin{array}{cl}
         S_1 \geqslant S_3 \geqslant S_5 \geqslant \ldots & \hbox{mon. csökken}\\
	 S_{2k+1} \geqslant S_2 & \hbox{alulról korlátos}
        \end{array}\right\} \Rightarrow$\\
       $\displaystyle \Rightarrow \; \exists \lim_{k\to\infty} S_{2k+1}=S^*$

 \item $\left.\begin{array}{cl}
         S_2 \leqslant S_4 \leqslant S_6 \leqslant \ldots & \hbox{mon. nő}\\
	 S_{2k} \leqslant S_1 & \hbox{felülről korlátos}
        \end{array}\right\} \Rightarrow$\\
       $\displaystyle \Rightarrow \; \exists \lim_{k\to\infty} S_{2k}=S_*$

\end{enumerate}
% meg kell törni a felsorolást, mert csak úgy lesz szép :)

\begin{enumerate}
\renewcommand{\theenumi}{\pscirclebox[boxsep=false,linewidth=0.3pt]{\arabic{enumi}}}
\setcounter{enumi}{2}
  
 \item $\begin{array}{ccccc}
S_{2k+1} &=& S_{2k} & + & c_{2k+1} \\
\downarrow && \downarrow&& \downarrow\\
S^* && S_* && 0
  \end{array}$ \\
 
  $\rnode{LeibnizSorHatErtek}{\pscirclebox[boxsep=false]{S}} = S^* = S_*$\\
  \hspace*{30pt} \rnode{LeibnizSorHatErtekSzoveg}{Ez lesz a határérték.} \psset{nodesep=5pt}\nccurve[angleA=-90,angleB=180]{->}{LeibnizSorHatErtek}{LeibnizSorHatErtekSzoveg}

\end{enumerate}

\subsection{Példa}

\[1-\frac{1}{2}+\frac{1}{3}-\frac{1}{4}+\ldots = \sum^\infty_{n=1} (-1)^{n+1} \frac{1}{n} \qquad \hbox{konvergens-e?}\]
Leibniz-kritérium ellenőrzése:\vspace*{8pt}\\
$\left.\begin{array}{l}
 1. \; \hbox{Váltakozó? } $\checkmark$\\
 2. \; c_n=\frac{1}{n} \geqslant c_{n+1}=\frac{1}{n+1} \; \checkmark\\
 3. \; \lim_{n\to\infty} c_n = \lim_{n\to\infty} \frac{1}{n} = 0 \; \checkmark\\
\end{array}\right\} \hbox{konvergens } \checkmark$

\section{Hiba, hibabecslés}

\begin{defi}
 Az $\displaystyle S_n = \sum^n_{k=1} a_k$ részletösszeg \textbf{hibája}:
\[H_n = |S-S_n|, \hbox{ ahol } S=\sum^\infty_{n=1} a_n \hbox{ az egzakt összeg.}\]
\end{defi}

\emph{Hibabecslés}: $H_n = |S-S_n| < $ \emph{\underline{becslés}}

\begin{tetel} $\boxed{$Hibabecslés Leibniz-típusú sor esetén$}$\\
Az $S_n$ közelítő összeg hibája $\leqslant$, mint az első elhagyott tag abszolút értéke:
\[|S-S_n| \leqslant |a_{n+1}| = c_{n+1}\]
\end{tetel}
\begin{biz}
Az előzőekben láttuk, hogy: $S_{2k} \leqslant S \leqslant S_{2k+1}$, ebből:
\[|S-S_{2k}| \leqslant |S_{2k+1}-S_{2k}| = c_{2k+1}\]
Hasonlóan: $S_{2k+2} \leqslant S \leqslant S_{2k+1}$, ebből:
\[|S-S_{2k+1}| \leqslant |S_{2k+2}-S_{2k+1}| = c_{2k+2}\]
Tehát valóban igaz, hogy:
\[|S-S_n| \leqslant c_{n+1} = |a_{n+1}|\]
\end{biz}

\subsection{Példák}

(1) $\displaystyle \sum^\infty_{n=1} (-1)^n \frac{1}{\sqrt{n}+3}$. Konvergens-e? Milyen $n$-re lesz $|S-S_n|\leqslant 10^{-3}$?\\
Leibniz-típusú-e?:\vspace*{8pt}\\
$\left.\begin{array}{l}
 1. \; \hbox{Váltakozó? } $\checkmark$\\
 2. \; c_n\geqslant c_{n+1} \hbox{, hiszen } \sqrt{n} \leqslant \sqrt{n+1} \; \checkmark\\
 3. \; c_n \xrightarrow{n\to\infty} 0 \hbox{, mert } \sqrt{n} \to \infty \; \checkmark\\
\end{array}\right\} \hbox{konvergens } \checkmark$\\
Hiba: $|S-S_n| \leqslant c_{n+1} = \dfrac{1}{\sqrt{n+1}+3} \leqslant 10^{-3}$\\
\[\sqrt{n+1}+3 \geqslant 1000 \quad \Rightarrow \quad n \geqslant 997^2-1\]

(2) $\displaystyle \sum^\infty_{k=2} \frac{\cos(k\pi)}{\ln k}$. Konvergens-e? Milyen $n$-re lesz $|S-S_n|\leqslant 10^{-3}$?\\
$\cos(k\pi) = \begin{cases}
		+1 \hbox{, ha $k$ páros}\\
		-1 \hbox{, ha $k$ páratlan}
              \end{cases} \quad \Rightarrow \cos(k\pi) = (-1)^k$\\
Tehát $\displaystyle \sum^\infty_{k=2} \frac{\cos(k\pi)}{\ln k} = \displaystyle \sum^\infty_{k=2} (-1)^k\cdot \frac{1}{\ln k}$\\
Leibniz-kritérium ellenőrzése:\vspace*{8pt}\\
$\left.\begin{array}{l}
 1. \; \hbox{Váltakozó? } $\checkmark$\\
 2. \; c_n\geqslant c_{n+1} \hbox{, hiszen $\ln n$ monoton nő} \; \checkmark\\
 3. \; c_n \xrightarrow{n\to\infty} 0 \hbox{, mert } \ln n \to \infty \; \checkmark\\
\end{array}\right\} \hbox{konvergens } \checkmark$\\
Hiba: $|S-S_n| \leqslant c_{n+1} = \dfrac{1}{\ln(n+1)} \leqslant 10^{-3}$\\
\[\ln(n+1) \geqslant 1000 \quad \Rightarrow \quad n \geqslant e^{1000}-1\]

(3) $\displaystyle \sum^\infty_{k=2} \frac{(-1)^{k-1}\cdot 2k}{k^2-1}$. Konvergens-e?\\
\begin{itemize*}
 \item Alternáló? $\checkmark$
 \item $c_k = |a_k| = \dfrac{2k}{k^2-1} = \dfrac{2}{k-\frac{1}{k}} \xrightarrow{k\to\infty} 0 \; \checkmark$
 \item $\rnode{Felteves}{c_k \overset{?}{\geqslant} c_{k+1}}$
    \[\frac{2k}{k^2-1} \overset{?}{\geqslant} \frac{2(k+1)}{(k+1)^2-1}\]
    \[(2k)((k+1)^2-1) \overset{?}{\geqslant} (2k+2)(k^2-1)\]
    \[\rnode{Belatva}{2k^2+2k+2 \overset{?}{\geqslant} 0 \; \checkmark} \nccurve[angleA=-90, angleB=-90, linewidth=0.5pt,nodesep=5pt,arrowsize=5pt 0.5]{->}{Belatva}{Felteves} \ncput*[npos=0.8]{\checkmark}
    % a vonalat törjük meg ott ahol a szöveg van.
    \ncput*[npos=0.53]{~} \ncput*[npos=0.50]{~} \ncput*[npos=0.47]{~} \ncput*[npos=0.44]{~}\]
\end{itemize*}
Tehát a fenti sor Leibniz-sor, tehát konvergens.\\
\\
\emph{Megjegyzés}: A konvergenciához elég, ha a Leibniz-kritérium $n>N_0$-ra teljesül.

\section{Abszolút és feltételes konvergencia}

\begin{defi}
 A $\displaystyle \sum^\infty_{n=1} a_n$ sor \textbf{abszolút konvergens}, ha $\displaystyle \sum^\infty_{n=1} |a_n|$ konvergens.
\end{defi}

\begin{defi}
 A $\displaystyle \sum^\infty_{n=1} a_n$ sor \textbf{feltételesen konvergens}, ha konvergens, de nem abszolút konvergens.
\end{defi}

Pl.: $\displaystyle \sum^\infty_{n=0} \left(\frac{-1}{2}\right)^n$ konvergens, sőt abszolút konvergens.\\
$\displaystyle \sum^\infty_{n=1} (-1)^{n+1}\cdot\frac{1}{n}$ konvergens (Leibniz-sor), de nem abszolút.\\

\begin{tetel}
 Ha $\displaystyle \sum^\infty_{n=1} a_n$ abszolút konvergens $\Rightarrow$ $\displaystyle \sum^\infty_{n=1} a_n$ konvergens.
\end{tetel}

\begin{biz}
\[\sum^\infty_{n=1} a_n \hbox{ abszolút konvergens } \overset{\hbox{\scriptsize def}}{\Longleftrightarrow} \displaystyle \sum^\infty_{n=1} |a_n| \hbox{ konvergens } \overset{\hbox{\scriptsize Cauchy-krit.}}{\Longrightarrow} \forall \varepsilon>0 \;:\; \exists N(\varepsilon)>0 \hbox{, hogy:}\]
\[\big|\underbrace{|a_m|+|a_{m+1}|+|a_{m+2}|+\ldots+|a_{m+k}|}_{\text{\small $\geqslant \; a_m+a_{m+1}+a_{m+2}+\ldots+a_{m+k}$}}\big|<\varepsilon \hbox{, ha } m>N(\varepsilon), k\in\mathbb{N} \qquad \Rightarrow\]
\[\Rightarrow \quad \forall \varepsilon>0 \;:\; \exists N(\varepsilon)>0 \hbox{, hogy: } |a_m+a_{m+1}+a_{m+2}+\ldots+a_{m+k}|<\varepsilon \hbox{, ha } m>N(\varepsilon), k\in\mathbb{N}\]
\[\Updownarrow \hbox{\scriptsize Cauchy-kritérium}\]
\[\sum^\infty_{n=1} a_n \hbox{ konvergens}\]
\end{biz}

\section{Pozitív tagú sorok}

\begin{defi}$\displaystyle \sum^\infty_{n} a_n$ pozitív tagú sor, ha $a_n > 0 \quad \forall n\in\mathbb{N}$\end{defi}

Tulajdonságok:
\begin{enumerate}
 \item pozitív tagú sorok részlet-összegei monoton nőnek: $S_n \leqslant S_{n+1} = S_n + \underbrace{a_{n+1}}_{>0} \quad \checkmark$
 
 \item pozitív tagú sor konvergens $\Longleftrightarrow$ a részlet-összeg sorozat korlátos. \\
      Biz $\Rightarrow$: Ha $\displaystyle \sum^\infty_{n} a_n$ konvergens, akkor $\displaystyle \exists \lim_{n\to\infty} S_n = S \quad \Rightarrow \quad S_n$ korlátos (minden konvergens sorozat korlátos).\\

      Biz $\Leftarrow$:\\
      $\left.\begin{array}{l}
	S_n < k \in \mathbb{R} \\
	S_n \nearrow
       \end{array}\right\} \quad \Rightarrow \quad \exists \displaystyle \lim_{n\to\infty} S_n = S = \sum^\infty_{n} a_n \quad \checkmark$
\end{enumerate}

Pozizív tagú sorok esetén:
\[\sum^\infty_{n} a_n \hbox{ konvergens } \quad \Longleftrightarrow \quad \sum^\infty_{n} a_n < \infty\]
\[\sum^\infty_{n} a_n \hbox{ divergens } \quad \Longleftrightarrow \quad \sum^\infty_{n} a_n = \infty\]

\begin{tetel}
\[\boxed{ \sum^\infty_{n=1} \frac{1}{n^\alpha} = \begin{cases}
   x\in\mathbb{R} \hbox{, ha } \alpha > 1 \\
   \infty \hbox{, ha } \alpha \leqslant 1
\end{cases}}\]
\end{tetel}
\begin{biz}$\varnothing$ Majd félév végén.\end{biz}

\begin{tetel}$\boxed{$Majoráns kritérium$}$\\
$\forall n\in\mathbb{N}$ esetén $0<a_n\leqslant c_n$ és $\displaystyle\sum^\infty_{n=1} c_n < \infty$ (konvergens), akkor $\displaystyle\sum^\infty_{n=1} a_n < \infty$ (konvergens).
\end{tetel}
\begin{biz}
\[0<a_n\leqslant c_n \quad \Rightarrow \quad S_n^{(a)} \leqslant S_n^{(c)} \leqslant \sum^\infty_{n=1} c_n = S^{(c)} \in \mathbb{R}\]
\[\Rightarrow \quad \exists \sum^\infty_{n=1} a_n = \lim_{n\to\infty} S_n^{(a)} = S^{(a)} \leqslant S^{(c)}\]
\end{biz}


\begin{tetel}$\boxed{$Minoráns kritérium$}$\\
 $\forall n\in\mathbb{N}$ esetén $0\leqslant d_n\leqslant a_n$ és $\displaystyle\sum^\infty_{n=1} d_n = \infty$ (divergens), akkor $\displaystyle\sum^\infty_{n=1} a_n = \infty$ (divergens).
\end{tetel}
\begin{biz}
 \[\begin{array}{ccl}
    S_n^{(d)} & \leqslant & S_n^{(a)}\\
    \downarrow & & \downarrow \\
    \infty & \Rightarrow & \infty \quad \hbox{\small(speciális rendőrelv)}
   \end{array}\]
\end{biz}

\subsection{Példa}

(1) $\displaystyle \sum^\infty_{n=1} \frac{1}{3n+2}$. Konvergens-e?\\
A konvergencia szükséges feltétele teljesül: $a_n \xrightarrow{n\to\infty} 0$. Minoráns kritérium:
\[\sum^\infty_{n=1} \frac{1}{3n+2} \geqslant \sum^\infty_{n=1} \frac{1}{3n+2n} = \underbrace{\sum^\infty_{n=1} \frac{1}{5}\cdot\frac{1}{n}}_{\hbox{divergens}}\]
Tehát az eredeti sor is divergens.\\

(2) $\displaystyle \sum^\infty_{n=2} \frac{1}{n^2-\sqrt{n}}$. Konvergens-e?\\
A konvergencia szükséges feltétele teljesül: $a_n \xrightarrow{n\to\infty} 0$. Majoráns kritérium:
\[\sum^\infty_{n=2} \frac{1}{n^2-\sqrt{n}} \leqslant \sum^\infty_{n=2} \frac{1}{n^2-\frac{1}{2}n^2} = \underbrace{\sum^\infty_{n=2} \frac{2}{n^2}}_{\hbox{konvergens}}\]
Tehát az eredeti sor is konvergens.\\

\subsection{Nem Leibniz-típusú pozitív tagú sorok hibabecslése}

Ötlet: a sort konvergens geometriai sorral majoráljuk:
\[H_n = |S-S_n| = |r_n| = \sum^\infty_{k=n+1} a_k \overset{a_k\leqslant Aq^k}{\leqslant}
  \sum^\infty_{k=n+1} A\cdot q^k = A\cdot q^{n+1} \frac{1}{1-q} \qquad \hbox{ha $|q|<1$}
\]

\subsubsection{Példák}

(1) $\displaystyle \sum^\infty_{n=1} \underbrace{\frac{2^n}{4^n-3}}_{0<a_n}$. Adjunk az $s\approx s_{1000}$ közelítés hibájára becslést!\\
\[H = \sum^\infty_{n=1001} \frac{2^n}{4^n-3} \leqslant \sum^\infty_{n=1001} \frac{2^n}{4^n-\frac{1}{2}4^n} = \sum^\infty_{n=1001} 2\cdot\left(\frac{1}{2}\right)^n = 2\cdot\left(\frac{1}{2}\right)^{1001}\cdot\frac{1}{1-\frac{1}{2}} = \left(\frac{1}{2}\right)^{999}\]
Tehát $H \leqslant \left(\dfrac{1}{2}\right)^{999}$\\
\\

(2) $\displaystyle \sum^\infty_{n=1} \underbrace{\frac{5^n+n^2\cdot 2^n}{3^n+8^n}}_{0<a_n}$. Adjunk az $s\approx s_{100}$ közelítés hibájára becslést!\\
\[H = \sum^\infty_{n=101} \frac{5^n+n^2\cdot 2^n}{3^n+8^n} \leqslant \sum^\infty_{n=101} \frac{5^n+5^n}{8^n} = \sum^\infty_{n=101} 2\cdot \left(\frac{5}{8}\right)^n = 2\cdot \left(\frac{5}{8}\right)^{101}\cdot\frac{1}{1-\frac{5}{8}} = \frac{16}{3}\cdot \left(\frac{5}{8}\right)^{101}\]
Tehát $H \leqslant \dfrac{16}{3}\cdot \left(\dfrac{5}{8}\right)^{101}$

\chapter{Valós egyváltozós függvények}

\begin{defi}
 Valós egyváltozójú \textbf{függvény}: egyértelmű reláció. $f: D_f \to R_f \quad \forall x\in D_f\subset \mathbb{R}$-hez hozzárendel pontosan egy $y\in R_f\subset\mathbb{R}$-et. $D_f$: értelmezési tartomány (domain), ÉT; $R_f:$ értékkészlet (range), ÉK.
\end{defi}

\begin{defi}
 Az $f: A\to B$ leképezés \textbf{szürjektív} (ráképezés), ha $\forall b\in B$-re $\exists a\in A$, hogy $f(a)=b$.
\end{defi}

\begin{defi}
 Az $f: A\to B$ leképezés \textbf{injektív}, ha $f(a_1) = f(a_2)$ esetén $a_1=a_2$, tehát $\forall b\in B$-re legfeljebb egy $A$-beli elemre ($a\in A$) teljesül, hogy $f(a)=b$.
\end{defi}

\begin{defi}
 Az $f: A\to B$ leképezés \textbf{bijektív} (egy-egy értelmű), ha $f$ szürjektív és injektív, azaz $\forall b\in B$ esetén $\exists!\; a\in A$, hogy $f(a)=b$. Ilyenkor $|A|=|B|$.
\end{defi}

\begin{defi}
 Legyen $f$ injektív. Ekkor $f$ \textbf{inverze} $f^{-1}: R_f \to A$, $b\mapsto f^{-1}(b) = a$, melyre $f(a)=b$.
\end{defi}

\section{Topológiai alapfogalmak}

Legyen $H\subset \mathbb{R}$
\begin{defi}
 $b\in H$ a $H$ \textbf{belső pont}ja, ha $\exists \varepsilon >0$, hogy $(b-\varepsilon,b+\varepsilon)=K_{\varepsilon}(b) \subset H$. Jelölés: $\Int(H)$
\end{defi}
\begin{defi}
 $k\in \mathbb{R}\setminus H$ a $H$ \textbf{külső pont}ja, ha $\exists \varepsilon >0$, hogy $(k-\varepsilon,k+\varepsilon)=K_{\varepsilon}(k) \subset \mathbb{R}\setminus H$.
\end{defi}

Tehát az előző kettőből: $k$ a $H$ külső pontja $\Leftrightarrow$ $k$ az $\mathbb{R}\setminus H$ belső pontja.

\begin{defi}
 $h\in \mathbb{R}$ a $H$ \textbf{határpont}ja, ha nem külső és nem belső pontja, azaz $\forall \varepsilon >0$ esetén $K_{\varepsilon}(h)\cap H\neq \emptyset$ és $K_{\varepsilon}(h)\cap (H\setminus \mathbb{R}) \neq \emptyset$. Jelölés: $\Front(H)$
\end{defi}

\begin{defi}
 $H\subset \mathbb{R}$ \textbf{nyílt}, ha $H$ minden pontja belsőpont. ($\Int(H)=H$)
\end{defi}

\begin{defi}
 $H\subset \mathbb{R}$ \textbf{zárt}, ha $\mathbb{R}\setminus H$ nyílt.
\end{defi}

Például $(a;b]$ se nem nyílt, se nem zárt.

\begin{tetel}
 $\boxed{\mathbb{R}$ és $\emptyset$ nyílt és zárt is.$}$
\end{tetel}
\addtocounter{biz}{1}

\begin{defi}
 $H\subset \mathbb{R}$ \textbf{kompakt}, ha $H$ korlátos és zárt.
\end{defi}

\section{Függvény tulajdonságok}

\begin{defiNL}
 $f$ \textbf{felülről korlátos}, ha $\exists K\in\mathbb{R}$, hogy $\forall x\in D_f$ esetén $f(x)\leqslant K$.\\
 $f$ \textbf{alulról korlátos}, ha $\exists k\in\mathbb{R}$, hogy $\forall x\in D_f$ esetén $f(x)\geqslant k$.\\
 $f$ \textbf{korlátos}, ha $\exists K\in\mathbb{R}$, hogy $\forall x\in D_f$ esetén $|f(x)|\leqslant k$.
\end{defiNL}

\begin{defiNL}
 $f$ \textbf{páros}, ha $f(x)=f(-x)$.\\
 $f$ \textbf{páratlan}, ha $f(x)=-f(-x)$.
\end{defiNL}

\begin{defiNL}
 $f$ \textbf{monoton nő}, ha $x_1<x_2$ esetén $f(x_1)\leqslant f(x_2)$.\\
 $f$ \textbf{monoton csökken}, ha $x_1<x_2$ esetén $f(x_1)\geqslant f(x_2)$.\\
 $f$ \textbf{szigorúan monoton nő}, ha $x_1<x_2$ esetén $f(x_1) < f(x_2)$.\\
 $f$ \textbf{szigorúan monoton csökken}, ha $x_1<x_2$ esetén $f(x_1) > f(x_2)$.
\end{defiNL}

\begin{defi}
 $f$ \textbf{periodikus}, ha $\exists T>0$, hogy $f(x+T)=f(T) \quad \forall x\in D_f$. A periódusa legyen a legkisebb ilyen $T$ érték.
\end{defi}

\begin{defiNL}
 A $t\in\mathbb{R}$ $\varepsilon>0$ sugarú környezete: $K_\varepsilon(t) = (t-\varepsilon, t+\varepsilon)$.\\
 A $t\in\mathbb{R}$ $\varepsilon>0$ pontozott sugarú környezete: $\dot{K}_\varepsilon(t) = K_\varepsilon(t)\setminus\{t\}$.\\
\end{defiNL}

\begin{defi}\label{Torlodasi1}
 $t\in\mathbb{R}$ a $H\subset \mathbb{R}$ \textbf{torlódási pontja}, ha $\forall\varepsilon>0$ esetén $|K_\varepsilon(t)\cap H|=\infty$, azaz $\forall K_\varepsilon(t)$ környezetbe végtelen sok $H$-beli pont esik.
\end{defi}

\begin{defi}\label{Torlodasi2}Alternatív definíció:\\
 $t\in\mathbb{R}$ a $H\subset\mathbb{R}$ \textbf{torlódási pontja}, ha $\forall\varepsilon>0$ esetén $\dot{K}_\varepsilon(t)\cap H\neq \emptyset$ (bármely pontozott környezetbe esik $H$-beli elem).
\end{defi}

\begin{tetel}
 A fenti két definíció ekvivalens (\ref{Torlodasi1} $\Leftrightarrow$ \ref{Torlodasi2})
\end{tetel}
\begin{biz}
 \begin{description*}
  \item[\ref{Torlodasi1} $\Rightarrow$ \ref{Torlodasi2}] 
    \[|K_\varepsilon(t)\cap H|=\infty \quad \Rightarrow \quad |\dot{K}_\varepsilon(t)\cap H|=\infty \quad \Rightarrow \quad \dot{K}_\varepsilon(t)\cap H \neq \emptyset\]
  \item[\ref{Torlodasi2} $\Rightarrow$ \ref{Torlodasi1}] 
    Egyre szűkülő környezetet veszünk: $\varepsilon > \varepsilon_2 > \ldots > \varepsilon_n > 0$
    \[\dot{K}_{\varepsilon}(t)\cap H \neq \emptyset \quad \Rightarrow \quad \exists h_1\in\dot{K}_{\varepsilon}(t)\cap H\]
    \[\dot{K}_{\varepsilon_2}(t)\cap H \neq \emptyset \quad \Rightarrow \quad \exists h_2\in\dot{K}_{\varepsilon_2}(t)\cap H\]
    \[\vdots\]
    \[\Longrightarrow \{h_n\}\subset\dot{K}_{\varepsilon}(t)\cap H \quad \Rightarrow \quad |K_\varepsilon(t)\cap H|=\infty\]
 \end{description*}
\end{biz}

\begin{tetel}
  $t\in\mathbb{R}$ a $H\subset \mathbb{R}$ torlódási pontja, ha $\exists h_n\in H, h_n \neq t$, hogy $\displaystyle \lim_{n\to\infty} h_n = t$.
\end{tetel}
\addtocounter{biz}{1}

\section{Függvények határértéke}

\begin{defi}
 Az $f$ függvény \textbf{határértéke} $x_0$-ban $A$, jelölve: $\displaystyle \lim_{x\to x_0} f(x) = A$, ha
 \begin{enumerate*}
  \item $x_0$ torlódási pontja $D_f$-nek és
  \item $\forall \varepsilon>0$ esetén $\exists \delta(\varepsilon)>0$, hogy
  % másképp: (\dot{K}_\delta(x_0)\cap D_f \neq \emptyset)
  $|f(x)-A|<\varepsilon \hbox{, ha } x\in D_f \hbox{ és } 0<|x-x_0|<\delta$
 \end{enumerate*}
\emph{Megjegyzés}: Nem kell, hogy $x_0\in D_f$. $x_0$-ban felvett függvényérték nem befolyásolja a határértéket.
\end{defi}

\begin{defi}
 Az $f$ függvény \textbf{jobboldali határértéke} $x_0$-ban $A$, ha
 \begin{enumerate*}
  \item $x_0$ torlódási pontja $D_f\cap(x_0,+\infty)$-nek és
  \item $\forall \varepsilon>0$ esetén $\exists \delta(\varepsilon)>0$, hogy
  $|f(x)-A|<\varepsilon \hbox{, ha } x\in D_f \hbox{ és } 0<x-x_0<\delta$
 \end{enumerate*}
\emph{Jelölés}: $\displaystyle \lim_{x\to x_0+0} f(x) = A$ vagy $f(x_0+0) = A$
\end{defi}

\begin{defi}
 Az $f$ függvény \textbf{baloldali határértéke} $x_0$-ban $A$, ha
 \begin{enumerate*}
  \item $x_0$ torlódási pontja $D_f\cap(-\infty,x_0)$-nak és
  \item $\forall \varepsilon>0$ esetén $\exists \delta(\varepsilon)>0$, hogy
  $|f(x)-A|<\varepsilon \hbox{, ha } x\in D_f \hbox{ és } 0<x_0-x<\delta$
 \end{enumerate*}
\emph{Jelölés}: $\displaystyle \lim_{x\to x_0-0} f(x) = A$ vagy $f(x_0-0) = A$
\end{defi}

\emph{Megjegyzés}: Legyen $x_0$ a $D_f$ belső pontja, ekkor:
\begin{tetel}\label{HatarertekJobbEsBal}
 $\displaystyle \lim_{x\to x_0} f(x)=A \; \Leftrightarrow \; f(x_0+0)=f(x_0-0)=A$
\end{tetel}
\addtocounter{biz}{1}

\subsubsection{Példák függvények határértékeinek, definícióval történő meghatározására}

\begin{enumerate}
\renewcommand{\theenumi}{\pscirclebox[boxsep=false,linewidth=0.5pt,framesep=1pt]{\arabic{enumi}}}

 \item Definícióval igazoljuk, hogy: \[\lim_{x\to 2} \sqrt{2x+5} = 3 \qquad (\delta(\varepsilon) = ?)\]
  $D_f = \left[-\frac{2}{5},\infty\right]$
  \[|\sqrt{2x+5}-3|=\left|\frac{(\sqrt{2x+5}-3)(\sqrt{2x+5}+3)}{\sqrt{2x+5}+3}\right| = \left|\frac{2x-4}{\sqrt{2x+5}+3}\right| = \frac{2|x-2|}{\sqrt{2x+5}+3} \leqslant\]
  \[\leqslant \frac{2}{3}\cdot |x-2| < \varepsilon \qquad \hbox{teljesül, ha } |x-x_0|<\delta(\varepsilon)\]
  Tehát $\boxed{\delta(\varepsilon) = \frac{3\varepsilon}{2}}$.

 \item Definícióval igazoljuk, hogy: \[\lim_{x\to 2} (x^2-3x+1) = -1 \qquad (\delta(\varepsilon) = ?)\]
  \[|x^2-3x+1--1| = |x^2-3x+2| = |x-1||x-2|\]
  Nyugodtan feltehetjük, hogy $\delta(\varepsilon)\leqslant 1$, ekkor $|x-1|< 2$, tehát:
  \[|x-1||x-2| < 2|x-2| < \varepsilon\]
  Innen: $\delta(\varepsilon)=\min\left\{\dfrac{\varepsilon}{2}, 1\right\}$. (Az 1 a fenti feltevés miatt kell)\\
  De akár feltehetjük, azt is, hogy $\delta(\varepsilon)\leqslant \dfrac{1}{2}$, ekkor $|x-1|< \dfrac{3}{2}$, így:
  \[|x-1||x-2| < \dfrac{3}{2}|x-2| < \varepsilon\]
  Tehát ebből $\delta(\varepsilon)=\min\left\{\dfrac{2\varepsilon}{3}, \dfrac{1}{2}\right\}$.

 \item Definícióval igazoljuk, hogy: \[\lim_{x\to 3} \left(\frac{x-3}{x^2-9}\right) = \frac{1}{6} \qquad (\delta(\varepsilon) = ?)\]
  $D_f = \mathbb{R}\setminus\{\pm3\}$
  \[\left|\frac{x-3}{x^2-9}-\frac{1}{6}\right| = \left|\frac{-x^2+6x-9}{6x^2-54}\right| = \left|\frac{(x-3)^2}{6(x+3)(x-3)}\right| = \frac{|x-3|}{6|x+3|}\]
  Tegyük fel, hogy $\delta(\varepsilon)\leqslant 1$, ekkor $\dfrac{1}{|x+3|}< \dfrac{1}{5}$ (hiszen $5 < |x+3| < 7$). Tehát:
  \[\frac{|x-3|}{6|x+3|} < \frac{|x-3|}{30} < \varepsilon\]
  Így $\delta(\varepsilon)=\min\left\{30\varepsilon,1\right\}$.
\end{enumerate}

\begin{tetel}$\boxed{$Átviteli-elv$} \qquad$ (Szükséges és elégséges feltétel határérték létezésére)\\
 \[\boxed{\lim_{x\to x_0} f(x) = A \quad \Leftrightarrow \quad
    \begin{array}{c}
      \forall \{x_n\} \xrightarrow{n\to\infty} x_0 \hbox{ sorozatra, melyre } x_n\in D_f\setminus\{x_0\}\\
      \displaystyle\lim_{n\to\infty} f(x_n) = A \quad \hbox{(sorozat határérték)}
    \end{array}}\]
\end{tetel}
\emph{Megjegyzés}: Főleg határérték \textbf{nem létezésére} használjuk, mert ahhoz, hogy bebizonyítsuk ennek segítségével, hogy létezik hatáérték, végtelen sorozatot kell megvizsgálni.
\begin{biz}
 \begin{description}
  \item[$\Rightarrow$] Teljesül, hogy $|f(x)-A|<\varepsilon$, ha $0<|x-x_0|<\delta(\varepsilon)$ és $\displaystyle \lim_{n\to\infty} x_n = x_0$. Ebből:
 \[|x_n-x_0|<\delta(\varepsilon) \quad \hbox{ha } n>N_1(\delta(\varepsilon))\]
 De ekkor:
 \[|f(x_n)-A|<\varepsilon \quad \hbox{ha } n>N_1(\delta(\varepsilon))[=N(\varepsilon)]\]
 Ez pedig pontosan azt jelenti, hogy $f(x_n) \to A$.
  \item[$\Leftarrow$] $\forall\{x_n\}\to x_0$-ra $f(x_n)\to A$. Következik-e, hogy $f(x)\to A$? Indirekt bizonyítjuk: Tegyük fel, hogy $\exists\varepsilon >0$, melyre $\nexists\delta(\varepsilon)$, hogy:
\[|f(x)-A|<\varepsilon\hbox{, ha } 0<|x-x_0|<\delta(\varepsilon)\]
Tehát bármilyen $\delta$-ra:
\[|f(x)-A|\geqslant\varepsilon\hbox{, ha } 0<|x-x_0|<\delta\]
Vagyis $\delta=\dfrac{1}{m}$-re ($m\in\mathbb{N}^+$) is igaz, vagyis $\exists x_m$, hogy:
\[0<|x_m-x_0|<\frac{1}{m} \qquad \hbox{ hiszen feltétel alapján } \forall x_m\to x_0\]
De $|f(x_m)-A| \geqslant \varepsilon$. Ez viszont ellentmondás, hiszen ekkor $f(x_m)\nrightarrow A$.
 \end{description}

\end{biz}

Példa: $\displaystyle \lim_{x\to 0} \cos\left(\frac{1}{x}\right) = ? \qquad D_f: \mathbb{R}\setminus\{0\}$\\
Átviteli elvvel igazoljuk, hogy \textbf{nem} létezik!\\
Vegyünk fel két olyan sorozatot, melyre $\cos\left(\frac{1}{x_n}\right) \equiv 1$, illetve $\cos\left(\frac{1}{y_n}\right) \equiv -1$.\\
Tehát $x_n = \dfrac{1}{2n\pi}; \quad y_n = \dfrac{1}{\pi+2n\pi}$
Mindkettőre: $x_n\xrightarrow{n\to\infty} 0$ és $y_n\xrightarrow{n\to\infty} 0$. \textbf{De}:
\[\underbrace{\lim_{n\to\infty} \cos\left(\frac{1}{x_n}\right) = 1 \qquad \lim_{n\to\infty} \cos\left(\frac{1}{y_n}\right) = -1}_{1 \neq -1}\]
Így az Átviteli-elv alapján \textbf{nem} létezik a határérték!

\subsection{Végesben vett határértékek}

\[\left.\begin{array}{rcl}
\left.\begin{array}{l}
\displaystyle\lim_{x\to x_0} f(x) = A \\
\displaystyle\lim_{x\to x_0+0} f(x) = A \\
\displaystyle\lim_{x\to x_0-0} f(x) = A
\end{array}\right\} &

\begin{array}{c}
 \forall\varepsilon>0 : \exists \delta(\varepsilon)>0 \hbox{, hogy}\\
 |f(x)-A|<\varepsilon \hbox{, ha}
\end{array} &

 \left\{\begin{array}{l}
0<|x-x_0|<\delta  \\[+10pt]
0<x-x_0<\delta  \\[+10pt]
0<x_0-x<\delta
\end{array}\right. \\\\[-5pt]

\left.\begin{array}{l}
\displaystyle\lim_{x\to x_0} f(x) = \infty \\
\displaystyle\lim_{x\to x_0+0} f(x) = \infty \\
\displaystyle\lim_{x\to x_0-0} f(x) = \infty
\end{array}\right\} &

\begin{array}{c}
 \forall\Omega>0 : \exists \delta(\Omega)>0 \hbox{, hogy}\\
 f(x)>\Omega \hbox{, ha}
\end{array} &

\left\{\begin{array}{l}
0<|x-x_0|<\delta  \\[+10pt]
0<x-x_0<\delta  \\[+10pt]
0<x_0-x<\delta
\end{array}\right.\\\\[-5pt]

\left.\begin{array}{l}
\displaystyle\lim_{x\to x_0} f(x) = -\infty \\
\displaystyle\lim_{x\to x_0+0} f(x) = -\infty \\
\displaystyle\lim_{x\to x_0-0} f(x) = -\infty
\end{array}\right\} &

\begin{array}{c}
 \forall\Omega>0 : \exists \delta(\Omega)>0 \hbox{, hogy}\\
 f(x)<-\Omega \hbox{, ha}
\end{array} &

 \left\{\begin{array}{l}
0<|x-x_0|<\delta  \\[+10pt]
0<x-x_0<\delta  \\[+10pt]
0<x_0-x<\delta
\end{array}\right.
\end{array}\right\} x\in D_f\]

\emph{Példa}: $\displaystyle \lim_{x\to 4+0} \dfrac{1}{8-2x} = -\infty \quad \delta(\Omega)=?$
\[\hbox{Kell, hogy: }\quad \frac{1}{8-2x} < -\Omega \quad \hbox{ha } 0<x-4<\delta(\Omega)\]
\[\frac{1}{-2(x-4)} < -\Omega\]
\[-2(x-4) > -\frac{1}{\Omega}\]
\[x-4 < \frac{1}{2\Omega}\]
Ez akkor igaz, ha $0<x-4<\delta(\Omega)$, tehát $\boxed{\delta(\Omega) = \frac{1}{2\Omega}}$.

\subsection{Végtelenben vett határértékek}

\[\begin{array}{rcl}
\left.\begin{array}{l}
\displaystyle\lim_{x\to +\infty} f(x) = A \\
\displaystyle\lim_{x\to -\infty} f(x) = A
\end{array}\right\} &

\begin{array}{c}
 \forall\varepsilon>0 : \exists P(\varepsilon)>0 \hbox{, hogy}\\
 |f(x)-A|<\varepsilon \hbox{, ha}
\end{array} &

 \left\{\begin{array}{l}
x>P(\varepsilon)  \\[+10pt]
x<-P(\varepsilon)
\end{array}\right\} x\in D_f \end{array}\]

\[\begin{array}{rcl}
\left.\begin{array}{l}
\displaystyle\lim_{x\to +\infty} f(x) = +\infty \\
\displaystyle\lim_{x\to -\infty} f(x) = +\infty \\
\displaystyle\lim_{x\to +\infty} f(x) = -\infty \\
\displaystyle\lim_{x\to -\infty} f(x) = -\infty \\
\end{array}\right\} &

\begin{array}{c} 
 \forall\Omega>0 \hbox{-hoz}\\
 \exists P(\Omega)>0 \hbox{, hogy}
\end{array} &

\left\{\begin{array}{l}
f(x)>\Omega \hbox{, ha } x>P(\Omega)  \\[+6pt]
f(x)>\Omega \hbox{, ha } x<-P(\Omega)  \\[+6pt]
f(x)<-\Omega \hbox{, ha } x>P(\Omega)  \\[+6pt]
f(x)<-\Omega \hbox{, ha } x<-P(\Omega)
\end{array}\right\} x\in D_f

\end{array} \]

\emph{Példa}: $\displaystyle \lim_{x\to \infty} \dfrac{2-3x}{x+1} = -3 \quad P(\varepsilon)=? \qquad D_f:\mathbb{R}\setminus\{-1\}$\\
\[\left|\frac{2-3x}{x+1}+3\right|=\left|\frac{2-3x+3x+3}{x+1}\right| = \frac{5}{|x+1|} \overset{x>-1}{=} \frac{5}{x+1} < \varepsilon \qquad \hbox{ha } x>\boxed{\frac{5}{\varepsilon}-1=P(\varepsilon)} \]\\

\emph{Megjegyzés}: Az átviteli-elv és a rendőr-elv is működik \emph{mindegyik} határérték típusra.\\

\emph{1. példa}: $\displaystyle \lim_{x\to \pm\infty} \dfrac{\sin x}{x^2+3} = 0$. Rendőr-elv alapján:
\[\underbrace{\frac{-1}{x^2+3}}_{\to 0} \leqslant \frac{\sin x}{x^2+3} \leqslant \underbrace{\frac{1}{x^2+3}}_{\to 0}\]
\emph{Megjegyzés}: Az, hogy a bal és jobb oldal miért tart 0-hoz (bár kézenfekvő), a következő fejezetben tanultak alapján indokolható.\\

\emph{2. példa}: $\displaystyle \nexists \lim_{x\to\infty} \sin x$. Átviteli-elv alapján:\\
\[\left.\begin{array}{ll}
    x_k = k\pi \xrightarrow{k\to\infty} \infty & \quad \Rightarrow \quad \sin(x_k) \equiv 0 \\[+5pt]
    y_k = \dfrac{\pi}{2}+2k\pi \xrightarrow{k\to\infty} \infty & \quad \Rightarrow \quad \sin(y_k) \equiv 1
  \end{array}\right\} \lim_{k\to\infty}\sin(x_k) \neq \lim_{k\to\infty}\sin(y_k) \Rightarrow \nexists \lim_{x\to\infty}\sin x
\]

\section{Műveletek függvényekkel}

Pontonként definiált műveletek; $x\in D = D_f = D_g \subset \mathbb{R}$
\[(f+g)(x) := f(x)+g(x)\]
\[(f\cdot g)(x) := f(x)\cdot g(x)\]
\[(cf)(x) := c\cdot f(x) \quad c\in\mathbb{R}\]
\[\left(\frac{f}{g}\right)(x) := \frac{f(x)}{g(x)} \qquad g(x)\neq 0\]
\[(f\circ g)(x) := f(g(x)) \qquad g(x)\in D_f\]

\subsection{Határértékre vonatkozó tételek}

\begin{tetel}
 Ha $\displaystyle \exists \lim_{x\to x_0} f(x) = A$, $\displaystyle \exists \lim_{x\to x_0} g(x) = B$ és $c\in\mathbb{R}$, akkor:
 \[\lim_{x\to x_0} (c\cdot f)(x) = c\cdot \lim_{x\to x_0} f(x) = c\cdot A\]
 \[\lim_{x\to x_0} (f+g)(x) = \lim_{x\to x_0} f(x) + \lim_{x\to x_0} g(x) = A+B\]
 \[\lim_{x\to x_0} (f\cdot g)(x) = \lim_{x\to x_0} f(x)\cdot \lim_{x\to x_0} g(x) = A\cdot B\]
 \[\lim_{x\to x_0} \left(\frac{f}{g}\right)(x) = \frac{\displaystyle\lim_{x\to x_0} f(x)}{\displaystyle\lim_{x\to x_0} g(x)} = \frac{A}{B} \qquad \hbox{ha } B\neq 0\]
\end{tetel}
\begin{biz}
 Mindegyiket az átviteli-elvvel könnyedén bizonyítható, pl. a szorzatra:\\
\[\lim_{x\to x_0} f(x) = A \quad \Leftrightarrow \quad \forall \{x_n\}\to x_0,\; x_n\in D_f\setminus\{0\} \hbox{ és } \exists \lim_{n\to\infty} f(x_n) = A\]
\[\lim_{x\to x_0} g(x) = B \quad \Leftrightarrow \quad \forall \{x_n\}\to x_0,\; x_n\in D_f\setminus\{0\} \hbox{ és } \exists \lim_{n\to\infty} g(x_n) = B\]
Tehát a sorozatokra tanult szabályok értelmében:
\[\lim_{n\to \infty} (f\cdot g)(x_n) = \lim_{n\to \infty} (f(x_n)\cdot g(x_n)) = \lim_{n\to \infty} f(x_n)\cdot \lim_{n\to \infty} g(x_n)\]
Most alkalmazva az átviteli-elv $\Rightarrow$ irányát:
\[\lim_{n\to \infty} f(x_n)\cdot \lim_{n\to \infty} g(x_n) = \lim_{x\to x_0} f(x)\cdot \lim_{x\to x_0} g(x) = A\cdot B\]
És az eredetire alkalmazva az átviteli-elv $\Leftarrow$ irányát:
\[\lim_{n\to \infty} (f\cdot g)(x_n) = \lim_{x\to x_0} (f\cdot g)(x) = A\cdot B\]
\end{biz}


\section{Folytonosság}

\begin{defi}
 Legyen $x_0 \in \Int D_f$. $f$ \textbf{folytonos} $x_0$-ban, akkor $\exists f(x_0)$ és $\exists \displaystyle \lim_{x\to x_0} f(x)$, illetve $\displaystyle f(x_0) = \lim_{x\to x_0} f(x)$.
\end{defi}

Alternatív definíciók:
\begin{defi}\label{FolytonossagEpsilon}
 Legyen $x_0 \in \Int D_f$. $f$ \textbf{folytonos} $x_0$-ban, ha $\forall\varepsilon > 0$ esetén $\exists \delta(\varepsilon)>0$, hogy $|f(x)-f(x_0)|<\varepsilon$, ha $|x-x_0|<\delta(\varepsilon)$.
\end{defi}
\begin{center}
\psset{xunit=1.0cm,yunit=1.0cm,algebraic=true,dotstyle=o,dotsize=3pt 0,linewidth=0.8pt,arrowsize=3pt 2,arrowinset=0.25}
\begin{pspicture*}(-1.36,-0.5)(5.49,4.43)
\psaxes[labelFontSize=\scriptstyle,xAxis=true,yAxis=true,labels=none,Dx=1,Dy=1,ticksize=-2pt 0,subticks=2]{->}(0,0)(-1.36,-0.5)(5.49,4.43)
\psplot[plotpoints=200]{-1.3552321774978247}{5.4890617887906625}{0.0740*x^3-0.2180*x^2+0.0991*x+1.9290}
\psline(3.61,2.92)(0,2.92)
\psline[linestyle=dashed,dash=4pt 4pt](4.1,3.77)(0,3.77)
\psline[linestyle=dashed,dash=4pt 4pt](0,2.07)(2.72,2.07)
\psline(3.61,2.92)(3.61,0)
\psline[linestyle=dashed,dash=2pt 2pt](3.21,0)(3.21,4.43)
\psline[linestyle=dashed,dash=2pt 2pt](4,0)(4,4.43)
\rput[bl](-1.28,1.05){$f$}
\psdots[dotstyle=*](4.1,3.77)

\psdots[dotstyle=*](0,2.92)
\rput[c](-0.6,2.92){$f(x_0)$}
\rput[c](0,2.92){\rnode{F}{}}

\psdots[dotstyle=*](0,3.77)
%\rput[bl](-0.31,3.73){$K$}
\psdots[dotstyle=*](0,2.07)
%\rput[bl](-0.23,2.17){$K'$}

\rput[c](0,3.77){\rnode{KF}{}}
\rput[c](0,2.07){\rnode{KA}{}}

\psbrace[ref=c,rot=0,nodesepA=5pt,nodesepB=0pt](F)(KF){$\varepsilon$}
\psbrace[ref=c,rot=0,nodesepA=5pt,nodesepB=0pt](KA)(F){$\varepsilon$}

\psdots[dotstyle=*](2.72,2.07)
\psdots[dotstyle=*](3.61,0)
\rput[c](3.61,-0.30){$x_0$}
\rput[c](3.61,0){\rnode{X}{}}
\psdots[dotstyle=*](4,0)
%\rput[bl](4.05,0.09){$N$}
\rput[c](4,0){\rnode{XJ}{}}
\psdots[dotstyle=*](3.21,0)
%\rput[bl](3.28,0.09){$N'$}
\rput[c](3.21,0){\rnode{XB}{}}

\psbrace[ref=c,rot=-90,nodesepA=0pt,nodesepB=7pt](XJ)(XB){$2\delta$}
%\psbrace[ref=c,rot=0,nodesepA=5pt,nodesepB=0pt](X)(XJ){$\delta$}

\psdots[dotstyle=*](4,3.57)
\psdots[dotstyle=*](3.21,2.45)
\end{pspicture*}
\end{center}

\begin{defi}
 $f$ \textbf{folytonos} $x_0$-ban, ha $\displaystyle f\left(\lim_{x\to x_0} x\right) = \lim_{x\to x_0} f(x)$
\end{defi}

\hspace*{0.25\textwidth}\rule{0.5\textwidth}{0.5pt}

\begin{defi}
 $f$ \textbf{jobbról folytonos} $x_0$-ban, ha $f(x_0) = f(x_0+0)$
\end{defi}

\begin{defi}
 $f$ \textbf{balról folytonos} $x_0$-ban, ha $f(x_0) = f(x_0-0)$
\end{defi}

\begin{tetel}
 $x_0 \in \rm{Int~Df}$. $f$ folytonos $x_0$-ban $\quad\Leftrightarrow\quad$ $f$ balről és jobbról folytonos $x_0$-ban.
\end{tetel}
\begin{biz}
 Határértékre vonatkozó hasonló tétel alapján következik.
\end{biz}

\begin{tetel}
 Ha $f, g$ folytonos $x_0$-ban, akkor $c\cdot f, f+g, f\cdot g$ is folytonos $x_0$-ban ($c\in\mathbb{R}$) és $g(x_0)\neq 0$ esetén $\dfrac{f}{g}$ is folytonos $x_0$-ban.
\end{tetel}
\begin{biz}
 Visszavezethető a függvény határértékre vonatkozó számolási szabályokra.\\
 Például a szorzás:
 \[\lim_{x\to x_0} (f\cdot g)(x) = \lim_{x\to x_0} f(x)\cdot \lim_{x\to x_0} g(x) = f(x_0)\cdot g(x_0) = (f\cdot g)(x_0)\]
\end{biz}

\begin{tetel} $\boxed{$Folytonos függvények kompozíciója folytonos$}$\\
 Ha $g$ folytonos $x_0$-ban és $f$ folytonos $g(x_0)$-ban, akkor $f\circ g$ folytonos $x_0$-ban.
\end{tetel}
\begin{biz}
 \[\lim_{x\to x_0} (f\circ g)(x) = \lim_{x\to x_0} f(g(x)) = f(\lim_{x\to x_0} g(x)) = f(g(x_0)) = (f\circ g)(x_0)\]
\end{biz}

\begin{lemma} Folytonossági lemmák\\
\begin{itemize*}
 \item Minden polinom folytonos $\forall x_0\in\mathbb{R}$-ben.
 \item Minden racionális tört függvény (polinom/polinom) folytonos mindenütt, ahol értelmezve van
\end{itemize*}
\end{lemma}

\subsection{Szakadási helyek}

Ha $f$ \emph{nem} folytonos $x_0$-ban, akkor $x_0$-ban \textbf{szakad}. Fajtái:
\begin{enumerate*}
 \item \emph{Elsőfajú}: ha $x_0$-ban szakad és $\exists f(x_0+0)\in\mathbb{R}$ és $\exists f(x_0-0)\in\mathbb{R}$.
    \begin{enumerate*}
     \item \emph{Megszüntethető}, ha $f(x_0+0) = f(x_0-0)$
     \item \emph{Nem megszüntethető}, ha $f(x_0+0) \neq f(x_0-0)$. Másnevén: \emph{véges ugrás}.
    \end{enumerate*}
 \item \emph{Másodfajú}: ha $x_0$ nem elsőfajú szakadási hely.
\end{enumerate*}

\subsection{Folytonos függvények tulajdonságai}

\begin{defi}
 $f$ folytonos az $(a;b)\subset \mathbb{R}$ intervallumon, ha $\forall x\in(a;b)$ ponton folytonos.
\end{defi}

\begin{defi}
 $f$ folytonos az $[a;b]\subset \mathbb{R}$ intervallumon, ha $(a;b)$-n folytonos és jobbról folytonos $a$-ban és balról folytonos $b$-ben.
\end{defi}

\begin{tetel}
 Ha $f$ folytonos $x_0$-ban és $f(x_0)>c$, akkor $\exists \delta>0$, hogy $x\in K_{\delta}(x_0)$ esetén $f(x)>c$.
\end{tetel}

\begin{center}
\psset{xunit=1.0cm,yunit=1.0cm,algebraic=true,dotstyle=o,dotsize=3pt 0,linewidth=0.8pt,arrowsize=3pt 2,arrowinset=0.25}
\begin{pspicture*}(-1,-0.7)(7.5,6.3)
\psaxes[xAxis=true,yAxis=true,labels=none,Dx=1,Dy=1,ticksize=-2pt 0,subticks=2]{->}(0,0)(-1,-3.94)(7.5,6.3)
\psplot[plotpoints=200]{-1}{7.5}{0.03396*x^4-0.53342*x^3+2.5576*x^2-3.43810*x+2.59061}
\psplot{-1}{7.5}{(--2.6-0*x)/1}
\rput[bl](-4.2,2.8){$c$}
\psline[linestyle=dashed,dash=5pt 5pt](3.5,-0.1)(3.5,6.3)
\psplot[linestyle=dashed,dash=5pt 5pt]{-0.1}{7.5}{(--4.11-0*x)/1}
\psline[linestyle=dotted](-1,4.87)(7.5,4.87)
\psline[linestyle=dotted](-1,3.36)(7.5,3.36)
\psline[linestyle=dotted](2.86,-3.94)(2.86,6.3)
\psline[linestyle=dotted](4.14,-3.94)(4.14,6.3)
\psdots[dotsize=4pt 0,dotstyle=*](3.5,4.11)
\psdots[dotstyle=*](3.5,0)
\rput[c](3.5,-0.4){$x_0$}
\psdots[dotstyle=*](0,4.11)
\rput[c](-0.35,4.11){$y_0$}
\rput[c](0,4.87){\rnode{AE}{}}
\rput[c](0,4.11){\rnode{E}{}}
\rput[c](0,3.36){\rnode{BE}{}}
\psbrace[ref=c,rot=0,nodesepA=5pt,nodesepB=0pt](E)(AE){$\varepsilon$}
\psbrace[ref=c,rot=0,nodesepA=5pt,nodesepB=0pt](BE)(E){$\varepsilon$}

\rput[c](2.86,0){\rnode{DA}{}}
\rput[c](4.14,0){\rnode{DB}{}}
\psbrace[ref=c,rot=-90,nodesepA=0pt,nodesepB=8pt](DB)(DA){$2\delta$}

\psdots[dotstyle=*](2.86,3.47)
\psdots[dotstyle=*](4.14,4.32)
\end{pspicture*}
\end{center}

\begin{biz}
 Mivel $f$ folytonos $x_0$-ban ezért a folytonosság definícióját (\ref{FolytonossagEpsilon}) alkalmazhatjuk. Legyen $\varepsilon = \dfrac{f(x_0)-c}{2}$. Ehhez létezik olyan $\delta(\varepsilon)$, hogy ha
\[|x-x_0|<\delta(\varepsilon), \hbox{ akkor } |f(x)-f(x_0)|=|f(x)-y_0|<\varepsilon\]
\[|f(x)-y_0|<\varepsilon \quad \Leftrightarrow \quad y_0-\varepsilon < f(x) <y_0+\varepsilon\]
Mivel $\varepsilon = \dfrac{y_0-c}{2}$, ezért:
\[c < c+\varepsilon = y_0-\varepsilon < f(x)\]
\end{biz}

\subsection{Tételek korlátos zárt intervallumon folytonos függvényekhez}

\begin{tetel}\label{Bolzano} $\boxed{$Bolzano-tétel$}$\\
 Ha $f$ folytonos $[a,b]$-n és $f(a)<c<f(b)$, akkor $\exists\xi\in(a,b)$, hogy $f(\xi)=c$
\end{tetel}
\begin{biz} (Vázlat) - Cantor-axióma (lásd \pageref{Cantor}. oldal)\\
 Legyen $a_0=a, b_0 = b$ és $I_0 = [a_0,b_0]$. Ha $f\left(\dfrac{a+b}{2}\right)=c \quad \Rightarrow $ kész $\; \xi = \dfrac{a+b}{2}$\\
 Ha $f\left(\dfrac{a_0+b_0}{2}\right)<c \quad \Rightarrow \quad I_1 := \left[\dfrac{a_0+b_0}{2};b_0\right]$\\
 Ha $f\left(\dfrac{a_0+b_0}{2}\right)>c \quad \Rightarrow \quad I_1 := \left[a_0;\dfrac{a_0+b_0}{2}\right]$\\
 És így tovább, tehát mindig elfelezzük az intervallumot és azzal az intervallummal dolgozunk tovább, ami tartalmazza $c$-t. Tehát $n$ lépés után: $f(a_n)<c; f(b_n)>c$\\
 Ha $f\left(\dfrac{a_n+b_n}{2}\right)=c \quad \Rightarrow $ kész $\; \xi = \dfrac{a_n+b_n}{2}$\\
 Ha $f\left(\dfrac{a_n+b_n}{2}\right)<c \quad \Rightarrow \quad I_{n+1} := \left[\dfrac{a_n+b_n}{2};b_n\right]$\\
 Ha $f\left(\dfrac{a_n+b_n}{2}\right)>c \quad \Rightarrow \quad I_{n+1} := \left[a_n;\dfrac{a_n+b_n}{2}\right]$\\
 \[I_0 \supset I_1 \supset I_2 \supset \ldots \supset I_n\]
 $|I_n| = \dfrac{(b-a)}{2^n} \xrightarrow{n\to\infty} 0$, de Cantor-axióma alapján nem üres $\Rightarrow$ csak 1 pontot tartalmazhat:
 \[\bigcap_{n} I_n = \{\xi\}\]
 Amit még be kell látni: $a_n \xrightarrow{n\to\infty} \xi$ és $b_n \xrightarrow{n\to\infty} \xi$. Ez vázlatosan:
 Mivel $f(a_n) < c$ és $f(b_n) > c$, de mind a kettő egy egyre szükülő intervallumok két vége, melyek egyetlen közöspontja $\xi$, ezért:
 \[\lim_{n\to\infty} f(a_n) = \lim_{n\to\infty} f(b_n) = \lim_{n\to\infty} f(\xi) = c\]
\end{biz}

\begin{tetel} Bolzano-tétel egy következménye\\
 Ha $f$ folytonos $[a,b]$-n és $f(a)<0<f(b)$, akkor $f$-nek van gyöke $(a, b)$-n. (Tehát a Bolzano-tételt $c=0$-ra alkalmazzuk)
\end{tetel}
\emph{Megjegyzés}: Nagyon hatékony gyökkereső algoritmusokat lehet ennek felhasználásával írni.
\addtocounter{biz}{1} % nem kell bizonyítani

\begin{tetel} Bolzano-tétel egy következménye\\
 Minden páratlan fokszámú polinomnak van legalább 1 valós gyöke. 
 \[p(x) = x^{2n+1} + a_{2n}x^{2n}+\ldots+a_0 \qquad (n\in\mathbb{Z^+})\]
 \[\lim_{x\to\pm\infty} p(x) = \pm\infty\]
\end{tetel}
\addtocounter{biz}{1} % nem kell bizonyítani

\begin{tetel}\label{W.I}$\boxed{$Weierstrass I. tétel$}$\\
 Korlátos zárt intervallumon\footnote{Ehelyett mondhatjuk, hogy \emph{kompakt halmazon} -- ekkor egy erősebb állítást kapunk} folytonos függvény korlátos.
\end{tetel}
\begin{biz}(*) Indirekt\\
 Tfh: $f$ folytonos $[a, b]$-n, de nem korlátos felülről, tehát \\
  1 nem felső korlát $\Rightarrow$ $\exists x_1\in[a, b] : f(x_n)>1$.\\
  2 nem felső korlát $\Rightarrow$ $\exists x_2\in[a, b] : f(x_n)>2$.\\
  $\vdots$\\
  $n\in\mathbb{N}$ nem felső korlát $\Rightarrow$ $\exists x_n\in[a, b] : f(x_n)>n$.\\
 $\forall x_n \in [a,b]$, tehát $\{x_n\}$ korlátos $\overset{\hbox{B. W. kiv. t.}}{\Longrightarrow} \exists x_{n_k}$ konvergens részsorozat: $\{x_{n_k}\} \to t\in[a, b]$. $f(x_{n_k}) \xrightarrow{k\to \infty} \infty$, de mivel $f(x_{n_k})$ folytonos, ezért $f(x_{n_k}) \xrightarrow{k\to \infty} f(t)$, de ez nem lehet, hiszen a határérték egyértelmű. \blitza
\end{biz}

\begin{tetel}\label{W.II}$\boxed{$Weierstrass II. tétel$}$\\
 Korlátos zárt intervallumon\footnote{Ehelyett mondhatjuk, hogy \emph{kompakt halmazon} -- ekkor egy erősebb állítást kapunk} folytonos függvény felveszi szélsőértékeit. Tehát
 \[m := \Inf\{f(x)\;|\;x\in[a,b]\} \in \mathbb{R} \qquad (\hbox{W. I. alapján korlátos})\]
 \[M := \Sup\{f(x)\;|\;x\in[a,b]\} \in \mathbb{R}\]
Ekkor $\exists \alpha, \beta\in[a,b]$, hogy $f(\alpha)=M, f(\beta)=b$.
\end{tetel}
\begin{biz} Megmutatjuk, hogy $\exists \alpha: f(\alpha)=M$. Hasonlóan lehetne $f(\beta)=m$-re.\\
Indirekt, tfh: $\nexists \alpha\in[a,b]: f(\alpha) = M \quad \Rightarrow \quad M-f(x)>0$, ha $x\in[a,b] \quad \Rightarrow$\\
$g(x) = \dfrac{1}{M-f(x)}$ folytonos $[a,b]$-ben $\overset{\hbox{W I. t.}}{\Longrightarrow}$ $g$ korlátos $[a,b]$-ben, tehát $\exists K$:
\[\frac{1}{M-f(x)} < K \qquad x\in[a, b], K>0\]
\[M-f(x) > \frac{1}{K} \qquad x\in[a, b], K>0\]
\[f(x) < M-\frac{1}{K} < M\]
De ez nem lehet igaz, hiszen $M$ a legkisebb felső korlátunk volt, mi viszont találtunk ennél kisebbet.
\end{biz}

\emph{Példa}: $f(x) = \dfrac{x^4+3x^2-4}{x^2+x-2}$.
\begin{enumerate*}
 \renewcommand{\theenumi}{\alph{enumi}}

 \item Hol és milyen szakadásai vannak?\\
  $f(x) = \dfrac{x^4+3x^2-4}{(x+2)(x-1)}$, tehát $x=-2$ és $x=1$-ben nem értelmezett a függvény, itt milyen szakadások vannak?\\
  $\displaystyle \lim_{x\to-2\pm 0} f(x) = \mp \infty \quad \Rightarrow \quad$ másodfajú szakadás.\\
  $\displaystyle \lim_{x\to1\pm 0} f(x) = \frac{10}{3} \quad \Rightarrow \quad$ megszüntethető szakadás.\\

 \item Van-e minimuma a $[-1, 0]$-n?\\
  $f$ folytonos $[-1, 0]$-n, mivel mindenhol folytonos kivéve $x=-2$ és $x=1$-ben. $[-1, 0]$ egy korlátos zárt intervallum, tehát az $f$ Weierstrass II. tétele alapján felveszi a szélső értékeket $\Rightarrow$ $\exists$ minimuma az $[-1, 0$]-en.
\end{enumerate*}

\subsection{Egyenletes folytonosság}

\emph{Egyenletes folytonosság} definícióját motiváljuk:
\begin{enumerate*}
\item Mutassuk meg, hogy $f(x)=x^2+2$ folytonos az [1,2] minden pontjában. $\delta(\varepsilon, x_0) = ?$\\
  $x_0 \in [1,2]$
  \[[f(x)-f(x_0)| = |x^2+2-x_0^2-2|= |x^2-x_0^2| = |x-x_0|\cdot|x+x_0| < \varepsilon\]
  \[|x-x_0|\cdot|x+x_0| \leqslant |x-x_0|\cdot|2+x_0| < \varepsilon \qquad \hbox{teljesül, ha:}\]
  \[|x-x_0| < \frac{\varepsilon}{2+x_0} = \delta(\varepsilon, x_0)\]
\item Létezik-e ``univerzális'' $x_0$-tól független $\delta(\varepsilon)$?
  Keressük a legkisebb $\delta(\varepsilon, x_0)$-t:
  \[\delta(\varepsilon) := \frac{\varepsilon}{2+2} = \underbrace{\frac{\varepsilon}{4}}_{< 0} \left[ \leqslant \frac{\varepsilon}{|2+x_0|} = \delta(\varepsilon, x_0) \right]\]
\end{enumerate*}

\begin{defi} $\boxed{$Egyenletes folytonosság$}$ $\qquad$ (Nem lokális tulajdonság!)\\
 Az $f$ az $A\subset D_f$ halmazon \textbf{egyenletesen folytonos}, ha $\forall \varepsilon>0 : \exists \delta(\varepsilon)>0$ ($x_0$-tól független, ``univerzális''), hogy
 \[|f(x_1)-f(x_0)|<\varepsilon, \hbox{ ha } |x_1-x_0|<\delta(\varepsilon) \hbox{ és } x_1,x_0\in A\]
\end{defi}
\emph{Megjegyzés}: Ha $f$ egyenletesen folytonos $A$-n, akkor folytonos $\forall x_0\in A$-ban.\\

\emph{Példa}: Egyenletesen folytonos-e az $f(x)=x^2+2$ az $[1,\infty)$-on?\\
Tekintsük az $x_n = n$ és $y_n = n+\dfrac{1}{n}$ sorozatokat.
\[|y_n-x_n| = \frac{1}{n} \xrightarrow{n\to\infty} 0\]
Tehát ha $n$-el tartunk végtelenbe, akkor a sorozatok $n$. tagjai egyre közelebbek lesznek egymáshoz. Ha vizsgáljuk az ezen számokhoz tartozó függvényértékek különbségét, akkor:
\[|f(y_n)-f(x_n)| = \left|\left(n+\frac{1}{n}\right)^2-n^2\right| = 2+\frac{1}{n} > 2\]
Tehát ha $n\to\infty$, akkor ezen két függvényértékek különbsége nagyobb lesz mint 2. Tehát ha $\varepsilon < 2$, akkor $\nexists \delta(\varepsilon)$, hiszen bármilyen kicsire is vesszük a $\delta(\varepsilon)$-t, láthatjuk, hogy a függényértékek különbsége nagyobb lesz, mint 2. Tehát \textbf{nem} egyenletesen folytonos a függvényünk az $[1, \infty)$ intervallumon.

\begin{tetel}
 Korlátos zárt intervallumon\footnote{Ehelyett mondhatjuk, hogy \emph{kompakt halmazon} -- ekkor egy erősebb állítást kapunk} folytonos függvény egyenletesen folytonos!
\end{tetel}
\addtocounter{biz}{1}

\subsubsection{Példák (ha a fentit tételt nem alkalmazhatjuk)}

\begin{enumerate*}
\item Egyenletesen folytonos-e az $f(x)=\dfrac{1}{x}$ az $[1,\infty)$-on?\\
A fenti tételt nem használhatjuk, hiszen nem korlátos az intervallum, de ez nem zárja ki azt, hogy a függvényünk egyenletesen folytonos, pusztán a tételből nem következik.\\
A kritikus rész az $1$ környezete, hiszen ha ott találunk $\delta(\varepsilon)$-t, akkor az jó lesz végig, hiszen utána ha távolodunk az origótól, akkor adott $\varepsilon$ esetén, a két helyettesítési pont egyre távolabb lesz egymástól, tehát jó lesz az előzőekben talált $\delta(\varepsilon)$. Legyen $\varepsilon > 0$ adott és $1\leqslant x_0 < x_1$.
\[|f(x_1)-f(x_0)|=\left|\frac{1}{x_1}-\frac{1}{x_0}\right|=\frac{|x_0-x_1|}{x_0x_1} = \frac{x_1-x_0}{\underbrace{x_0x_1}_{\geqslant 1}} \leqslant \frac{x_1-x_0}{1}\]
\[\frac{x_1-x_0}{1} < \varepsilon \quad \hbox{ha } |x_1-x_0|<\boxed{\delta(\varepsilon)=\varepsilon}\]
Tehát $\delta(\varepsilon)=\varepsilon$ jó választás; egyenletesen folytonos $f(x)$ az $[1, \infty)$-n.\\

\item Egyenletesen folytonos-e az $f(x)=\sqrt{x}$ az $[0, \infty)$ intervallumon?\\
Itt is a kritikus helyzetet keressük ez pedig az, ha mindkét szám helyettesítési érték közel van a 0-hoz. Legyen $\varepsilon > 0$ adott és $0\leqslant x_0 < x_1$
\[|f(x_1)-f(x_0)|=|\sqrt{x_1}-\sqrt{x_0}|=(\sqrt{x_1}-\sqrt{x_0})\cdot\frac{\sqrt{x_1}+\sqrt{x_0}}{\sqrt{x_1}+\sqrt{x_0}}=\frac{x_1-x_0}{\sqrt{x_1}+\sqrt{x_0}} \leqslant\]
\[ \leqslant \frac{x_1-x_0}{\sqrt{x_1-x_0}+\sqrt{x_0-x_0}}=\sqrt{x_1-x_0} < \varepsilon\]
Ez akkor teljesül, ha $|x_1-x_0|<\boxed{\delta(\varepsilon) = \varepsilon^2}$\\

\item Egyenletesen folytonos-e az $f(x)=\dfrac{1}{x}$ az $(0, 1]$ intervallumon (korlátos, de nem zárt)?\\
Vegyünk fel két pontsorozatot: $x_n = \dfrac{1}{n} \xrightarrow{n\to\infty} 0$ és $y_n = \dfrac{1}{n+1} \xrightarrow{n\to\infty} 0$. Ezek ha meggondoljuk, egyre közelebb lesznek egymáshoz.
\[|f(x_n)-f(y_n)|=|n-(n+1)|=1 < \varepsilon\]
Tehát, ha $\varepsilon < 1$, akkor $\nexists \delta(\varepsilon)$, mert ha létezne, akkor vegyünk, olyan $n$-et, hogy $|x_n-y_n|<\delta(\varepsilon)$, de ekkor $|f(x_n)-f(y_n)|=1 > \varepsilon$ \blitza.
\end{enumerate*}

\section{Függvények differenciálása}

\begin{defi} Differenciahányados: $\displaystyle \frac{\Delta f}{\Delta x} = \frac{f(x_0+\Delta x)-f(x_0)}{\Delta x} = \tg \alpha$ \end{defi}
\begin{center}
\psset{xunit=1.0cm,yunit=1.0cm,algebraic=true,dotstyle=o,dotsize=3pt 0,linewidth=0.8pt,arrowsize=3pt 2,arrowinset=0.25}
\begin{pspicture*}(-2.3,-1.12)(5.49,4.43)
\psaxes[labelFontSize=\scriptstyle,xAxis=true,yAxis=true,labels=none,Dx=1,Dy=1,ticksize=-2pt 0,subticks=2]{->}(0,0)(-1.36,-1.12)(5.49,4.43)
\psplot[plotpoints=200]{-1.3552321774978247}{4.3}{0.0193*x^3+0.1060*x^2-0.04600*x+0.9393}
\psline(3.86,3.45)(1.57,1.2)

\pscustom{\parametricplot{0.0}{0.7762613335289333}{0.9*cos(t)+1.57|0.9*sin(t)+1.2}\lineto(1.57,1.2)\closepath}
\rput[bl](1.95,1.35){$\alpha$}

\psline(3.86,3.45)(3.86,0)
\psline(1.57,1.2)(1.57,0)
\psline(1.57,1.2)(0,1.2)
\psline(3.86,3.45)(0,3.45)
\psline(1.57,1.2)(3.86,1.2)
%\rput[bl](-1.28,0.92){$f$}

\psdots[dotstyle=*](1.57,1.2)
\rput(1.57,1.2){\rnode{E}{}}
%\rput[bl](1.36,1.35){$E$}
\psdots[dotstyle=*](3.86,3.45)
\rput(3.86,3.45){\rnode{F}{}}
%\rput[bl](3.64,3.59){$F$}
\psdots[dotstyle=*](3.86,1.2)
\rput(3.86,1.2){\rnode{G}{}}
%\rput[bl](3.92,1.29){$G$}

\psbrace[ref=c,rot=90,nodesepA=0pt,nodesepB=-7pt](E)(G){$\Delta x$}
\psbrace[ref=c,rot=0,nodesepA=10pt,nodesepB=0pt](G)(F){$\Delta f$}

\psdots[dotstyle=*](3.86,0)
\rput[c](3.86,-0.3){$x_0+\Delta x$}
\psdots[dotstyle=*](1.57,0)
\rput[c](1.57,-0.3){$x_0$}

\psdots[dotstyle=*](0,1.2)
\rput[r](-0.2,1.2){$f(x_0)$}
\psdots[dotstyle=*](0,3.45)
\rput[r](-0.2,3.45){$f(x_0+\Delta x)$}
\end{pspicture*}
\end{center}

Ha $\Delta x\to 0$, akkor az $x_0$-ban vett érintőt kapjuk: $\displaystyle \lim_{\Delta x\to 0} \frac{\Delta f}{\Delta x} = f'(x)$.

\begin{defi}
 Differenciálhányados: $x_0 \in \Int D_f$
\[f'(x_0) = \frac{df(x)}{dx} = \lim_{\Delta x\to 0} \frac{f(x_0+\Delta x)-f(x_0)}{\Delta x} = \lim_{x\to x_0} \frac{f(x)-f(x_0)}{x-x_0}\]
\end{defi}

\begin{defi}
 Derivált függvény: $f': x_0 \mapsto f'(x_0)$. (Azaz minden pontban deriváljuk az eredeti függvényt)
\end{defi}

\begin{wrapfigure}{l}{0.43\textwidth}
   \vspace{-35pt}
\begin{center}
\psset{xunit=1.0cm,yunit=1.0cm,algebraic=true,dotstyle=o,dotsize=3pt 0,linewidth=0.8pt,arrowsize=3pt 2,arrowinset=0.25}
\begin{pspicture*}(-1.36,-1.12)(5.49,4.43)
\psaxes[labelFontSize=\scriptstyle,xAxis=true,yAxis=true,labels=none,Dx=1,Dy=1,ticksize=-2pt 0,subticks=2]{->}(0,0)(-1.36,-1.12)(5.49,4.43)

\psplot[linewidth=1.6pt]{-1.36}{5.49}{(--0.53--0.43*x)/1}
\pscustom{\parametricplot{0.0}{0.40461146164215567}{1.5*cos(t)+1.57|1.5*sin(t)+1.2}\lineto(1.57,1.2)\closepath}
\rput[bl](2.62,1.22){$\beta$}

\rput[bl](1.95,1.27){\psframebox*{\small$\alpha$}}
\psplot[plotpoints=200]{-1.3552321774978247}{5.4890617887906625}{0.0193*x^3+0.1060*x^2-0.04600*x+0.9393}
\pscustom{\parametricplot{0.0}{0.7762613335289333}{0.9*cos(t)+1.57|0.9*sin(t)+1.2}\lineto(1.57,1.2)\closepath}

\psline(3.86,3.45)(1.57,1.2)
\psline[linestyle=dotted](3.86,3.45)(3.86,0)
\psline[linestyle=dotted](1.57,1.2)(1.57,0)
\psline[linestyle=dotted](1.57,1.2)(0,1.2)
\psline[linestyle=dotted](3.86,3.45)(0,3.45)
\psline[linestyle=dotted](1.57,1.2)(3.86,1.2)

\rput[bl](-1.28,0.92){$f$}
\psdots[dotstyle=*](1.57,1.2)
%\rput[bl](1.36,1.35){$E$}
\psdots[dotstyle=*](3.86,3.45)
%\rput[bl](3.64,3.59){$F$}
\psdots[dotstyle=*](3.86,1.2)
%\rput[bl](3.92,1.29){$G$}
\rput(1.57,1.2){\rnode{E}{}}
\rput(3.86,3.45){\rnode{F}{}}
\rput(3.86,1.2){\rnode{G}{}}

\psbrace[ref=c,rot=90,nodesepA=0pt,nodesepB=-7pt](E)(G){\small$\Delta x$}
\psbrace[ref=c,rot=0,nodesepA=10pt,nodesepB=0pt](G)(F){\small$\Delta f$}

\psdots[dotstyle=*](3.86,0)
\rput[c](3.86,-0.3){$x$}
\psdots[dotstyle=*](1.57,0)
\rput[c](1.57,-0.3){$x_0$}

\rput[bl](-0.92,0.36){$e$}
\end{pspicture*}
\end{center}
\vspace{-130pt}
\end{wrapfigure}

Húr meredeksége: $\tg \alpha = \dfrac{f(x_0+\Delta x)-f(x_0)}{\Delta x}$\\
$f'(x_0) = \tg \beta = \displaystyle \lim_{\Delta x\to 0} \frac{f(x_0+\Delta x)-f(x_0)}{\Delta x}$\\
Derivált függvény: $f'(x) = \displaystyle \lim_{h\to 0} \frac{f(x+h)-f(x)}{h}$\\

\vspace{80pt}

\begin{defi}
 Jobb-oldali derivált: $f'_+(x_0) = \displaystyle \lim_{h\to 0+0} \frac{f(x_0+h)-f(x_0)}{h}$
\end{defi}

\begin{defi}
 Bal-oldali derivált: $f'_-(x_0) = \displaystyle \lim_{h\to 0-0} \frac{f(x_0+h)-f(x_0)}{h}$
\end{defi}

\begin{tetel}
 \[\boxed{\exists f'(x_0) = A \quad \Leftrightarrow \quad \exists f'_-(x_0), \exists f'_+(x_0) \hbox{ és } f'_-(x_0)=f'_+(x_0)=A}\]
\end{tetel}
\begin{biz}
 Legyen $g(h) = \dfrac{f(x_0+h)-f(x_0)}{h}$, ekkor az állítást a következő képpen foglmazhatjuk át felhasználva a megfelelő definíciókat:
 \[\exists\lim_{h\to 0} g(h) = A \qquad \Leftrightarrow \qquad \underbrace{\lim_{h\to 0+0} g(h)}_{=g(0+0)}=\underbrace{\lim_{h\to 0-0} g(h)}_{=g(0-0)}=\lim_{h\to 0} g(h)\]
 Ez pedig igaz \aref{HatarertekJobbEsBal} tétel (\pageref{HatarertekJobbEsBal}. oldal) alapján.
\end{biz}
Példa: $f(x) = |x|$. $\nexists f'(0)$, hiszen $f'_+(x_0) = 1$, de $f'_-(x_0) = -1$. Ekkor 0-ban \emph{töréspont}ja van a függvénynek.

\begin{defi}
 $f(x)$ \textbf{differenciálható} $(a;b)$-n, ha $\forall x\in(a;b)$ esetén $\exists f'(x)$.
\end{defi}

\begin{defi}
 $f(x)$ \textbf{differenciálható} $[a;b]$-n, ha differenciálható $(a;b)$-n és $\exists f'_-(a)$ és $\exists f'_+(b)$.
\end{defi}

\subsection{``Ismert'' függvények deriváltja}

$f(x)\equiv c \in\mathbb{R}$; $f'(x)= \displaystyle \lim_{h\to 0} \frac{f(x+h)-f(x)}{h} = \lim_{h\to 0} \frac{c-c}{h} = 0$\\

$g(x)=x$; $g'(x) = \displaystyle \lim_{h\to 0} \frac{g(x+h)-g(x)}{h} = \lim_{h\to 0} \frac{x+h-x}{h} = 1$\\

$i(x)=x^2$; $i'(x) = \displaystyle \lim_{h\to 0} \frac{(x+h)^2-x^2}{h} = \lim_{h\to 0} \frac{h^2+2xh}{h} = \lim_{h\to 0} h+2x = 2x$\\

$j(x)=x^n, n\in\mathbb{N}$. Ennek a deriváltjához a binomiális-tételt alkalmazzuk:

\begin{lemma} $\boxed{$Binomiális-tétel$}$
\[(a+b)^n = \binom{n}{0}a^n + \binom{n}{1}a^{n-1}\cdot b + \binom{n}{2}a^{n-2}\cdot b^2 + \ldots + \binom{n}{n-1}a\cdot b^{n-1} + \binom{n}{n}b^n\]
\end{lemma}

$j'(x) = \displaystyle \lim_{h\to 0} \frac{(x+h)^n-x^n}{h}=\lim_{h\to 0} \frac{1}{h}\left( n\cdot x^{n-1}\cdot h + \binom{n}{2}x^{n-2}\cdot h^2 + \ldots + h^n \right) = n\cdot x^{n-1}$

Tehát természetes számokra igaz. Bármely valós számra megcsinálhatjuk a definíció segedelmével a bizonyítást, tehát igaz lesz az állítás valók számokra is. Nézzünk meg egy két példát:

$j(x) := \sqrt{x} = x^{\frac{1}{2}}$.
\[(\sqrt{x})' = \lim_{h\to 0} \frac{\sqrt{x+h}-\sqrt{x}}{h}\cdot\frac{\sqrt{x+h}+\sqrt{x}}{\sqrt{x+h}+\sqrt{x}} = \lim_{h\to 0} \frac{x+h-x}{h\cdot (\sqrt{x+h}+\sqrt{x})} = \]
\[= \lim_{h\to 0} \frac{h}{h\cdot (\sqrt{x+h}+\sqrt{x})} = \lim_{h\to 0} \frac{1}{\cdot (\sqrt{x+h}+\sqrt{x})} = \frac{1}{2\sqrt{x}} = \frac{1}{2} \cdot x^{-\frac{1}{2}} \quad \checkmark\]

$j(x) := \dfrac{1}{x} = x^{-1}$.
\[\left(\frac{1}{x}\right)' = \lim_{h\to 0} \frac{1}{h}\cdot\left(\frac{1}{x+h}-\frac{1}{x}\right) = \lim_{h\to 0} \frac{1}{h}\cdot\left(\frac{x}{x(x+h)}-\frac{x+h}{x(x+h)}\right) =\]
\[ = \lim_{h\to 0} \frac{-1}{x(x+h)} = \frac{-1}{x^2} = -1\cdot x^{-2} \quad \checkmark\]

Tehát: $\boxed{(x^n)' = n\cdot x^{n-1} \quad n\in\mathbb{R}}$ (később explicit bizonyítjuk)

\begin{tetelAbraval}
 \[\boxed{\lim_{x\to 0} \frac{\sin x}{x} = 1}\]
\end{tetelAbraval}
\begin{wrapfigure}{r}{0.2\textwidth}
\psset{xunit=2.0cm,yunit=2.0cm,algebraic=true,dotstyle=o,dotsize=3pt 0,linewidth=0.8pt,arrowsize=3pt 2,arrowinset=0.25}
\begin{pspicture*}(-0.3,-0.3)(1.4,1.4)
\psaxes[xAxis=true,yAxis=true,labels=none,Dx=0.2,Dy=0.2,ticksize=-2pt 0,subticks=2]{->}(0,0)(-0.3,-3)(1.4,1.4)
\pscircle[linewidth=0.4pt](0,0){2}
\parametricplot[linewidth=1.6pt]{0.0}{0.5881287812822467}{1*1*cos(t)+0*1*sin(t)+0|0*1*cos(t)+1*1*sin(t)+0}
\psline(0.83,0.55)(0,0)
\psline(1,0)(0,0)
\psline(0.83,0.55)(0.83,0)
\psline(1,-0.29)(1,1.11)
\psline(0.83,0.55)(1,0.67)
\psdots[dotstyle=*](1,0)
\rput[bl](1.05,0.05){$A$}
\psdots[dotstyle=*](0.83,0.55)
\rput[bl](0.70,0.61){$B$}
\psdots[dotstyle=*](0,0)
\rput[bl](-0.22,-0.22){$O$}
\psdots[dotstyle=*](0.83,0)
\rput[c](0.83,-0.15){$D$}
\psdots[dotstyle=*](1,0.67)
\rput[bl](1.04,0.69){$C$}
\end{pspicture*}
\vspace{-20pt}
\end{wrapfigure}

% Itt nem rakunk biz environmentet, mert a lebegő ábra nem lesz jó. Csak a headert írjuk ki.
\begin{bizAbraval} 
 \[T_{OAC} = \frac{1}{2}\cdot \frac{\sin x}{\cos x} \qquad T_{OAB\hbox{ \small körcikk}}=\frac{1}{2}\cdot 1\cdot x \qquad T_{OBD} = \frac{1}{2}\cdot\sin x\cos x\]
 \[T_{OBD} \leqslant T_{OAB\hbox{ \small körcikk}} \leqslant T_{OAC}\]
 \[ \frac{1}{2}\cdot\sin x\cos x \leqslant \frac{x}{2} \leqslant \frac{1}{2}\frac{\sin x}{\cos x} \qquad /\cdot\frac{2}{\sin x}\]
 \[ \cos x \leqslant \frac{x}{\sin x} \leqslant \frac{1}{\cos x}\]
 \[ \begin{array}{ccc}
     \dfrac{1}{\cos x} & \geqslant \dfrac{\sin x}{x} \geqslant & \cos x \\
      \downarrow & & \downarrow \\
       1/1 & & 1
    \end{array}\]
 Rendőr szabály használva: $\dfrac{\sin x}{x} \to 1$
\end{bizAbraval}

\subsubsection{Szinusz és koszinusz deriváltjai}

\[(\sin x)' = \lim_{h\to 0} \frac{\sin(x+h)-\sin x}{h} = \lim_{h\to 0} \frac{\sin x\cos h+\sin h\cos x - \sin x}{h} = \]
\[= \lim_{h\to 0} \left(\sin x\cdot \underbrace{\frac{\cos h - 1}{h}}_{\to 0}+\cos x\cdot \underbrace{\frac{\sin h}{h}}_{\to 1}\right) = \boxed{\cos x}\]

\[(\cos x)' = \lim_{h\to 0} \frac{\cos(x+h)-\cos x}{h} =\lim_{h\to 0} \frac{\cos x\cos h-\sin x\sin h - \cos x}{h} = \]
\[= \lim_{h\to 0} \left(\cos x\cdot \underbrace{\frac{\cos h - 1}{h}}_{\to 0}-\sin x\cdot \underbrace{\frac{\sin h}{h}}_{\to 1}\right) = \boxed{-\sin x}\]

\begin{wrapfigure}{r}{0.41\textwidth}
\vspace{10pt}
\psset{xunit=1.0cm,yunit=1.0cm,algebraic=true,dotstyle=o,dotsize=3pt 0,linewidth=0.8pt,arrowsize=3pt 2,arrowinset=0.25}
\begin{pspicture*}(-1.36,-1.12)(5.49,4.43)
\psaxes[labelFontSize=\scriptstyle,xAxis=true,yAxis=true,labels=none,Dx=1,Dy=1,ticksize=-2pt 0,subticks=2]{->}(0,0)(-1.36,-1.12)(5.49,4.43)
\psplot[plotpoints=200]{-1.3552321774978247}{5.4890617887906625}{0.0193*x^3+0.1060*x^2-0.04600*x+0.9393}
\psline(3.86,3.45)(1.57,1.2)
\psline[linestyle=dotted](3.86,3.45)(3.86,0)
\psline[linestyle=dotted](1.57,1.2)(1.57,0)
\psline[linestyle=dotted](1.57,1.2)(3.86,1.2)
\psplot[linewidth=1.6pt]{-1.36}{5.49}{(--0.53--0.43*x)/1}
\pscustom{\parametricplot{0.0}{0.40461146164215567}{1.12*cos(t)+1.57|1.12*sin(t)+1.2}\lineto(1.57,1.2)\closepath}
\rput[bl](-1.28,0.92){$f$}

\rput(1.57,1.2){\rnode{E}{}}
\rput(3.86,3.45){\rnode{F}{}}
\rput(3.86,1.2){\rnode{G}{}}
\rput(3.86,2.18){\rnode{M}{}}

\psbrace[ref=c,rot=90,nodesepA=0pt,nodesepB=-7pt](E)(G){\small$h$}
\psbrace[ref=c,rot=0,nodesepA=10pt,nodesepB=0pt](G)(F){\small$\Delta f$}
\psbrace[ref=c,rot=180,nodesepA=-11pt,nodesepB=-2pt](M)(G){\small$A h$}

\psdots[dotstyle=*](3.86,0)
\rput[c](3.86,-0.3){$x_0+h$}

\psdots[dotstyle=*](1.57,0)
\rput[c](1.57,-0.3){$x_0$}

\rput[bl](-0.92,0.36){$e$}
\end{pspicture*}
\end{wrapfigure}

\begin{tetelAbraval} $\boxed{$Szükséges és elégséges feltétel a differenciálhatóságra$}$\\

Legyen $x_0 \in \rm{Inf~Df.}$
\[\exists f'(x_0) \quad \Leftrightarrow \quad \Delta f = f(x_0+h)-f(x_0) = \]
\[ = \underbrace{A\cdot h}_{\hbox{főrész}} + \underbrace{\varepsilon(h)\cdot h}_{\hbox{elenyésző rész}} \qquad A\in\mathbb{R}\]
\[\varepsilon(h) \xrightarrow{h\to 0} 0 \quad \hbox{ ekkor } \quad A = f'(x_0)\]

Ezért hívjuk a \emph{főrész}t másképpen \emph{linearizált növekmény}nek, hiszen ha $h\to 0$, akkor $A\cdot h = f'(x_0)\cdot h$.
\end{tetelAbraval}

\begin{bizAbraval}
$\Rightarrow$
    \[\hbox{Tudjuk, hogy } \; \exists \lim_{h\to 0} \frac{f(x_0+h)-f(x_0)}{h}=f'(x_0) \in \mathbb{R}\]
    \[\frac{f(x_0+h)-f(x_0)}{h} = f'(x_0) + \underbrace{\varepsilon(h)}_{\to 0} \qquad /\cdot h\]
    \[f(x_0+h)-f(x_0) = \underbrace{f'(x_0)}_{A}\cdot h + \varepsilon(h)\cdot h\]
$\Leftarrow$
    \[f(x_0+h)-f(x_0) = A\cdot h + \underbrace{\varepsilon(h)}_{\xrightarrow{h\to 0} 0}\cdot h \qquad /: h\]
    \[\frac{f(x_0+h)-f(x_0)}{h} = A + \varepsilon(h)\]
    \[\lim_{h\to 0} \frac{f(x_0+h)-f(x_0)}{h} = A = f'(x_0) \in \mathbb{R}\]
\end{bizAbraval}

Következmény:
\begin{tetel} $\boxed{$Deriválhatóság szükséges feltétele: folytonosság$}$
 \[\exists f'(x_0) \quad \Rightarrow \quad \hbox{$f$ folytonos $x_0$-ban}\]
\end{tetel}
\begin{biz}
 \[\exists f'(x_0) \Leftrightarrow f(x_0+h)-f(x_0) = f'(x_0)\cdot h + \varepsilon(h)\cdot h\]
 \[\lim_{x\to x_0} f(x) = \lim_{h\to 0} f(x_0+h) = \lim_{h\to 0} f(x_0) + \underbrace{f'(x_0)\cdot h}_{\to 0} + \underbrace{\varepsilon(h)\cdot h}_{\to 0} = f(x_0)\]
 Ha megnézzük akkor a fenti sor a folytonosság definíciója, tehát $f$ folytonos $x_0$-ban.
\end{biz}

\emph{Megjegyzés}: A feltétel nem elégséges! (Például: $f(x)=|x|$, hiszen folytonos 0-ban, de nem itt differenciálható)

\begin{defi}
 \textbf{Függvény differenciálja}: $f$ függvény elsőrendű differenciálja az $x_0$ helyen a $h$ növekmény mellett:
\[df(x_0, h) = \underbrace{f'(x_0)\cdot h}_{\hbox{linearizált főrész}}\]
 Egyéb jelölések:
 \[df = f'(x)\cdot dx = df(x,dx)\]
 \[\frac{df}{dx} = f'(x)\]
\end{defi}
Pl.: $d(x^3) = 3x^2\, dx$\\
$d(x^2) = (x+dx)^2-x^2 = x^2+2x\,dx + dx^2-x^2 \approx 2x\cdot dx$ (elsőrendben számolunk!)

Közelítéseknél lehet jól alkalmazni:
\[\Delta f = f(x_0+h)-f(x_0) \approx df(x_0, dx) = f'(x_0)\cdot dx\]

\subsection{Érintő egyenlete}

Egy érintő egyenlete az $x_0$ pontban:
\[y = f'(x_0)\cdot(x-x_0) + f(x_0) \]
Nem nehéz meggondolni miért: a meredeksége a függvény deriváltja az $x_0$ pontban, és rajta van az $(x_0, f(x_0))$ pont (az érintési pont).

\subsection{Deriválási szabályok}

\begin{tetel} Ha $\exists f'(x_0)$ és $c\in\mathbb{R}$
 \[\boxed{\exists(c\cdot f)'(x_0) = c\cdot f'(x_0)}\]
\end{tetel}
\begin{biz}
 \[(c\cdot f)'(x_0) = \lim_{h\to 0} \frac{(cf)(x_0+h)-(cf)(x_0)}{h} = \lim_{h\to 0} \frac{c\cdot f(x_0+h)-c\cdot f(x_0)}{h} =\]
 \[ = c\cdot\lim_{h\to 0} \frac{f(x_0+h)-f(x_0)}{h} = c\cdot f'(x_0)\]
\end{biz}

\begin{tetel} Ha $\exists f'(x_0)$ és $\exists g'(x_0)$
 \[\boxed{\exists(f+g)'(x_0) = f'(x_0)+g'(x_0)}\]
\end{tetel}
\begin{biz}
 \[(f+g)'(x_0) = \lim_{h\to 0} \frac{(f+g)(x_0+h)-(f+g)(x_0)}{h} = \]
 \[ = \lim_{h\to 0} \frac{f(x_0+h)+g(x_0+h)-(f(x_0)+g(x_0))}{h} =\]
 \[ = \lim_{h\to 0} \frac{f(x_0+h)-f(x_0)}{h} \; + \; \frac{g(x_0+h)-g(x_0)}{h} = f'(x_0) + g'(x_0)\]
\end{biz}

\begin{tetel} Ha $\exists f'(x_0)$ és $\exists g'(x_0)$; \textbf{Leibniz-szabály}:
 \[\boxed{\exists(f\cdot g)'(x_0) = f'(x_0)\cdot g(x_0) + g'(x_0)\cdot f(x_0)}\]
\end{tetel}
\begin{biz}
 \[(f\cdot g)'(x_0) = \lim_{h\to 0} \frac{(f\cdot g)(x_0+h)-(f\cdot g)(x_0)}{h} = \]
 \[ = \lim_{h\to 0} \frac{f(x_0+h)\cdot g(x_0+h)-f(x_0)\cdot g(x_0)}{h} =\]
 \[ = \lim_{h\to 0} \frac{f(x_0+h)\cdot g(x_0+h)-f(x_0)\cdot g(x_0)+\overbrace{-f(x+h)g(x)+f(x+h)g(x)}^{=0 \quad \hbox{trükk $\ddot\smile$}}}{h} =\]
 \[ = \lim_{h\to 0} \frac{f(x_0+h)\cdot \Big(g(x_0+h)-g(x)\Big)+g(x_0)\cdot\Big(f(x+h)-f(x_0)\Big)}{h} = \]
 \[ = f(x_0)\cdot g'(x_0)+g(x_0)\cdot f'(x_0)\]
\end{biz}

\begin{tetel} Ha $\exists g'(x_0)$ és $g(x_0) \neq 0$
 \[\boxed{\exists\left(\frac{1}{g}\right)'(x_0) = \frac{-g'(x_0)}{g^2(x_0)}}\]
\end{tetel}
\begin{biz}
 \[\left(\frac{1}{g}\right)'(x_0) = \lim_{h\to 0} \frac{1}{h}\cdot\left(\frac{1}{g(x_0+h)}-\frac{1}{g(x_0)}\right) = \lim_{h\to 0} \frac{1}{h}\cdot \frac{g(x_0)-g(x_0+h)}{g(x_0+h)g(x_0)} = \]
 \[ = \lim_{h\to 0} \frac{-\Big(g(x_0+h)-g(x_0)\Big)}{h}\cdot\frac{1}{g(x_0+h)g(x_0)} = \frac{-g'(x_0)}{g^2(x_0)}\]
\end{biz}

\begin{tetel} Ha $\exists f'(x_0)$, $\exists g'(x_0)$ és $g(x_0) \neq 0 \qquad $(Az előző két tétel következménye)
  \[\boxed{\exists\left(\frac{f}{g}\right)'(x_0) = \frac{f'(x_0)\cdot g(x_0)-g'(x_0)\cdot f(x_0)}{g^2(x_0)}}\]
\end{tetel}
\begin{biz}
 \[\left(\frac{f}{g}\right)'(x_0) = \left(f\cdot \frac{1}{g}\right)'(x_0) = f'(x_0)\cdot\frac{1}{g}(x_0)+f(x_0)\cdot\left(\frac{-g'(x_0)}{g^2(x_0)}\right) = \]
 \[ = \frac{f'(x_0)\cdot g(x_0)-g'(x_0)\cdot f(x_0)}{g^2(x_0)}\]
\end{biz}

\subsubsection{Példák}

$\displaystyle \left(x^3+\sqrt{x}-\frac{2}{x^3}\right)' = 3x^2+\frac{1}{2}\cdot x^{\frac{-1}{2}}-2\cdot (-3)\cdot x^{-4}$\\
$\displaystyle (\tg x)' = \left(\frac{\sin x}{\cos x}\right)' = \frac{\cos x\cos x-\sin x\cdot -\sin x}{\cos^2 x} = \frac{\cos^2 x+\sin^2 x}{\cos^2 x} = \frac{1}{\cos^2 x}$

\begin{tetel} $\boxed{$Összetett függvény deriválása; \textbf{láncszabály}$}$\\
 Ha $g$ differenciálható $x_0$ helyen és $f$ differenciálható $g(x_0)$ helyen, akkor
 \[\exists (f\circ g)'(x_0) = f'\circ g(x_0)\cdot g'(x_0) = f'(g(x_0))\cdot g'(x_0)\]
\end{tetel}
\begin{biz}
 \[(f\circ g)'(x_0) = \lim_{h\to 0} \frac{(f\circ g)(x_0+h)-(f\circ g)(x_0)}{h} = \]
 \[ = \lim_{h\to 0} \frac{f(g(x_0+h))-f(g(x_0))}{g(x_0+h)-g(x_0)}\cdot \underbrace{\frac{g(x_0+h)-g(x_0)}{h}}_{\to g'(x_0)}\]
 $\Delta g = g(x_0+h)-g(x_0)$. Mivel $h\to 0$, ezért $\Delta g \to 0$ ($g$ folytonos) tehát:
 \[\lim_{\Delta g\to 0} \frac{f(g(x_0)+\Delta g)-f(g(x_0))}{\Delta g} \cdot \lim_{h\to 0} \frac{g(x_0+h)-g(x_0)}{h} = f'(g(x_0))\cdot g'(x_0)\]
\end{biz}

\subsubsection{Példák}
$(\sin 2x)' = \cos (2x) \cdot 2$.\\
Másképp: $(\sin 2x)' = (2\cos x\sin x)' = -2\cdot sin^2 x + 2\cos^2 x = 2(\cos^2 x-\sin^2 x) = 2\cos 2x$.\\

$\left(\sqrt{\tg(3x^3+2x)}\right)' = \dfrac{1}{2}(\tg(3x^3+2x))^{-\frac{1}{2}}\cdot \dfrac{1}{\cos^2 (3x^3+2x)}\cdot (9x^2+2)$

\subsection{Inverz függvény deriválása}

\begin{defi}
 $f$ az $I$ intervallumon \textbf{szigorúan monoton nő}, ha $x_1 < x_2$, $x_1, x_2 \in I$ esetén $f(x_1) < f(x_2)$.
\end{defi}

\begin{defi}
 $f$ az $I$ intervallumon \textbf{szigorúan monoton csökken}, ha $x_1 < x_2$, $x_1, x_2 \in I$ esetén $f(x_1) > f(x_2)$.
\end{defi}

\begin{tetel}
 Ha $f$ szig. mon. az $I$ intervallumon, akkor $\exists f^{-1}$ (azaz $f$ \textbf{injektív}).
\end{tetel}
\begin{biz}
 Tegyük fel, hogy $f$ szig. mon. nő. Ha $f(x_1) = f(x_2)$, akkor $x_1 = x_2$, különben nem teljesül a szig. monotonitás. Tehát nem lesz olyan érték, amit a függvény kétszer vesz fel, tehát az inverze is függvény marad.
\end{biz}

\begin{tetel}
 Ha $D_f = I$ intervallum és $f$ szig. mon. $I$-n ($\Rightarrow \exists f^{-1}$) és $f$ folytonos $I$-n, akkor $f(I)$ is intervallum és ezen $f^{-1}$ folytonos.
\end{tetel}
\addtocounter{biz}{1}

\begin{tetel} $\boxed{$Inverz függvény deriváltja$}$\\
Legyen
\begin{itemize*}
 \item $f$ szig. mon. az $I$-n ($\rightsquigarrow \exists f^{-1}$)
 \item $f$ differenciálható $I$-n ($\rightsquigarrow f$ folytonos és $f^{-1}$ is)
 \item $f'(x) \neq 0$, ha $x\in I$
\end{itemize*}
Ekkor $f^{-1}$ differenciálható $I$ belsejében és
\[(f^{-1})'(x) = \frac{1}{f'(f^{-1}(x))}\]

\end{tetel}
\begin{biz}
 \[f^{-1}(f(x)) = x \qquad \overset{\frac{d}{dx}}{\Longrightarrow} \qquad (f^{-1})'(f(x))\cdot f'(x)= 1\]
 \[(f^{-1})'(f(x)) = \frac{1}{f'(x)} \qquad \Rightarrow \qquad (f^{-1})'(y) = \frac{1}{f'(f^{-1}(y))}\]
\end{biz}

\subsubsection{Szögfüggvények inverzei}

\begin{wrapfigure}{r}{0.3\textwidth}
   \vspace{-35pt}
\begin{center}

\psset{xunit=1.0cm,yunit=1.0cm,algebraic=true,dotstyle=o,dotsize=3pt 0,linewidth=0.8pt,arrowsize=3pt 2,arrowinset=0.25}
\begin{pspicture*}(-2.31,-2.41)(2.42,2.52)
\psaxes[xAxis=true,yAxis=true,Dx=1,Dy=1,labels=none,ticksize=-2pt 0,subticks=2]{->}(0,0)(-2.31,-2.41)(2.42,2.52)
\psplot[plotpoints=200]{-2.3138312601182194}{2.4205618377538336}{sin(x)}
\psplot[linewidth=1.6pt,plotpoints=200]{-1}{1}{asin(x)}
\psline[linestyle=dotted](-1,1)(1,1)
\psline[linestyle=dotted](1,1)(1,-1)
\psline[linestyle=dotted](1,-1)(-1,-1)
\psline[linestyle=dotted](-1,-1)(-1,1)
\psline[linestyle=dotted](-1.57,1.57)(1.57,1.57)
\psline[linestyle=dotted](1.57,1.57)(1.57,-1.57)
\psline[linestyle=dotted](1.57,-1.57)(-1.57,-1.57)
\psline[linestyle=dotted](-1.57,-1.57)(-1.57,1.57)
\psplot[linewidth=0.4pt,linestyle=dashed,dash=4pt 4pt]{-2.31}{2.42}{(-0--1*x)/1}
\psdots[dotstyle=*](1,1.57)
\psdots[dotstyle=*](-1,-1.57)
\rput[bl](-2.24,-0.6){$\sin x$}
\rput[bl](-1.5,-2){$\arcsin x$}
\rput[c](1,-0.3){1}
\rput[c](1.57,-0.6){$\dfrac{\pi}{2}$}
\end{pspicture*}

\end{center}
   \vspace{-70pt}
\end{wrapfigure}

A $\sin x$ függvény szig. mon. nő a $\left[-\dfrac{\pi}{2};\dfrac{\pi}{2}\right]$-ben, tehát itt invertálható: $\sin^{-1} x = \arcsin x$. $\arcsin x: [-1; 1] \mapsto \left[-\dfrac{\pi}{2};\dfrac{\pi}{2}\right]$
\[(\arcsin x)' = \dfrac{1}{\sin'(\arcsin x)} = \dfrac{1}{\cos(\underbrace{\arcsin x}_{\alpha})} \qquad (\sin \alpha = x)\]
Mivel $\cos^2 \alpha = (1-\sin^2 \alpha)$, ezért $\cos\alpha = \pm\sqrt{1-\sin^2\alpha}$, de mivel $\alpha \in \left[-\dfrac{\pi}{2};\dfrac{\pi}{2}\right]$, ezért $\cos\alpha \geqslant 0$, tehát:
\[(\arcsin x)' = \dfrac{1}{\sqrt{1-\sin^2\alpha}} = \boxed{\frac{1}{\sqrt{1-x^2}}} \quad \hbox{ha } x\in(-1; 1) \]\\

\newpage

\begin{wrapfigure}{r}{0.3\textwidth}
   \vspace{-20pt}
\begin{center}

\psset{xunit=1.0cm,yunit=1.0cm,algebraic=true,dotstyle=o,dotsize=3pt 0,linewidth=0.8pt,arrowsize=3pt 2,arrowinset=0.25}
\begin{pspicture*}(-2.31,-2.41)(2.42,2.52)
\psaxes[xAxis=true,yAxis=true,labels=none,Dx=1,Dy=1,ticksize=-2pt 0,subticks=2]{->}(0,0)(-2.31,-2.41)(2.42,2.52)
\psplot[plotpoints=200]{-1.5707}{1.5707}{tan(x)}
\psplot[linewidth=1.6pt,plotpoints=200]{-2.3138312601182194}{2.4205618377538336}{ATAN(x)}
\psline[linestyle=dotted](-1.57,1.57)(1.57,1.57)
\psline[linestyle=dotted](1.57,1.57)(1.57,-1.57)
\psline[linestyle=dotted](1.57,-1.57)(-1.57,-1.57)
\psline[linestyle=dotted](-1.57,-1.57)(-1.57,1.57)
\psplot[linewidth=0.4pt,linestyle=dashed,dash=4pt 4pt]{-2.31}{2.42}{(-0-1*x)/-1}
\rput[bl](-0.9,-2.3){$\tg x$}
\rput[bl](-2.24,-0.75){$\arctg x$}
\rput[c](1.57,-0.6){$\dfrac{\pi}{2}$}
\end{pspicture*}

\end{center}
   \vspace{-70pt}
\end{wrapfigure}

A $\tg x$ függvény szig. mon. nő a $\left(-\dfrac{\pi}{2};\dfrac{\pi}{2}\right)$-ben, tehát itt invertálható: $\tg^{-1} x = \arctg x$. $\arctg x: \mathbb{R} \mapsto \left(-\dfrac{\pi}{2};\dfrac{\pi}{2}\right)$.
\[(\arctg x)' = \frac{1}{\tg' (\arctg x)} = \frac{1}{\frac{1}{\cos^2 (\arctg x)}} = \cos^2 (\underbrace{\arctg x}_{\alpha})\]
Mivel $\tg \alpha = x$ és $\cos^2 \alpha = \dfrac{\cos^2 \alpha}{\cos^2 \alpha + \sin^2 \alpha} = \dfrac{1}{1+\tg^2 \alpha}$, ezért:
\[(\arctg x)' = \cos^2 \alpha = \frac{1}{1+\tg^2 \alpha} = \boxed{\frac{1}{1+x^2}}\]\\

\begin{wrapfigure}{r}{0.3\textwidth}
   \vspace{-30pt}
\begin{center}

\psset{xunit=1.0cm,yunit=1.0cm,algebraic=true,dotstyle=o,dotsize=3pt 0,linewidth=0.8pt,arrowsize=3pt 2,arrowinset=0.25}
\begin{pspicture*}(-2.27,-1.67)(2.42,3.3)
\psaxes[xAxis=true,yAxis=true,labels=none,Dx=1,Dy=1,ticksize=-2pt 0,subticks=2]{->}(0,0)(-2.28,-1.67)(2.45,3.26)
\psplot[plotpoints=200]{-2.2846965333620837}{2.4496965645099693}{cos(x)}
\psplot[linewidth=1.6pt,plotpoints=200]{-0.9999952942890197}{0.9999980577682834}{acos(x)}
\psline[linestyle=dotted](-1,3.14)(1,3.14)
% \psline[linestyle=dotted](1,-1)(-1,-1)
\psline[linestyle=dotted](-1,0)(-1,3.14)
\psline[linestyle=dotted](1,3.14)(1,0)
\psline[linestyle=dotted](-1.57,1.57)(1.57,1.57)
\psline[linestyle=dotted](1.57,1.57)(1.57,-1.57)
\psline[linestyle=dotted](1.57,-1.57)(-1.57,-1.57)
\psline[linestyle=dotted](-1.57,-1.57)(-1.57,1.57)
\psplot[linewidth=0.4pt,linestyle=dashed,dash=4pt 4pt]{-2.27}{2.42}{(-0--1*x)/1}
\psdots[dotstyle=*](1,0)
\psdots[dotstyle=*](-1,3.14)
\rput[bl](-2.21,-0.9){$\cos x$}
\rput[bl](-2.15,2.0){$\arccos x$}
\rput[c](1,-0.3){1}
\rput[l](0.25,3.14){$\pi$}
\rput[c](1.57,-0.6){$\dfrac{\pi}{2}$}
\end{pspicture*}

\end{center}
   \vspace{-70pt}
\end{wrapfigure}

A $\cos x$ függvény szig. mon. csökken a $\left[0;\pi\right]$ intervallumon, tehát itt invertálható: $\cos^{-1} x = \arccos x$. $\arccos x: [-1; 1] \mapsto \left[0; \pi\right]$
\[(\arccos x)' = \frac{1}{\cos'(\arccos x)} = \frac{1}{-\sin(\underbrace{\arccos x}_{\alpha})} \qquad (\cos \alpha = x)\]
Mivel $\sin^2 \alpha = (1-\cos^2 \alpha)$, ezért $\sin\alpha = \pm\sqrt{1-\cos^2\alpha}$, de mivel $\alpha \in [0;\pi]$, ezért $\sin\alpha \leqslant 0$, tehát:
\[(\arccos x)' = \frac{1}{-\sqrt{1-\cos^2(\alpha)}} = \boxed{\frac{1}{-\sqrt{1-x^2}}} \quad \hbox{ha } x\in(-1; 1) \]\\

\begin{wrapfigure}{r}{0.3\textwidth}
   \vspace{-30pt}
\begin{center}

\psset{xunit=1.0cm,yunit=1.0cm,algebraic=true,dotstyle=o,dotsize=3pt 0,linewidth=0.8pt,arrowsize=3pt 2,arrowinset=0.25}
\begin{pspicture*}(-1.4,-1.1)(3.34,4.02)
\psaxes[xAxis=true,yAxis=true,labels=none,Dx=1,Dy=1,ticksize=-2pt 0,subticks=2]{->}(0,0)(-1.4,-0.91)(3.34,4.02)
\psplot[plotpoints=200]{0.0001}{3.1410}{1/tan(x)}
\psplot[linewidth=1.6pt,plotpoints=200]{-1.3960873672999445}{3.338305730572108}{3.1416/2-ATAN(x)}
\psline[linestyle=dotted](-1.57,1.57)(1.57,1.57)
\psline[linestyle=dotted](1.57,1.57)(1.57,-1.57)
\psline[linestyle=dotted](1.57,-1.57)(-1.57,-1.57)
\psline[linestyle=dotted](-1.57,-1.57)(-1.57,1.57)
\psplot[linewidth=0.4pt,linestyle=dashed,dash=4pt 4pt]{-1.4}{3.34}{(-0-1*x)/-1}
\rput[bl](0.5,3.4){$\ctg x$}
\rput[bl](1.8,0.6){$\arcctg x$}
\rput[c](1.57,-0.6){$\dfrac{\pi}{2}$}
\end{pspicture*}

\end{center}
   \vspace{-20pt}
\end{wrapfigure}

A $\ctg x$ függény szig. mon. csökken a $\left[0;\pi\right]$ intervallumon, tehát itt invertálható: $\ctg^{-1} x = \arcctg x$. $\arcctg x: \mathbb{R} \mapsto [0;\pi]$
\[(\arcctg x)' = \frac{1}{\ctg'(\arcctg x)} = \frac{1}{-\frac{1}{\sin^2(\arcctg x)}} = -\sin^2(\underbrace{\arcctg x}_{\alpha})\]
\[(\arcctg x)' = -\sin^2 \alpha = \frac{-\sin^2 \alpha}{sin^2\alpha + \cos^2\alpha} = -\frac{1}{1+\ctg^2 \alpha}\]
Mivel $\ctg \alpha = x$, ezért:
\[(\arcctg x)' = -\frac{1}{1+\ctg^2 \alpha} = \boxed{-\frac{1}{1+x^2}}\]\\

\newpage

\subsection{Exponenciális függvények}

\begin{wrapfigure}{r}{0.37\textwidth}
   \vspace{-40pt}
\begin{center}
\psset{xunit=1.0cm,yunit=1.0cm,algebraic=true,dotstyle=o,dotsize=3pt 0,linewidth=0.8pt,arrowsize=3pt 2,arrowinset=0.25}
\begin{pspicture*}(-2.71,-1.5)(3.22,5.05)
\psaxes[xAxis=true,yAxis=true,labels=none,Dx=1,Dy=1,ticksize=-2pt 0,subticks=2]{->}(0,0)(-2.71,-0.3)(3.22,5.05)
\psplot[plotpoints=200]{-2.7071500713260512}{3.2217668235475654}{3^x}
\psplot[plotpoints=200]{-2.7071500713260512}{3.2217668235475654}{(1/2)^x}
\psplot[plotpoints=200]{-2.7071500713260512}{3.2217668235475654}{130^x}
\psplot[plotpoints=200]{-2.7071500713260512}{3.2217668235475654}{10^x}
\psplot[plotpoints=200]{-2.7071500713260512}{3.2217668235475654}{(1/10)^x}
\psline[linestyle=dotted](1,-0.3)(1,5.05)
\psplot[linestyle=dashed,dash=2pt 2pt]{-2.71}{3.22}{(--1-0*x)/1}
\rput[bl](-2.10,4.50){$b_1^x$}
\rput[bl](-1.2,4.50){$b_2^x$}
\rput[bl](0.2,4.5){$a_3^x$}
\rput[bl](1.55,4.5){$a_1^x$}
\rput[bl](0.76,4.5){$a_2^x$}
\psdots[dotstyle=*](1,0.5)
\rput[bl](1.05,0.5){$(1;b_1)$}
\psdots[dotstyle=*](1,0.1)
\rput[bl](1.05,-0.6){$(1;b_2)$}
\psdots[dotstyle=*](1,3)
\rput[bl](1.1,2.8){$(1;a_1)$}
\rput[bl](-2.65,-0.9){$a_3>a_2>a_1>1$}
\rput[bl](-2.45,-1.4){$0<b_2<b_1<1$}
\end{pspicture*}
\end{center}
   \vspace{-80pt}
\end{wrapfigure}

$f: \mathbb{R} \mapsto (0; \infty)$. $f(x) = a^x$ ($a\in\mathbb{R} > 0$)

\subsubsection{Tulajdonságok}
\begin{enumerate*}
 \item $f(x) = a^x$ folytonos $\mathbb{R}$-ben
 \item $a^0 = 1; \quad a^1 = a$
 \item $f(x) = a^x \quad \begin{cases}
                           \hbox{szig. mon. nő, ha } a>1\\
			   \hbox{konstans 1, ha } a=1\\
			   \hbox{szig. mon. csökken, ha } 0<a<1.
                         \end{cases}$
 \item $a^{x+y} = a^x\cdot a^y$
 \item $(a^x)^y = a^{x\cdot y}$
 \item $\displaystyle \lim_{x\to +\infty} a^x =
  \begin{cases}
    0, \hbox{ ha } 0<a<1\\
    1, \hbox{ ha } a=1\\
    \infty, \hbox{ ha } a>1
  \end{cases}$\\
  $\displaystyle \lim_{x\to -\infty} a^x =
  \begin{cases}
    \infty, \hbox{ ha } 0<a<1\\
    1, \hbox{ ha } a=1\\
    0, \hbox{ ha } a>1
  \end{cases}$
\end{enumerate*}

\begin{tetel} Ha $a=e$, akkor $x\mapsto e^x$ meredeksége 0-ban 1. Tehát $g(x) := e^x$, ekkor $g'(0) = 1$, azaz $\displaystyle \lim_{h\to 0} \frac{e^h-e^0}{h} = 1.$
\end{tetel}
\addtocounter{biz}{1}

\[(e^x)' = \lim_{h\to 0} \frac{e^{x+h}-e^{x}}{h} = \lim_{h\to 0} e^x\cdot \underbrace{\frac{e^h-1}{h}}_{\to 1} = \boxed{e^x}\]

\subsection{Logaritmus függvények}

\begin{wrapfigure}{r}{0.37\textwidth}
   \vspace{-40pt}
\begin{center}
\psset{xunit=1.0cm,yunit=1.0cm,algebraic=true,dotstyle=o,dotsize=3pt 0,linewidth=0.8pt,arrowsize=3pt 2,arrowinset=0.25}
\begin{pspicture*}(-0.45,-3.60)(5.48,3.07)
\psaxes[xAxis=true,yAxis=true,labels=none,Dx=1,Dy=1,ticksize=-2pt 0,subticks=2]{->}(0,0)(-0.45,-3.06)(5.48,3.07)
\psplot[plotpoints=200]{0.05}{5.479708147148083}{ln(x)}
\psplot[plotpoints=200]{0.004}{5.479708147148083}{log(x)}
\psplot[plotpoints=200]{1.569731460386284E-6}{5.479708147148083}{ln(x)/ln(1/2)}
\psplot[linewidth=0.4pt,linestyle=dashed,dash=4pt 4pt]{-0.45}{5.48}{(--1-0*x)/1}
\psplot[plotpoints=200]{1.569731460386284E-6}{5.479708147148083}{ln(x)/ln(1/10)}
\rput[tl](-0.4,-0.05){$\log(x)$}
\rput[bl](0.32,-2.03){$\ln(x)$}
\psdots[dotstyle=*](0.5,1)
\rput[bl](0.53,1.2){$(b_1, 1)$}
\psdots[dotstyle=*](2.72,1)
\rput[bl](2.16,1.17){$(e, 1)$}
\psdots[dotstyle=*](0.1,1)
\rput[bl](0.13,0.4){$(b_2, 1)$}
\rput[tl](4,-0.7){$\log_{b_2} x$}
\rput[tl](4,-2.4){$\log_{b_1} x$}
\rput[bl](0.4,-3.4){$0<b_2<b_1<1$}
\end{pspicture*}
\end{center}
   \vspace{-80pt}
\end{wrapfigure}

A logaritmus függvény az exponenciális függvény inverze, azaz: $y = a^x \; \Leftrightarrow \; x = \log_a y$.\\
Tehát $f: (0;\infty)\mapsto \mathbb{R}$. $f(x) = \log_a x$ ($a\in\mathbb{R}; a>0; a\neq 1$)

\subsubsection{Tulajdonságok}
\begin{enumerate*}
 \item $f(x) = \log_a x$ folytonos $(0;\infty)$-n
 \item $\log_a 1 = 0; \quad \log_a a = 1$
 \item $f(x) = \log_a x \quad \begin{cases}
                           \hbox{szig. mon. nő, ha } a>1\\
			   \hbox{szig. mon. csökken, ha } 0<a<1.
                         \end{cases}$
 \item $\log_a xy = \log_a x + \log_a y$
 \item $\log_a x^y = y\cdot\log_a x$
 \item $\displaystyle \lim_{x\to \infty} \log_a x =
  \begin{cases}
    -\infty, \hbox{ ha } 0<a<1\\
    \infty, \hbox{ ha } a>1
  \end{cases}$\\
  $\displaystyle \lim_{x\to 0+0} \log_a x =
  \begin{cases}
    \infty, \hbox{ ha } 0<a<1\\
    -\infty, \hbox{ ha } a>1
  \end{cases}$
 \item $\log_b c = \dfrac{\log_a c}{\log_a b}$
\end{enumerate*}

\subsubsection{Derivált függvények}
\[(\log_e x)' = (\ln x)' = \frac{1}{\exp'(\ln x)} = \frac{1}{e^{\ln x}} = \frac{1}{x}\]
\[(\log_a x)' = \left(\frac{\ln x}{\ln a}\right)' = \frac{1}{x\cdot \ln a}\]
\[(a^x)' = ((e^{\ln a})^x)' = (e^{x\cdot \ln a})' = e^{x\cdot\ln a}\cdot \ln a = a^x\cdot \ln a\]

\subsection{Hatvány függvények}

\begin{wrapfigure}{r}{0.37\textwidth}
   \vspace{-40pt}
\begin{center}
\psset{xunit=1.0cm,yunit=1.0cm,algebraic=true,dotstyle=o,dotsize=3pt 0,linewidth=0.8pt,arrowsize=3pt 2,arrowinset=0.25}
\begin{pspicture*}(-0.68,-0.68)(5.25,5.25)
\psaxes[xAxis=true,yAxis=true,labels=none,Dx=1,Dy=1,ticksize=-2pt 0,subticks=2]{->}(0,0)(-0.68,-0.68)(5.25,5.25)
\psline[linestyle=dotted](1,-0.68)(1,5.46)
\psplot[plotpoints=200]{0.0}{5.246630333098997}{x^2}
\psplot[plotpoints=200]{0.0}{5.246630333098997}{x^-1}
\psplot[plotpoints=200]{0.0}{5.246630333098997}{x^0}
\psplot[plotpoints=200]{0.0}{5.246630333098997}{x^1}
\rput[tl](0.35,4.58){$x^{-1}$}
\rput[tl](2.14,4.45){$x^2$}
\rput[tl](3.2,3.04){$x$}
\rput[tl](3.45,1.58){$x^0$}
\psdots(0,0)
\psdots(0,1)
\end{pspicture*}
\end{center}
   \vspace{-80pt}
\end{wrapfigure}

$f: \mathbb{R}^+ \mapsto \mathbb{R}^+$. $f(x) = x^a$ $(a\in\mathbb{R}, x>0)$.

\subsubsection{Tulajdonságok}

$f(x) = x^a \quad \begin{cases}
                           \hbox{szig. mon. nő, ha } a>0\\
			   \hbox{szig. mon. csökken, ha } 0<a.
                         \end{cases}$

\subsubsection{Derivált függvények}
\[(x^a)' = ((e^{\ln x})^a)' = (e^{\ln x\cdot a})' = x^a\cdot \frac{a}{x} = a\cdot x^{a-1}\]
\[(x^x)' = ((e^{\ln x})^x)' = (e^{\ln x\cdot x})' = x^x\cdot \left(\frac{x}{x}+\ln x\right) = (1+\ln x)\cdot x^x\]

\subsection{Hiperbolikus függvények}

\[\begin{array}{cc}
    \sh x = \dfrac{e^x-e^{-x}}{2} = -\sh(-x) \quad&\quad \th x = \dfrac{e^x-e^{-x}}{e^x+e^{-x}}\\
    \ch x = \dfrac{e^x+e^{-x}}{2} = \ch(-x) \quad&\quad \cth x = \dfrac{e^x+e^{-x}}{e^x-e^{-x}}
  \end{array}
\]

\begin{center}
\psset{xunit=1.6cm,yunit=1.6cm,algebraic=true,dotstyle=o,dotsize=3pt 0,linewidth=0.8pt,arrowsize=3pt 2,arrowinset=0.25}
\begin{pspicture*}(-2.3,-2)(2.3,2)
\psaxes[xAxis=true,yAxis=true,labels=none,Dx=0.5,Dy=0.5,ticksize=-2pt 0,subticks=2]{->}(0,0)(-2.3,-2)(2.3,2)
\psplot[linewidth=1.2pt,plotpoints=200]{-2.250668536409383}{2.3424243891457657}{COSH(x)}
\psplot[linewidth=1.2pt,plotpoints=200]{-2.250668536409383}{2.3424243891457657}{SINH(x)}
\psplot[linestyle=dotted,plotpoints=200]{-2.250668536409383}{2.3424243891457657}{2.72^x/2}
\psplot[linestyle=dotted,plotpoints=200]{-2.250668536409383}{2.3424243891457657}{2.72^(-(x))/2}
\psplot[linestyle=dotted,plotpoints=200]{-2.250668536409383}{2.3424243891457657}{-2.72^x/2}
\psplot[linestyle=dotted,plotpoints=200]{-2.250668536409383}{2.3424243891457657}{-2.72^(-(x))/2}
\rput[bl](-1.06,1.71){$\ch(x)$}
\rput[bl](-1.75,-1.45){$\sh(x)$}
\rput[bl](-1.6,1.3){$\dfrac{e^{-x}}{2}$}
\rput[c](-2,0.4){$\dfrac{e^x}{2}$}
\rput[c](-2,-0.4){$\dfrac{-e^{x}}{2}$}
\rput[bl](-1.1,-2){$\dfrac{-e^{-x}}{2}$}
\rput[c](-0.2,0.5){$\frac{1}{2}$}
\psline(-0.05,0.5)(0.05,0.5)
\rput[c](-0.2,1){$1$}
\psline[linestyle=dashed,dash=7pt 10pt,linewidth=0.5pt](-2.3,1)(2.3,1)
\psline[linestyle=dashed,dash=7pt 10pt,linewidth=0.5pt](-2.3,-2.3)(2.3,2.3)
\end{pspicture*}
\hspace{30pt}
\psset{xunit=1.0cm,yunit=1.0cm,algebraic=true,dotstyle=o,dotsize=3pt 0,linewidth=0.8pt,arrowsize=3pt 2,arrowinset=0.25}
\begin{pspicture*}(-3.8,-3)(3.8,3)
\psaxes[labelFontSize=\scriptstyle,xAxis=true,yAxis=true,labels=none,Dx=1,Dy=1,ticksize=-2pt 0,subticks=2]{->}(0,0)(-3.8,-3)(3.8,3)
\psplot[linestyle=dotted,plotpoints=200]{-3.634730388502775}{4.241344738015167}{COSH(x)}
\psplot[linestyle=dotted,plotpoints=200]{-3.634730388502775}{4.241344738015167}{SINH(x)}
\psplot[linewidth=1.2pt,plotpoints=200]{-3.634730388502775}{4.241344738015167}{TANH(x)}
\psplot[linewidth=1.2pt,linestyle=dashed,dash=6pt 3pt,plotpoints=200]{-3.634730388502775}{-0.001}{1/TANH(x)}
\psplot[linewidth=1.2pt,linestyle=dashed,dash=6pt 3pt,plotpoints=200]{0.001}{4.241344738015167}{1/TANH(x)}
\rput[tl](0.6,2.8){$\cth(x)$}
\rput[bl](-2.7,2.1){$\ch(x)$}
\rput[bl](-2.8,-2.6){$\sh(x)$}
\rput[bl](-3.29,-0.74){$\th(x)$}
\end{pspicture*}
\end{center}

\begin{tetel}
\[\boxed{\ch^2 x - \sh^2 x = 1}\]
\end{tetel}
\begin{biz}
\[\frac{(e^x+e^{-x})^2-(e^x-e^{-x})^2}{4} = \frac{e^{2x}+e^{-2x}+2-(e^{2x}+e^{-2x}-2)}{4} = \frac{4}{4} = 1\]
\end{biz}

\begin{tetel}
\[\boxed{\sh(\alpha+\beta) = \sh\alpha\ch\beta + \ch\alpha\sh\beta}\]
\end{tetel}

\begin{biz}
\[\sh\alpha\ch\beta + \ch\alpha\sh\beta = \frac{(e^\alpha-e^{-\alpha})(e^\beta+e^{-\beta})}{4} + \frac{(e^\alpha+e^{-\alpha})(e^\beta-e^{-\beta})}{4} = \]
\[ = \frac{2e^\alpha e^\beta - 2e^{-\alpha} e^{-\beta}}{4} = \frac{e^{\alpha+\beta} - e^{-(\alpha+\beta)}}{2} = \sh(\alpha+\beta)\]
\end{biz}

\begin{tetel}
\[\boxed{\ch(\alpha+\beta) = \ch\alpha\ch\beta + \sh\alpha\sh\beta}\]
\end{tetel}
\begin{biz}
\[\ch\alpha\ch\beta + \sh\alpha\sh\beta = \frac{(e^\alpha+e^{-\alpha})(e^\beta+e^{-\beta})}{4}  +\frac{(e^\alpha-e^{-\alpha})(e^\beta-e^{-\beta})}{4} = \]
\[ = \frac{2e^\alpha e^\beta + 2e^{-\alpha} e^{-\beta}}{4} = \frac{e^{\alpha+\beta} + e^{-(\alpha+\beta)}}{2} = \ch(\alpha+\beta)\]
\end{biz}

\subsubsection{Derivált függvények}

\[\sh' x = \left(\dfrac{e^x-e^{-x}}{2}\right)' = \dfrac{e^x+e^{-x}}{2} = \ch x\]
\[\ch' x = \left(\dfrac{e^x+e^{-x}}{2}\right)' = \dfrac{e^x-e^{-x}}{2} = \sh x\]
\[\th' x = \left(\frac{\sh x}{\ch x}\right)' = \frac{\ch^2 x-\sh^2 x}{\ch^2 x} = \frac{1}{\ch^2 x}\]
\[\cth' x = \left(\frac{\ch x}{\sh x}\right)' = \frac{\sh^2 x-\ch^2 x}{\sh^2 x} = \frac{-1}{\sh^2 x}\]

\subsubsection{Inverzek}

$y = \sh x = \dfrac{e^x-e^{-x}}{2} = \dfrac{q-\frac{1}{q}}{2} \quad /\cdot 2q \qquad (q=e^x >0)$\\
$2qy = q^2-1 \quad \Leftrightarrow \quad q^2-2qy-1=0$. Ebből $q_{1/2} = y\pm\sqrt{y^2+1}$, de mivel $q>0$, ezért $q=y+\sqrt{y^2+1}$. Tehát $e^x = y+\sqrt{y^2+1}$, így $x = \ln(y+\sqrt{y^2+1})$.\\
Innen látszik, hogy $\sh x$ inverzét felírhatjuk a logaritmus segítségével, tehát
\[\arsh x = \ln(x+\sqrt{x^2+1})\]
Hasonlóképpen $\arch x$ is levezethető:\\
$y = \ch x = \dfrac{e^x+e^{-x}}{2} = \dfrac{q+\frac{1}{q}}{2} \quad /\cdot 2q \qquad (q=e^x >0)$\\
$2qy = q^2+1 \quad \Leftrightarrow \quad q^2-2qy+1=0$. Ebből $q_{1/2} = y\pm\sqrt{y^2-1}$, de mivel $q>0$, ezért $q=y+\sqrt{y^2-1}$. Tehát $e^x = y+\sqrt{y^2-1}$, így $x = \ln(y+\sqrt{y^2-1})$. Tehát $\ch$ inverze:\\
\[\arch x = \ln(x + \sqrt{x^2-1})\]
Nézzük meg $\arth x$-et:\\
$y = \th x = \dfrac{e^x-e^{-x}}{e^x+e^{-x}} = \dfrac{q-\frac{1}{q}}{q+\frac{1}{q}} = \dfrac{q^2-1}{q^2+1} \qquad (q=e^x >0)$\\
$(q^2+1)y = q^2-1 \quad \Leftrightarrow \quad (y-1)q^2+y+1=0$. Ebből $q^2 = \dfrac{1+y}{1-y} \quad \Leftrightarrow \quad q=\sqrt{\dfrac{1+y}{1-y}}$. Tehát $e^x = \sqrt{\dfrac{1+y}{1-y}}$, így $x = \ln \sqrt{\dfrac{1+y}{1-y}}$. Tehát $\th$ inverze:\\
\[\arth x = \frac{1}{2} \ln \left(\frac{1+x}{1-x}\right)\]
Végül $\arcth x$:\\
$y = \cth x = \dfrac{e^x+e^{-x}}{e^x-e^{-x}} = \dfrac{q+\frac{1}{q}}{q-\frac{1}{q}} = \dfrac{q^2+1}{q^2-1} \qquad (q=e^x >0)$\\
$(q^2-1)y = q^2+1 \quad \Leftrightarrow \quad (y-1)q^2-y-1=0$. Ebből $q^2 = \dfrac{y+1}{y-1} \quad \Leftrightarrow \quad q=\sqrt{\dfrac{y+1}{y-1}}$. Tehát $e^x = \sqrt{\dfrac{y+1}{y-1}}$, így $x = \ln \sqrt{\dfrac{y+1}{y-1}}$. Tehát $\cth$ inverze:\\
\[\arcth x = \frac{1}{2} \ln \left(\frac{x+1}{x-1}\right)\]

\subsubsection{Inverzek deriváltjai}

\[\arsh' x = \frac{1}{\sh'(\underbrace{\arsh x}_{y})} = \frac{1}{\ch(y)} = \frac{1}{\sqrt{1+\sh^2 y}} = \frac{1}{\sqrt{1+x^2}}\]
\[\arch' x = \frac{1}{\ch'(\underbrace{\arch x}_{y})} = \frac{1}{\sh(y)} = \frac{1}{\sqrt{\ch^2 y-1}} = \frac{1}{\sqrt{x^2-1}} \quad (\ch y\geqslant 1 \Rightarrow x>1)\]
\[\arth' x = \left(\frac{1}{2} \ln\left(\frac{1+x}{1-x}\right)\right)' = \frac{1}{2}\cdot \frac{1-x}{1+x}\cdot \frac{(1-x)+(1+x)}{(1-x)^2} = \frac{1}{1-x^2} \qquad x\neq \pm 1\]
\[\arcth' x = \left(\frac{1}{2} \ln\left(\frac{x+1}{x-1}\right)\right)' = \frac{1}{2}\cdot \frac{x-1}{1+x}\cdot \frac{(x-1)-(x+1)}{(x-1)^2} = \frac{1}{1-x^2} \qquad x\neq \pm 1\]

\subsection{Differenciálszámítás középérték-tételei}

\begin{defi} Legyen $x_0\in \Int D_f$.\\
$f$-nek az $x_0$ helyen \textbf{lokális maximuma} van, ha $\exists\varepsilon>0$, hogy $f(x)\leqslant x_0$ $\forall x\in K_\varepsilon(x_0)$ esetén.
\end{defi}

\begin{defi} Legyen $x_0\in \Int D_f$.\\
$f$-nek az $x_0$ helyen \textbf{lokális minimuma} van, ha $\exists\varepsilon>0$, hogy $f(x)\geqslant x_0$ $\forall x\in K_\varepsilon(x_0)$ esetén.
\end{defi}

\textit{Megjegyzés}: $f(x)\equiv c\in\mathbb{R}$ esetén $f$-nek minden pontjában lokális minimuma és maximuma is van.

\begin{tetel} \label{LokalisSzelsoertekSzukFeltetel} $\boxed{$Lokális szélsőérték szükséges feltétele$}$\\
Ha $f$ differenciálható $x_0$-ban és ott lokális szélsőértéke van, akkor $f'(x_0) = 0$.
\end{tetel}
\begin{biz}
 Tfh $x_0$-ban lokális maximuma van (konretizálva egyszerűbb a bizonyítás -- hasonlóan megy ha lokális minimuma van).\\
 \[f'_-(x_0) = \lim_{h\to 0-0} \frac{\overbrace{f(x_0+h)-f(x_0)}^{\leqslant 0}}{\underbrace{h}_{\leqslant 0}} \geqslant 0 \qquad \hbox{hiszen ha } |h|<\varepsilon \Rightarrow f(x_0)\geqslant f(x_0+h)\]
 \[f'_+(x_0) = \lim_{h\to 0+0} \frac{\overbrace{f(x_0+h)-f(x_0)}^{\leqslant 0}}{\underbrace{h}_{\geqslant 0}} \leqslant 0 \]
 Mivel deriválható $f$ az $x_0$ pontban, ezért:
 \[\underbrace{f'_-(x_0)}_{\geqslant 0} = \underbrace{f'_+(x_0)}_{\leqslant 0} = f'(x_0) = 0\]
\end{biz}

\begin{tetel} $\boxed{$Rolle-tétel$}$\\
Ha $f$ folytonos $[a,b]$-n és differenciálható $(a,b)$-n és $f(a)=f(b)$, akkor $\exists\xi\in(a,b)$, hogy $f'(\xi) = 0$
\end{tetel}
\begin{center}
\psset{xunit=1.0cm,yunit=1.0cm,algebraic=true,dotstyle=o,dotsize=3pt 0,linewidth=0.8pt,arrowsize=3pt 2,arrowinset=0.25}
\begin{pspicture*}(-0.44,-0.92)(6.44,4.74)
\psaxes[labelFontSize=\scriptstyle,xAxis=true,yAxis=true,labels=none,Dx=1,Dy=1,ticksize=-2pt 0,subticks=2]{->}(0,0)(-0.44,-0.92)(6.44,4.74)
\psline[linestyle=dashed,dash=5pt 5pt](1,-0.1)(1,4.74)
\psline[linestyle=dashed,dash=5pt 5pt](5,-0.1)(5,4.74)
\psplot[plotpoints=200]{0.9}{5.1}{-0.37357*x^4+4.57626*x^3-19.485324*x^2+33.325163*x-16.042529}
%\psplot[plotpoints=200]{0.9}{5.1}{-0.27172*x^4+3.2893*x^3-13.81963*x^2+23.40797*x-10.64591}
%\psline[linestyle=dotted](3.02,-0.1)(3.02,4.74)
%\psline[linestyle=dotted](4.46,-0.1)(4.46,4.74)
%\psline[linestyle=dotted](2.99,-0.92)(2.99,4.74)
%\psline[linestyle=dotted](4.37,-0.92)(4.37,4.74)
%\psline[linestyle=dotted](1.63,-0.92)(1.63,4.74)
%\psline(2.52,2)(3.52,2)
%\psline(3.94,3.16)(4.96,3.16)
%\psline(2.47,2.23)(3.52,2.23)
%\psline(3.91,3.16)(4.83,3.16)
%\psline(1.15,3.125)(2.1,3.125)

\psline[linestyle=dotted](3.12,-0.92)(3.12,4.74)
\psline[linestyle=dotted](4.47,-0.92)(4.47,4.74)
\psline[linestyle=dotted](1.6,-0.92)(1.6,4.74)
\psline(2.6,1.84)(3.64,1.84)
\psline(4.01,3.17)(4.93,3.17)
\psline(1.23,3.69)(1.97,3.69)

% a
\psdots[dotstyle=*](1,2)
\psdots[dotstyle=*](1,0)
\rput[c](1,-0.4){$a$}
% b
\psdots[dotstyle=*](5,2)
\psdots[dotstyle=*](5,0)
\rput[c](5,-0.4){$b$}

\psdots[dotstyle=*](1.6,-0)
\rput[c](1.6,-0.4){$\xi_1$}
\psdots[dotstyle=*](3.12,-0)
\rput[c](3.12,-0.4){$\xi_2$}
\psdots[dotstyle=*](4.47,-0)
\rput[c](4.47,-0.4){$\xi_3$}
\end{pspicture*}
\end{center}
\begin{biz}
 Weierstrass II. tétele alapján ($f$ korlátos, zárt intervallumon folytonos) felveszi szélsőértékeit.
 \begin{enumerate*}
  \item ha $f$ az intervallum végpontjaiban veszi fel a maximumát és minimumát is, akkor $f(x)\equiv c$, ekkor $\forall \xi\in(a,b)$-re $f'(\xi) = 0$.
  \item ha az egyik szélső érték hely nem a végpont, akkor itt $f'(\xi) = 0$
 \end{enumerate*}

\end{biz}


\begin{tetel} $\boxed{$Lagrange-tétel$}$\\
Ha $f$ folytonos $[a,b]$-n ($a, b \in \mathbb{R}$) és $f$ differenciálható $(a,b)$-n, akkor $\exists \xi\in(a, b)$, hogy $f'(\xi) = \dfrac{f(b)-f(a)}{b-a} = \hbox{intervallum végpontjai közti húr meredeksége}$
\end{tetel}
\begin{center}
\psset{xunit=1.0cm,yunit=1.0cm,algebraic=true,dotstyle=o,dotsize=3pt 0,linewidth=0.8pt,arrowsize=3pt 2,arrowinset=0.25}
\begin{pspicture*}(-0.54,-1.22)(6.26,4.73)
\psaxes[xAxis=true,yAxis=true,labels=none,Dx=2,Dy=2,ticksize=-2pt 0,subticks=2]{->}(0,0)(-0.54,-1.22)(6.26,4.73)
\psline[linestyle=dashed,dash=6pt 6pt](1,-0.1)(1,4.73)
\psline[linestyle=dashed,dash=6pt 6pt](5,-0.1)(5,4.73)
\psplot[plotpoints=200]{0.9}{5.1}{0.3978*x^3-3.41400*x^2+8.497662*x-3.521252}
\psline[linestyle=dotted](4.03,-0.1)(4.03,4.73)
\psline[linestyle=dotted](1.7,-0.1)(1.7,4.73)
\psline(1.19,2.83)(2.21,3.19)
\psline(3.35,1.08)(4.7,1.55)
\psline(1,1.96)(5,3.34)
\psdots[dotstyle=*](1,1.96)
\psdots[dotstyle=*](5,3.34)
\psdots[dotstyle=*](1.7,0)
\rput[c](1.7,-0.4){$\xi_1$}
\psdots[dotstyle=*](4.03,0)
\rput[c](4.03,-0.4){$\xi_2$}
\psdots[dotstyle=*](1,-0)
\rput[c](1,-0.4){$a$}
\psdots[dotstyle=*](5,-0)
\rput[c](5,-0.4){$b$}
\end{pspicture*}
\end{center}
\begin{biz}(*)\\
 Húr egyenlete: $h(x) = \dfrac{f(b)-f(a)}{b-a}\cdot(x-a)+f(a)$\\
 $g(x) := f(x)-h(x)$-re alkalmazhatjuk a Rolle-tételt:
\[g'(x) = f'(x)-h'(x) = f'(x)-\frac{f(b)-f(a)}{b-a} \quad \overset{\hbox{Rolle-t.}}{\Longrightarrow} \quad \]
\[\Longrightarrow \quad \exists \xi\in(a, b):g'(\xi) = 0 \quad \Leftrightarrow \quad f'(\xi) = \frac{f(b)-f(a)}{b-a}\]
\end{biz}

\begin{tetel}
 Legyen $f$ folytonos $[a, b]$-n ($a, b \in \mathbb{R}$) és differenciálható $(a,b)$-n, továbbá $f'(x)=0$, ha $x\in(a,b)$. Ekkor $\exists c\in\mathbb{R}$, hogy $f(x)=c$ $\forall x\in[a,b]$.
\end{tetel}
\begin{biz}
 Legyen $x_1, x_2\in(a,b)$, $a<x_1<x_2<b$. Lagrange-tételt alkalmazva:
\[\exists \xi\in(x_1,x_2) \hbox{ hogy } \underbrace{f'(\xi)}_{=0}=\dfrac{f(x_2)-f(x_1)}{x_2-x_1} = 0 \qquad x_2-x_1\neq 0 \quad \Rightarrow \quad f(x_2)-f(x_1) = 0\]
Tehát $f(x_2)=f(x_1) \quad \forall x_2,x_1\in(a,b)$.
\end{biz}

\begin{tetel}\label{IntAlapI}$\boxed{$Integrálszámítás alaptétele$}$\\
 Legyen $f$ és $g$ folytonos $[a, b]$-n ($a,b\in\mathbb{R}$) és differenciálható $(a,b)$-n, továbbá $f'(x)=g'(x)$ ($x\in(a,b)$), ekkor $\exists c\in\mathbb{R}$, hogy $f(x)=g(x)+c \quad \forall x\in[a, b]$.
\end{tetel}
\begin{biz}
 $h(x) := f(x)-g(x)$, ekkor $h'(x) = f'(x)-g'(x) = 0 \quad \forall x\in(a,b)$, alkalmazva az előző tételt: $h(x) = c\in\mathbb{R} \quad \forall x\in(a,b)$, tehát $f(x)=g(x)+c$.
\end{biz}

\subsubsection{Példák a fenti tételek alkalmazására}

\begin{enumerate*}
 \item Alkalmazható-e a Rolle-tétel az $f(x)=e^{-|x|}$-re az $I=[-2,+2]$-n?
  \begin{itemize*}
   \item $f$ folytonos $[-2,+2]$-n $\quad \checkmark$
   \item $f(-2)=f(2) \quad \checkmark$
   \item \textbf{de} $f$ nem differenciálható $(-2,2)$-ben, hiszen töréspontja van $x=0$-ban.
  \end{itemize*}
 \item Alkalmazható-e a Lagrange-tétel az $f(x)=\dfrac{1}{(x-1)^2}$-re az $I=[-1,0]$-n? $(\xi=?)$
  \begin{itemize*}
   \item $f$ folytonos $[-1,0]$-n $\quad \checkmark$
   \item $f$ differenciálható $(-1,0)$-n $\quad \checkmark$
  \end{itemize*}
  A húr meredeksége: $\dfrac{f(0)-f(-1)}{0-(-1)} = \dfrac{\frac{1}{1}-\frac{1}{4}}{1} = \dfrac{3}{4}$. Keressük $\xi$-t, melyre $f'(\xi) = \dfrac{3}{4}$:
 \[f'(\xi) = \Big((\xi-1)^{-2}\Big)' = -2\cdot (\xi-1)^{-3} = \dfrac{3}{4}\]
\[(\xi-1)^{-3} = \dfrac{-3}{8} \quad \rightarrow \quad \xi = \sqrt[3]{\dfrac{8}{-3}}+1 = \dfrac{2}{-\sqrt[3]{3}}+1\]
 \item Igazoljuk, hogy ha $0<a<b<\dfrac{\pi}{2}$, akkor
\[\frac{b-a}{\cos^2 a}<\tg b - \tg a < \frac{b-a}{\cos^2 b}\]
$f(x) = \tg x$-re az $I=\left[0,\dfrac{\pi}{2}\right]$-n alkalmazhatjuk a Lagrange-tételt, hiszen folytonos $I$-n és differenciálható $I$ belsejében. Tehát $\exists \xi \in (a, b) \subset \left(0,\dfrac{\pi}{2}\right)$, hogy:
\[\tg' \xi = \frac{1}{\cos^2 \xi} = \frac{\tg b - \tg a}{b-a}\]
Mivel $0<a<b<\dfrac{\pi}{2}$ és $\xi\in(a, b)$, ezért:
\[\cos a>\cos \xi > \cos b > 0\]
\[\frac{1}{\cos^2 a} > \frac{1}{\cos^2 \xi} > \frac{1}{\cos^2 b} \]
\[\frac{1}{\cos^2 a} > \frac{\tg b - \tg a}{b-a} > \frac{1}{\cos^2 b} \]
\[\frac{b-a}{\cos^2 a} > \tg b - \tg a > \frac{b-a}{\cos^2 b} \quad \checkmark \]
\end{enumerate*}


\begin{tetel}$\boxed{$L'Hospital szabály$}$ $\qquad$ $\left(\dfrac{0}{0}, \dfrac{\infty}{\infty} \hbox{ típusú határérték számításához használható} \right)$\\
Legyen $f$ és $g$ differenciálható $\dot{K}_\varepsilon(x_0)$, ahol $\varepsilon>0$, illetve $g(x)\neq 0$, $g'(x)\neq 0$, ahol $x\in\dot{K}_\varepsilon(x_0)$. Továbbá $\displaystyle \exists\lim_{x\to x_0} f(x)=0$ és $\displaystyle \exists\lim_{x\to x_0} g(x)=0$. Ekkor ha $\displaystyle \exists\lim_{x\to x_0} \frac{f'(x)}{g'(x)}=\beta$, akkor $\displaystyle \exists\lim_{x\to x_0} \frac{f(x)}{g(x)}=\beta$.
\end{tetel}
\addtocounter{biz}{1}
\emph{Megjegyzés}: Bizonyítása nem tananyag. De $x\to x_0$ helyett állhat $x\to x_0+0$, $x\to x_0-0$, $x\to \infty$ és $x\to -\infty$ is. Hasonlóan $\beta$ lehet valós és $\pm\infty$ is! Továbbá akkor is igaz a szabály, ha $\displaystyle \lim_{x\to x_0} f(x) = \lim_{x\to x_0} g(x) = \infty$.\\

A fenti tétel segedelmével az alábbi határozatlan alakok könnyedén meghatározhatóak, de olyan alakra kell őket hozni, hogy a tételt alkalmazni lehessen:
\begin{itemize*}
 \item $\dfrac{0}{0}; \dfrac{\infty}{\infty}$: közvetlen L'Hospital szabály
 \item $0\cdot \infty: f\cdot g = \dfrac{f}{1/g} \hbox{ vagy } \dfrac{g}{\underbrace{1/f}_{\to \pm\infty}} \quad$ (2. esetben előjel vizsgálat szükséges!)
 \item $\infty - \infty: f - g = \dfrac{1}{1/f}-\dfrac{1}{1/g} = \dfrac{1/g-1/f}{1/f\cdot 1/g}$
 \item $0^0, 1^\infty, \infty^0: f^g = e^{g\cdot\ln f}$ és ekkor $g\cdot\ln f$ határértékét kel vizsgálni.
\end{itemize*}

\subsubsection{Példák a L'Hospital szabály alkalmazására}

\begin{enumerate}
 \item $\displaystyle \lim_{x\to\infty} \underbrace{x^2}_{\to \infty}\cdot \underbrace{e^{-x}}_{\to 0} = \lim_{x\to\infty} \frac{x^2}{e^x} \overset{L'H}{=} \lim_{x\to\infty} \frac{2x}{e^x} \overset{L'H}{=} \lim_{x\to\infty} \frac{2}{e^x} = 0$\\
 A fenti feladatnál mindig feltesszük, hogy létezik az adott határérték, ezért alkalmazhatjuk a L'Hospitalt, de amikor a végére jutunk, láthatjuk, hogy valóban létezik, ezért az előzőek is léteztek $\checkmark$.

 \item $\displaystyle \lim_{x\to 1-0} \underbrace{\ln x}_{\to 0}\cdot \underbrace{\tg \left(\frac{\pi}{2}x\right)}_{\to \infty} = \lim_{x\to 1-0} \frac{\ln x}{\ctg \left(\frac{\pi}{2}x\right)} \overset{L'H}{=} \lim_{x\to 1-0} \frac{\frac{1}{x}}{-\frac{1}{\sin^2 \left(\frac{\pi}{2}x\right)}\cdot \frac{\pi}{2}} = \lim_{x\to 1-0} -\frac{2\sin^2 \left(\frac{\pi}{2}x\right)}{\pi\cdot x} = -\frac{2\sin^2 \left(\frac{\pi}{2}\right)}{\pi\cdot 1} = \frac{-2}{\pi}$

 \item $\displaystyle \lim_{x\to 0+0} \left(\underbrace{\frac{1}{x}}_{\to +\infty} - \underbrace{\frac{1}{e^x-1}}_{\to +\infty} \right) = \lim_{x\to 0+0} \frac{e^x-1-x}{x\cdot(e^x-1)} \overset{L'H}{=} \lim_{x\to 0+0} \frac{e^x-1}{x\cdot(e^x)+e^x-1} \overset{L'H}{=} $\vspace*{8pt}\\
 $\displaystyle \overset{L'H}{=} \lim_{x\to 0+0} \frac{e^x}{x\cdot(e^x)+e^x+e^x} = \lim_{x\to 0+0} \frac{e^x}{e^x}\cdot\frac{1}{x+2} = \frac{1}{2}$

 \item $\displaystyle \lim_{x\to 0} (\cos 3x)^{\frac{1}{x^2}} = \lim_{x\to 0} \exp\left(\frac{1}{x^2}\cdot \ln(\cos 3x)\right) = \exp\left(\lim_{x\to 0} \frac{\ln(\cos 3x)}{x^2} \right)$ Kitevőt vizsgáljuk:\vspace*{8pt}\\
 $\displaystyle \lim_{x\to 0} \frac{\ln(\cos 3x)}{x^2} \overset{L'H}{=} \lim_{x\to 0} \frac{\frac{1}{\cos 3x}\cdot -\sin 3x\cdot 3}{2x} = \lim_{x\to 0} \frac{-3\cdot \tg 3x}{2x} \overset{L'H}{=} \lim_{x\to 0} \frac{-9\cdot \frac{1}{\cos^2 3x}}{2} = \frac{-9}{2}$\vspace*{8pt}\\
 Tehát $\displaystyle \lim_{x\to 0} (\cos 3x)^{\frac{1}{x^2}} = e^{\frac{-9}{2}}$

 \item $\displaystyle \lim_{x\to\infty} \frac{\sh 3x}{e^{3x}} \overset{L'H}{=} \lim_{x\to\infty} \frac{3\ch 3x}{3e^{3x}} \overset{L'H}{=} \lim_{x\to\infty} \frac{\sh 3x}{e^{3x}}$. Ez nem vezet eredményre.\vspace*{8pt}\\
 $\displaystyle \lim_{x\to\infty} \frac{\sh 3x}{e^{3x}} = \lim_{x\to\infty} \frac{e^{3x}-e^{-3x}}{2\cdot e^{3x}} = \lim_{x\to\infty} \left(\frac{1}{2}-\underbrace{\frac{e^{-6x}}{2}}_{\to 0}\right) = \frac{1}{2}$
\end{enumerate}

A L'Hospital szabállyal eddig explicit be nem látott tételek igazolhatóak:\\
Láttuk, hogy $\sqrt[n]{n} \xrightarrow{n\to\infty} 1 \quad n\in\mathbb{N}$. Ezt bármilyen $n:= x\in\mathbb{R}$-re beláthatjuk:
\[\lim_{x\to\infty} \sqrt[x]{x} = \lim_{x\to\infty} x^{\frac{1}{x}} = \lim_{x\to\infty} \exp\left(\frac{1}{x}\cdot\ln x\right) = \exp\left(\lim_{x\to\infty} \frac{1}{x}\cdot\ln x\right)\]
\[\lim_{x\to\infty} \frac{\ln x}{x} \overset{L'H}{=} \lim_{x\to\infty} \frac{\frac{1}{x}}{~1~} = 0\]
Tehát $\displaystyle \lim_{x\to\infty} \sqrt[x]{x} = e^0 = 1$.\\

Hasonlóan bizonyíthatjuk $\displaystyle \left(1+\frac{1}{x}\right)^x \xrightarrow{x\to\infty} e \quad x\in\mathbb{R}$:
\[\lim_{x\to\infty} \left(1+\frac{1}{x}\right)^x = \lim_{x\to\infty} \exp\left[x\cdot\ln\left(1+\frac{1}{x}\right)\right] =  \exp\left[\lim_{x\to\infty} x\cdot\ln\left(1+\frac{1}{x}\right)\right] \]
\[\lim_{x\to\infty} x\cdot\ln\left(1+\frac{1}{x}\right) = \lim_{x\to\infty} \frac{\ln\left(1+\frac{1}{x}\right)}{\frac{1}{x}} \overset{L'H}{=} \lim_{x\to\infty} \frac{\frac{1}{1+\frac{1}{x}}\cdot \frac{-1}{x^2}}{-\frac{1}{x^2}} = \lim_{x\to\infty} \frac{1}{1+\frac{1}{x}} = 1\]
Tehát $\displaystyle \lim_{x\to\infty} \left(1+\frac{1}{x}\right)^x = e^1 = e$.

\subsection{Nyílt intervallumon differenciálható függvények tulajdonsá\-gai}

Legyen $I=(a,b)$ $\qquad a,b\in\mathbb{R}\cup\{\pm\infty\}$.

\begin{defi}
 $f$ \textbf{monoton nő} $I$-n, ha $\forall x_1, x_2\in I$ esetén, ha $x_1<x_2$, akkor $f(x_1)\leqslant f(x_2)$.
\end{defi}
\begin{defi}
 $f$ \textbf{szigorúan monoton nő} $I$-n, ha $\forall x_1, x_2\in I$ esetén, ha $x_1<x_2$, akkor $f(x_1)< f(x_2)$.
\end{defi}

\begin{defi}
 $f$ \textbf{monoton csökken} $I$-n, ha $\forall x_1, x_2\in I$ esetén, ha $x_1<x_2$, akkor $f(x_1)\geqslant f(x_2)$.
\end{defi}
\begin{defi}
 $f$ \textbf{szigorúan monoton csökken} $I$-n, ha $\forall x_1, x_2\in I$ esetén, ha $x_1<x_2$, akkor $f(x_1)>f(x_2)$.
\end{defi}

\begin{defi}
 $f$ \textbf{alulról konvex} $I$-n, ha $\forall x_1, x_2\in I, x_1<x_2$ esetén $x_1,x_2$ pontokban állított húr a függvény grafikonja fölött halad. Azaz: $h(x) \geqslant f(x) \quad x\in[x_1, x_2]$.
\end{defi}
\begin{defi}
 $f$ \textbf{alulról konkáv} $I$-n, ha $\forall x_1, x_2\in I, x_1<x_2$ esetén $x_1,x_2$ pontokban állított húr a függvény grafikonja alatt halad. Azaz: $h(x) \leqslant f(x) \quad x\in[x_1, x_2]$.
\end{defi}

\begin{defi}
 $f$-nek $x_0$-ban \textbf{inflexiós pontja} van, ha $f$ folytonos $x_0$-ban és $x_0$-ban konvex és konkáv szakaszok találkoznak.
\end{defi}

\begin{tetel} Ha $f$ differenciálható $I$-n, akkor:
 \[\left.\begin{array}{llcc}
 1. & \hbox{$f$ mon. nő} & \Leftrightarrow & f'(x)\geqslant 0 \\
 2. & \hbox{$f$ szig. mon. nő} & \Leftarrow & f'(x)> 0 \\
 3. & \hbox{$f$ mon. csökken} & \Leftrightarrow & f'(x)\leqslant 0 \\
 4. & \hbox{$f$ szig. mon. csökken} & \Leftarrow & f'(x)< 0 \\
         \end{array} \right\} \forall x\in I\]
\end{tetel}
\begin{bizNL}
 1. $\Rightarrow$ $f'(x) = f'_+(x) = \displaystyle \lim_{h\to 0+0} \frac{\overbrace{f(x+h)-f(x)}^{\geqslant 0}}{\underbrace{h}_{>0}} \geqslant 0 \quad \checkmark$\\
 $\Leftarrow$ Lagrange-tétel alapján $\exists\xi\in(x_1,x_2) \quad a< x_1<x_2< b$, hogy
 \[f'(\xi) = \frac{f(x_2)-f(x_1)}{x_2-x_1} \hbox{ ez a feltétel szerint } \geqslant 0 \quad \Rightarrow f(x_2)-f(x_1) \geqslant 0 \quad \checkmark\]

 2. $\Leftarrow$ Szintén Lagrange-tétel $[x_1, x_2]\subset I$-re
 \[f'(\xi) = \frac{f(x_2)-f(x_1)}{x_2-x_1} > 0 \quad \Rightarrow f(x_2)-f(x_1) > 0 \quad \checkmark\]
\end{bizNL}
\emph{Megjegyzés}: A 2-es és 4-es állítás megfordítása \textbf{nem} igaz. Például: $f(x) = x^3$ szig. mon. nő $\mathbb{R}$-en, de $f'(x) = 3x^2$-nek van zérushelye: $f'(0)=0$.

\begin{tetel} Legyen $f$ differenciálható $I$-n.
 \[\begin{array}{llcl}
 1. & \hbox{$f$ konvex $I$-n} & \quad\Leftrightarrow\quad & \hbox{$f'$ monoton nő $I$-n}\\
 2. & \hbox{$f$ konkáv $I$-n} & \quad\Leftrightarrow\quad & \hbox{$f'$ monoton csökken $I$-n}\\
         \end{array}\]
\end{tetel}
\addtocounter{biz}{1}

\begin{tetel}\label{MasodikDerivKonvexitas} Legyen $f$ kétszer differenciálható $I$-n.
 \[\left. \begin{array}{llcl}
 1. & \hbox{$f$ konvex $I$-n} & \quad\Leftrightarrow\quad & f''(x)\geqslant 0\\
 2. & \hbox{$f$ konkáv $I$-n} & \quad\Leftrightarrow\quad & f''(x)\leqslant 0\\
         \end{array} \right\}\forall x\in I\]
\end{tetel}
\begin{biz}
 Előző két tétel alapján.
\end{biz}

\subsubsection{Példák}

\begin{wrapfigure}{r}{0.38\textwidth}
   \vspace{-30pt}
\begin{center}
\psset{xunit=0.5cm,yunit=0.166cm,algebraic=true,dotstyle=o,dotsize=3pt 0,linewidth=0.8pt,arrowsize=3pt 2,arrowinset=0.25}
\begin{pspicture*}(-2.83,-19.74)(9.69,13.28)
\psaxes[xAxis=true,yAxis=true,Dx=2,labels=y,Dy=5,ticksize=-2pt 0,subticks=2]{->}(0,0)(-2.83,-19.74)(9.69,13.28)
\psplot[plotpoints=200]{-2.8319242492997763}{9.687167447284619}{x^3/3-7*x^2/2+6*x}
\psline[linestyle=dotted](1,-19.74)(1,13.28)
\psline[linestyle=dotted](3.5,-19.74)(3.5,13.28)
\psline[linestyle=dotted](6,-19.74)(6,13.28)
\rput[bl](-1.32,-19.31){$f$}
\psdots[dotstyle=*](1,0)
\rput[c](1,-2.5){1}
\psdots[dotstyle=*](6,0)
\rput[c](6,-2.5){6}
\psdots[dotstyle=*](3.5,0)
\rput[c](3.5,-2.5){$\frac{7}{2}$}
\psdots[dotstyle=*](1,2.83)
\psdots[dotstyle=*](3.5,-7.58)
\psdots[dotstyle=*](6,-18)
\end{pspicture*}\vspace*{5pt}
Ábrázolva $f(x)$-et (analízis után)
\end{center}
\vspace{-20pt}
\end{wrapfigure}

$f(x) = \dfrac{x^3}{3}-\dfrac{7}{2}x^2+6x$. $D_f = \mathbb{R}$, végtelenszer differenciálható. Melyek azok a legbővebb intervallumok, ahol $f$ monoton nő/csökken, konvex/konkáv?\\
$f'(x) = x^2-7x+6 = (x-6)(x-1)$. Parabola, ami $1< x < 6$ esetén negatív, egyébként pozitív. Táblázatba foglalva:
\[
\begin{array}{c||c|c|c|c|c}
x & (-\infty, 1) & 1 & (1,6) & 6 & (6, \infty)\\\hline\hline
f' & + & 0 & - & 0 & +\\\hline
f & \nearrow & \hbox{lok. max.} & \searrow & \hbox{lok. min.} & \nearrow
\end{array}
\]
$f''(x) = 2x-7$. Tehát:
\[
\begin{array}{c||c|c|c}
x & (-\infty, \frac{7}{2}) & \frac{7}{2} & (\frac{7}{2},\infty)\\\hline\hline
f'' & - & 0 & +\\\hline
f & \cap & \hbox{infl. pont} & \cup
\end{array}
\]

\begin{wrapfigure}{r}{0.25\textwidth}
   \vspace{-15pt}
\begin{center}
\psset{xunit=2.0cm,yunit=1cm,algebraic=true,dotstyle=o,dotsize=3pt 0,linewidth=0.8pt,arrowsize=3pt 2,arrowinset=0.25}
\begin{pspicture*}(-0.85,-1.29)(1.25,2.86)
\psaxes[xAxis=true,yAxis=true,Dx=1,Dy=1,ticksize=-2pt 0,subticks=2]{->}(0,0)(-0.85,-1.29)(1.25,2.86)
\psplot[plotpoints=200]{-0.8500570203526164}{1.1237182975885984}{2.718281828^(2*x)-(4*x+1)}
\psline[linestyle=dotted](0.35,-1.29)(0.35,2.86)
\psplot[linestyle=dashed,dash=3pt 3pt,plotpoints=200]{-0.8500570203526164}{1.1237182975885984}{-4*x-1}
\rput[bl](-0.7,2.36){$f$}
\psdots[dotstyle=*](0.35,0)
\rput[c](0.35,0.3){$\ln \sqrt{2}$}
\psdots[dotstyle=*](0.35,-0.39)
\rput[bl](-0.85,1.5){$g$}
\end{pspicture*}
\end{center}
\vspace{-50pt}
\end{wrapfigure}
Másik példa: $f(x)=e^{2x}-(4x+1)$. Monotonitás? Konvexitás?\\
$f'(x) = 2\cdot e^{2x}-4$. $f'(x) = 0 \Leftrightarrow x = \ln \sqrt{2}$\\
$f''(x) = 4\cdot e^{2x} > 0$.
\[
\begin{array}{c||c|c|c}
x & (-\infty, \ln\sqrt{2}) & \ln\sqrt{2} & (\ln\sqrt{2},\infty)\\\hline\hline
f' & - & 0 & +\\\hline
f & \searrow & \hbox{lok. min} & \nearrow
\end{array} \qquad\qquad
\begin{array}{c||c}
x & \mathbb{R}\\\hline\hline
f'' & +\\\hline
f & \cup
\end{array}
\]

Az ábrán $g(x)=-4x-1$ az $f(x)$ függvény egy lineáris asszimptotája ($-\infty$-ben ehhez tart).

\subsection{Differenciálható függvények lokális tulajdonságai}

\begin{defi}
 $f$ az $x_0$-ban $\left\{\begin{array}{c}\hbox{\textbf{lokálisan növekedő}}\\\hbox{\textbf{lokálisan csökkenő}}\end{array}\right\}$, ha $\exists\varepsilon>0$, hogy $\left\{\begin{array}{c}f(x)\leqslant f(x_0)\\f(x)\geqslant f(x_0)\end{array}\right\}$, ha $x\in(x_0-\varepsilon,x_0]$ és $\left\{\begin{array}{c}f(x)\geqslant f(x_0)\\f(x)\leqslant f(x_0)\end{array}\right\}$, ha $x\in[x_0,x_0+\varepsilon)$
\end{defi}

\begin{tetel} \label{LokalisanNoCsokken}
 Legyen $f$ az $x_0$-ban deriválható. Ekkor
\[\begin{array}{llcl}
 1. & \hbox{Ha $f$ lokálisan nő $x_0$-ban} & \quad\Rightarrow\quad & f'(x_0) \geqslant 0\\
 2. & \hbox{Ha $f'(x_0)>0$} & \quad\Rightarrow\quad & \hbox{$f$ lokálisan nő $x_0$-ban}\\
 1'. & \hbox{Ha $f$ lokálisan csökken $x_0$-ban} & \quad\Rightarrow\quad & f'(x_0) \leqslant 0\\
 2'. & \hbox{Ha $f'(x_0)<0$} & \quad\Rightarrow\quad & \hbox{$f$ lokálisan csökken $x_0$-ban}\\
         \end{array}\]
\end{tetel}
\begin{biz}
 \begin{enumerate*}
  \item
    \[\exists f'(x_0) = f'_+(x_0) = \lim_{h\to 0+0} \frac{\overbrace{f(x_0+h)-f(x_0)}^{\geqslant 0 \quad (h\in(0,\varepsilon))}}{\underbrace{h}_{>0}} \geqslant 0\]
  \item $f'(x_0)=\displaystyle \lim_{h\to 0+0} \frac{f(x_0+h)-f(x_0)}{h} = A > 0$, tehát $\exists \varepsilon_1>0$, hogy ha $h\in(0,\varepsilon_1)$ akkor:
 \[\frac{f(x_0+h)-f(x_0)}{h}>\frac{A}{2}>0 \qquad \hbox{mivel $h>0$, ezért}\]
 \[f(x_0+h)-f(x_0) > 0 \quad \Rightarrow \quad \boxed{f(x_0+h)>f(x_0)}\]
   Hasonlóan: $f'(x_0)=\displaystyle \lim_{h\to 0-0} \frac{f(x_0+h)-f(x_0)}{h} = A > 0$, tehát $\exists \varepsilon_2>0$, hogy ha $h\in(-\varepsilon_2,0)$ akkor:
 \[\frac{f(x_0+h)-f(x_0)}{h}>\frac{A}{2}>0 \qquad \hbox{mivel $h<0$, ezért}\]
 \[f(x_0+h)-f(x_0) < 0 \quad \Rightarrow \quad \boxed{f(x_0+h)<f(x_0)}\]
 Tehát $\exists\varepsilon = \min\{\varepsilon_1, \varepsilon_2\} > 0$, hogy $f(x_0)<f(x_0+h)$, és $f(x_0-h)<f(x_0)$, ha $h\in(0,\varepsilon)$ tehát $f$ lokálisan nő $x_0$-ban.
 \end{enumerate*}
 \emph{Megjegyzés}: A vesszős állítások hasonlóan bizonyíthatóak.
\end{biz}

\begin{tetel}
 Legyen $f$ differenciálható $x_0$ egy környezetében.
 \begin{description*}
  \item[~~~1] Ha $f$-nek $x_0$-ban lokális minimuma van $\quad \Rightarrow \quad f'(x_0) = 0$ (lokális szélsőérték szük. felt.).
  \item[2/a] Ha $f'(x_0)=0$ és $f'(x)$ $\rnode{NegPoz}{\underbrace{\hbox{negatívból pozitívba vált}}} \quad \Rightarrow \quad$ $f$-nek $x_0$-ban lokális minimuma van.
  \item[2/b] $\rnode{Kov}{\left.\begin{array}{l}f'(x)<0, \hbox{ ha } x\in(x_0-\varepsilon,x_0)\\ f'(x)>0, \hbox{ ha } x\in(x_0;x_0+\varepsilon)\end{array}\right\}}$ \nccurve[angleA=-90, angleB=0,nodesepA=3pt, nodesepB=0pt]{->}{NegPoz}{Kov} $\qquad \Longrightarrow$ $f'(x)$ lokálisan növekedő $x_0$-ban.\\[+5pt]
  \hspace*{-7pt}Tehát ha $f$ deriválható $x_0$ környékén és $\exists f''(x_0)$ $\quad \Rightarrow \quad$\\
  \hspace*{10pt} ha $f'(x_0)=0$ és $f''(x_0)>0$ $\quad \Rightarrow \quad$ $f$-nek lokális minimuma van $x_0$-ban \\
  \hspace*{10pt} ha $f'(x_0)=0$ és $f''(x_0)<0$ $\quad \Rightarrow \quad$ $f$-nek lokális maximuma van $x_0$-ban
 \end{description*}
\end{tetel}
\begin{biz}
 \begin{description*}
  \item[~~~1] Lásd \ref{LokalisSzelsoertekSzukFeltetel} tétel (\pageref{LokalisSzelsoertekSzukFeltetel}. oldal).
  \item[2/a] Tehát $\exists\varepsilon$, hogy $f$ szig. mon. csökkenő, ha $x\in(x_0-\varepsilon, x_0)$ és $f$ szig. mon. nő, ha $x\in(x_0, x_0+\varepsilon)$, illetve $x_0$-ban lokális szélsőértéke van, tehát $x_0$-ban lokális minimuma van.
  \item[2/b] \Aref{LokalisanNoCsokken}. tétel alapján ha $f''(x_0)>0$, akkor $f'$ lokálisan nő $x_0$-ban. Mivel $f'(x_0)=0$, ezért a lokálisan növekedés miatt $f'$ az $x_0$ pontban előjelet vált (negatívból pozitívba), ezért itt lokális minimuma van. Ugyanígy bizonyítható a másik fele.
 \end{description*}
\end{biz}

\begin{tetel} $\boxed{$Elégséges feltétel inflexiós pont létezésére$}$
 \begin{enumerate*}
  \item Ha $f$ kétszer differenciálható $x_0$ egy környezetében és $f''(x_0)=0$, és\\
  $f''(x)$ előjelet vált $x_0$-ban $\quad$ vagy $\quad$ $f''(x)$ lokálisan nő v. csökken $x_0$-ban\\
  $\Rightarrow$ $f$-nek $x_0$-ban inflexiós pontja van.
  \item Ha $f$ kétszer differenciálható $x_0$ egy környezetében és $f''(x_0)=0$ és $\exists f'''(x_0)\neq 0$, akkor $f$-nek $x_0$-ban inflexiós pontja van.
 \end{enumerate*}
\end{tetel}
\addtocounter{biz}{1}

\subsection{Implicit deriválás}

Explicit kapcsolat: $y(x)=\ldots$\\
Implicit kapcsolat: $f(x,y)=0$\\
\emph{Cél}: Implicit kapcsolat segítségével adjuk meg az $y'(x_0)$ deriváltat egy adott $x_0$ pontban.

\subsubsection{Példák}

\emph{1. példa}: $y(x)$ folytonos, kétszer deriválható és kielégíti a következő implicit egyenletet:
\[y+x\cdot \ln y + 2x^2-x+\ln(1+x)=1\]
Az $(x_0, y_0)=(0,1)$ pontban milyen lokális tulajdonságok vannak az $y(x)$ függvény grafikonjában? ($y'(x_0)=?, y''(x_0)=?$)\\

$(x_0, y_0)=(0,1)$ valóban kielégíti a fenti egyenletet. $y$ is valamilyen függvény, hiszen függ $x$-től:
\[y(x)+x\cdot \ln y(x) + 2x^2-x+\ln(1+x)=1 \qquad /\dfrac{d}{dx}\]
\[y'(x)+\ln y(x)+x\cdot \frac{1}{y(x)}\cdot y'(x) + 4x-1+\frac{1}{1+x}=0\]
\[y'(x)= \frac{1-4x-\frac{1}{1+x}-\ln y(x)}{1+x\cdot \frac{1}{y(x)}}\]
\[y'(x)= \left(\frac{y(x)}{y(x)+x}\right)\cdot\left(1-4x-\frac{1}{1+x}-\ln y(x)\right)\]
Nézzük meg, hogy akkor mi lehet az $(0,1)$ pontban ($x=0, y(x)=1$):
\[y'(0)= \left(\frac{1}{1+0}\right)\cdot\left(1-0-\frac{1}{1+0}-\ln y(1)\right) = 0\]
Tehát itt lehet lokális szélső érték, határozzuk meg a 2. deriváltat:
\[\left(y'(x)+\ln y(x)+x\cdot \frac{1}{y(x)}\cdot y'(x) + 4x-1+\frac{1}{1+x}\right)'=0\]
\[y''(x)+\frac{y'(x)}{y(x)}+\frac{y'(x)}{y(x)}+x\cdot \left(-\frac{1}{y^2(x)}\right)\cdot y'^2(x)+x\cdot \frac{1}{y(x)}\cdot y''(x) + 4-\frac{1}{(1+x)^2}=0\]
$x=0$, $y(x)=1$, $y'(x)=0$, tehát:
\[y''(x)+2\cdot\frac{0}{1}+0\cdot \left(-\frac{1}{1}\right)\cdot 0+0\cdot \frac{1}{1}\cdot y''(x) + 4-\frac{1}{(1+0)^2}=0\]
\[y''(x) = -3 \quad \Rightarrow \quad \hbox{lokális maximuma van $(0,1)$-ben az $y(x)$-nek}\]\\

\emph{2. példa}: $x\cdot \sh x - y\cdot \ch y = 0$. $y(x)$ kétszer folytonosan deriválható. Milyen lokális tulajdonságai vannak $y$-nak $x_0=0$-ban? Először is ekkor mennyi az $y_0$?
\[0\cdot \sh 0 - y_0\cdot \ch y_0 = 0 \quad \Rightarrow \quad y_0\cdot \ch y_0 = 0\]
Mivel $\ch y_0 \geqslant 1$, ezért $y_0 = 0$.
Lederiváljuk:
\[\sh x+x\cdot\ch x - y'\cdot \ch y - y\cdot \sh y\cdot y' = 0\]
$x = 0, y=0$, tehát:
\[\sh 0+0\cdot\ch 0 - y'\cdot \ch 0 - 0\cdot \sh 0\cdot y' = 0 \quad \Rightarrow \quad y'(0) = 0\]
Tehát $(0,0)$-ban lehet szélsőérték, nézzük meg a második deriváltat:
\[\ch x+\ch x+x\cdot\sh x - y''\cdot \ch y - y'\cdot \sh y\cdot y' - y'\cdot \sh y\cdot y' - y\cdot \ch y\cdot y'^2 - y\cdot \sh y\cdot y'' = 0\]
Beírva az ismert adatokat: $x = 0, y=0, y'=0$:
\[2\cdot\ch 0+0\cdot\sh 0 - y''\cdot \ch 0 - 2\cdot 0^2\cdot \sh 0 - 0\cdot \ch 0\cdot 0^2 - 0\cdot \sh 0\cdot y'' = 0\]
\[2=y''(0) > 2 \quad \Rightarrow \quad \hbox{lokális minimuma van $(0,0)$-ban}\]

\section{Teljes függvény vizsgálat}

\subsection{Teendők}
\begin{enumerate}
 \item $D_f$, zérushelyek, periocitás, paritás, szakadási helyek, határértékek ($\pm\infty$, szakadási helyeken, $D_f$ végpontjaiban)
 \item $f'(x)$: monotonitás, lokális szélsőértékek
 \item $f''(x)$: konvexitás, inflexiós pontok
 \item (Lineáris asszimptoták vizsgálatat -- nem tanultuk!)
 \item $R_f$, grafikon felrajzolása
\end{enumerate}

\subsection{Konkrét példákon való függvény vizsgálat}

\[\boxed{f(x)=x^2\cdot \ln x}\]
$D_f = (0;\infty)$. Nem periodikus, nem páros/páratlan. Folytonos $D_f$-en, tehát nincs szakadási hely. Zérushely:
\[x^2\cdot \ln x = 0 \quad \Leftrightarrow \quad x=1\]
Határértékek:
\[\lim_{x\to 0} \left(x^2\cdot \ln(x)\right) = \lim_{x\to 0} \frac{\ln(x)}{x^{-2}} \overset{L'H}{=} \lim_{x\to 0} \frac{\frac{1}{x}}{-2x^{-3}} = \lim_{x\to 0} \frac{x^2}{-2} = \infty\]
\[\lim_{x\to \infty} \left(x^2\cdot \ln(x)\right) = \infty\]
Deriváltakat felírjuk, majd keressük a zérushelyeket:
\[f'(x) = (x^2\cdot \ln x)' = 2x\cdot \ln x+x = x(2\ln x +1) = 0\]
Mivel $x\in(0;\infty)$, ezért $f'(x)=0 \quad\Leftrightarrow\quad 2\ln x +1 = 0 \quad\Rightarrow\quad x=e^{-\frac{1}{2}}$.
\[f''(x) = (2x\cdot \ln x+x)' = 2\cdot \ln x+2+1 = 0 \quad\Leftrightarrow\quad x=e^{-\frac{3}{2}}\]

Táblázatba foglalva:
\[
\begin{array}{c||c|c|c}
x & (0, e^{-\frac{1}{2}}) & e^{-\frac{1}{2}} & (e^{-\frac{1}{2}},\infty)\\\hline\hline
f' & - & 0 & +\\\hline
f & \searrow & \hbox{lok. min} & \nearrow
\end{array} \qquad\qquad
\begin{array}{c||c|c|c}
x & (0, e^{-\frac{3}{2}}) & e^{-\frac{3}{2}} & (e^{-\frac{3}{2}},\infty)\\\hline\hline
f'' & - & 0 & +\\\hline
f & \cap & \hbox{infl. pont} & \cup
\end{array}
\]
Felrajzolva (az analízis és a rajz után $R_f$ könnyen meghatározható: $R_f=\left[\frac{-1}{2e},\infty\right)$):
\begin{center}
\psset{xunit=3.33cm,yunit=10.0cm,algebraic=true,dotstyle=o,dotsize=3pt 0,linewidth=0.8pt,arrowsize=3pt 2,arrowinset=0.25}
\begin{pspicture*}(-0.34,-0.25)(1.21,0.16)
\psaxes[xAxis=true,yAxis=true,labels=none,Dx=0.2,Dy=0.1,ticksize=-2pt 0,subticks=2]{->}(0,0)(-0.34,-0.21)(1.21,0.16)
\psplot[plotpoints=200]{3.81244408197498E-7}{1.2148616706225237}{x^2*ln(x)}
\psline[linestyle=dashed,dash=1pt 1pt](0.22313,-0.21)(0.22313,0.16)
\psline[linestyle=dashed,dash=1pt 1pt](0.60653,-0.21)(0.60653,0.16)
\psdots[dotstyle=*](0.22,-0.074680603)
\rput[c](0.22,0.04){$e^{-\frac{1}{2}}$}
\psdots[dotstyle=*](0.61,-0.183939721)
\rput[c](0.61,0.04){$e^{-\frac{3}{2}}$}
\rput[c](1,0.03){$1$}
\psdots[dotstyle=*](1,0)
\rput[c](-0.15,-0.183939721){$\frac{-1}{2e}$}
\psplot[linestyle=dotted]{-0.34}{1.21}{(-0.183939--0*x)/1}
\psdots(0,-0)
\end{pspicture*}
\end{center}

\[\boxed{g(x)=\sqrt[3]{(x^2-1)^2}} \qquad D_f = \mathbb{R}\]
A függvény nem periodikus, de páros, hiszen $f(-x) = f(x)$. Folytonos $\mathbb{R}$-en. Zérushelyek:
\[f(x)=0 \quad \Leftrightarrow \quad (x^2-1)^2 = 0 \quad \Leftrightarrow \quad x=\pm 1\]
Határértékek:
\[\lim_{x\to\pm\infty} \sqrt[3]{(x^2-1)^2} = \infty\]
Derivált függvény:
\[f'(x)=\frac{2}{3}\cdot(x^2-1)^{\frac{-1}{3}}\cdot 2x = \frac{4x}{3\sqrt[3]{x^2-1}} \quad \hbox{ha } x\neq \pm 1\]
\[f'(x) = 0 \quad \Leftrightarrow \quad x = 0\]
Derivált függvény $x=\pm 1$-ben nem értelmezett, de definíció alapján megnézhetjük, hogy van-e differenciálhányados vagy nincs:
\[\lim_{h\to 0+0} \frac{f(\pm1+h)-f(\pm1)}{h} = \lim_{h\to 0+0} \frac{\sqrt[3]{((\pm1+h)^2-1)^2}-0}{h} = \lim_{h\to 0+0} \frac{\sqrt[3]{(h^2 \pm 2h)^2}-0}{h} = \]
\[= \lim_{h\to 0+0} \sqrt[3]{\frac{h^4 + 4h^2 \pm 4h^3}{h^3}} =  \lim_{h\to 0+0} \sqrt[3]{h + \frac{4}{h} \pm 4} = \infty \]
Tehát $\nexists f'(1)$ és $\nexists f'(-1)$. Táblázat felrajzolásához segítségül hívunk egy számegyenest, hogy a törtben melyik tag, milyen előjelű:

\begin{center}
 \psset{xunit=1.0cm,yunit=1.0cm,algebraic=true,dotstyle=o,dotsize=3pt 0,linewidth=0.8pt,arrowsize=3pt 2,arrowinset=0.25}
\begin{pspicture*}(-4.8,-3.4)(3.2,0.78)
\psaxes[xAxis=true,yAxis=false,Dx=1,Dy=1,ticksize=-2pt 0,subticks=2]{->}(0,0)(-2.84,-3.14)(2.92,0.78)
\psplot{0}{2.92}{(-3.7-0*x)/3.7}
\psplot[linestyle=dashed,dash=5pt 5pt]{-2.84}{0}{(--4-0*x)/-4}
\psline[linestyle=dashed,dash=5pt 5pt](-1,-2)(1,-2)
\psplot{-2.84}{-1}{(--4.84-0*x)/-2.42}
\psplot{1}{2.92}{(-4-0*x)/2}
\psline[linestyle=dotted](-1,-3.14)(-1,0.78)
\psline[linestyle=dotted](1,-3.14)(1,0.78)
\psline[linestyle=dotted](-0,-3.14)(-0,0.78)
\psdots[dotstyle=*](0,-1)
\psdots(-1,-2)
\psdots(1,-2)
\rput[Br](-3.0,-1.1){$4x$}
\rput[Br](-3.0,-2.1){$(x^2-1)^\frac{1}{3}$}
\rput[Br](3.1,-2.8){$f'$}
\rput[c](-2.0,-2.7){\begin{footnotesize}$\pscirclebox[boxsep=false, framesep=0.1pt, linewidth=0.4pt]{-}$\end{footnotesize}}
\rput[c](-0.5,-2.7){\begin{footnotesize}$\pscirclebox[boxsep=false, framesep=0.1pt, linewidth=0.4pt]{+}$\end{footnotesize}}
\rput[c](0.5,-2.7){\begin{footnotesize}$\pscirclebox[boxsep=false, framesep=0.1pt, linewidth=0.4pt]{-}$\end{footnotesize}}
\rput[c](2,-2.7){\begin{footnotesize}$\pscirclebox[boxsep=false, framesep=0.1pt, linewidth=0.4pt]{+}$\end{footnotesize}}
\end{pspicture*}\\[+5pt]
$\begin{array}{c||c|c|c|c|c|c|c}
x & (-\infty, -1) & -1 & (-1,0) & 0 & (0, 1) & 1 & (1,\infty)\\\hline\hline
f' & - & \nexists & + & 0 & - & \nexists & +\\\hline
f & \searrow & \hbox{lok. min} & \nearrow & \hbox{lok. max} & \searrow & \hbox{lok. min} & \nearrow
\end{array}$
\end{center}

Második derivált:
\[f''(x)=\left(\frac{4x}{3\sqrt[3]{x^2-1}}\right)' = \frac{12\sqrt[3]{x^2-1}-8x^2\cdot(x^2-1)^{\frac{-2}{3}}}{9(\sqrt[3]{x^2-1})^2} = \frac{12(x^2-1)^{\frac{1}{3}}-8x^2\cdot(x^2-1)^{\frac{-2}{3}}}{9(x^2-1)^{\frac{2}{3}}} = \]
\[= \frac{12-8x^2\cdot(x^2-1)^{\frac{-3}{3}}}{9(x^2-1)^{\frac{1}{3}}} = 0 \quad \Leftrightarrow \quad 12-8x^2\cdot(x^2-1)^{\frac{-3}{3}} = 0\]
\[12-\frac{8x^2}{x^2-1} = 0\]
\[12x^2 - 12 = 8x^2 \quad \Leftrightarrow \quad 4x^2 = 12\]
\[x = \pm\sqrt{3}\]
\[\begin{array}{c||c|c|c|c|c|c|c|c|c}
x & (-\infty, -\sqrt{3}) & -\sqrt{3} & (-\sqrt{3},-1)  & -1 & (-1,1)  & 1 & (1, \sqrt{3}) & \sqrt{3} & (\sqrt{3},\infty)\\\hline\hline
f'' & + & 0 & - & \nexists & - & \nexists & - & 0 & + \\\hline
f & \cup & \hbox{infl. pont} & \cap &  & \cap & & \cap & \hbox{infl. pont} & \cup
\end{array}\]

\begin{center}
\psset{xunit=1.0cm,yunit=1.0cm,algebraic=true,dotstyle=o,dotsize=3pt 0,linewidth=0.8pt,arrowsize=3pt 2,arrowinset=0.25}
\begin{pspicture*}(-2.69,-0.8)(2.66,2.83)
\psaxes[xAxis=true,yAxis=true,Dx=1,labels=y,Dy=1,ticksize=-2pt 0,subticks=2]{->}(0,0)(-2.69,-0.8)(2.66,2.83)
\psplot[plotpoints=200]{-2.6907634872151776}{2.6614039072200266}{((x^2-1)^2)^(1/3)}
\psline[linestyle=dashed,dash=3pt 3pt](-1.73,-0.1)(-1.73,2.83)
\psline[linestyle=dashed,dash=3pt 3pt](1.73,-0.1)(1.73,2.83)
\psdots[dotstyle=*](-1.73,1.59)
\psdots[dotstyle=*](1.73,1.59)
\psdots[dotstyle=*](0,1)
\rput[c](1,-0.41){$1$}
\rput[c](-1,-0.41){$-1$}
\rput[c](1.73,-0.41){$\sqrt{3}$}
\end{pspicture*}
\end{center}

$R_f = [0, \infty)$. 

\subsection{Folytonos függvények szélsőértékei zárt intervallumon (abszolút szélsőértékhely)}

Zárt intervallumon folytonos függvénynek van minimuma és maximuma (lásd Weierstrass II. tétele (jegyzetben \aref{W.II}. tétel \apageref{W.II}. oldalon). Vizsgálandó pontok:
\begin{itemize*}
 \item Derivált zérushelyei
 \item Ahol a függvény nem deriválható
 \item intervallum végpontjai.
\end{itemize*}

\subsubsection{Példa}

$f(x)=\sqrt[3]{2x-8}-\frac{2}{3}x+3$. $I=\left[0; \frac{9}{2}\right]$.\\
Intervallum végpontjaiban: $f(0) = -2+3 = 1$. $f\left(\frac{9}{2}\right)=1-3+3 = 1$.\\
Derivált: $f'(x) = \frac{2}{3}\cdot(2x-8)^{\frac{-2}{3}}-\frac{2}{3}$. $\nexists f'(x)$, ha $x=4$. $f(4)=\frac{1}{3}$.\\
$f'(x) = 0$, ha:
\[\frac{2}{3}\cdot(2x-8)^{\frac{-2}{3}}-\frac{2}{3} = 0\]
\[\sqrt[3]{\frac{1}{(2x-8)^2}} = 1\]
\[2x-8 = \pm1 \qquad \Leftrightarrow \qquad x_1 = \frac{7}{2};\; x_2 = \frac{9}{2}\]
$f\left(\frac{7}{2}\right) = -\frac{1}{3}$.\\
Tehát a függvény abszolút minimuma $I$-n: $\Inf\left\{1, -\frac{1}{3}, \frac{1}{3}\right\} = -\frac{1}{3}$, amit akkor vesz fel, ha $x = \frac{7}{2}$. Abszolút maximuma: $\Sup\left\{1, -\frac{1}{3}, \frac{1}{3}\right\} = 1$, amit akkor vesz fel ha $x=0$ vagy ha $x=\frac{9}{2}$.

\chapter{Polár koordináták}

Egy adott $P$ pont megadása a két koordinátával: $(r, \varphi)$, ahol $r$ az origótól mért távolság ($OP$ szakasz hossza), $\varphi$ pedig az $x$ tengely és az $OP$ szakasz irányított szögtávolsága. Tehát: $(r, \varphi) \in [0,\infty)\times[0, 2\pi)$. Ezzel a koordinátázással az origó kivételével a sík összes pontját egy-egy értelműen megfeleltethetjük, az origónál $r=0$, de $\varphi$ tetszőleges.

\section{Ortogonális koordinátarendszer}

\begin{wrapfigure}{r}{0.37\textwidth}
   \vspace{-25pt}
\begin{center}
\psset{xunit=0.5cm,yunit=0.5cm,algebraic=true,dotstyle=o,dotsize=3pt 0,linewidth=0.8pt,arrowsize=3pt 2,arrowinset=0.25}
\begin{pspicture*}(-6,-6)(6,6)
\psaxes[xAxis=true,yAxis=true,Dx=2,Dy=2,labels=none,ticksize=-2pt 0,subticks=1]{->}(0,0)(-4.8,-4.8)(4.8,4.8)
\pscircle[linewidth=0.3pt](0,0){1}
\pscircle[linewidth=0.3pt](0,0){2}
\psline[linewidth=0.3pt](3,5.2)(-0,-0)
\psline[linewidth=0.3pt](5.2,3)(-0,-0)
\psline[linewidth=0.3pt](6,0)(-0,-0)
\psline[linewidth=0.3pt](5.2,-3)(-0,-0)
\psline[linewidth=0.3pt](3,-5.2)(-0,-0)
\psline[linewidth=0.3pt](-3,-5.2)(-0,-0)
\psline[linewidth=0.3pt](-5.2,-3)(-0,-0)
\psline[linewidth=0.3pt](-6,0)(-0,-0)
\psline[linewidth=0.3pt](-5.2,3)(-0,-0)
\psline[linewidth=0.3pt](-3,5.2)(-0,-0)
\psline[linewidth=0.3pt](0,6)(-0,-0)
\psline[linewidth=0.3pt](-0,-0)(-0,-6)
\parametricplot[linewidth=0.3pt]{4.1887902047863905}{5.7020646194377225}{1*0.42*cos(t)+-0*0.42*sin(t)+2|0*0.42*cos(t)+1*0.42*sin(t)+3.46}
\parametricplot[linewidth=0.3pt]{3.665191429188092}{5.186302670253404}{1*0.42*cos(t)+-0*0.42*sin(t)+3.46|0*0.42*cos(t)+1*0.42*sin(t)+2}
\psdots[dotstyle=*,dotsize=0.9pt](3.4,1.75)
\psdots[dotstyle=*,dotsize=0.9pt](2.06948, 3.21554)
%\rput[c](3.40,1.73){\footnotesize $\cdot$}
\end{pspicture*}
\end{center}
\vspace{-50pt}
\end{wrapfigure}

Amennyiben $\varphi$-t állandónak vesszük $r$ pedig befutja a $[0,\infty)$ intervallumot, akkor egy az origón átmenő félegyenest kapunk. Ha pedig $\varphi$ futja be a $[0, 2\pi)$ intervallumot, $r$ pedig állandó, akkor egy $r$ sugarú origó középpontú kört kapunk.\\

Amennyiben egy a derékszögű koordinátarendszerben megadott $(x,y)$ pontot, szeretnénk polár koordinátákkal felírni, akkor:
\[r = \sqrt{x^2+y^2} \qquad\quad \tg\varphi = \frac{y}{x} \quad \ctg\varphi = \frac{x}{y}\]
Visszafele pedig:
\[\left.\begin{array}{rcl}x & = & r\cdot\cos\varphi\\ y & = & r\cdot\sin\varphi \end{array}\right\}\]

\section{Görbék paraméteres megadása}

Eddig a következő megadást használtuk: $y(x) = f(x)$, ahol a görbe a függvény grafikonja. Ennek feltétele, hogy $\forall x_0$-hoz legfeljebb egy $y=f(x_0)$ érték tartozzon. Ehelyett a paraméteres megadást használjuk:

\[\left.\begin{array}{rcl}x &=& \xi(t)\\y &=& \eta(t)\end{array}\right\} \qquad
\begin{array}{l}
  t\in[t_1, t_2]\subset\mathbb{R}\\
  \xi[t_1, t_2] \to \mathbb{R}\\
  \eta[t_1, t_2] \to \mathbb{R}\\
\end{array}\]
Tehát a görbe:
\[\Big\{\big(\xi(t), \eta(t)\big)\;\big|\;t\in[t_1, t_2]\Big\}\subset \mathbb{R}^2\]

\subsection{Kör}

\begin{wrapfigure}{l}{0.25\textwidth}
\vspace{-25pt}
\begin{center}
\psset{xunit=0.5cm,yunit=0.5cm,algebraic=true,dotstyle=o,dotsize=3pt 0,linewidth=0.8pt,arrowsize=3pt 2,arrowinset=0.25}
\begin{pspicture*}(-4,-4)(4,4)
\psaxes[xAxis=true,yAxis=true,Dx=5,Dy=5,labels=none,ticksize=-2pt 0,subticks=1]{->}(0,0)(-4,-4)(4,4)
\parametricplot[plotpoints=200]{0}{6.2832}{3*cos(t)|3*sin(t)}
\end{pspicture*}
\end{center}
\vspace{-20pt}
\end{wrapfigure}

$t \in [0, 2\pi)$\\
\[x(t) = \xi(t)= R\cdot \cos t\]
\[y(t) = \eta(t)= R\cdot \sin t\]
Ahol $R$ a kör sugara.
\vspace{35pt}

\subsection{Ellipszis}

\begin{wrapfigure}{l}{0.25\textwidth}
\vspace{-25pt}
\begin{center}
\psset{xunit=0.5cm,yunit=0.5cm,algebraic=true,dotstyle=o,dotsize=3pt 0,linewidth=0.8pt,arrowsize=3pt 2,arrowinset=0.25}
\begin{pspicture*}(-4,-2.7)(4,2.7)
\psaxes[xAxis=true,yAxis=true,Dx=5,Dy=5,labels=none,ticksize=-2pt 0,subticks=1]{->}(0,0)(-4,-2.7)(4,2.7)
\parametricplot[plotpoints=200]{0}{6.2832}{3*cos(t)|1.7*sin(t)}
\rput[c](0,0){\rnode{o}{}}
\rput[c](3,0){\rnode{a}{}}
\rput[c](0,1.7){\rnode{b}{}}
\psbrace[linewidth=0.3pt,ref=c,rot=90,nodesepA=0pt,nodesepB=-5pt](o)(a){\footnotesize$a$}
\psbrace[linewidth=0.3pt,ref=c,rot=0,nodesepA=5pt,nodesepB=0pt](o)(b){\footnotesize$b$}
\end{pspicture*}
\end{center}
\vspace{10pt}
\end{wrapfigure}

$t \in [0, 2\pi)$\\
\[x(t) = \xi(t)= a\cdot \cos t\]
\[y(t) = \eta(t)= b\cdot \sin t\]
Ahol $a$ és $b$ rendre a nagy- és kistengely hosszának a fele.

\subsection{Archimédeszi spirál}

\begin{wrapfigure}{l}{0.365\textwidth}
\vspace{-20pt}
\begin{center}
\psset{xunit=0.5cm,yunit=0.5cm,algebraic=true,dotstyle=o,dotsize=3pt 0,linewidth=0.8pt,arrowsize=3pt 2,arrowinset=0.25}
\begin{pspicture*}(-6,-6)(6,6)
\psaxes[xAxis=true,yAxis=true,Dx=7,Dy=7,labels=none,ticksize=-2pt 0,subticks=1]{->}(0,0)(-6,-6)(6,6)
\parametricplot[plotpoints=200]{0}{15.708}{2/(6.2832)*t*cos(t)|2/(6.2832)*t*sin(t)}
\rput[c](0,0){\rnode{xA}{}}
\rput[c](2,0){\rnode{xB}{}}
\rput[c](4,0){\rnode{xC}{}}

\rput[c](0,0.5){\rnode{yA}{}}
\rput[c](0,2.5){\rnode{yB}{}}
\rput[c](0,4.5){\rnode{yC}{}}

\psbrace[linewidth=0.3pt,ref=c,rot=90,nodesepA=0pt,nodesepB=-7pt](xA)(xB){\footnotesize$A$}
\psbrace[linewidth=0.3pt,ref=c,rot=90,nodesepA=0pt,nodesepB=-7pt](xB)(xC){\footnotesize$A$}

\psbrace[linewidth=0.3pt,ref=c,rot=0,nodesepA=5pt,nodesepB=0pt](yA)(yB){\footnotesize$A$}
\psbrace[linewidth=0.3pt,ref=c,rot=0,nodesepA=5pt,nodesepB=0pt](yB)(yC){\footnotesize$A$}
\end{pspicture*}
\end{center}
\vspace{-80pt}
\end{wrapfigure}

\[x(t) = \xi(t)= \frac{A}{2\pi}\cdot t\cdot \cos t\]
\[y(t) = \eta(t)= \frac{A}{2\pi}\cdot t\cdot \sin t\]
Az ábrán látható spirálhoz $t\in[0,5\pi]$ paramétert használtunk.\\
\vspace{40pt}

\section{Görbék invertálhatósága, differenciálása}

\begin{wrapfigure}{l}{0.35\textwidth}
\vspace{-25pt}
\begin{center}
\psset{xunit=0.5cm,yunit=0.5cm,algebraic=true,dotstyle=*,dotsize=3pt 0,linewidth=0.8pt,arrowsize=3pt 2,arrowinset=0.25}
\begin{pspicture*}(-6,-4)(6,6)
\psaxes[xAxis=true,yAxis=true,Dx=7,Dy=7,labels=none,ticksize=-2pt 0,subticks=1]{->}(0,0)(-6,-4)(6,6)
\parametricplot[plotpoints=200]{1.2}{4.2}{2*t*cos(t)+3|t*sin(t)+2}
\psdots(3.86966,3.11845) % t2
\rput[c](4.3,3.6){$t_2$}
\psdots(-2.63156,2.69382) % t0
\rput[c](-2.9,3.2){$t_0$}
\psdots(-1.11819,-1.66062) % t1
\rput[c](-1.1,-2.3){$t_1$}
\psline[linewidth=0.7pt,linestyle=dashed](3.86966,-4)(3.86966,5)
\psline[linewidth=0.7pt,linestyle=dashed](-3.57,-4)(-3.57,5)
\end{pspicture*}
\end{center}
\vspace{-35pt}
\end{wrapfigure}

\[t\in[t_1,t_2]\subset\mathbb{R} \qquad \hbox{Legyen } t_0 \in I\]
\begin{tetelAbraval}
 Ha $I'\subset I$-n a $\xi(t)$ invertálható, akkor $\xi(I')\subset \mathbb{R}$-en felírható a görbe $y(x)$ alakban.
\end{tetelAbraval}
\begin{bizAbraval}
\[y(x) = \eta\Big(\underbrace{\xi^{-1}(x)}_{t\in I'}\Big) \qquad x\in\xi(I')\]
\end{bizAbraval}

\newpage % Jobban néz ki, ha itt lapot dobunk.

Következmény: $\boxed{$Az invertálás elégséges feltétele$}$
\begin{itemize*}
 \item $\exists \dot{\xi}$ és nem vált előjelet $t_0$ egy környezetében.
 \item $\dot{\xi}$ folytonos $t_0$-ban és $\dot{\xi}(t_0) \neq 0$
\end{itemize*}
Ahol $\dot{\xi} = \dfrac{d\xi}{dt}$\\

Ilyenkor (ha az intervallumon invertálható a függvény) tudjuk vizsgálni az $y$ egy $x_0$-ban vett deriváltját, de előtte nézzük meg mégegyszer az $y$ értékét $x_0$-ban:
\[y(x_0) = \eta(\xi^{-1}(x_0)) = \eta(t_0) \qquad (x_0 = \xi(t_0))\]

Elsőrendű derivált (ha $\exists \dot{\eta}(t_0)$ és $\exists \dot{\xi}(t_0)$, illetve $\dot{\xi}(t_0)\neq 0$):
\[\frac{dy(x_0)}{dx_0} = y'(x_0) = \dot{\eta}(\xi^{-1}(x_0))\cdot\frac{1}{\dot{\xi}(\xi^{-1}(x_0))} = \boxed{\frac{\dot{\eta}(t_0)}{\dot{\xi}(t_0)}}\]

Másodrendű derivált:
\[\frac{d^2y(x_0)}{dx^2_0} = y''(x_0) = \left(\frac{\dot{\eta}(\xi^{-1}(x_0))}{\dot{\xi}(\xi^{-1}(x_0))}\right)' = \frac{\Big(\dot{\eta}(\xi^{-1}(x_0))\Big)'\dot{\xi}(\xi^{-1}(x_0))}{\dot{\xi}^2(\xi^{-1}(x_0))} - \]
\[ - \frac{\dot{\eta}(\xi^{-1}(x_0))\Big(\dot{\xi}(\xi^{-1}(x_0))\Big)'}{\dot{\xi}^2(\xi^{-1}(x_0))} = \frac{\ddot{\eta}(\xi^{-1}(x_0))\dot{\xi}(\xi^{-1}(x_0))}{\dot{\xi}^2(\xi^{-1}(x_0))}\cdot\frac{1}{\dot{\xi}(\xi^{-1}(x_0))} - \]
\[- \frac{\dot{\eta}(\xi^{-1}(x_0))\ddot{\xi}(\xi^{-1}(x_0))}{\dot{\xi}^2(\xi^{-1}(x_0))}\cdot\frac{1}{\dot{\xi}(\xi^{-1}(x_0))} = \boxed{\frac{\ddot{\eta}\cdot\dot{\xi}-\dot{\eta}\cdot\ddot{\xi}}{\dot{\xi}^3}\Big(t_0\Big)}\]

\chapter{Integrálszámítás}

\section{Határozatlan integrál}

\subsection{Primitív függvény}

\begin{defi}
 $F$ az $f$ függvény \textbf{prmitív függvénye} az $I\subset \mathbb{R}$ intervallumon, ha $\forall x\in I$ esetén $\exists F'(x) = f(x)$.
\end{defi}
Például:\\
$F(x) = \cos(2x) \; \Rightarrow \; F'(x)=-2\sin(2x)$.\\
$G(x)=2\cos^2 x \;\Rightarrow\; G'(x) = -4\cos x\sin x = -2\sin (2x)$. Tehát $f(x)=-2\sin (2x)$-nek $F$ és $G$ is primitív függvénye.

\begin{tetel}
 Ha $F$ és $G$ az $f$ primitív függvénye $I$-n, akkor $\exists c\in\mathbb{R}$, hogy $F(x)=G(x)+c$, $\forall x\in I$-re.
\end{tetel}
\begin{bizNoNL}
 \[F'(x)=G'(x)=f(x) \quad \underset{\hbox{I.}}{\overset{\hbox{Int. szám.}}{\Rightarrow}} \quad F(x)-G(x) = c\]
Megjegyzés: Csak intervallumon!
\end{bizNoNL}

\begin{defi}
 Az $f$ függvény \textbf{határozatlan integrálja} az $I\subset \mathbb{R}$ intervallumon az $f$ primitív függvényeinek összessége:
\[\integ{f(x)} = \Big\{F(x) \,\big|\, F'(x) = f(x) \quad \forall x\in I\Big\}\]
\end{defi}

\subsubsection{Példa}

$f(x) = \dfrac{1}{x} \quad D_f = \mathbb{R}\setminus\{0\}$.
\[\integ{\dfrac{1}{x}} = \left.\begin{cases}
  \ln x + c_1, \quad \hbox{ha } x>0\\
  \ln(-x) + c_2, \quad \hbox{ha } x<0\\
\end{cases}\right\} = \ln|x| + c
\]

\subsection{Határozatlan integrál tulajdonságai}

\[\integ{\Big(f(x)+g(x)\Big)} = \integ{f(x)}+\integ{g(x)}\]
\[\integ{c\cdot f(x)} = c\cdot\integ{f(x)} \qquad c\in\mathbb{R}\]
\[\integ{\Big(f(\varphi(x))\cdot \varphi'(x)\Big)} = F(\varphi(x))+c \qquad F'(t) = f(t)\]
\[\integ{x^n} = \frac{x^{n+1}}{n+1}+c \qquad n\neq -1\]
\[\integ{\frac{1}{x}} = \ln|x| + c\]
\[\integ{\varphi^\alpha(x)\cdot \varphi'(x)} = \frac{\varphi^{\alpha+1}(x)}{\alpha+1}+c \qquad \alpha \neq -1\]
\[\integ{\frac{\varphi'(x)}{\varphi(x)}} = \ln|\varphi(x)| + c\]
\[\integ{e^{\varphi(x)}\cdot\varphi'(x)} = e^{\varphi(x)} + c\]
\[\integ{f(ax+b)} = \frac{F(ax+b)}{a}+c \qquad a\neq 0\]
A fentiek a már ismeretes deriválás szabályaival levezethetőek.

\subsubsection{Példák}

$\displaystyle \integ{\sin(3x)} = \frac{-\cos (3x)}{3}+c$\\
$\displaystyle \integ{\frac{6\cdot e^3x-2e^{-x}}{e^{2x}}} = 6\integ{e^x}-2\integ{e^{-3x}} = 6e^x+\frac{2}{3}e^{-3x}+c$\\
$\displaystyle \integ{x^3} = \frac{x^4}{4} + c$\\
$\displaystyle \integ{\Big(\sh x\cdot (3+4\ch x)^3\Big)} = \frac{(3+4\ch x)^4}{16} + c$\\
$\displaystyle \integ{\tg x} = \integ{\frac{\sin x}{\cos x}} = -\ln|\cos x|+c$\\[+4pt]
$\displaystyle \integ{\tg^2 x} = \integ{\frac{\sin^2 x}{\cos^2 x}} = \integ{\frac{1-\cos^2 x}{\cos^2 x}} = \integ{\frac{1}{\cos^2 x}} - \integ{} = \tg x - x +c$\\[+4pt]
$\displaystyle \integ{\frac{\tg^2 x}{\cos^2 x}} = \integ{\tg^2 x\cdot\frac{1}{\cos^2 x}} = \frac{\tg^3 x}{3} + c$\\
$\displaystyle \integ{\Big(\sin x\cdot \cos^3 x\Big)} = -\frac{\cos^4 x}{4} + c$\\
$\displaystyle \integ{\Big(x^2\sh(3x^3-5)\Big)} = \frac{1}{9}\ch(3x^3-5) + c$\\
$\displaystyle \integ{\frac{3}{2+5x}} = \frac{3}{5}\ln|2+5x| + c$\\[+4pt]
$\displaystyle \integ{\frac{3}{(2+5x)^2}} = \frac{3}{5}\cdot\frac{-1}{2+5x} + c$\\
$\displaystyle \integ{\frac{3}{2+5x^2}} = \integ{\frac{3}{2}\cdot\frac{1}{1+\frac{5}{2}x^2}} = \frac{3}{2}\cdot\integ{\frac{1}{1+\left(\sqrt{\frac{5}{2}}x\right)^2}} = \frac{3}{2}\cdot\sqrt{\frac{2}{5}}\cdot\arctg\left(\sqrt{\frac{5}{2}}x\right) +c$\\[+4pt]
$\displaystyle \integ{\frac{1}{\sqrt{1-4x^2}}} = \integ{\frac{1}{\sqrt{1-(2x)^2}}} = \frac{\arcsin(2x)}{2} + c$\\[+4pt]
$\displaystyle \integ{\frac{1}{\sqrt{1+(4x)^2}}} = \frac{1}{4}\arsh(4x) + c$\\[+4pt]
$\displaystyle \integ{\frac{1}{\sqrt{2-8x^2}}} = \frac{1}{\sqrt{2}}\integ{\frac{1}{\sqrt{1-(2x)^2}}} = \frac{1}{2\sqrt{2}}\arcsin(2x) + c$\\[+4pt]
$\displaystyle \integ{\frac{x}{\sqrt{3+5x^2}}} = \frac{2(3+5x^2)^{\frac{1}{2}}}{10} + c$\\[+4pt]
$\displaystyle \integ{\frac{1}{\sqrt{3-2x-x^2}}} = \integ{\frac{1}{\sqrt{4-(x+1)^2}}} = \integ{\frac{1}{2\sqrt{1-\left(\frac{x+1}{2}\right)^2}}} = \arcsin\left(\frac{x+1}{2}\right)+ c$

\subsection{Integrálási módszerek}

\newcounter{integcount}
\newcounter{subcount}
\newcounter{subsubcount}
\begin{list}{\arabic{integcount}.}{\usecounter{integcount}\leftmargin=1.5em} % azért kell list, hogy lehessen a margint szabályozni
 \item $\boxed{\sin(ax)\cos(bx); \quad \sin(ax)\sin(bx); \quad \cos(ax)\cos(bx)}$\\[+3pt]
  A jól ismert addíciós tételek segítségével:
  \[\sin(\alpha\pm\beta) = \sin(\alpha)\cos(\beta)\pm\cos(\alpha)\sin(\beta)\]
  \[\cos(\alpha\pm\beta) = \cos(\alpha)\cos(\beta)\mp\sin(\alpha)\sin(\beta)\]
  Felírhatjuk a következőket:
  \[\begin{array}{rcl}
      \sin(\alpha+\beta)+\sin(\alpha-\beta) &=& 2\sin(\alpha)\cos(\beta)\\[+4pt]
      \cos(\alpha+\beta)+\cos(\alpha-\beta) &=& 2\cos(\alpha)\cos(\beta)\\[+4pt]
      \cos(\alpha+\beta)-\cos(\alpha-\beta) &=& -2\sin(\alpha)\sin(\beta)
    \end{array}\]
  Ezek alapján könnyű a fenti típusú integrálokat kiszámolni, hiszen összegre/különb\-ségre alakíthatjuk őket, és ezeket tagonként integrálhatjuk, pl:
  \[\integ{\sin(2x)\sin(3x)} = -\frac{1}{2}\integ{\Big(\cos(3x+2x)-\cos(3x-2x)\Big)} = \]
  \[ = -\frac{1}{2}\left(\frac{\sin(5x)}{5}-\sin x\right) +c = -\frac{\sin(5x)}{10}+\frac{\sin x}{2} +c\]
  \[\integ{\cos(3x)\cos(5x)} = \frac{1}{2}\integ{\Big(\cos(5x+3x)+\cos(5x-3x)\Big)} = \frac{\sin(8x)}{16}+\frac{\sin(2x)}{4} + c\]
 \item $\boxed{\sin^n x\cdot \cos^m x}$
  \begin{list}{\alph{subcount})}{\usecounter{subcount}\leftmargin=1.5em}
    \item $sin^n x$ vagy $cos^m x$ és $n,m$ páratlan
    \[sin^{2k+1} x = \sin x\cdot (sin^2 x)^k = \sin x\cdot(1-\cos^2 x)^k = \sum_{l} a_l\cdot\underbrace{\sin x\cdot \cos^l x}_{\varphi'\cdot\varphi^l}\]
    Hasonlóan $\cos^{2k+1}$ is felírható. Példa:
      \[\integ{\cos^5 x} = \integ{\cos x(1-\sin^2 x)^2} = \integ{\cos x} + \integ{\cos x\sin^4 x} - \]
      \[-2\integ{\cos x\sin^2 x} = \sin x + \frac{sin^5 x}{5} - 2\cdot\frac{\sin^3}{3} + c\]
    \item $sin^n x$ vagy $cos^m x$ és $n,m$ páros
      \[\sin^2 x = \frac{1-\cos 2x}{2} \qquad\qquad \cos^2 x = \frac{1+\cos 2x}{2}\]
    A fentieket felhasználva, tehát:
      \[\sin^{2k} x = (\sin^2 x)^k = \frac{(1-\cos 2x)^k}{2^k} \quad\hbox{illetve: }\qquad \cos^{2k} x = \frac{(1+\cos 2x)^k}{2^k}\]
    \[\integ{\underbrace{\sin^6 x}_{(\sin^2 x)^3}} = \frac{1}{8}\integ{(1-\cos 2x)^3} = \frac{1}{8}\left(\integ{}+3\integ{\underbrace{\cos^2 2x}_{\frac{1+\cos 4x}{2}}}- 3\integ{\cos 2x}-\right.\]
    \[\left. - \integ{\cos^3 2x}\right) =  \frac{1}{8}\left(x+\frac{3}{2}\integ{(1+\cos 4x)} -\frac{3}{2}\sin 2x -\integ{\cos^3 2x} \right) = \]
    \[\frac{x}{8}+\frac{3x}{16}+\frac{3}{16\cdot 4}\sin 4x -\frac{3}{16}\sin 2x -\frac{1}{8}\integ{\cos 2x(1-\sin^2 2x)} = \frac{5x}{16}+\frac{3}{64}\sin 4x -\]
    \[-\frac{3}{16}\sin 2x-\frac{1}{16}\sin 2x+\frac{1}{8\cdot 3\cdot 2}\sin^3 2x + c\]
    \item $sin^n x\cdot cos^m x$, ha $n$ és $m$ közül valamelyik páratlan, lásd a). Példa:
    \[\integ{\cos^3 x\sin^2 x} = \integ{\cos x(1-\sin^2 x)\sin^2 x} = \]
    \[= \integ{\cos x\sin^2}- \integ{\cos x\sin^4 x} = \frac{\sin^3}{3}-\frac{\sin^5 x}{5} + c\]
    \item $sin^n x\cdot cos^m x$, ha $n$ és $m$ páros, lásd b). Példa:
    \[\integ{\sin^2 x\cos^4 x} = \integ{(1-\cos^2 x)\cos^4 x} = \integ{\cos^4 x}-\integ{\cos^6 x} = \]
    \[ = \integ{\frac{(1+\cos 2x)^2}{4}}-\integ{\frac{(1+\cos 2x)^3}{8}} = \frac{1}{4}\integ{\left(1+\cos^2 2x+2\cos 2x\right)}-\]
    \[-\frac{1}{8}\integ{1+\cos^3 2x+3\cos^2 2x+3\cos 2x} = \frac{1}{8}\integ{\left(1-\cos^2 2x+\cos 2x-\cos^3 2x\right)} =\]
    \[= \frac{x}{8}-\frac{1}{8\cdot 2}\integ{(1+\cos 4x)}+\frac{1}{8\cdot 2}\sin 2x-\frac{1}{8}\integ{\cos 2x(1-\sin^2 2x)} = \]
    \[= \frac{x}{8}-\frac{x}{16}-\frac{1}{16\cdot 4}\sin 4x+\frac{1}{8\cdot 2}\sin 2x-\frac{1}{8\cdot 2}\sin 2x+\frac{1}{8\cdot 3\cdot 2}\sin^3 2x + c = \]
    \[= \frac{x}{16} -\frac{1}{64}\sin 4x+\frac{1}{48}\sin^3 2x + c\]
  \end{list}
 \item $\boxed{$Parciális integrálás$}$\\[+3pt]
  \[\boxed{\integ{u(x)\cdot v'(x)} = u(x)\cdot v(x)-\integ{u'(x)\cdot v(x)}}\]
  Azért gyorsan nézzük meg, hogy ez honnan jött ki:
    \[(u\cdot v)' = u'\cdot v+u\cdot v'\]
    \[u\cdot v' = (u\cdot v)'-u'\cdot v \qquad /\textstyle\int\]
    \[\integ{u\cdot v'} = u\cdot v-\integ{u'\cdot v}\]
  Mikor célszerű használni? Ha $v(x)$-nek ismert az integráltja és $u(x)$-nek a deriváltja. Példák:
  \begin{list}{\alph{subcount})}{\usecounter{subcount}\leftmargin=1.5em}
    \item $\displaystyle \integ{\underbrace{\hbox{polinom}}_{u}\cdot\underbrace{\begin{cases}\hbox{exp.}\\\hbox{trig.}\\\hbox{hiper.}\end{cases}}_{v'}}$
      \[\integ{(x^2+x+1)\sh 2x} = (x^2+x+1)\frac{\ch 2x}{2}-\integ{(2x+1)\frac{\ch 2x}{2}} = \]
      \[(x^2+x+1)\frac{\ch 2x}{2} - (2x+1)\frac{\sh 2x}{4}+\integ{2\frac{\sh 2x}{4}} = \]
      \[ = (2x^2+2x+3)\frac{\ch 2x}{4} - (2x+1)\frac{\sh 2x}{4} + c\]
    \item $\displaystyle \integ{\underbrace{\hbox{polinom}}_{v'}\cdot\underbrace{\begin{cases}\log\\\arsh, \arch\\\arcsin, \arccos\end{cases}}_{u}}$
    \[\integ{\ln x} = \integ{1\cdot \ln x} = x\ln x - \integ{x\cdot \frac{1}{x}} = x(\ln x - 1) + c\]
    \[\integ{x\cdot \arctg x} = \frac{x^2}{2}\cdot\arctg x - \integ{\frac{x^2}{2}\cdot\frac{1}{1+x^2}} = \frac{x^2}{2}\cdot\arctg x - \frac{1}{2}\integ{\frac{x^2+1-1}{1+x^2}} = \]
    \[= \frac{x^2}{2}\cdot\arctg x - \frac{1}{2}\integ{\left(1-\frac{1}{1+x^2}\right)} = \frac{x^2}{2}\cdot\arctg x - \frac{x}{2}+\frac{1}{2}\arctg x + c\]
    \item $\displaystyle \integ{\begin{cases}e^x\\\sh, \ch\\\sin, \cos\end{cases}\cdot\begin{cases}e^x\\\sh, \ch\\\sin, \cos\end{cases}}$
    \[\integ{e^{3x}\cdot \sh(2x)} = e^{3x}\cdot \frac{\ch(2x)}{2}-\integ{3\cdot e^{3x}\cdot \frac{\ch(2x)}{2}} = \]
    \[ e^{3x}\cdot \frac{\ch(2x)}{2}-\frac{3}{2}\left(e^{3x}\cdot\frac{\sh(2x)}{2}-\integ{3\cdot e^{3x}\cdot\frac{\sh(2x)}{2}} \right)\]
    Tehát azt kaptuk, hogy:
    \[\integ{e^{3x}\cdot \sh(2x)} = e^{3x}\cdot \frac{\ch(2x)}{2}-\frac{3}{4}e^{3x}\cdot\sh(2x) +\frac{9}{4}\integ{ e^{3x}\cdot\sh(2x)}\]
    \[\integ{e^{3x}\cdot \sh(2x)} = -\frac{4}{5}\left(e^{3x}\cdot \frac{\ch(2x)}{2}-\frac{3}{4}e^{3x}\cdot\sh(2x)\right) + c\]
    \[\integ{e^{3x}\cdot \sh(2x)} = -\frac{2}{5}\cdot e^{3x}\cdot\ch(2x)+\frac{3}{5}\cdot e^{3x}\cdot\sh(2x) + c\]
  \end{list}
 \item $\boxed{$Racionális törtfüggvények integrálása$}$
  \begin{defi}
   Racionális törtfüggvény a $\displaystyle \frac{p(x)}{q(x)}$, ahol $p(x)$, $q(x)$ polinomok. Valódi racionális törtről beszélünk, ha a számláló foka kisebb, mint a nevezőé ($\deg p(x) < \deg q(x)$).
  \end{defi}
  Lépések:
  \begin{list}{\roman{subcount})}{\usecounter{subcount}\leftmargin=1.5em}
    \item Ha nem valódi a racionális tört, akkor felírjuk polinom + valódi rac. tört alakban. Ekkor a polinomot a megszokott módon integráljuk a valódi racionális törtet pedig a következő lépés alapján.
      \[\frac{p_n(x)}{q_m(x)} = r_k(x) + \frac{\tilde{p}_{\tilde{n}}(x)}{q_m(x)}\]
    \item Ha valódi racionális törtünk van, akkor a nevezőt ($q(x)$-et) valós gyöktényezők (valós gyök: $(x-\alpha)$ vagy ha komplex gyökpár, akkor: $(x^2+ax+b)$) szorzatára bontjuk.
    \item Parciális törtekre való bontás, esetek:
    \begin{list}{\alph{subsubcount})}{\usecounter{subsubcount}\leftmargin=1.5em}
      \item $n$ darab csupa egyszeres valós gyöktényező van, pl: $q(x)=(x-\alpha_1)(x-\alpha_2)\cdot\ldots\cdot(x-\alpha_n)$, $\alpha_i \neq \alpha_j$, ha $i\neq j$.
      \[\frac{p(x)}{q(x)} = \frac{A_1}{x-\alpha_1}+\frac{A_2}{x-\alpha_2}+\ldots+\frac{A_n}{x-\alpha_n} \qquad A_1,\ldots,A_n\in\mathbb{R}\]
      \item Ha $(x-\alpha)$ $n$-szeres valós gyöke $q(x)$-nek; $q(x)=(x-\alpha)^n\cdot\ldots$
      \[\frac{p(x)}{q(x)} = \frac{B_1}{x-\alpha}+\frac{B_2}{(x-\alpha)^2}+\frac{B_3}{(x-\alpha)^3}+\ldots+\frac{B_n}{(x-\alpha)^n}+\ldots\]
      \item Ha $(x^2+Ax+B)$ egyszeres, másodfokú tovább nem bontható gyöktényező:
      \[\frac{p(x)}{q(x)} = \frac{Cx+D}{x^2+Ax+B}+\ldots\]
      \item (*) Többszörös komplex gyökpár $q(x)=(x^2+Ax+B)^n\cdot\ldots$
      \[\frac{p(x)}{q(x)} = \frac{C_1x+D_1}{x^2+Ax+B}+\frac{C_2x+D_2}{(x^2+Ax+B)^2}+\ldots+\frac{C_nx+D_n}{(x^2+Ax+B)^n}+\ldots\]
    \end{list}
  \end{list}
  \begin{tetel}
   Az így kapható parciális tört felbontás egyértelmű.
  \end{tetel}
  \begin{biz}
    $\emptyset$. Megjegyzés: Legyen $\deg q = n$. Ekkor $\deg q \leqslant n-1$, tehát $n$ darab együttható ($0, 1, \ldots, n-1$) és a parciális tört felbontásban is $n$ darab szabad paraméter van.
  \end{biz}
  Láttuk, hogy hogy lehet a valódi racionális törtet parciális törtekre bontani, ezek után már csak ezeket kell a tanult szabályok alapján integrálni. Nézzünk egy példát:
   \[\integ{\frac{x^4+2x^2-1}{x^4-1}} = \integ{\left(1+\frac{2x^2}{x^4-1}\right)} = x+\integ{\frac{2x^2}{x^4-1}}\]
  Tehát először a nem valódiból valódi törtet csináltunk, majd ennek a nevezőjének keressük a gyökeit:
  \[(x^4-1)=(x^2+1)(x^2-1)= (x^2+1)(x+1)(x-1)\]
  \[\frac{2x^2}{x^4-1} = \frac{Ax+B}{x^2+1}+\frac{C}{x+1}+\frac{D}{x-1}\]
  Határozzuk meg $A, B, C$ és $D$ értékét:
  \[2x^2 = (Ax+B)(x^2-1)+C(x-1)(x^2+1)+D(x+1)(x^2+1)\]
  \[2x^2 = x^3(A+C+D)+x^2(B-C+D)+x(-A+C+D)+(-B-C+D)\]
  \[\left.\begin{array}{rrrrcc}
    A&&+C&+D&=&0\\
    &+B&-C&+D&=&2\\
    -A&&+C&+D&=&0\\
    &-B&-C&+D&=&0
  \end{array}\right\}\]
  \[\left.\begin{array}{rcccc}2C&+&2D &=& 0\\
    -2C&+&2D &=& 2\end{array}\right\} \quad \Rightarrow\quad D = \frac{1}{2};\; C = -\frac{1}{2} \quad\Rightarrow\quad A=0;\; B=1\]
  Tehát:
  \[\integ{\frac{x^4+2x^2-1}{x^4-1}} = x+\integ{\frac{1}{x^2+1}}-\integ{\frac{1}{2x+2}}+\integ{\frac{1}{2x-2}} = \]
  \[= x+\arctg x-\frac{1}{2}\ln|x+1|+\frac{1}{2}\ln|x-1|+c\]
  Másik példa:
  \[\integ{\frac{3x+1}{x^2-2x+5}} = ?\]
  Itt a nevező nem bontható tovább, ez már tulajdonképpen 1 darab parciális tört. $(x^2-2x+3)' = 2x-2$, tehát ezt kéne belecsempészni a számlálóba:
  \[\integ{\frac{3x+1}{x^2-2x+5}} = \integ{\frac{\frac{3}{2}(2x-2)+4}{x^2-2x+5}} = \integ{\frac{\frac{3}{2}(2x-2)}{x^2-2x+5}} + 4\cdot\integ{\frac{1}{(x-1)^2+4}} = \]
  \[\frac{3}{2}\cdot\ln(x^2-2x+5)+4\cdot\frac{1}{4}\cdot\integ{\frac{1}{\left(\frac{x-1}{2}\right)^2+1}} = \frac{3}{2}\cdot\ln(x^2-2x+5)+2\arctg\left(\frac{x-1}{2}\right)+c\]
\end{list}

\section{Határozott integrál (Riemann-integrál)}

Motiváció: Legyen $[a,b]\subset\mathbb{R}$ \textbf{korlátos intevallum} és legyen $f:[a,b]\to\mathbb{R}$ valós, \textbf{korlátos függvény}. Szeretnénk az $[a,b]$ intervallum függvény alatti területét kiszámolni:

\begin{center}
\psset{xunit=1.0cm,yunit=1.0cm,algebraic=true,dotstyle=o,dotsize=3pt 0,linewidth=0.8pt,arrowsize=3pt 2,arrowinset=0.25}
\begin{pspicture*}(-0.44,-0.96)(5.5,4.3)
\psaxes[xAxis=true,yAxis=true,labels=none,Dx=1,Dy=1,ticksize=-2pt 0,subticks=2]{->}(0,0)(-0.44,-0.96)(5.5,4.3)
\pscustom[fillstyle=hlines,hatchwidth=0.5pt,hatchsep=8pt]{\psplot{2}{4}{0.00665*x^4-0.06438*x^3-0.01052*x^2+1.31665*x+0.942082}\lineto(4,0)\lineto(2,0)\closepath}
\psplot[plotpoints=200]{0.3}{5}{0.00665*x^4-0.06438*x^3-0.01052*x^2+1.31665*x+0.942082}
\rput(3,1.8){\psframebox*{T}}
\rput[b](2,-0.50){$a$}
\rput[b](4,-0.50){$b$}
\rput[b](0.7,2){$f$}
\end{pspicture*}
\end{center}

\begin{defi}
 $[a,b]$ egy \textbf{felosztása} $F_n=\{I_k\}^n_{k=1};$ $I_k = [x_{k-1}, x_k]$. Véges sok osztópont:
  \[a = x_0 < x_1 < \ldots < x_{n-1} < x_n = b\]
\end{defi}

Legyen $m_k = \underset{x\in I_k}{\Inf}\big\{f(x)\big\}$ (ezek léteznek, hiszen $f$ korlátos).\\
Legyen $M_k = \underset{x\in I_k}{\Sup}\big\{f(x)\big\}$.\\
Ekkor az $F$ felosztáshoz tartozó alsó ($s_F$) ill. felső ($S_F$) közelítő összeg:
\[s_F = \sum^{n}_{k=1} m_k\cdot(x_k-x_{k-1}) = \sum^{n}_{k=1} m_k\cdot|I_k|\]
\[S_F = \sum^{n}_{k=1} M_k\cdot(x_k-x_{k-1}) = \sum^{n}_{k=1} M_k\cdot|I_k|\]
Ezeket az ábrán jelölve (tetején sráffal jelölt téglalapok területeinek összege):

\begin{center}
\psset{xunit=1.0cm,yunit=1.0cm,algebraic=true,dotstyle=o,dotsize=3pt 0,linewidth=0.8pt,arrowsize=3pt 2,arrowinset=0.25}
\begin{pspicture*}(-0.44,-0.96)(6.92,4.3)
\psaxes[xAxis=true,yAxis=true,labels=none,Dx=1,Dy=1,ticksize=-2pt 0,subticks=2]{->}(0,0)(-0.44,-0.96)(6.92,4.3)
\psplot[linewidth=1.2pt,plotpoints=200]{-0.44}{6.92}{0.01561*x^4 - 0.1861*x^3 + 0.48715*x^2 + 0.60522*x + 1.21954}

\pspolygon[linestyle=none,fillstyle=hlines](1.25,1.55)(1.25,1.35)(0.42,1.35)(0.42,1.55)
\pspolygon[linestyle=none,fillstyle=vlines](1.25,2.21)(1.25,2.41)(0.42,2.41)(0.42,2.21)
\pspolygon[linestyle=none,fillstyle=hlines](1.25,2.21)(2,2.21)(2,2.41)(1.25,2.41)
\pspolygon[linestyle=none,fillstyle=vlines](2,2.94)(1.25,2.94)(1.25,3.14)(2,3.14)
\pspolygon[linestyle=none,fillstyle=hlines](3,3.14)(3,2.94)(2,2.94)(2,3.14)
\pspolygon[linestyle=none,fillstyle=vlines](2,3.46)(3,3.46)(3,3.66)(2,3.66)
\pspolygon[linestyle=none,fillstyle=hlines](4.42,2.76)(4.42,2.56)(5.24,2.56)(5.24,2.76)
\pspolygon[linestyle=none,fillstyle=vlines](4.42,3.1)(5.24,3.1)(5.24,3.3)(4.42,3.3)
\pspolygon[linestyle=none,fillstyle=hlines](6,2.42)(6,2.22)(5.24,2.22)(5.24,2.42)
\pspolygon[linestyle=none,fillstyle=vlines](6,2.76)(6,2.56)(5.24,2.56)(5.24,2.76)

\psline(3,0)(3,3.66)
\psline(2,0)(2,3.66)
\psline(2,3.14)(3,3.14)
\psline(2,3.66)(3,3.66)
\psline(1.25,3.14)(1.25,0)
\psline(0.42,2.41)(0.42,0)
\psline(0.42,1.55)(1.25,1.55)
\psline(1.25,2.41)(2,2.41)
\psline(2,3.14)(1.25,3.14)
\psline(1.25,2.41)(0.42,2.41)
\psline(6,0)(6,2.76)
\psline(5.24,0)(5.24,3.3)
\psline(4.42,0)(4.42,3.3)
\psline(4.42,3.3)(5.24,3.3)
\psline(5.24,2.76)(4.42,2.76)
\psline(6,2.76)(5.24,2.76)
\psline(6,2.42)(5.24,2.42)
\psdots[dotstyle=*](7.54,4.18)
\psdots[dotstyle=*](0.42,0)
\psdots[dotstyle=*](1.25,0)
\psdots[dotstyle=*](2,0)
\psdots[dotstyle=*](3,0)
\psdots[dotstyle=*](4.42,0)
\psdots[dotstyle=*](5.24,0)
\psdots[dotstyle=*](6,0)
\rput[bl](-0.34,0.55){$f$}
\rput[b](0.42,-0.50){$x_0$}
\rput[b](1.25,-0.50){$x_1$}
\rput[b](2,-0.50){$x_2$}
\rput[b](3,-0.50){$x_3$}
\rput[b](3.6,-0.50){$\ldots$}
\rput[b](4.42,-0.50){$x_{n-2}$}
\rput[b](5.24,-0.90){$x_{n-1}$}
\rput[b](6,-0.50){$x_{n}$}
\end{pspicture*}
\end{center}

\begin{tetel}\label{AlsoKisebbMintFelsoKozOsszeg}
 $s_f \leqslant S_f$, hiszen $m_k \leqslant M_k$. $\quad \checkmark$
\end{tetel}
\addtocounter{biz}{1}

\begin{tetel}\label{FelosztasFinomitasa}
 Legyen $F^*$ az $F$ felosztás egy új osztóponttal vett finomítása. Ekkor $s_{F^*} \geqslant s_F$ és $S_{F^*} \leqslant S_F$.
\end{tetel}
\begin{biz}
 Legyen az új $x^*$ osztópont $x_k$ és $x_{k+1}$ között. Ekkor:
  \[\left.\begin{array}{r}
    m_k' = \underset{x\in [x_k, x^*]}{\Inf}\big\{f(x)\big\}\\\\[-5pt]
    m_k'' = \underset{x\in [x^*, x_{k+1}]}{\Inf}\big\{f(x)\big\}
   \end{array}\right\} \begin{array}{l}
     m_k' \geqslant m_k\\\\[-5pt]
     m_k'' \geqslant m_k
   \end{array}
\]
Hasonlóan $M_k' \leqslant M_k$, illetve $M_k'' \leqslant M_k$. Tehát igaz az állítás.
\end{biz}

\begin{tetel}
 Legyen $F_1$, $F_2$ két véges felosztás. Ekkor:
  \[s_{F_1} \leqslant S_{F_2}; \qquad s_{F_2} \leqslant S_{F_1}\]
\end{tetel}
\begin{bizNoNL}
 \[s_{F_1} \leqslant s_{F_1\cup F_2} \leqslant S_{F_1\cup F_2} \leqslant S_{F_2}\]
Először \aref{FelosztasFinomitasa}. tételt (felosztás finomítása) használjuk, majd \aref{AlsoKisebbMintFelsoKozOsszeg}. tételt, és majd megint \aref{FelosztasFinomitasa}-et.
\end{bizNoNL}

\begin{defi}
 Legyen $F$ véges felosztás, \textbf{Darbaux-féle alsó integrál}:
  \[h := \Sup s_F = \int^{b}_{\underline{x=a}} f(x)\; dx\]
  Ha $[a,b]$ véges és $f$ ezen korlátos, akkor $\exists h$.
\end{defi}
\begin{defi}
 Legyen $F$ véges felosztás, \textbf{Darbaux-féle felső integrál}:
  \[H := \Inf S_F = \int^{\overline{b}}_{x=a} f(x)\; dx\]
  Ha $[a,b]$ véges és $f$ ezen korlátos, akkor $\exists H$.
\end{defi}

\begin{defi}
 Az $f$ korlátos függvény az $[a,b]$ korlátos intervallumon \textbf{Riemann-szerint integrálható}, ha $h=H$ és ekkor $h = H = \displaystyle \int^{b}_{x=a} f(x)\; dx$. Jelölés: $f\in R[a,b]$ (Megjegyzés: $R[a,b]$ az $[a,b]$ korlátos intervallumon Riemann-szerint integrálható függvények halmaza).
\end{defi}

\textbf{Példák a definíció alkalmazására}:\\

\emph{Példa 1}: Legyen $f(x)\equiv c\in\mathbb{R}$ ($\rightsquigarrow m_k = \underset{x\in I_k}{\Inf}\big\{f(x)\big\} = M_k = \underset{x\in I_k}{\Sup}\big\{f(x)\big\}=c$). Legyen $F$ felosztása $[a,b]$-nek.
\[s_F = \sum^{n}_{k=1} m_k(x_k-x_{k-1})= c\sum^{n}_{k=1} |I_k| = c(b-a)\]
\[S_F = \sum^{n}_{k=1} M_k(x_k-x_{k-1})=c(b-a)\]
Ebből következően nyílván $h=H$, tehát:
\[\hatInteg{a}{b}{c} = c(b-a)\]

\emph{Példa 2}: Dirichlet-függvény:
\[D(x) = \begin{cases}1, \hbox{ ha } x\in \mathbb{Q}\\ 0, \hbox{ ha }x\in \mathbb{R}\setminus\mathbb{Q}\end{cases}\]
Tetszőleges $F$ felosztásra:
\[\left.\begin{array}{r}
    s_F = \displaystyle\sum^{n}_{k=1} \underbrace{m_k}_{0}|I_k| = 0\\
    S_F = \displaystyle\sum^{n}_{k=1} \underbrace{M_k}_{1}|I_k| = b-a
   \end{array}\right\} \left.\begin{array}{l}
    h = \Sup s_F = 0\\
    H = \Inf S_F = b-a
   \end{array}\right\} \quad h \neq H\]
Tehát $D(x)$ Riemann-szerint \textbf{nem} integrálható!

Láthatjuk, hogy néhány esetben kézenfekvő a definíció használata, de általánosabb esetekben nehézkes. De ennek kiküszöbölésére a következő tétel segítséget nyújt, ami összekapcsolja a határozott integrált a határozotlan ingtegrállal:

\begin{tetel} $\boxed{$Newton-Leibniz formula$}$\\[+2pt]
 Ha $f$ Riemann-integrálható $[a,b]$-n és $F$ az $f$ primitív függvénye az $[a,b]$-n, akkor:
  \[\hatInteg{a}{b}{f(x)} = \Big[F(x)\Big]^{b}_{x=a} = F(b)-F(a)\]
\end{tetel}
\begin{biz}
 Majd később visszatérünk rá.
\end{biz}

\begin{defi}
 $F_n$ felosztás \textbf{finomsága}: $\Delta F_n=\underset{k=1,\ldots,n}{\max}\{(x_k-x_{k-1})\}$.
\end{defi}
\begin{defi}
 \textbf{Minden határon túl finomodó felosztás sorozat} (röv: m.h.t.f.f.s.):
  \[\{F_n\}_{n\in\mathbb{N}};\qquad \Delta F_n \xrightarrow{n\to\infty} 0\]
 Például: $n$ osztópontos egyenletes felosztás.
\end{defi}

\subsection{Riemann-integrálhatóság szükséges és elégséges feltételei}

\begin{tetel}(Segéd tétel)\\
 Ha $F_n$ m.h.t.f.f.s. $[a,b]$-n, akkor
  \[\exists \lim_{n\to\infty} s_{F_n}(f) = h(f); \quad \exists \lim_{n\to\infty} S_{F_n}(f) = H(f)\]
\end{tetel}\addtocounter{biz}{1}

\begin{tetelAbraval}
 \begin{enumerate*}
  \item Ha $\displaystyle\exists\hatInteg{a}{b}{f(x)} \quad \Rightarrow \quad \forall F_n$ m.h.t.f.f.s. esetén $\displaystyle\exists \lim_{n\to\infty} s_{F_n}(f)=\lim_{n\to\infty} S_{F_n}(f) = I = \hatInteg{a}{b}{f(x)}$.
   \item Ha $\exists F_n$ m.h.t.f.f.s., hogy $\displaystyle\lim_{n\to\infty} s_{F_n}(f)=\lim_{n\to\infty} S_{F_n}(f) \quad \Rightarrow \quad$ $f\in R[a,b]$ és $\displaystyle\hatInteg{a}{b}{f(x)}=\lim_{n\to\infty} s_{F_n}(f)=\lim_{n\to\infty} S_{F_n}(f)$.
 \end{enumerate*}
\end{tetelAbraval}
\begin{bizAbraval}
 \begin{enumerate*}
  \item Az előző tétel alapján $\exists h, H$, de mivel $\displaystyle\exists\hatInteg{a}{b}{f(x)}$, ezért $h=H=I$.
  \item Mivel $h=H$, ezért definíció alapján $f\in R[a,b]$.
 \end{enumerate*}
\end{bizAbraval}

\begin{defi}
 \textbf{Oszcillációs összeg}:
\[O_{F_n}(f) = S_{F_n}(f)-s_{F_n}(f)=\sum^{n}_{k=1}(\underbrace{M_k - m_k}_{\geqslant 0})(\underbrace{x_k-x_{k-1}}_{\geqslant 0}) \geqslant 0\]
\end{defi}

\begin{tetel}\label{OszcillaciosOsszegTart0}
 \[\exists\hatInteg{a}{b}{f(x)} \quad \Leftrightarrow \quad \forall\varepsilon>0 \hbox{ esetén } \exists F \hbox{ felosztás, hogy } O_F(f)<\varepsilon\]
\end{tetel}
\begin{biz}
 \begin{description*}
  \item[($\Rightarrow$)] $h=H$, tehát $\exists F_1: I-\dfrac{\varepsilon}{2} < s_{F_1}$, $\exists F_2: I+\dfrac{\varepsilon}{2} > S_{F_2}$. Vegyük $F_1\cup F_2$ felosztást, ekkor ugye: $s_{F_1\cup F_2} \geqslant s_{F_1}$, illetve $S_{F_1\cup F_2} \leqslant S_{F_2}$. Tehát:
  \[O_{F_1\cup F_2}(f) = S_{F_1\cup F_2} - s_{F_1\cup F_2} < S_{F_2} - s_{F_1} < \varepsilon\]
   \item[($\Leftarrow$)] $\forall \varepsilon > 0$-ra:
    \[H-h \leqslant S_F - s_f = O_f < \varepsilon\]
    Tehát $H-h=0 \quad \Rightarrow \quad H=h=I$.
 \end{description*}
\end{biz}

Tekintsük az $F_n$ felosztását $[a,b]$-nek, és vegyünk fel minden $[x_{k-1},x_k]$-n ($k=1,\ldots,n$) egy $\xi_k$ reprezentáns pontot.

\begin{defi}
 \textbf{Integrál közelítő összeg}:
  \[\sigma_{F_n}(f) = \sum^{n}_{k=1} \underbrace{\overbrace{f(\xi_k)}^{\leqslant M_k}}_{\geqslant m_k}(x_k-x_{k-1})\]
  \[s_{F_n}(f) \leqslant \sigma_{F_n}(f) \leqslant S_{F_n}(f)\]
\end{defi}

\begin{tetelAbraval}
 \begin{enumerate*}
  \item Ha $\displaystyle\exists\hatInteg{a}{b}{f(x)} \quad \Rightarrow \quad \forall F_n$ m.h.t.f.f.s. esetén $\displaystyle\exists \lim_{n\to\infty} \sigma_{F_n}(f) = \hatInteg{a}{b}{f(x)}$.
   \item Ha $\exists F_n$ m.h.t.f.f.s., hogy $\displaystyle\lim_{n\to\infty} \sigma_{F_n}(f)$ a $\{\xi_k\}$ reprezentáns pontok választásától függetlenül $\quad \Rightarrow \quad$ $f\in R[a,b]$ és $\displaystyle\hatInteg{a}{b}{f(x)}=\lim_{n\to\infty} \sigma_{F_n}(f)$.
 \end{enumerate*}
\end{tetelAbraval}
\begin{bizAbraval}
 \begin{enumerate*}
  \item Rendőr-elv alapján:
    \[\underbrace{s_{F_n}}_{\to h} \leqslant \sigma_{F_n} \leqslant \underbrace{S_{F_n}}_{\to H}\]
    De mivel $\displaystyle\exists\hatInteg{a}{b}{f(x)}$, ezért $h=H=I$, tehát $\sigma_{F_n}=I$.
  \item $\emptyset$
 \end{enumerate*}
 \vspace{-30pt}
\end{bizAbraval}

\subsection{Elégséges feltételek a Riemann-integrálhatóságra}

\begin{tetel}
 Ha $f$ monoton az $[a,b]\subset\mathbb{R}$ intervallumon, akkor $f\in R[a,b]$.
\end{tetel}
\begin{biz}
 Tfh. $f$ monoton nő. Legyen $F_n$ az $[a,b]$ intervallumon való egyenletes felosztás: $|I_k|=x_k-x_{k-1}=\dfrac{b-a}{n}$. Ekkor:
\[O_{F_n}(f) = \sum^{n}_{k=1} (M_k - m_k)(x_k-x_{k-1}) = \dfrac{b-a}{n}\sum^{n}_{k=1} (M_k - m_k) = \dfrac{b-a}{n}\sum^{n}_{k=1} (f(x_k) - f(x_{k-1}))\]
Ha jól meggondoljuk ez egy teleszkópikus összeg, amiből azt kapjuk, hogy:
\[O_{F_n}(f) = \dfrac{b-a}{n}\sum^{n}_{k=1} (f(x_k) - f(x_{k-1})) = \dfrac{b-a}{n}\cdot(f(b)-f(a)) \xrightarrow{n\to \infty} 0 \]
\Aref{OszcillaciosOsszegTart0}. tételnél láthattuk, hogy $O_{F_n}(f) \to 0 \Leftrightarrow f\in R[a,b]$.
\end{biz}

\begin{tetel}
 Ha $f\in C[a,b]$ ($\overset{W. I.}{\rightsquigarrow} f$ korlátos $[a,b]$-n), akkor $f\in R[a,b]$.
\end{tetel}
\begin{biz}
 Weierstrass II. tétele alapján (\ref{W.II}. tétel; \pageref{W.II}. oldal) a függvény felveszi szélsőértékeit a kérdéses intervallumokon ($I_k$-n $\xi_k$ helyen veszi fel maximumát, $\eta_k$ helyen a minimumát), tehát:
 \[O_{F_n}(f) = \sum^{n}_{k=1} (M_k - m_k)(x_k-x_{k-1}) = \sum^{n}_{k=1} \Big(f(\xi_k) - f(\eta_k)\Big)(x_k-x_{k-1})\]
 Zárt intervallumon folytonos függvény egyenletesen folytonos, tehát:
  \[|f(\xi_k) - f(\eta_k)| < \varepsilon, \quad \hbox{ ha } \quad |\xi_k - \eta_k|<\delta(\varepsilon)\]
 Adott $\varepsilon>0:\exists \delta(\varepsilon) > 0$. Legyen $\Delta F_n < \delta(\varepsilon)$, ekkor
  \[\xi_k-\eta_k < \delta(\varepsilon) \quad \Rightarrow \quad |f(\xi_k)-f(\eta_k)| < \varepsilon\]
 Tehát:
  \[O_{F_n} \leqslant \varepsilon \sum^{n}_{k=1}(x_k-x_{k-1}) = \varepsilon\cdot(b-a) \xrightarrow{\varepsilon\to 0} 0\]
\end{biz}

\begin{tetel}
 Ha $f$ egy pont kivételével folytonos $[a,b]$-n és $f$ korlátos, akkor $f\in R[a,b]$.
\end{tetel}
\begin{biz}
 Legyen $x^*\in [a,b]$ a szakadási pont, illetve $x^*\in (x_{k-1}, x_k)$. Legyen $O_I$ az $[a,x_{k-1}]$ intervallumon, $O_{II}$ pedig az $[x_k, b]$ intervallumon vett oszcillációs összeg. Legyen $|f(x^*)|<K$. Ekkor adott $\varepsilon>0$-hoz válasszuk úgy az $F$ felosztás finomságát, hogy $O_I < \frac{\varepsilon}{3}$, $O_{II} < \frac{\varepsilon}{3}$, az $[x_{k-1}, x_{k}]$ intervallumhoz tartozó oszcillációs összeg helyett vegyünk egy intervallum szélességnyi $2K$ magas téglalap területét. Legyen $x_k-x_{k-1}<\frac{\varepsilon}{6K}$, ekkor:
  \[O_F = O_I+O_{II}+(x_{k}-x_{k-1})2K < \frac{\varepsilon}{3}+\frac{\varepsilon}{3}+\frac{\varepsilon}{3} = \varepsilon\]
\end{biz}

\begin{tetel}
 Ha $f$ véges pont kivételével folytonos $[a,b]$-n és $f$ korlátos, akkor $f\in R[a,b]$.
\end{tetel}
\addtocounter{biz}{1}

\begin{tetel}
 Ha $f\in R[a,b]$ és $f(x)=g(x)$ véges sok $x$ kivételével, akkor $g\in R[a,b]$ és $\hatInteg{a}{b}{f(x)}=\hatInteg{a}{b}{g(x)}$
\end{tetel}
\addtocounter{biz}{1}

\begin{tetel} $\boxed{$Newton-Leibniz formula$}$\\[+2pt]
 Ha $f$ Riemann-integrálható $[a,b]$-n és $F$ az $f$ primitív függvénye az $[a,b]$-n, akkor:
  \[\hatInteg{a}{b}{f(x)} = \Big[F(x)\Big]^{b}_{x=a} = F(b)-F(a)\]
\end{tetel}
\begin{biz}
Legyen $\phi_n$ az $[a,b]$ egy m.h.t.f.f.s; osztópontok: $a=x_0<x_1<\ldots<x_n=b$.
 \[F(b)-F(a)= \sum^{n}_{k=1} \Big(F(x_k)-F(x_{k-1})\Big) \overset{\hbox{\small Lagrange-t}}{=} \sum^{n}_{k=1} F'(\xi_k)\cdot(x_k-x_{k-1})\]
 Mivel $F$ az $f$ primitív függvénye $[a,b]$-n, ezért $F'(x)=f(x) \quad \forall x\in[a,b]$:
 \[F(b)-F(a)=\sum^{n}_{k=1} f(\xi_k)\cdot(x_k-x_{k-1}) = \sigma_{\phi_n}(f) \quad \underset{\hbox{\footnotesize m.h.t.f.f.s.}}{\xrightarrow{n\to\infty}} \quad \hatInteg{a}{b}{f(x)}\]
\end{biz}

Például:
\[\hatInteg{0}{\pi}{\sin x} = \Big[-\cos x\Big]^{\pi}_{x=0} = -\cos \pi - (-\cos 0) = 2\]

\subsection{Riemann-integrál tulajdonságai}

Eddig mindig úgy vettük, hogy $a<b$, és $[a,b]$ intervallumon számoltuk ki a határozott integrált. Lehet másik irányba is:
\begin{defi}
 \[\hatInteg{b}{a}{f(x)} := -\hatInteg{a}{b}{f(x)}\]
\end{defi}

Tulajdonságok:
\begin{enumerate*}
 \item Intervallum additív: $\qquad a<c<b$ és $f\in R[a,b]$:
  \[\hatInteg{a}{b}{f(x)} = \hatInteg{a}{c}{f(x)}+\hatInteg{c}{b}{f(x)}\]
 \item Az integrál lineáris vektorteret alkot $R[a,b]$-n, azaz ha $f, g \in R[a,b]$ és $\alpha, \beta \in \mathbb{R}$, akkor $\alpha f+\beta g \in R[a,b]$ és
  \[\hatInteg{a}{b}{(\alpha f + \beta g)(x)} = \alpha\hatInteg{a}{b}{f(x)}+\beta\hatInteg{a}{b}{g(x)}\]
 \item Az integrál monoton: Ha $f(x) \geqslant g(x) \; \forall x\in [a,b]$-re, akkor $\hatInteg{a}{b}{f(x)} \geqslant \hatInteg{a}{b}{g(x)}$.
\end{enumerate*}

\subsection{Az integrálszámítás középértéktétele}

Legyen $f$ korlátos függvény $[a,b]\subset\mathbb{R}$-en, ekkor:
\[(b-a)\underset{x\in[a,b]}{\Inf}\{f(x)\} \leqslant \hatInteg{a}{b}{f(x)} \leqslant (b-a)\underset{x\in[a,b]}{\Sup}\{f(x)\}\]
Hiszen:
\[(b-a)\underset{x\in[a,b]}{\Inf}\{f(x)\} \quad \leqslant \quad \sigma_F(f)=\sum^{n}_{k=1}f(\xi_k)\cdot(x_k-x_{k-1}) \quad \leqslant\quad (b-a)\underset{x\in[a,b]}{\Sup}\{f(x)\}\]
Mivel $\underset{x\in[a,b]}{\Inf}\{f(x)\} \leqslant f(\xi_k) \leqslant \underset{x\in[a,b]}{\Sup}\{f(x)\}$.

\begin{defi}
 \textbf{Integrál közép}:
\[\kappa := \frac{\hatInteg{a}{b}{f(x)}}{b-a}\]
\end{defi}

\begin{tetel}
 \[\underset{x\in[a,b]}{\Inf}\{f(x)\} \leqslant \kappa \leqslant \underset{x\in[a,b]}{\Sup}\{f(x)\}\]
\end{tetel}
\begin{biz}
 A fenti egyenletet leosztva $(b-a)$-val kapjuk az állítást.
\end{biz}

\begin{tetel}
 Ha $f$ folytonos $[a,b]$-n, akkor $\exists \xi \in (a,b)$, hogy $f(\xi)=\kappa$.
\end{tetel}
\begin{biz}
 Lásd Bolzano-tétel (\ref{Bolzano}. tétel; \pageref{Bolzano}. oldal)
\end{biz}

\begin{defi}
 $f$ függvény \textbf{pozitív/negatív} része:
  \[f^+(x) = \begin{cases}f(x), \hbox{ ha } f(x)\geqslant 0\\ 0, \hbox{ ha } f(x)\leqslant 0\end{cases}\]
  \[f^-(x) = \begin{cases}f(x), \hbox{ ha } f(x)\leqslant 0\\ 0, \hbox{ ha } f(x)\geqslant 0\end{cases}\]
\end{defi}

\begin{lemma}
 \[f(x)=f^+(x)+f^-(x)\]
 \[|f(x)|=f^+(x)-f^-(x)\]
\end{lemma}

\begin{tetel}
 Ha $f\in R[a,b]$ ($b>a$), akkor:
\begin{enumerate*}
 \item $f^+, f^-, |f| \in R[a,b]$
 \item $\displaystyle\left|\hatInteg{a}{b}{f(x)}\right| \leqslant \hatInteg{a}{b}{\Big|f(x)\Big|}$
\end{enumerate*}
\end{tetel}
\begin{biz}
 \begin{enumerate*}
 \item \[O_F(f^+) \leqslant O_F(f) < \varepsilon\]
       \[O_F(f^-) \leqslant O_F(f) < \varepsilon\]
      Tehát az $O_F(f)$-hez tartozó $\varepsilon$-hoz jó $\delta(\varepsilon)$ jó az $O_F(f^+)$-hoz és $O_F(f^-)$-hoz is.
  \[\hatInteg{a}{b}{|f(x)|} = \hatInteg{a}{b}{\Big(f^+(x)-f^-(x)\Big)} = \hatInteg{a}{b}{f^+(x)}-\hatInteg{a}{b}{f^-(x)}\]
 \item \[-|f(x)| \leqslant f(x) \leqslant |f(x)|\]
  Mivel az integrál monoton, ezért:
  \[-\hatInteg{a}{b}{|f(x)|} \leqslant \hatInteg{a}{b}{f(x)} \leqslant \hatInteg{a}{b}{|f(x)|}\]
\end{enumerate*}
\end{biz}

\subsection{Integrál függvény}

\begin{defi}
 Legyen $f\in R[a,b]$ ($\rightsquigarrow f\in R[c,d]$, ha $[c,d]\subset [a,b]$). Ekkor $f$ \textbf{integrálfüggvénye}:
  \[F(x) := \intfv{a}{x}{f(t)} \qquad x\in[a,b]\]
\end{defi}

\begin{tetel} $\boxed{$Az integrálszámítás II. alaptétele$}$
 \begin{enumerate*}
  \item $F(x)$ folytonos $[a,b]$-n
  \item Ha $f$ folytonos $x_0\in[a,b]$-ben, akkor $F$ differenciálható $x_0$-ban és $F'(x_0)=f(x_0)$
 \end{enumerate*}
\end{tetel}
\emph{Következmény}: Ha $f\in C[a,b]\subset R[a,b]$ akkor $\forall x\in [a,b]$ esetén $F'(x)=f(x)$ (végpontokban féloldalú folytonosság/derivált).

\begin{biz}
  \begin{enumerate*}
  \item Legyen $x_0\in [a,b]$
  \[|F(x)-F(x_0)| = \left|\intfv{a}{x}{f(t)}-\intfv{a}{x_0}{f(t)}\right| = \]
  \[ = \left|\intfv{a}{x_0}{f(t)}+\intfv{x_0}{x}{f(t)}-\intfv{a}{x_0}{f(t)}\right| = \left|\intfv{x_0}{x}{f(t)}\right| \]
  Mivel $f\in R[a,b]$, ezért $f$ korlátos, tehát $\exists K:|f(t)|\leqslant K$:
  \[|F(x)-F(x_0)| = \left|\intfv{x_0}{x}{f(t)}\right| \leqslant K|x-x_0| < \varepsilon \quad \forall\varepsilon>0 \]
  Ez teljesül, ha $|x-x_0|<\delta(\varepsilon)$, tehát $\delta(\varepsilon) := \dfrac{\varepsilon}{K}$. Tehát folytonos $\checkmark$.
  \item $f$ folytonos $x_0\in [a,b]$-ben.
    \[F'(x_0) = \lim_{h\to 0} \frac{F(x_0+h)-F(x_0)}{h} \overset{?}{=} f(x_0)\]
    \[\left|\frac{F(x_0+h)-F(x_0)}{h}-f(x_0)\right|=\left|\frac{\intfv{x_0}{x_0+h}{f(t)}}{h}-f(x_0)\cdot\frac{h}{h}\right| = \]
    \[= \left|\frac{\intfv{x_0}{x_0+h}{f(t)}}{h}-\frac{\intfv{x_0}{x_0+h}{f(x_0)}}{h}\right| = \left|\frac{\intfv{x_0}{x_0+h}{f(t)-f(x_0)}}{h}\right| = \]
    \[ = \frac{\left|\intfv{x_0}{x_0+h}{f(t)-f(x_0)}\right|}{|h|} \leqslant \frac{\intfv{x_0}{x_0+h}{\left|f(t)-f(x_0)\right|}}{|h|}\]
    Mivel $f$ folytonos $x_0$-ban, ezért $|f(t)-f(x_0)|<\varepsilon$, ha $|t-x_0|<\delta(\varepsilon)$, tehát:
    \[\frac{\intfv{x_0}{x_0+h}{\left|f(t)-f(x_0)\right|}}{|h|} < \frac{\intfv{x_0}{x_0+h}{\varepsilon}}{|h|} = \varepsilon\]
    Vagyis:
     \[\left|\frac{F(x_0+h)-F(x_0)}{h}-f(x_0)\right| < \varepsilon\]
    Tehát $F'(x_0) = f(x_0)$.
 \end{enumerate*}
\vspace{-10pt}
\end{biz}

\subsubsection{Példák}

$F(x) = \intfv{0}{x}{\cos^3(t)}$\\
$G(x) = \intfv{x^2}{e^x}{\cos^3(t)}$\\

$F'(x) = \cos^3 x$, hiszen $\cos^3 x$ folytonos.\\
$G(x) = \intfv{x^2}{e^x}{\cos^3(t)} = \intfv{0}{e^x}{\cos^3(t)} - \intfv{0}{x^2}{\cos^3(t)} = F(e^x)-F(x^2)$.\\
$G'(x) = \Big(F(e^x)-F(x^2)\Big)' = f'(e^x)\cdot e^x - f'(x^2)\cdot 2x = e^x\cdot\cos^3 e^x - 2x\cdot \cos^3 x^2$.

\subsection{Parciális integrálás}

\[\integ{u(x)\cdot v'(x)} = u(x)\cdot v(x)-\integ{u'(x)\cdot v(x)}\]
\[\hatInteg{a}{b}{u(x)\cdot v'(x)} = \Big[u(x)\cdot v(x)\Big]^{b}_{a}-\hatInteg{a}{b}{u'(x)\cdot v(x)}\]

\subsection{Integrálás helyettesítéssel}

\begin{tetel}
Ha $x=\varphi(t)$, akkor:
\[\integDT{f(\varphi(t))\cdot \varphi'(t)} = F(\varphi(t))+c = F(x)+c = \integ{f(x)}\]
ahol $F$ az $f$ egy primitív függvénye, tehát:
\[\integ{f(x)}\Big|_{x=\varphi(t)} = \integDT{f(\underbrace{\varphi(t)}_{x})\cdot\underbrace{\varphi'(t)}_{\frac{dx}{dt}}}\]
\end{tetel}
Feltételek: $\varphi$ folytonosan differenciálható, szig. monoton.
\begin{bizNoNL}
 \[\integ{f(x)} = F(x)+c\]
 \[\integ{f(x)}\Big|_{x=\varphi(t)} = F(\varphi(t))+c\]
 \[\frac{d}{dt}\Big(F(\varphi(t))+c\Big) = F'(\varphi(t))\cdot \varphi'(t) = f(\varphi(t))\cdot \varphi'(t)\]
\end{bizNoNL}

\begin{tetel}
Ha $x=\varphi(t)$, akkor:
\[\hatInteg{a}{b}{f(x)} = \int^{\beta}_{\alpha}f(\varphi(t))\cdot\varphi'(t)\; dt\]
\end{tetel}
\begin{biz}
 \[\hatInteg{a}{b}{f(x)} = \Big[F(x)\Big]^{b}_{a} = F(b)-F(a)\]
 \[\int^{\beta}_{\alpha}f(\varphi(t))\cdot\varphi'(t)\; dt = \Big[F(\varphi(t))\Big]^{\beta}_{\alpha} = F(\overbrace{\varphi(\beta)}^{b})-F(\overbrace{\varphi(\alpha)}^{a})\]
\end{biz}

\subsubsection{Helyettesítéssel megadható integrálok}

\begin{list}{\arabic{integcount}.}{\usecounter{integcount}\leftmargin=1.5em}
 \item $\boxed{\integ{R\left(x, \sqrt{ax^2+bx+c}\right)}}\;$ Módszer: teljes négyzetté alakítunk:
  \[\sqrt{ax^2+bx+c} \rightarrow \begin{cases}
    c\sqrt{1-A^2(x)} \qquad A(x) := \sin t\\
    c\sqrt{1+A^2(x)} \qquad A(x) := \sh t\\
    c\sqrt{A^2(x)-1} \qquad A(x) := \ch t
  \end{cases}\]
  Pl: $\displaystyle \integ{\sqrt{x^2+2x+2}} = \integ{\sqrt{(x+1)^2+1}}$. Legyen $\sh t = x+1$, ekkor $dx = \ch t\cdot dt$
  \[\integ{\sqrt{(x+1)^2+1}} = \integDT{\sqrt{\sh^2 t + 1}\cdot\ch t} = \integDT{\ch^2 t} = \integDT{\left(\frac{e^t+e^{-t}}{2}\right)^2} = \]
  \[= \frac{1}{4}\integDT{ e^{2t} }+\frac{1}{2}\integDT{} +\frac{1}{4}\integDT{ e^{-2t} } =  \frac{e^{2t}}{8}+\frac{t}{2}-\frac{e^{-2t}}{8}+c = \frac{t}{2}+\frac{1}{4}\sh(2t) + c  \]
  Visszatérünk $x$-re:
  \[\frac{t}{2}+\frac{1}{4}\underbrace{\sh(2t)}_{\sh t\ch t} + c = \frac{\arsh(x+1)}{2}+\frac{1}{2}(x+1)\sqrt{x^2+2x+2}+c\]

 \item $\boxed{\integ{R(e^x)}}\;$
  Itt $t := e^x$, ekkor $dx = \dfrac{dt}{t}$.\\
  Pl: $\displaystyle \integ{\frac{1}{e^{2x}-2e^x}} = \integDT{\frac{1}{t^2-2t}\cdot \frac{1}{t}} = \integDT{\frac{1}{t^2(t-2)}}$.
  Ezt parciális törtekre bontjuk:
  \[\integDT{\frac{1}{t^2(t-2)}} = \integDT{\frac{A}{t}}+\integDT{\frac{B}{t^2}}+\integDT{\frac{C}{t-2}}\]
  \[1 = A(t)(t-2)+B(t-2)+C(t^2) = t^2(A+C)+t(-2A+B)-2B\]
  Ebből $B=-\frac{1}{2}$, $A=-\frac{1}{4}$ és $C=\frac{1}{4}$. Tehát:
  \[-\integDT{\frac{1}{4t}}-\integDT{\frac{1}{2t^2}}+\integDT{\frac{1}{4t-8}} = -\frac{1}{4}\ln|t|-\frac{1}{2}\frac{t^{-1}}{-1}+\frac{1}{4}\ln|t-2|+c\]
  Visszatérünk $x$-re:
  \[-\frac{1}{4}\ln|t|-\frac{1}{2}\frac{t^{-1}}{-1}+\frac{1}{4}\ln|t-2|+c = -\frac{1}{4}x+\frac{1}{2e^x}+\frac{1}{4}\ln|e^x-2|+c\]

 \item $\boxed{\integ{R\left(x, \sqrt[n]{ax+b}\right)}}\;$ Ekkor $t := \sqrt[n]{ax+b}$.\\
  Pl: $\displaystyle \integ{x\sqrt{5x+3}}$. Így $t = \sqrt{5x+3}$, $x = \dfrac{t^2-3}{5}$, illetve $dx=\dfrac{2t}{5}dt$. Tehát:
  \[\integ{x\sqrt{5x+3}} = \integDT{\frac{t^2-3}{5}\cdot t\cdot\dfrac{2t}{5}} = \frac{2}{25}\integDT{(t^2-3)t^2} = \frac{2}{25}\left(\integDT{t^4}-\integDT{3t^2}\right) =\]
  \[= \frac{2}{25}\cdot\frac{t^5}{5}-\frac{6}{25}\cdot\frac{t^3}{3}+c = \frac{2}{125}(5x+3)^{\frac{5}{2}}-\frac{2}{25}(5x+3)^{\frac{3}{2}}+c\]

 \item $\boxed{\integ{R\big(x^{\frac{p_1}{q_1}}, x^{\frac{p_2}{q_2}}, \ldots\big)}}\;$ Itt $t := x^\frac{1}{q}$, ahol $q$ a $q_i$ legkisebb közös többszöröse.\\
  Pl: $\displaystyle \integ{\frac{1+x^\frac{3}{2}}{3-x^\frac{1}{3}}}$. Így $t = x^\frac{1}{6}$, $dx = 6t^5 dt$.
  \[6\integ{\frac{1+x^\frac{3}{2}}{3-x^\frac{1}{3}}} = \integDT{\frac{1+t^9}{3-t^2}\cdot 6t^5} = \]
  \[= 6\integDT{\left(-729 - 3 t - 243 t^2 - t^3 - 81 t^4 - 27 t^6 - 9 t^8 - 3 t^{10} - t^{12} + \frac{9 (243 + t)}{3 - t^2}\right)} = \]
  \[= \underbrace{6\Big(-729t - 3\frac{t^2}{2} - 81 t^3 - \frac{t^4}{4} - 81\cdot \frac{t^5}{5} - 27 \frac{t^7}{7} - t^9 - 3\cdot\frac{t^{11}}{11} - \frac{t^{13}}{13}\Big)}_{G} + 6\integDT{\frac{9 (243 + t)}{3 - t^2}} = \]
  \[G + 6\integDT{\frac{(-2t)\cdot(-\frac{9}{2})}{3 - t^2}} + 18\cdot 243\cdot\integ{\frac{1}{1-\left(\frac{t}{\sqrt{3}}\right)^2}} = \]
  \[= G - 27\ln |3 - t^2| + 18\sqrt{3}\cdot 243\cdot \arth\frac{t}{\sqrt{3}} +c \]
  Visszaírva $x$-et:
  \[4374\sqrt[6]{x} - 9\sqrt[3]{x} - 486 \sqrt{x} - \frac{3}{2}\cdot x^\frac{2}{3} - \frac{486}{5}\cdot x^\frac{5}{6} - \frac{162}{7} x^\frac{7}{6} - 6\cdot x^\frac{3}{2} - \frac{18}{11}\cdot x^\frac{11}{6} - \frac{6}{13}x^\frac{13}{6} -\]
  \[-27\ln |3 - \sqrt[3]{x}| + 4374\sqrt{3}\cdot \arth\frac{\sqrt[6]{x}}{\sqrt{3}} +c \]

 \item $\boxed{\integ{R(\sin x, \cos x)}}\;$ Itt $t := \tg \dfrac{x}{2}$, $x= 2\arctg t$ és $dx = \dfrac{2}{1+t^2}dt$
  \[\sin x = \frac{2\sin\frac{x}{2}\cos\frac{x}{2}}{\sin^2 \frac{x}{2}+\cos^2 \frac{x}{2}} = \frac{2\tg\frac{x}{2}}{\tg^2\frac{x}{2}+1} = \frac{2t}{t^2+1}\]
  \[\cos x = \frac{\cos^2\frac{x}{2}-\sin^2\frac{x}{2}}{\sin^2 \frac{x}{2}+\cos^2 \frac{x}{2}} = \frac{1-\tg^2\frac{x}{2}}{\tg^2\frac{x}{2}+1} = \frac{1-t^2}{t^2+1}\]
  Pl: $\displaystyle \integ{\frac{1}{2-\cos x}}$:
  \[\integ{\frac{1}{2-\cos x}} = \integDT{\frac{1}{2-\frac{1-t^2}{t^2+1}}\cdot \dfrac{2}{1+t^2}} = \integDT{\frac{2}{2+2t^2-1+t^2}} = \integDT{\frac{2}{1+3t^2}} = \]
  \[ = 2\integDT{\frac{1}{1+(\sqrt{3}t)^2}} = \frac{2}{\sqrt{3}}\arctg(\sqrt{3}t) + c = \frac{2}{\sqrt{3}}\arctg(\sqrt{3}\tg\frac{x}{2}) + c\]
\end{list}

\section{Improprius integrál}

Eddig korlátos intervallumon korlátos függvényt integráltunk.

\subsection{Ha az intervallum nem korlátos}

\begin{defi} Legyen $f\in R[a,\omega] \quad \forall \omega > a$ esetén, ekkor $\displaystyle \hatInteg{a}{\infty}{f(x)}=\lim_{\omega\to\infty} \hatInteg{a}{\omega}{f(x)}$, ha a limesz létezik.\end{defi}

\begin{defi} Legyen $f\in R[\omega, b] \quad \forall \omega < b$ esetén, ekkor $\displaystyle \hatInteg{-\infty}{b}{f(x)}=\lim_{\omega\to-\infty} \hatInteg{\omega}{b}{f(x)}$, ha a limesz létezik.\end{defi}

\begin{defi} Az $\displaystyle \hatInteg{a}{b}{f(x)}$ improprius integrál konvergens, ha $\forall c\in[a,b]$ esetén
 \[\hatInteg{a}{b}{f(x)} = \hatInteg{a}{c}{f(x)}+\hatInteg{c}{b}{f(x)}\]
\end{defi}

Pl: $\displaystyle \hatInteg{-\infty}{\infty}{x}$ nem konvergens, hiszen $\nexists \hatInteg{0}{\infty}{x}$.

\subsection{Ha a függvény nem korlátos}

\begin{defi}
 Ha $a$-ban nem korlátos, de $\forall\delta>0$ esetén $f\in R[a+\delta, b]$:
  \[\hatInteg{a}{b}{f(x)} = \lim_{\delta\to 0+0} \hatInteg{a+\delta}{b}{f(x)}\]
\end{defi}

\begin{defi}
 Ha $b$-ben nem korlátos, de $\forall\delta>0$ esetén $f\in R[a, b-\delta]$:
  \[\hatInteg{a}{b}{f(x)} = \lim_{\delta\to 0+0} \hatInteg{a}{b-\delta}{f(x)}\]
\end{defi}

\begin{defi}
 Ha $c\in(a,b)$-ben nem korlátos, de $\forall\delta>0$ esetén $f\in R[a, c-\delta]$ és $f\in R[c+\delta, b]$:
  \[\hatInteg{a}{b}{f(x)} = \hatInteg{a}{c}{f(x)} + \hatInteg{c}{b}{f(x)} = \lim_{\delta\to 0+0} \hatInteg{a}{c-\delta}{f(x)} + \lim_{\delta\to 0+0} \hatInteg{c+\delta}{b}{f(x)}\]
\end{defi}

\subsubsection{Példák}

1.
\[\hatInteg{0}{1}{\frac{(\arcsin x)^{\frac{1}{3}}}{\sqrt{1-x^2}}} = \lim_{\delta\to 0+0}\hatInteg{0}{1-\delta}{\frac{(\arcsin x)^{\frac{1}{3}}}{\sqrt{1-x^2}}} = \lim_{\delta\to 0+0}\left[\frac{(\arcsin x)^{\frac{4}{3}}}{\frac{4}{3}}\right]^{1-\delta}_{0} = \]
\[ = \frac{3}{4}\lim_{\delta\to 0+0}(\arcsin^\frac{4}{3} (1-\delta)-\arcsin^\frac{4}{3} 0) = \frac{3}{4}\left(\frac{\pi}{2}\right)^\frac{4}{3}\]\\

2.
\[\hatInteg{3}{7}{\frac{x}{\sqrt{x-3}}} = \lim_{\delta\to 0+0}\hatInteg{3+\delta}{7}{\frac{x}{\sqrt{x-3}}} = \]
Legyen $t=\sqrt{x-3}$, figyeljünk arra, hogy az intervallumokat is transzformáljuk:
\[= \lim_{\delta\to 0+0}\int^{2}_{0+\delta} \frac{t^2+3}{t}\cdot 2t\, dt = 2\lim_{\delta\to 0+0}\int^{2}_{0+\delta} (t^2+3)\, dt = 2\lim_{\delta\to 0+0}\left[\frac{t^3}{3}+3t\right]^{2}_{0+\delta} = \]
\[= 2\lim_{\delta\to 0+0}\left(\frac{8}{3}+6-\frac{\delta^3}{3}+3\delta\right) = \frac{16}{3}+12\]\\

3. Milyen $\alpha$-ra konvergens a $\displaystyle \hatInteg{1}{\infty}{\frac{1}{x^{\alpha}}}$?\\
Ha $\alpha = 1$
\[\hatInteg{1}{\infty}{\frac{1}{x}} = \lim_{\omega\to\infty} (\ln \omega - \ln 1) = \infty, \quad \hbox{tehát divergens}\]
Ha $\alpha \neq 1$
\[\hatInteg{1}{\infty}{x^{-\alpha}} = \lim_{\omega\to\infty} \left[\frac{x^{-\alpha+1}}{-\alpha+1}\right]^\omega_{1} = \lim_{\omega\to\infty} \left(\frac{\omega^{-\alpha+1}}{-\alpha+1}-\frac{1}{-\alpha+1}\right) = \begin{cases}\dfrac{1}{\alpha-1}, \hbox{ ha } \alpha > 1\\\infty, \hbox{ ha } \alpha < 1\end{cases}\]
Tehát $\displaystyle \hatInteg{1}{\infty}{\frac{1}{x^{\alpha}}}$ konvergens, ha $\alpha > 1$, divegens, ha $\alpha \leqslant 1$.\\\\

4. Milyen $\alpha$-ra konvergens a $\displaystyle \hatInteg{0}{1}{\frac{1}{x^{\alpha}}}$?\\
Ha $\alpha = 1$
\[\hatInteg{0}{1}{\frac{1}{x}} = \lim_{\delta\to 0+0} (\ln 1 - \ln \delta) = \infty, \quad \hbox{tehát divergens}\]
Ha $\alpha \neq 1$
\[\hatInteg{0}{1}{x^{-\alpha}} = \lim_{\delta\to 0+0} \left[\frac{x^{-\alpha+1}}{-\alpha+1}\right]^1_{\delta} = \lim_{\delta\to 0+0} \left(\frac{1}{-\alpha+1}-\frac{\delta^{-\alpha+1}}{-\alpha+1}-\right) = \begin{cases}\dfrac{1}{1-\alpha}, \hbox{ ha } \alpha < 1\\\infty, \hbox{ ha } \alpha > 1\end{cases}\]
Tehát $\displaystyle \hatInteg{0}{1}{\frac{1}{x^{\alpha}}}$ konvergens, ha $\alpha < 1$, divegens, ha $\alpha \geqslant 1$.\\\\

5. Milyen $\alpha$-ra konvergens a $\displaystyle \hatInteg{0}{\infty}{\frac{1}{x^{\alpha}}}$?\\
\[\hatInteg{0}{\infty}{\frac{1}{x^{\alpha}}} = \underbrace{\hatInteg{0}{1}{\frac{1}{x^{\alpha}}}}_{\hbox{\small div., ha }\alpha \geqslant 1} + \underbrace{\hatInteg{1}{\infty}{\frac{1}{x^{\alpha}}}}_{\hbox{\small div., ha }\alpha \leqslant 1} \quad \hbox{tehát divergens}\]

\subsection{Improprius integrál néhány tulajdonsága}

\begin{tetel}
 $f\in R[a, \omega] \quad \forall\omega > a$ esetén.\\
 $\displaystyle \hatInteg{a}{\infty}{f(x)}$ improprius integrál konvergens $\Leftrightarrow \forall\varepsilon>0$ esetén $\exists \Omega(\varepsilon)\in\mathbb{R}$, hogy $\displaystyle\left|\hatInteg{\omega_1}{\omega_2}{f(x)}\right|<\varepsilon$, ha $\omega_1, \omega_2, > \Omega$.
\end{tetel}
\addtocounter{biz}{1}

\begin{tetel}
 Legyen $F(x) = \intfv{a}{x}{f(t)}$ integrál függvény. $F$ folytonos $[a,\infty)$-n.\\
 $\displaystyle \hatInteg{a}{\infty}{f(x)} \Leftrightarrow \lim_{x\to\infty} F(x) \overset{\hbox{\small Cauchy}}{\Leftrightarrow} \forall\varepsilon>0$ esetén $\exists \Omega(\varepsilon)\in\mathbb{R}$, hogy $\displaystyle\left|\hatInteg{\omega_1}{\omega_2}{f(x)}\right|=|F(\omega_2)-F(\omega_1)|<\varepsilon$, ha $\omega_1, \omega_2, > \Omega$.
\end{tetel}
\addtocounter{biz}{1}

\begin{defi}
 $\displaystyle \hatInteg{a}{\infty}{f(x)}$ \textbf{abszolút konvergens}, ha $\exists \hatInteg{a}{\infty}{|f(x)|} < \infty$.
\end{defi}
\begin{defi}
 $\displaystyle \hatInteg{a}{\infty}{f(x)}$ \textbf{feltételesen konvergens}, ha konvergens, de nem abszolút konvergens.
\end{defi}

\begin{tetel}
 Ha $\displaystyle \hatInteg{a}{\infty}{f(x)}$ abszolút konvergens, akkor konvergens.
\end{tetel}
\begin{bizNoNL} (Cauchy-kritérium)\\
\[\left|\hatInteg{\omega_1}{\omega_2}{f(x)}\right|\leqslant \hatInteg{\omega_1}{\omega_2}{|f(x)|} <\varepsilon, \hbox{ ha } \omega_1,\omega_2 > \Omega(\varepsilon) \]
\end{bizNoNL}

\subsection{Majoráns kritérium}

\begin{tetel}
 $f\in R[a,b]$, illetve $g\in R[a,b] \quad \forall b> a$ esetén és $|f(x)|\leqslant g(x) \quad \forall x>a$ esetén, ha $\displaystyle \hatInteg{a}{b}{g(x)}$ konvergens, akkor $\displaystyle \hatInteg{a}{b}{|f(x)|}$ is az, illetve
 \[\displaystyle \hatInteg{a}{\infty}{|f(x)|} \leqslant \hatInteg{a}{\infty}{g(x)}\]
\end{tetel}
\begin{bizNoNL} (Cauchy-kritérium)\\
 \[\hatInteg{\omega_1}{\omega_2}{|f(x)|} \leqslant \hatInteg{\omega_1}{\omega_2}{g(x)} < \varepsilon \quad \hbox{ha } \omega_1, \omega_2 > \Omega(\varepsilon)\]
 Teáht $\displaystyle\hatInteg{\omega_1}{\omega_2}{|f(x)|}$ konvergens és az integrál monotonitásából következik, hogy ha $|f(x)|\leqslant g(x)$, akkor:
  \[\hatInteg{a}{\omega}{|f(x)|} \leqslant \hatInteg{a}{\omega}{g(x)} \quad / \lim_{\omega\to\infty}\]
  \[\hatInteg{a}{\infty}{|f(x)|} \leqslant \hatInteg{a}{\infty}{g(x)}\]
\end{bizNoNL}

\subsection{Minoráns kritérium}

\begin{tetel}
 Ha $h$ és $f\in R[a,\omega] \quad \forall \omega > a$ esetén és $\displaystyle \hatInteg{a}{\infty}{h(x)} = \infty$ és $f(x) \geqslant h(x) \;\forall x>a$ esetén akkor $\displaystyle \hatInteg{a}{\infty}{f(x)} = \infty$.
\end{tetel}
\begin{biz}
 Ha $\displaystyle \hatInteg{a}{\infty}{f(x)} < \infty$ lenne, akkor a majoráns kritérium miatt $\hatInteg{a}{\infty}{h(x)} < \infty$ lenne, ami ellentmondás.
\end{biz}

\section{Az integrálás néhány alkalmazása}

\subsection{Terület számítás}

Pl: Ellipszis területe. Egyenlete: $\dfrac{x^2}{a^2}+\dfrac{y^2}{b^2} = 1$. Számoljuk ki a besatírozott területet, és ezt 4-el megszorozva megkaphatjuk a teljes területét.

\begin{wrapfigure}{r}{0.35\textwidth}
   \vspace{-35pt}
  \begin{center}
\psset{xunit=0.75cm,yunit=0.75cm,algebraic=true,dotstyle=o,dotsize=3pt 0,linewidth=0.8pt,arrowsize=3pt 2,arrowinset=0.25}
\begin{pspicture*}(-3.40,-2.45)(3.50,2.55)
\psaxes[labelFontSize=\scriptstyle,xAxis=true,yAxis=true,labels=none,Dx=1,Dy=1,ticksize=-2pt 0,subticks=2]{->}(0,0)(-3.40,-2.45)(3.50,2.55)
\rput{0}(0,0){\psellipse(0,0)(3,2)}
\psdots[dotstyle=*](0,2)
\rput[c](-0.4,1){$b$}
\psdots[dotstyle=*](3,0)
\rput[b](1.5,-0.5){$a$}
\psdots[dotstyle=*](-0,-0)
\pscustom[fillstyle=hlines,hatchwidth=0.5pt,hatchsep=8pt,linestyle=none]{\psplot[plotpoints=200]{0.0}{3.0}{2*sqrt(1-x^2/9)}\lineto(0,0)\lineto(0,2)\closepath}
\end{pspicture*}
\end{center}
\vspace{-25pt}
\end{wrapfigure}

\[\frac{x^2}{a^2}+\frac{y^2}{b^2} = 1\]
\[y = b\sqrt{1-\frac{x^2}{a^2}}\]
Ezzel megkaptuk az ellipszis felső ágának egyenletét. Magyarán az ellipszis területe:
\[4\hatIntegLimits{0}{a}{\left(b\sqrt{1-\frac{x^2}{a^2}}\right)}{x} = 4b\hatIntegLimits{0}{a}{\sqrt{1-\left(\frac{x}{a}\right)^2}}{x}\]
Legyen $\sin t := \dfrac{x}{a}$. Ekkor $dx = a\cos t\cdot dt$, a határok pedig: $0, \dfrac{\pi}{2}$, Tehát:
\[4b\hatIntegLimits{0}{\frac{\pi}{2}}{\sqrt{1-\sin t^2}\cdot a\cos t}{t} = 4ab\hatIntegLimits{0}{\frac{\pi}{2}}{\cos^2 t}{t} = 4ab\hatIntegLimits{0}{\frac{\pi}{2}}{\frac{1+\cos 2t}{2}}{t} = 4ab\left[\frac{1}{2}t+\frac{\sin 2t}{4}\right]^\frac{\pi}{2}_0 = \]
\[= 4ab\left(\frac{\pi}{4}+\frac{\sin \pi}{4}-\frac{\sin 0}{4}\right) = \boxed{ab\pi}\]

\subsection{Forgástest térfogatának kiszámolása}

\begin{center}
\psset{xunit=1.0cm,yunit=1.0cm,algebraic=true,dotstyle=o,dotsize=3pt 0,linewidth=0.8pt,arrowsize=3pt 2,arrowinset=0.25}
\begin{pspicture*}(-0.31,-3.65)(9.24,3.43)
\psaxes[labelFontSize=\scriptstyle,xAxis=true,yAxis=true,labels=none,Dx=2,Dy=2,ticksize=-2pt 0,subticks=2]{->}(0,0)(-0.31,-3.65)(9.24,3.43)
\psplot[plotpoints=200]{1.5}{7.5}{-0.02723*x^4+0.48653*x^3-2.93082*x^2+7.032741*x-4.24393}
\psline[linestyle=dotted](1.5,-3.65)(1.5,3.43)
\psline[linestyle=dotted](7.5,-3.65)(7.5,3.43)
\psline[linestyle=dashed,dash=8pt 8pt](3.4,1.27)(3.4,0)
\psline[linestyle=dashed,dash=8pt 8pt](4.38,0)(4.38,1.19)
\psplot[linestyle=dashed,dash=4pt 4pt,plotpoints=200]{1.5}{7.5}{-(-0.02723*x^4+0.48653*x^3-2.93082*x^2+7.032741*x-4.24393)}
\rput{90}(1.5,0){\psellipse[linestyle=dashed,dash=8pt 8pt](0,0)(1.21,0.27)}
\rput{90}(7.5,0){\psellipse[linestyle=dashed,dash=8pt 8pt](0,0)(2.73,0.48)}
\psdots[dotstyle=*](3.4,0)
\rput[bl](3.21,-0.61){$x$}
\psdots[dotstyle=*](4.38,0)
\rput[bl](3.89,-0.68){$x+dx$}
\psdots[dotstyle=*](3.4,1.27)
\psdots[dotstyle=*](4.38,1.19)
\rput[b](1.5,-0.4){$a$}
\rput[b](7.5,-0.4){$b$}
\end{pspicture*}
\end{center}
Ha megnézzük akkor az $x$ és $x+dx$ közti rész, ha megforgatjuk, akkor jó közelítéssel egy hengert ad ki ($dx$ tetszőleges kicsiny), ennek a térfogata:
\[dV(x) = \underbrace{\pi\cdot f^2(x)}_{\hbox{alapterület}}\cdot \underbrace{dx}_{\hbox{magasság}}\]
A teljes térfogat:
\[V = \hatIntegLimits{a}{b}{\pi\cdot f^2(x)}{x}\]
Például a kúp térfogata: Legyen a magassága $m$, alapterülete $r$. A kúpot kaphatjuk például az $y= \frac{r}{m}x$ egyenletű egyenes megforgatásásval:
\[V = \hatIntegLimits{0}{m}{\pi\cdot \frac{r^2}{m^2}x^2}{x} = \pi\frac{r^2}{m^2}\left[\frac{x^3}{3}\right]^{m}_{0} = \pi\frac{r^2}{m^2}\cdot\frac{m^3}{3} = \boxed{\frac{r^2\pi\cdot m}{3}}\]

\subsection{Ívhossz}

\begin{wrapfigure}{r}{0.40\textwidth}
   \vspace{-35pt}
  \begin{center}
\psset{xunit=1.0cm,yunit=1.0cm,algebraic=true,dotstyle=o,dotsize=3pt 0,linewidth=0.8pt,arrowsize=3pt 2,arrowinset=0.25}
\begin{pspicture*}(-0.48,-1.22)(5.8,5.12)
\psaxes[labelFontSize=\scriptstyle,xAxis=true,yAxis=true,labels=none,Dx=1,Dy=1,ticksize=-2pt 0,subticks=2]{->}(0,0)(-0.48,-1.22)(5.8,5.12)
\psplot[plotpoints=200]{1.0}{5.0}{(x-2)^2/3+1.5}
\psline[linestyle=dotted](1.24,-1.22)(1.24,5.12)
\psline[linestyle=dotted](4.56,-1.22)(4.56,5.12)
\psline[linestyle=dashed,dash=5pt 5pt](2.6,1.62)(2.6,-0)
\psline[linestyle=dashed,dash=5pt 5pt](3.54,-0)(3.54,2.29)
\psline[linestyle=dashed,dash=5pt 5pt](2.6,1.62)(3.54,1.62)
\psdots[dotstyle=*](1.24,0)
\rput[c](1.24,-0.4){$a$}
\psdots[dotstyle=*](4.56,0)
\rput[c](4.56,-0.4){$b$}
\psdots[dotstyle=*](2.6,1.62)
\rput[bl](2.40,1.74){$A$}
\psdots[dotstyle=*](3.54,2.29)
\rput[bl](3.3,2.42){$B$}
\psdots[dotstyle=*](2.6,-0)
\rput[b](2.5,-0.5){$x$}
\psdots[dotstyle=*](3.54,-0)
\rput[b](3.6,-0.52){$x+dx$}
\psdots[dotstyle=*](3.54,1.62)
\rput[bl](3.68,1.82){$f'(x) dx$}
\rput[bl](2.85,1.20){$dx$}
\end{pspicture*}
\end{center}
\vspace{-25pt}
\end{wrapfigure}

Ha megnézzük az ábrát, akkor $A$ és $B$ közti ív jó közelítéssel egy szakasz, Pithagorasz-tétel alapján:
\[dS(x)^2 = dx^2 + (f'(x)dx)^2\]
\[dS(x) = dx\sqrt{1+f'^2(x)}\]
A teljes ívre pedig:
\[S = \hatIntegLimits{a}{b}{\sqrt{1+f'^2(x)}}{x}\]
Pl. a kör kerülete. Kör egyenlete: $x^2+y^2 = r^2$, ekkor a felső ív függvénye:
\[y=\sqrt{r^2-x^2}\]
\[y'=\frac{-2x}{2\sqrt{r^2-x^2}} \]
Ha ennek a felének a hosszát kiszámoljuk, majd beszorozzuk 4-el, akkor megkapjuk a kör kerületét:
\[K = 4\hatIntegLimits{0}{r}{\sqrt{1+\left(\frac{-x}{\sqrt{r^2-x^2}}\right)^2}}{x} = 4\hatIntegLimits{0}{r}{\sqrt{\frac{r^2}{r^2-x^2}}}{x} = 4\hatIntegLimits{0}{r}{\sqrt{\frac{1}{1-\left(\frac{x}{r}\right)^2}}}{x} = \]
\[4\left[\frac{\arcsin \frac{x}{r}}{\frac{1}{r}}\right]^{r}_{0} = 4r\left(\arcsin 1-\arcsin 0\right) = 4r\frac{\pi}{2} = \boxed{2r\pi}\]

\section{Integrál kritérium}

\begin{tetel} $\boxed{$Integrál kritérium$}$\\[+3pt]
 Legyen $f\in R[1, a] \quad \forall a>1$  esetén, és $f$ monoton csökkenő és $f(x)\geqslant 0 \; \forall x\geqslant 1$ esetén. Legyen $a_n = f(n)$. Ekkor
  \[\sum^{\infty}_{n=1} a_n \quad \hbox{ és } \hatIntegLimits{1}{\infty}{f(x)}{x} \quad \hbox{ ekvikonvergensek, azaz }\]
  \[\sum^{\infty}_{n=1} a_n < \infty \quad \Leftrightarrow \quad \hatIntegLimits{1}{\infty}{f(x)}{x} < \infty \quad \hbox{ és }\]
  \[\sum^{\infty}_{n=1} a_n = \infty \quad \Leftrightarrow \quad \hatIntegLimits{1}{\infty}{f(x)}{x} = \infty\]
\end{tetel}
\begin{center}
\psset{xunit=1.0cm,yunit=1.0cm,algebraic=true,dotstyle=o,dotsize=3pt 0,linewidth=0.8pt,arrowsize=3pt 2,arrowinset=0.25}
\begin{pspicture*}(-0.7,-0.74)(8.14,5.16)
\psaxes[labelFontSize=\scriptstyle,xAxis=true,yAxis=true,labels=none,Dx=1,Dy=1,ticksize=-2pt 0,subticks=2]{->}(0,0)(-0.7,-0.74)(8.14,5.16)
\psplot[plotpoints=200]{0.5}{8.0}{3/x+1}
\psline[linestyle=dotted](1,4)(2,4)
\psline[linestyle=dashed,dash=3pt 3pt](2,2.5)(1,2.5)
\psline[linestyle=dashed,dash=3pt 3pt](2,2)(3,2)
\psline[linestyle=dotted](2,2.5)(3,2.5)
\psline[linestyle=dotted](3,2)(4,2)
\psline[linestyle=dashed,dash=3pt 3pt](3,1.75)(4,1.75)
\psline[linestyle=dashed,dash=3pt 3pt](4,1.43)(7,1.43)
\psline[linestyle=dotted](4,1.75)(7,1.75)
\psline(1,0)(1,4)
\psline(2,4)(2,0)
\psline(3,2.5)(3,0)
\psline(4,2)(4,0)
\psline(7,1.75)(7,0)
\psdots[dotstyle=*](1,0)
\rput[c](1,-0.4){1}
\psdots[dotstyle=*](2,0)
\rput[c](2,-0.4){2}
\psdots[dotstyle=*](3,0)
\rput[c](3,-0.4){3}
\psdots[dotstyle=*](4,0)
\rput[c](4,-0.4){4}
\psdots[dotstyle=*](7,0)
\rput[c](7,-0.4){$n$}
% \psdots[dotstyle=*](1,4)
% \psdots[dotstyle=*](2,2.5)
% \psdots[dotstyle=*](3,2)
% \psdots[dotstyle=*](4,1.75)
% \psdots[dotstyle=*](7,1.43)
% \psdots[dotstyle=*](2,4)
% \psdots[dotstyle=*](1,2.5)
% \psdots[dotstyle=*](2,2)
% \psdots[dotstyle=*](3,2.5)
% \psdots[dotstyle=*](3,1.75)
% \psdots[dotstyle=*](4,2)
% \psdots[dotstyle=*](4,1.43)
% \psdots[dotstyle=*](7,1.75)
\end{pspicture*}
\end{center}
\begin{bizNoNL}
 \[\sum^{n}_{k=2} a_k \leqslant \hatIntegLimits{1}{\infty}{f(x)}{x} \leqslant \sum^{n-1}_{k=1} a_k\]
 Illetve: 
  \[\hatIntegLimits{1}{n+1}{f(x)}{x} \leqslant \sum^{n}_{k=1} a_k \leqslant a_1+\hatIntegLimits{1}{n}{f(x)}{x}\]
 Mindkét esetben minoráns és majornás kritérium alapján igaz az állítás.
\end{bizNoNL}

Pl:
\[\sum^{\infty}_{n=1} \frac{1}{n} \quad \underbrace{\sim}_{\hbox{\footnotesize ekvikonv.}} \quad \hatIntegLimits{1}{\infty}{\frac{1}{x}}{x} = \infty\]
Tehát 
\[\sum^{\infty}_{n=1} \frac{1}{n} = \infty\]

\subsection{Hibabecslés}

\[R_n = S- S_n = \sum^{\infty}_{k=n+1} a_k \leqslant \hatIntegLimits{n}{\infty}{f(x)}{x}\]

\chapter{Számsorozatok nagyságrendje}

\begin{defi} \textbf{Ekvivalencia reláció}:
 \begin{itemize*}
  \item $a\sim b \Leftrightarrow b\sim a$ (szimmetrikus)
  \item $a\sim a$ (reflexív)
  \item $a\sim b, b\sim a \Rightarrow a\sim b$ (tranzitív)
 \end{itemize*}
\end{defi}

\begin{defi}
 $a_n = O(b_n)$, ha $\exists c\in\mathbb{R}$ és $N\in\mathbb{N}$, hogy
  \[|a_n| \leqslant c|b_n| \quad \hbox{ ha } n> N\]
\end{defi}

\begin{defi}
 $a_n = \Omega(b_n)$, ha $\exists c\in\mathbb{R}^+$ és $N\in\mathbb{N}$, hogy
  \[c|b_n| \leqslant a_n \quad \hbox{ ha } n> N\]
\end{defi}

\begin{defi}
 $a_n = \Theta(b_n) \quad \Leftrightarrow \quad b_n = \Theta(a_n)$, ha 
  \[a_n=O(b_n) \quad\hbox{ és }\quad a_n = \Omega(b_n)\]
 Tehát $a_n = \Theta(b_n) \quad \Leftrightarrow \quad \exists c, d > 0, N\in\mathbb{N}$ , hogy
  \[c|b_n| \leqslant a_n \leqslant d|b_n| \quad \hbox{ ha } n>N\]
\end{defi}

\begin{defi}
 $a_n \sim b_n$ (asszimptotikusan egyenlő), ha
  \[\lim_{n\to\infty} \frac{a_n}{b_n} = 1\]
\end{defi}

\begin{lemma} $a_n>0, b_n>0$\\
 $\Theta$ és $\sim$ ekvivalencia relációk és
\[a_n \sim b_n \quad \Rightarrow \quad a_n = \Theta(b_n)\]
\end{lemma}
\textbf{Bizonyítás}:
\[1-\varepsilon \leqslant \frac{a_n}{b_n} \leqslant 1+\varepsilon \quad \hbox{ ha } n>N\]
\[(1-\varepsilon)b_n \leqslant a_n \leqslant (1+\varepsilon)b_n \quad \hbox{ ha } n>N\]

\subsection{Műveleti szabályok}

\begin{tetel} Ha $a_n = \Theta(b_n)$ és $c_n = \Theta(d_n)$\\
  \[a_n+c_n = \Theta(b_n+d_n)\]
  \[a_n\cdot c_n = \Theta(b_n\cdot d_n)\]
  \[\frac{a_n}{c_n} = \Theta\left(\frac{b_n}{d_n}\right)\]
\emph{Megjegyzés}: különbségre nincs szabály! És hasonló szabályok igazak a $\sim$-re.
\end{tetel}
\addtocounter{biz}{1}

\begin{tetel} $a_n>0$, $b_n>0$ és $a_n \sim b_n$, akkor
  \[\sum^{\infty}_{n=1} a_n \quad \hbox{ekvikonvergens} \quad \sum^{\infty}_{n=1} b_n \;\hbox{-el}\]
\end{tetel}
\addtocounter{biz}{1}

\begin{tetel}$\boxed{$Stirling-formula$}$
  \[n! \sim \left(\frac{n}{e}\right)^n\sqrt{2\pi n}\]
\end{tetel}
\addtocounter{biz}{1}
\end{document}
