\documentclass[12pt]{article}
\usepackage[T1]{fontenc}
\usepackage{lmodern}
\usepackage[utf8]{inputenc}
\usepackage{amsmath}
\usepackage[a4paper]{geometry}
\usepackage{pstricks}
\usepackage{pstricks-add}
\title{Házi feladat}
\date{}
\author{Kriván Bálint - CBVOEN}

\begin{document}
 
  \maketitle

  \parindent 0pt
 
Ahogy azt az órai feladatban láttuk, az $u(x,t)$ felírható a következő alakban:
\[u(x,t) = \sum_{k=1}^\infty \sin\left(\frac{k\pi}{l}\cdot x\right)\left[ A_k \cos\left(\frac{kc\pi}{l}t\right) + B_k \sin\left(\frac{kc\pi}{l}t\right) \right]\]

Jelen esetben $l = 0.2$, $c = 1$. Továbbá tudjuk, hogy a kezdeti alakunk a következő függvénnyel írható fel:

\[f^*(x) = \left\{ \begin{array}{ll}0.05x & \hbox{ha } x\leq 0.1\\0.01-0.05x & \hbox{ha } 0.1 < x\end{array} \right.\]
Természetesen a húrunk a $0$ és a $0.2$ között van kifeszítve, ezért a függvényt csak $[0, 0.2]$-n értelmezzük.

Terjesszük ki a függvényünket úgy, hogy egy páratlan függvényt kapjunk, legyen ez az $f(x)$:

\begin{center}
\psset{xunit=12cm,yunit=120cm,algebraic=true,dotstyle=o,dotsize=3pt 0,linewidth=0.8pt,arrowsize=3pt 2,arrowinset=0.25}
\begin{pspicture*}(-0.53,-0.013)(0.53,0.013)
\psaxes[labelFontSize=\scriptstyle,xAxis=true,yAxis=true,Dx=0.1,Dy=0.005,ticksize=-2pt 0,subticks=2]{->}(0,0)(-0.54,-0.03)(0.54,0.03)
\psplot[linewidth=1.6pt,plotpoints=200]{0.1}{0.2}{0.01-0.05*x}
\psplot[linewidth=1.6pt,plotpoints=200]{-0.2}{-0.1}{-(0.01)-0.05*x}
\psplot[linestyle=dashed,dash=1pt 1pt,plotpoints=200]{-0.3}{-0.2}{-(0.01)-0.05*x}
\psplot[linewidth=1.6pt,plotpoints=200]{-0.1}{0.1}{0.05*x}
\psplot[linestyle=dashed,dash=1pt 1pt,plotpoints=200]{-0.5}{-0.3}{0.05*x+0.02}
\psplot[linestyle=dashed,dash=1pt 1pt,plotpoints=200]{0.2}{0.3}{0.01-0.05*x}
\psplot[linestyle=dashed,dash=1pt 1pt,plotpoints=200]{0.3}{0.5}{0.05*x-0.02}
\rput[bl](0.22,0){$r$}
\end{pspicture*} 
\end{center}

Húr kezdeti alakja:
\[u(x, 0) = f(x) \qquad \hbox{ha $0 \leq x \leq 0.2$}\]
Fejtsük Fourier-sorba $f(x)$-et ($a_k = 0$, hiszen páratlan függvény):
\[f(x) = \sum_{k=1}^{\infty} b_k \sin\left(k\cdot\frac{2\pi}{T}\cdot x\right)\]
Illetve:
\[b_k = \frac{2}{T} \int_{-T/2}^{T/2} f(x) \sin\left(k\cdot\frac{2\pi}{T}\cdot x\right)\, dx\]
Jelen esetben $T=0.4$, tehát:
\[b_k = \frac{1}{0.2} \int_{-0.2}^{0.2} f(x) \sin\left(k\cdot\frac{\pi}{0.2}\cdot x\right)\, dx = \frac{1}{0.1} \int_{0}^{0.2} f(x) \sin\left(k\cdot\frac{\pi}{0.2}\cdot x\right)\, dx =\]
\[= \frac{1}{0.1} \left[ \int_{0}^{0.1} 0.05x \sin\left(k\cdot\frac{\pi}{0.2}\cdot x\right)\, dx + \int_{0.1}^{0.2} (0.01-0.05x) \sin\left(k\cdot\frac{\pi}{0.2}\cdot x\right)\, dx \right] = \]
Paricális integrálás segítségével az alábbihoz jutunk:
\[ = \frac{1}{0.1} \left[ \frac{0.05 \sin\left(0.1\cdot\frac{\pi}{0.2}\cdot k\right)- 0.005 k \cos\left(0.1\cdot\frac{\pi}{0.2}\cdot k\right)}{\frac{\pi^2}{0.2^2}k^2} + \right.\]
\[\left. + \frac{0.05\sin\left(0.1\cdot\frac{\pi}{0.2}\cdot k\right)-0.05\sin\left(0.2\cdot\frac{\pi}{0.2}\cdot k\right)+0.005 k \cos\left(0.1\cdot\frac{\pi}{0.2}\cdot k\right)}{\frac{\pi^2}{0.2^2} k^2} \right] = \]
\[= \frac{1}{0.1} \left[ \frac{0.1 \sin\left(0.1\cdot\frac{\pi}{0.2}\cdot k\right) -0.05\sin\left(0.2\cdot\frac{\pi}{0.2}\cdot k\right)}{\frac{\pi^2}{0.2^2} k^2} \right] = \]
\[= \frac{0.2^2}{0.1} \left[ \frac{0.1 \sin\left(\frac{\pi}{2}\cdot k\right) -0.05\sin\left(\pi k\right)}{\pi^2 k^2} \right] = \]
Mivel $\sin(\pi k)$ az minden $k$ egész számra 0, ezért:
\[= \boxed{\frac{4}{\pi^2 k^2} \sin\left(\frac{\pi}{2}\cdot k\right) = b_k}\]
Vagyis:
\[f(x) = \sum_{k=1}^{\infty} \frac{4}{\pi^2 k^2} \sin\left(\frac{\pi}{2}\cdot k\right) \sin\left(k\cdot\frac{\pi}{0.2}\cdot x\right)\]

Tekintsük $u(x,t)$-t:
\[u(x,t) = \sum_{k=1}^\infty \sin\left(\frac{k\pi}{l}\cdot x\right)\left[ A_k \cos\left(\frac{kc\pi}{l}t\right) + B_k \sin\left(\frac{kc\pi}{l}t\right) \right]\]
A húr kezdeti alakja kifejezhető $u(x,t)$-vel, ha $t = 0$:
\[u(x,0) = \sum_{k=1}^\infty \sin\left(\frac{k\pi}{l}\cdot x\right)\left[ A_k \cos0 + B_k \sin0 \right] = \sum_{k=1}^\infty A_k\sin\left(\frac{k\pi}{l}\cdot x\right)\]
Mivel $f(x) = u(x, 0)$ (ha $0\leq x\leq 0.2$), láthatjuk, hogy:
\[A_k = b_k = \frac{4}{\pi^2 k^2} \sin\left(\frac{\pi}{2}\cdot k\right)\]

Továbbá azt is tudjuk, hogy:
\[\frac{\delta u}{\delta t}(x, 0) = 0 \qquad \hbox{hiszen csak elengedjük a húrt}\]
Mivel
\[\frac{\delta u}{\delta t}(x, t) = \sum_{k=1}^\infty \sin \left(\frac{k\pi}{l}x\right)\Big(-A_k\alpha_k\sin(\alpha_k t)+B_k\alpha_k\cos(\alpha_k t)\Big) \qquad \alpha_k = \frac{kc\pi}{l}\]
ezért:
\[\frac{\delta u}{\delta t}(x, 0) = \sum_{k=1}^\infty B_k\alpha_k\sin \left(\frac{k\pi}{l}x\right) = 0\]
Mivel $\forall k$-ra teljesülni kell, ezért $B_k = 0$. Ezzel meg is kaptuk $u(x, t)$-t:

\[\boxed{\sum_{k=1}^\infty \sin\left(\frac{k\pi}{l}\cdot x\right)\left[ \frac{4}{\pi^2 k^2} \sin\left(\frac{\pi}{2}\cdot k\right) \cos\left(\frac{kc\pi}{l}t\right) \right]}\]
Beírva a megadott értékeket:
\[\boxed{u(x, t) = \sum_{k=1}^\infty \sin\left(\frac{k\pi}{0.2}\cdot x\right)\left[ \frac{4}{\pi^2 k^2} \sin\left(\frac{\pi}{2}\cdot k\right) \cos\left(\frac{k\pi}{0.2}t\right) \right]}\]

\end{document}
​
