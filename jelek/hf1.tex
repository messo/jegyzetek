\documentclass[12pt,a4paper]{article}

\usepackage[utf8]{inputenc}
\usepackage[T1]{fontenc}
\usepackage[magyar]{babel}
\usepackage{lmodern}
\usepackage{amsmath,amssymb}
\usepackage{fullpage}
\usepackage[a4paper,top=2cm,left=2cm,right=2cm,bottom=3.5cm]{geometry}
\usepackage{pstricks-add}
\usepackage{fancyhdr}
\usepackage{graphicx}

\title{\vspace{-2ex}Jelek és rendszerek 1. házi}
\author{Kriván Bálint \texttt{CBVOEN}\\
Gyakorlat vezető: Farkasvölgyi Andrea}

\pagestyle{fancy}
\lhead{\sc Kriván Bálint \texttt{CBVOEN}}
\chead{}
\rhead{Jelek és rendszerek 1. házi}
\lfoot{}
\cfoot{\thepage}
\rfoot{}
\headheight 15pt
\headsep 10pt

\parindent 0pt

\newlength{\wordwidth}

\newcommand\uuline[1]{\underline{\underline{#1}}}

\newcommand\uline[1]{\underline{#1}}

\begin{document}
\maketitle

\thispagestyle{fancy}

\textbf{1.1 (a)}

Akkor gerjesztés-válasz stabilis az FI illetve a DI rendszer, ha $h(t)$ ill. $h[k]$ abszolút integrálható ill. összegezhető. Vagyis:

\[\int_{-\infty}^{\infty}\Big|h(t)\Big| dt < \infty \quad \hbox{és} \quad \sum_{k=-\infty}^{\infty} \Big|h[k]\Big| < \infty\]

Nézzük először az FI-t:

\[\int_{-\infty}^{\infty}\Big|h(t)\Big| dt = \int_{-\infty}^{\infty}\Big|4\delta(t)+\varepsilon(t)\{6e^{-0,2t}+(-4)\cdot e^{-0,3t}\}\Big| dt \leqslant \]
\[\leqslant \int_{-\infty}^{\infty}\Big|4\delta(t)\Big| dt +\int_{-\infty}^{\infty}\Big|\varepsilon(t)\{6e^{-0,2t}+(-4)\cdot e^{-0,3t}\}\Big| dt \leqslant\]
\[\leqslant 4 +\int_{0}^{\infty}\Big|6e^{-0,2t}+(-4)\cdot e^{-0,3t}\Big| dt \leqslant 4 +\int_{0}^{\infty}\Big|6e^{-0,2t}\Big| + \int_{0}^{\infty}\Big|4e^{-0,3t}\Big| dt = \]
\[= 4 +\left[\frac{6e^{-0,2t}}{-0,2}\right]^{\infty}_{0} + \left[\frac{4e^{-0,3t}}{-0,3}\right]^{\infty}_{0} = 4 +\left[0-\frac{6}{-0,2}\right] + \left[0-\frac{4}{-0,3}\right] < \infty\]
Tehát az FI rendszer GV stabilis. Nézzük a DI rendszert:

\[\sum_{k=-\infty}^{\infty} \Big|h[k]\Big| = \sum_{k=-\infty}^{\infty} \Big|4\varepsilon[k](0,5)^k\cos(0,3k+(-0,5))\Big|\]
A $\cos$-os tagot becsülhetjük felülről 1-gyel:
\[\sum_{k=-\infty}^{\infty} \Big|4\varepsilon[k](0,5)^k\cos(0,3k+(-0,5))\Big| \leqslant \sum_{k=-\infty}^{\infty} \Big|4\varepsilon[k](0,5)^k\Big| = \sum_{k=0}^{\infty} 4(0,5)^k = 4\cdot \frac{1}{1-0,5} = 8 < \infty\]
Tehát a DI rendszer is GV stabilis.\\

\textbf{1.1 (b)}

FI-nél elegendó valamelyik $e$ hatvány kitevőjét pozitívra cserélni, így nem lesz abszolút integrálható. DI-nél pedig a $0,5^k$-ból lecserélni a $0,5$-ös alapot 1-nél nagyobb számra, így nem lesz abszolút összegezhető.\\

\textbf{1.1 (c)}

Általánosan:
\[y[k] = \sum_{i=-\infty}^{\infty} u[k-i]h[i]\]
Mivel belépő gerjesztésről és belépő impulzusválaszról van szó, ezért:
\[y[k] = \varepsilon[k]\sum_{i=0}^{k} u[k-i]h[i] = \varepsilon[k]\sum_{i=0}^{k}\Big( 4(0,5)^i\cdot\cos(0,3i+(-0,5))\cdot(7-8(0,5)^{k-i}) \Big) =\]
\[= \varepsilon[k]\sum_{i=0}^{k}\Big(\big(28(0,5)^i-32 (0,5)^k\big)\cdot  \cos(0,3i-0,5)\Big) = \]
Nézzük meg $k=0$-ra:
\[y[0] = \left\{\big(28-32\big)\cdot  \cos(-0,5)\right\} = \left\{-4 \cos(-0,5)\right\} \approx -3,51\]
$k=1$-re:
\[y[1] = \left\{\big(28-32 (0,5)\big)\cdot  \cos(-0,5)+\big(28(0,5)-32 (0,5)\big)\cdot  \cos(0,3-0,5)\right\} = \]
\[= \left\{12\cdot \cos(-0,5)-2\cdot  \cos(-0,2)\right\} \approx 8,57\]
$k=2$-re:
\[y[2] = \left\{\big(28-32 (0,5)^2\big)\cdot  \cos(-0,5)+\big(28(0,5)-32 (0,5)^2\big)\cdot  \cos(0,3-0,5)+\right.\]
\[\left.+\big(28(0,5)^2-32 (0,5)^2\big)\cdot  \cos(0,6-0,5)\right\}=\]
\[= \left\{20\cdot\cos(-0,5)+6\cdot  \cos(-0,2)-1\cdot  \cos(0,1)\right\} \approx 22,437\]

\textbf{1.1 (d)}

Konvolúcióval számolhatjuk, kezdjük az FI-vel:

\[y(t) = \int_{-\infty}^{\infty} h(\tau)u(t-\tau) d\tau = 4\int_{-\infty}^{\infty} 4\delta(\tau)+\varepsilon(\tau)\{6e^{-0,2\tau}+(-4)\cdot e^{-0,3\tau}\} d\tau =\]
\[= 16 + 4\int_{0}^{\infty} 6e^{-0,2\tau}+(-4)\cdot e^{-0,3\tau} d\tau = 16 + 24\left[\frac{e^{-0,2\tau}}{-0,2}\right]_0^{\infty}-16\left[\frac{e^{-0,3\tau}}{-0,3}\right]_0^{\infty}=\]
\[= 16 + 24\left(0-\frac{1}{-0,2}\right)-16\left(0-\frac{1}{-0,3}\right) = 136 - 16\cdot\frac{3}{10} = 82,6667\]

DI rendszer esetében:

\[y[k] = \sum_{t=-\infty}^{\infty} h[t]u[k-t] = \sum_{t=-\infty}^{\infty} 4\varepsilon[t](0,5)^t\cos(0,3t+(-0,5))5(4)^{k-t} =\]
\[= 20\cdot4^k\cdot \sum_{t=0}^{\infty} (0,5)^t\cos(0,3t-0,5)\left(\frac{1}{4}\right)^{t} = 20\cdot4^k\cdot \sum_{t=0}^{\infty} \left(\frac{1}{8}\right)^{t}\cos(0,3t-0,5) = \]
\[20\cdot4^k\cdot \sum_{t=0}^{\infty} \hbox{Re}\left(\left(\frac{1}{8}\right)^{t} e^{(0,3t-0,5)i}\right) = 20\cdot4^k\cdot \hbox{Re}\left(e^{-0,5i} \sum_{t=0}^{\infty} \left(\frac{1}{8}\cdot e^{0,3i}\right)^{t}\right) =\]
\[ = 20\cdot4^k\cdot \hbox{Re}\left(e^{-0,5i} \frac{1}{1-\frac{1}{8}\cdot e^{0,3i}}\right) = 20\cdot4^k\cdot \hbox{Re}\left(1.01764 - 0.501752i\right) = 20,3528\cdot4^k\]

\textbf{1.2 (a)}

Kezdjük a DI rendszerrel, a feladat alapján:
\[\uuline{A} = \left[\begin{matrix}-1 & -0,7\\0,8 & 0,8\end{matrix}\right] \quad \uline{B}=\left[\begin{matrix}0,7\\0,6\end{matrix}\right] \quad \uline{C}^T = \left[\begin{matrix}-0,4 & 0,5\end{matrix}\right] \quad D = 1,3\]

Kezdjük az $\uuline{A}$ mátrix sajátértékeinek kiszámolásával:
\[|\uuline{A}-\lambda \uuline{E}| = \det\left[\begin{matrix}-1-\lambda & -0,7\\0,8 & 0,8-\lambda\end{matrix}\right] = (-1-\lambda)(0,8-\lambda)+0,7\cdot 0,8 = \lambda^2 +0,2\lambda -0,24\]
Ebből $\lambda_1 = 0,4$ és $\lambda_2 = -0,6$. Ezután kiszámoljuk a Lagrange-mátrixokat:
\[\uuline{L}_1 = \frac{\uuline{A}-\lambda_2\uuline{E}}{\lambda_1-\lambda_2} = \frac{1}{0,4+0,6}\left[\begin{matrix}-1+0,6 & -0,7\\0,8 & 0,8+0,6\end{matrix}\right] = \left[\begin{matrix}-0,4 & -0,7\\0,8 & 1,4\end{matrix}\right]\]
\[\uuline{L}_2 = \frac{\uuline{A}-\lambda_1\uuline{E}}{\lambda_2-\lambda_1} = \frac{1}{-0,6-0,4}\left[\begin{matrix}-1-0,4 & -0,7\\0,8 & 0,8-0,4\end{matrix}\right] = \left[\begin{matrix}1,4 & 0,7\\-0,8 & -0,4\end{matrix}\right]\]
Ezek alapján egy adott $k$-ra meg tudjuk mondani $\uline{C}^T\uuline{A}^k\uline{B}$ értékét, hiszen:
\[\uline{C}^T\uuline{A}^k\uline{B} = \uline{C}^T(\lambda_1^k\uuline{L}_1+\lambda_2^k\uuline{L}_2)\uline{B} = \uline{C}^T\lambda_1^k\uuline{L}_1+\uline{C}^T\lambda_2^k\uuline{L}_2\uline{B}=\]
\[= \lambda_1^k\left[\begin{matrix}-0,4 & 0,5\end{matrix}\right]\cdot\left[\begin{matrix}-0,4 & -0,7\\0,8 & 1,4\end{matrix}\right]\cdot \left[\begin{matrix}0,7\\0,6\end{matrix}\right] + \lambda_2^k\left[\begin{matrix}-0,4 & 0,5\end{matrix}\right]\cdot\left[\begin{matrix}1,4 & 0,7\\-0,8 & -0,4\end{matrix}\right]\cdot \left[\begin{matrix}0,7\\0,6\end{matrix}\right] = \]
\[= \lambda_1^k\left[\begin{matrix}0,56 & 0,98\end{matrix}\right]\cdot\left[\begin{matrix}0,7\\0,6\end{matrix}\right] + \lambda_2^k\left[\begin{matrix}-0,96 & -0,48\end{matrix}\right]\cdot \left[\begin{matrix}0,7\\0,6\end{matrix}\right] = \]
\[= 0,98\lambda_1^k-0,96\lambda_2^k \]

Ebből pedig már behelyettesíthetünk a képletbe, ami a következő alakot ölti:
\[h[k] = D\delta[k]+\varepsilon[k-1]\uline{C}^T\uuline{A}^{k-1}\uline{B} = \]
\[= 1,3\delta[k]+\varepsilon[k-1](0,98\lambda_1^{k-1}-0,96\lambda_2^{k-1}) = \]
\[= \boxed{1,3\delta[k]+\varepsilon[k-1]\left(0,98\cdot 0,4^{k-1}-0,96\cdot (-0,6)^{k-1}\right)}\]

Ábrázoltuk \aref{fig:di}. ábrán.\\

\begin{figure}[h]
\begin{center}
\psset{xunit=1.0cm,yunit=3.0cm,algebraic=true,dotstyle=o,dotsize=3pt 0,linewidth=0.5pt,arrowsize=3pt 2,arrowinset=0.25}
\begin{pspicture*}(-0.75,-0.35)(8.58,1.61)
\psaxes[labelFontSize=\scriptstyle,xAxis=true,yAxis=true,Dx=2,Dy=0.5,ticksize=-2pt 0,subticks=2]{->}(0,0)(-0.75,-0.29)(8.58,1.61)
%\psline[linewidth=1pt,linestyle=dotted](0,1.3)(1,0.02)
%\psline[linewidth=1pt,linestyle=dotted](1,0.02)(2,0.97)
%\psline[linewidth=1pt,linestyle=dotted](2,0.97)(3,-0.19)
%\psline[linewidth=1pt,linestyle=dotted](3,-0.19)(4,0.27)
%\psline[linewidth=1pt,linestyle=dotted](4,0.27)(5,-0.1)
%\psline[linewidth=1pt,linestyle=dotted](5,-0.1)(6,0.08)
%\psline[linewidth=1pt,linestyle=dotted](6,0.08)(7,-0.04)
%\psline[linewidth=1pt,linestyle=dotted](7,-0.04)(8,0.03)
\psdots[dotstyle=*](0,1.3)
\psline[linewidth=1pt](0,0)(0,1.3)
\psdots[dotstyle=*](1,0.02)
\psline[linewidth=1pt](1,0)(1,0.02)
\psdots[dotstyle=*](2,0.97)
\psline[linewidth=1pt](2,0)(2,0.97)
\psdots[dotstyle=*](3,-0.19)
\psline[linewidth=1pt](3,0)(3,-0.19)
\psdots[dotstyle=*](4,0.27)
\psline[linewidth=1pt](4,0)(4,0.27)
\psdots[dotstyle=*](5,-0.1)
\psline[linewidth=1pt](5,0)(5,-0.1)
\psdots[dotstyle=*](6,0.08)
\psline[linewidth=1pt](6,0)(6,0.08)
\psdots[dotstyle=*](7,-0.04)
\psline[linewidth=1pt](7,0)(7,-0.04)
\psdots[dotstyle=*](8,0.03)
\psline[linewidth=1pt](8,0)(8,0.03)
\end{pspicture*}
\caption{$h[k]$ ábrázolva $k=0$-tól $k=8$-ig}
\label{fig:di}
\end{center}
\end{figure}

Most tekintsük az FI rendszert, ebben az alábbiak a mátrixok/vektorok:
\[\uuline{A} = \left[\begin{matrix}-1,2 & -0,5\\2 & -1,2\end{matrix}\right] \quad \uline{B}=\left[\begin{matrix}1,2\\1,3\end{matrix}\right] \quad \uline{C}^T = \left[\begin{matrix}0,8 & -0,6\end{matrix}\right] \quad D = 0,45\]
Meghatározzuk $\uuline{A}$ sajátértékeit:
Kezdjük az $\uuline{A}$ mátrix sajátértékeinek kiszámolásával:
\[|\uuline{A}-\lambda \uuline{E}| = \det\left[\begin{matrix}-1,2-\lambda & -0,5\\2 & -1,2-\lambda\end{matrix}\right] = (-1,2-\lambda)^2+0,5\cdot 2 = \lambda^2 +2,4\lambda +2,44\]
Ebből $\lambda_{1,2} = -1,2\pm i$. Ezután kiszámoljuk a Lagrange-mátrixokat:
\[\uuline{L}_1 = \frac{\uuline{A}-\lambda_2\uuline{E}}{\lambda_1-\lambda_2} = \frac{1}{2i}\left[\begin{matrix}-1,2-(-1,2-i) & -0,5\\2 & -1,2-(-1,2-i)\end{matrix}\right] = \frac{1}{2i}\left[\begin{matrix}i & -0,5\\2 & i\end{matrix}\right] = \left[\begin{matrix}0,5 & 0,25i\\-i & 0,5\end{matrix}\right]\]
\[\uuline{L}_2 = \frac{\uuline{A}-\lambda_1\uuline{E}}{\lambda_2-\lambda_1} = \frac{1}{-2i}\left[\begin{matrix}-1,2-(-1,2+i) & -0,5\\2 & -1,2-(-1,2+i)\end{matrix}\right] =\]
\[= \frac{1}{-2i}\left[\begin{matrix}-i & -0,5\\2 & -i\end{matrix}\right] = \left[\begin{matrix}0,5 & -0,25i\\i & 0,5\end{matrix}\right]\]
Ezek alapján felírható $\uline{C}^Te^{\uuline{A}t}\uline{B}$:
\[\uline{C}^Te^{\uuline{A}t}\uline{B} = \uline{C}^T\left(e^{\lambda_1 t}\uuline{L_1}+e^{\lambda_2 t}\uuline{L_2}\right)\uline{B}\]
Behelyettesítve a számokat:
\[\uline{C}^Te^{\uuline{A}t}\uline{B} = \left[\begin{matrix}0,8 & -0,6\end{matrix}\right]\cdot\left(e^{(-1,2+i)t}\left[\begin{matrix}0,5 & 0,25i\\-i & 0,5\end{matrix}\right]+e^{(-1,2-i)t}\left[\begin{matrix}0,5 & -0,25i\\i & 0,5\end{matrix}\right]\right)\cdot\left[\begin{matrix}1,2\\1,3\end{matrix}\right]\]
Ezt kiszámolva a következőt kapjuk:
\[\uline{C}^Te^{\uuline{A}t}\uline{B} = (0,09+0,98 i) e^{(-1.2+i) t} + (0,09-0,98 i)e^{(-1,2-i) t}\]
Impulzusválasz kiszámolásához, nincs más dolgunk, mint a lenti képletbe behelyettesíteni:
\[h(t) = D\delta(t)+\varepsilon(t)\uline{C}^Te^{\uuline{A}t}\uline{B} = 0,45\delta(t) + \varepsilon(t)\left((0,09+0,98 i) e^{(-1,2+i) t} + (0,09-0,98 i)e^{(-1,2-i) t}\right)\]
\[h(t) = 0,45\delta(t) + \varepsilon(t)e^{-1,2t}\left((0,09+0,98 i) e^{it} + (0,09-0,98 i)e^{-it}\right) = \]
\[= 0,45\delta(t) + \varepsilon(t)e^{-1,2t}\left(0,9841e^{1,479i} e^{it} + 0,9841e^{-1,479i}e^{-it}\right)\]
\[= 0,45\delta(t) + \varepsilon(t)e^{-1,2t}\left(0,9841e^{(1,479+t)i} + 0,9841e^{-(1,479+t)i}\right) = \]
Látható, hogy a két tagban az $e$ kitevője egymás $-1$-szerese, így ha átírjuk az $e^{\phi i}$ alakot $cos(\phi)+i\cdot\sin(\phi)$ alakba, akkor a $\sin$-os tag kiesik, így:
\[= 0,45\delta(t) + \varepsilon(t)e^{-1,2t}\left(2\cdot 0,9841\cos(1,479+t)\right) =\]
\[= \boxed{0,45\delta(t) + \varepsilon(t)\cdot 1,9682\cdot e^{-1,2t}\cos(1,479+t)}\]

Ábrázoltuk \aref{fig:fi}. ábrán.\\

\begin{figure}[h]
\begin{center}
\psset{xunit=1.5cm,yunit=7.5cm,algebraic=true,dotstyle=o,dotsize=3pt 0,linewidth=0.5pt,arrowsize=3pt 2,arrowinset=0.25}
\begin{pspicture*}(-0.68,-0.54)(6.21,0.69)
\psaxes[labelFontSize=\scriptstyle,xAxis=true,yAxis=true,Dx=1,Dy=0.2,ticksize=-2pt 0,subticks=2]{->}(0,0)(-0.68,-0.54)(6.21,0.69)
\psplot[linewidth=1.2pt,plotpoints=200]{-0.6791189685899286}{0.0}{0}
\psplot[linewidth=1.2pt,plotpoints=200]{0.0}{6.207677723309362}{1.97*EXP(-(1.2)*x)*COS(1.48+x)}
\psline[linewidth=1.2pt,arrowinset=0]{->}(0,0)(0,0.45)
\psline[linewidth=1.2pt](0,0.45)(0,0.7)
\end{pspicture*}
\caption{$h(t)$ ábrázolva}
\label{fig:fi}
\end{center}
\end{figure}

\textbf{1.2 (b)}

Feladat szövegéből:
\[\uline{x}[k+1] = \left[\begin{matrix}-1 & -0,7\\0,8 & 0,8\end{matrix}\right]\uline{x}[k] + \left[\begin{matrix}0,7\\0,6\end{matrix}\right]u[k]\]
\[y[k] = \left[\begin{matrix}-0,4 & 0,5\end{matrix}\right]\uline{x}[k] + 1,3u[k]\]

Impulzusválaszt keresünk, így $u[0]=1$, többi időpillantban pedig 0, azaz $u[k]=0$, ha $k>0$ ($k\in \mathbb{N}$).
Tudjuk, hogy $\uline{x}[0]=\uline{0}$, így:
\[h[0] = 1,3 \qquad \uline{x}[1] = \left[\begin{matrix}0,7\\ 0,6\end{matrix}\right]\]
Innen:
\[h[1] = \left[\begin{matrix}-0,4 & 0,5\end{matrix}\right]\cdot\left[\begin{matrix}0,7\\ 0,6\end{matrix}\right] = 0,02 \qquad \uline{x}[2]=\left[\begin{matrix}-1 & -0,7\\0,8 & 0,8\end{matrix}\right]\cdot\left[\begin{matrix}0,7\\ 0,6\end{matrix}\right]=\left[\begin{matrix}-1,12\\ 1,04\end{matrix}\right]\]
És végül:
\[h[2] = \left[\begin{matrix}-0,4 & 0,5\end{matrix}\right]\cdot\left[\begin{matrix}-1,12\\ 1,04\end{matrix}\right] = 0,968\]
Behelyettesítéssel ellenőrizzük:
\[h[k] = 1,3\delta[k]+\varepsilon[k-1]\left(0,98\cdot 0,4^{k-1}-0,96\cdot (-0,6)^{k-1}\right)\]
\[h[0] = 1,3\]
\[h[1] = \left(0,98\cdot 0,4^{0}-0,96\cdot (-0,6)^{0}\right)=0,02\]
\[h[2] = \left(0,98\cdot 0,4^{1}-0,96\cdot (-0,6)^{1}\right)=0,968\]

Ugyanazon eredményeket kaptuk, ez jót jelent.\\

\textbf{1.2 (c)}

1.1 (d)-hez hasonlóan, itt is konvolúcióval számolunk, kezdjük az FI-vel:
\[u(t) = 9\{\varepsilon(t)-\varepsilon(t-1,6)\}\]
\[y(x) = \int_{-\infty}^{\infty} h(t)u(x-t) dt\]
\[y(x) = \int_{-\infty}^{\infty} \left(0,45\delta(t) + \varepsilon(t)\cdot 1,9682\cdot e^{-1,2t}\cos(1,479+t)\right)9\{\varepsilon(x-t)-\varepsilon(x-t-1,6)\} dt = \]
\[= 0,45\cdot9\{\varepsilon(x)-\varepsilon(x-1,6)\} - \int_0^{\infty}\left(1,9682\cdot e^{-1,2t}\cos(1,479+t)\right)9\{\varepsilon(x-t)-\varepsilon(x-t-1,6)\} dt = \]
\[= 4,05\{\varepsilon(x)-\varepsilon(x-1,6)\} - \int_0^{\infty}\left(1,9682\cdot e^{-1,2t}\cos(1,479+t)\right)9\varepsilon(x-t) dt +\]
\[+ \int_0^{\infty}\left(1,9682\cdot e^{-1,2t}\cos(1,479+t)\right)9\varepsilon(x-t-1,6) dt = \]
Az első integrálnál az $\varepsilon(x-t)$-s tényező maitt $x-t > 0$, ezért az első integrál felső határa $x$ lesz, a másodiknál hasonló okok miatt $x-1,6$, tehát: 
\[= 4,05\{\varepsilon(x)-\varepsilon(x-1,6)\} - \varepsilon(x)\int_0^{x}17,7138\cdot e^{-1,2t}\cos(1,479+t) dt + \]
\[+ \varepsilon(x-1,6)\int_0^{x-1,6}17,7138\cdot e^{-1,2t}\cos(1,479+t) dt\]
Az integrálok elé, azért került egy $\varepsilon(x)$ és $\varepsilon(x-1,6)$, mert ha a határok kisebbek, mint 0, akkor az integrál 0, ezt ezzel könnyen jelölhetjük.
\[= 4,05\{\varepsilon(x)-\varepsilon(x-1,6)\} - \varepsilon(x)\left[e^{-1,2 t} \big(9,34 \sin(t)+6,43 \cos(t)\big)\right]_0^{x} + \]
\[+ \varepsilon(x-1,6)\left[e^{-1,2 t} \big(9,34 \sin(t)+6,43 \cos(t)\big)\right]_0^{x-1,6} = \]
\[= \boxed{\varepsilon(x)\left(9,34 e^{-1,2 x} \sin(x)+6,43 e^{-1,2 x} \cos(x)-2,38
\right) +}\]
\[+ \boxed{\varepsilon(x-1,6)\left(-64,96 e^{-1,2 x} \cos(x)+41,98 e^{-1,2 x} \sin(x)-2,38
\right)}\]

Nézzük a DI rendszert:
\[u[k] = \varepsilon[k]\{7 + (-8)\cdot(0,5)^k\}\]
\[y[k] = \sum_{t=-\infty}^{\infty} h[t]u[k-t] = \]
\[= \sum_{t=-\infty}^{\infty}\left\{1,3\delta[t]+\varepsilon[t-1]\left(0,98\cdot 0,4^{t-1}-0,96\cdot (-0,6)^{t-1}\right)\right\}\left(\varepsilon[k-t]\{7 + (-8)\cdot(0,5)^{k-t}\}\right) = \]
\[= 1,3\left(\varepsilon[k]\{7 + (-8)\cdot(0,5)^{k}\}\right) + \]
\[+ \sum_{t=-\infty}^{\infty}\varepsilon[t-1]\left(0,98\cdot 0,4^{t-1}-0,96\cdot (-0,6)^{t-1}\right)\left(\varepsilon[k-t]\{7 + (-8)\cdot(0,5)^{k-t}\}\right) = \]

A rövidség kedvéért, nem írom le a $\varepsilon[k-1]$-et a szumma elé a továbbiakban, de majd a végén ne felejtsük el, hiszen $k<1$ esetén nincs értelme a szummának, így annak értéke 0.

\[1,3\left(\varepsilon[k]\{7 + (-8)\cdot(0,5)^{k}\}\right) + \]
\[+ \sum_{t=1}^{k}\left(0,98\cdot 0,4^{t-1}-0,96\cdot (-0,6)^{t-1}\right)\left(7 + (-8)\cdot(0,5)^{k-t}\right) = \]

\[= 1,3\left(\varepsilon[k]\{7 + (-8)\cdot(0,5)^{k}\}\right) + \]
\[+ \sum_{t=1}^{k}\left(6,86\cdot 0,4^{t-1}-6,72\cdot (-0,6)^{t-1}\right)+ \sum_{t=1}^{k}\left(-7,84\cdot 0,4^{t-1}+7,68\cdot (-0,6)^{t-1}\right)(0,5)^{k-t} = \]
\[= 1,3\left(\varepsilon[k]\{7 + (-8)\cdot(0,5)^{k}\}\right) + \sum_{t=1}^{k}\left(6,86\cdot 0,4^{t-1}\right)-\sum_{t=1}^{k}\left(6,72\cdot (-0,6)^{t-1}\right)+\]
\[+ (0,5)^k\sum_{t=1}^{k}\left(-7,84\cdot 0,4^{-1}\cdot 0,4^{t}\cdot(0,5)^{-t}+7,68\cdot(-0,6)^{-1}\cdot (-0,6)^{t}\cdot(0,5)^{-t}\right)\]

\[= 1,3\left(\varepsilon[k]\{7 + (-8)\cdot(0,5)^{k}\}\right) + \frac{6,86\cdot (0,4^k-1)}{0,4-1}-\frac{6,72\cdot ((-0,6)^k-1)}{-0,6-1}+\]
\[+ (0,5)^k\sum_{t=1}^{k}\left(-19,6\cdot 0,8^{t}\right)+(0,5)^k\sum_{t=1}^{k}\left(-12,8\cdot (-1,2)^{t}\right) = \]

\[= 1,3\left(\varepsilon[k]\{7 + (-8)\cdot(0,5)^{k}\}\right) + \frac{6,86\cdot (0,4^k-1)}{0,4-1}-\frac{6,72\cdot ((-0,6)^k-1)}{-0,6-1}+\]
\[+ (0,5)^k\frac{-19,6\cdot 0,8\cdot (0,8^{k}-1)}{0,8-1}+(0,5)^k\frac{-12,8\cdot (-1,2)\cdot ((-1,2)^{k}-1)}{-1,2-1} = \]

\[= 1,3\left(\varepsilon[k]\{7 + (-8)\cdot(0,5)^{k}\}\right) - 11,43\cdot (0,4^k-1)+4,2\cdot ((-0,6)^k-1)+\]
\[+ (0,5)^k\cdot 78,4\cdot (0,8^{k}-1)+(0,5)^k\cdot (-6,98)\cdot ((-1,2)^{k}-1) = \]
\[= 1,3\left(\varepsilon[k]\{7 + (-8)\cdot(0,5)^{k}\}\right) + 66,97\cdot 0,4^k -2,78\cdot (-0,6)^k - 71,42(0,5)^k + 7,23\]
Mielőtt elfelejtenénk rakjuk vissza a $\varepsilon[k-1]$-et:
\[= 1,3\left(\varepsilon[k]\{7 + (-8)\cdot(0,5)^{k}\}\right) + \varepsilon[k-1]\left(66,97\cdot 0,4^k -2,78\cdot (-0,6)^k - 71,42(0,5)^k + 7,23\right)\]
Ezt már nagyon nem lehet tovább pofozgatni:
\[y[k] = \boxed{\varepsilon[k]\left(9,1 -10,4\cdot(0,5)^{k}\right) + \varepsilon[k-1]\left(66,97\cdot 0,4^k -2,78\cdot (-0,6)^k - 71,42(0,5)^k + 7,23\right)}\]

\textbf{1.3 (a)}

Miután elneveztük az integrátorok/késleltetők kimenetét, kiszámítjuk a kimeneteket/bementeket, ezek alapján könnyedén felírhatjuk a normál alakokat (lásd \ref{fig:halozat}. ábra). DI rendszernek:

\begin{figure}
\begin{center}
\includegraphics{3as.ps}
\caption{Hálózat állapotváltozós leírásának meghatározása}
\label{fig:halozat}
\end{center}
\end{figure}

\[\uline{x}[k+1] = \left[\begin{matrix}0,8 & -2\\0,56 & -0,4\end{matrix}\right]\uline{x}[k] + \left[\begin{matrix}-2\\-0,4\end{matrix}\right]u[k]\]
\[y[k] = \left[\begin{matrix}0,84 & 0,6\end{matrix}\right]\uline{x}[k] + 0,6u[k]\]

FI rendszernek:

\[\uline{x}'(t) = \left[\begin{matrix}0,8 & -2\\0,56 & -0,4\end{matrix}\right]\uline{x}(t) + \left[\begin{matrix}-2\\-0,4\end{matrix}\right]u(t)\]
\[y(t) = \left[\begin{matrix}0,84 & 0,6\end{matrix}\right]\uline{x}(t) + 0,6u(t)\]

\textbf{1.3 (b)}

Aszimptotikus stabilitáshoz, meghatározzuk az $\uuline{A}$ mátrix sajátértékeit:
\[|\uuline{A}-\lambda\uuline{E}| = \det\left[\begin{matrix}0,8-\lambda & -2\\0,56 & -0,4-\lambda\end{matrix}\right] = (0,8-\lambda)(-0,4-\lambda)+1,12=\lambda^2-0,4\lambda+0,8\]
Ebből $\lambda_{1,2} = 0,2 \pm 0,872i$.\\

Mivel $|\lambda_{1,2}| = \sqrt{0,2^2+0,872^2}=0,8946 < 1$, tehát a sajátértékek az egységsugarú körön belül vannak, ezért a DI-rendszer aszimptotikusan stabil. Viszont mivel nem a bal félsíkon helyezkednek el ($\hbox{Re}(\lambda_{1,2})>0$), ezért az FI-rendszer nem aszimptotikusan stabil.

\end{document}