\documentclass[12pt,a4paper]{article}

\usepackage[utf8]{inputenc}
\usepackage[T1]{fontenc}
\usepackage[magyar]{babel}
\usepackage{lmodern}
\usepackage{amsmath,amssymb}
\usepackage{fullpage}
\usepackage[a4paper,top=2cm,left=2cm,right=2cm,bottom=3.5cm]{geometry}
\usepackage{pstricks-add}
\usepackage{fancyhdr}
\usepackage{graphicx}

\title{\vspace{-2ex}Jelek és rendszerek 2. házi}
\author{Kriván Bálint \texttt{CBVOEN}\\
Gyakorlat vezető: Farkasvölgyi Andrea}

\pagestyle{fancy}
\lhead{\sc Kriván Bálint \texttt{CBVOEN}}
\chead{}
\rhead{Jelek és rendszerek 2. házi}
\lfoot{}
\cfoot{\thepage}
\rfoot{}
\headheight 15pt
\headsep 10pt

\parindent 0pt

\newlength{\wordwidth}

\newcommand\uuline[1]{\underline{\underline{#1}}}

\newcommand\uline[1]{\underline{#1}}
\DeclareMathOperator{\adj}{adj}
\DeclareMathOperator{\arc}{arc}

\begin{document}
\maketitle

\thispagestyle{fancy}

\textbf{2 (a)}

Nézzük először az FI-t:

\[\uline{x}'(t) = \left[\begin{matrix}-1,2 & -0,5\\2 & -1,2\end{matrix}\right]\uline{x}(t) + \left[\begin{matrix}1,2\\1,3\end{matrix}\right]u(t)\]
\[y(t) = \left[\begin{matrix}0,8 & -0,6\end{matrix}\right]\uline{x}(t) + 0,45u(t)\]

Használva a szokványos jelöléseket:
\[\uuline{A} = \left[\begin{matrix}-1,2 & -0,5\\2 & -1,2\end{matrix}\right] \quad \uline{B}=\left[\begin{matrix}1,2\\1,3\end{matrix}\right] \quad \uline{C}^T = \left[\begin{matrix}0,8 & -0,6\end{matrix}\right] \quad D = 0,45\]

Tudjuk, hogy:
\[H(j \omega) = \frac{\uline{C}^T \adj(j \omega \uuline{E}-\uuline{A})\uline{B}}{\det(j \omega \uuline{E}-\uuline{A})} + D\]

Felírva $j \omega \uuline{E}-\uuline{A}$ mátrixot, majd ennek adjungáltját és determinánsát:
\[j \omega \uuline{E}-\uuline{A} = \left[\begin{matrix}j\omega+1,2 & 0,5\\-2 & j\omega+1,2\end{matrix}\right]\]
\[\adj(j \omega \uuline{E}-\uuline{A}) = \left[\begin{matrix}j\omega+1,2 & -0,5\\2 & j\omega+1,2\end{matrix}\right]\]
\[\det(j \omega \uuline{E}-\uuline{A}) = (j\omega+1,2)^2+1 = (j\omega)^2+2,4j\omega+2,44\]

Innen már kiszámolhatjuk $H(j \omega)$-t:
\[H(j \omega) = \frac{\left[\begin{matrix}0,8 & -0,6\end{matrix}\right] \left[\begin{matrix}j\omega+1,2 & -0,5\\2 & j\omega+1,2\end{matrix}\right]\left[\begin{matrix}1,2\\1,3\end{matrix}\right]}{(j\omega)^2+2,4j\omega+2,44} + 0,45 =\]
\[= \frac{\left[\begin{matrix}(0,8j\omega - 0,24) & (-0,6j\omega-1,12)\end{matrix}\right] \left[\begin{matrix}1,2\\1,3\end{matrix}\right]}{(j\omega)^2+2,4j\omega+2,44} + 0,45 = \]
\[= \frac{0,18j\omega-1,744}{(j\omega)^2+2,4j\omega+2,44} + 0,45 = \boxed{\frac{0,45(j\omega)^2 +1,26j\omega-0,646}{(j\omega)^2+2,4j\omega+2,44}} \]

Nézzük a DI-t:

\[\uline{x}[k+1] = \left[\begin{matrix}-1 & -0,7\\0,8 & 0,8\end{matrix}\right]\uline{x}[k] + \left[\begin{matrix}0,7\\0,6\end{matrix}\right]u[k]\]
\[y[k] = \left[\begin{matrix}-0,4 & 0,5\end{matrix}\right]\uline{x}[k] + 1,3u[k]\]

Ebből:
\[\uuline{A} = \left[\begin{matrix}-1 & -0,7\\0,8 & 0,8\end{matrix}\right] \quad \uline{B}=\left[\begin{matrix}0,7\\0,6\end{matrix}\right] \quad \uline{C}^T = \left[\begin{matrix}-0,4 & 0,5\end{matrix}\right] \quad D = 1,3\]

Tudjuk, hogy:
\[H(e^{j \vartheta}) = \frac{\uline{C}^T \adj(e^{j \vartheta} \uuline{E}-\uuline{A})\uline{B}}{\det(e^{j \vartheta} \uuline{E}-\uuline{A})} + D\]

Felírva $e^{j\vartheta} \uuline{E}-\uuline{A}$ mátrixot, majd ennek adjungáltját és determinánsát:
\[e^{j\vartheta} \uuline{E}-\uuline{A} = \left[\begin{matrix}e^{j\vartheta}+1 & 0,7\\-0,8 & e^{j\vartheta}-0,8\end{matrix}\right]\]
\[\adj(e^{j\vartheta} \uuline{E}-\uuline{A}) = \left[\begin{matrix}e^{j\vartheta}-0,8 & -0,7\\0,8 & e^{j\vartheta}+1\end{matrix}\right]\]
\[\det(e^{j\vartheta} \uuline{E}-\uuline{A}) = e^{2j\vartheta}+ 0,2e^{j\vartheta} -0,24\]

Ebből kiszámíthatjuk $H(e^{j \vartheta})$-t:
\[H(e^{j \vartheta}) = \frac{\left[\begin{matrix}-0,4 & 0,5\end{matrix}\right] \left[\begin{matrix}e^{j\vartheta}-0,8 & -0,7\\0,8 & e^{j\vartheta}+1\end{matrix}\right] \left[\begin{matrix}0,7\\0,6\end{matrix}\right]}{e^{2j\vartheta}+ 0,2e^{j\vartheta} -0,24} + 1,3 = \]
\[= \frac{\left[\begin{matrix}(-0,4e^{j\vartheta}+0,72) & (0,5e^{j\vartheta}+0,78)\end{matrix}\right] \left[\begin{matrix}0,7\\0,6\end{matrix}\right]}{e^{2j\vartheta}+ 0,2e^{j\vartheta} -0,24} + 1,3 = \]
\[= \frac{0,02e^{j\vartheta}+0,972}{(e^{j\vartheta})^2+ 0,2e^{j\vartheta} -0,24} + 1,3 = \boxed{\frac{1,3e^{2j\vartheta}+0,28e^{j\vartheta}+0,66}{e^{2j\vartheta}+ 0,2e^{j\vartheta}-0,24}}\]

\textbf{2 (b)}

Kezdjük az FI jellel:
\[u(t) = -2,75t + 11\varepsilon(t)\cdot t -16,5 \quad \hbox{ha } -6\leqslant t \leqslant 2 \quad \hbox{illetve, } u(t+8) = u(t)\]

A komplex Fourier-sor $p$. együtthatójának kiszámítása:
\[\overline{U}^C_p = \frac{1}{T}\int_{-T/2}^{T/2} u(t) e^{-j\omega_0 p t} \; dt\]
Ahol $T=8$, illetve $\omega_0 = \frac{2\pi}{T} = \frac{\pi}{4}$. Jelenleg egyszerűbb, ha az integrálási határt nem $-4$-től $4$-ig, hanem $-6$-tól $2$-ig válasszuk:
\[\overline{U}^C_p = \frac{1}{8}\int_{-6}^{2} u(t) e^{-j\omega_0 p t} \; dt = \frac{1}{8}\int_{-6}^{0} (-2,75t-16,5) e^{-j\omega_0 p t} \; dt + \frac{1}{8}\int_{0}^{2} (8,25t-16,5) e^{-j\omega_0 p t} \; dt =\]
\[= \frac{1}{8}\int_{-6}^{0} (-2,75t) e^{-j\omega_0 p t} \; dt + \frac{1}{8}\int_{0}^{2} (8,25t) e^{-j\omega_0 p t} \; dt + \frac{1}{8}\int_{-6}^{2} (-16,5) e^{-j\omega_0 p t} \; dt = \]
\[= \frac{-2,75}{8}\int_{-6}^{0} t e^{-j\omega_0 p t} \; dt + \frac{8,25}{8}\int_{0}^{2} t e^{-j\omega_0 p t} \; dt + \frac{1}{8}\left[\frac{(-16,5) e^{-j\omega_0 p t}}{-j\omega_0 p }\right]_{-6}^{2} = \]
\[= \frac{-2,75}{8}\left(\left[\frac{t e^{-j\omega_0 p t}}{-j\omega_0 p}\right]_{-6}^{0} - \left[ \frac{e^{-j\omega_0 p t}}{(j\omega_0 p)^2} \right]_{-6}^{0} \right) + \frac{8,25}{8}\left(\left[\frac{t e^{-j\omega_0 p t}}{-j\omega_0 p}\right]_{0}^{2} - \left[ \frac{e^{-j\omega_0 p t}}{(j\omega_0 p)^2} \right]_{0}^{2} \right) + \]
\[ + \frac{1}{8}\left[\frac{(-16,5) e^{-j\omega_0 p t}}{-j\omega_0 p }\right]_{-6}^{2} = \]
\[= \frac{-2,75}{8}\left(\frac{(-6) e^{6j\omega_0 p }}{j\omega_0 p} - \frac{1-e^{6j\omega_0 p}}{(j\omega_0 p)^2} \right) + \frac{8,25}{8}\left( \frac{2 e^{-2j\omega_0 p}}{-j\omega_0 p} - \frac{e^{-2j\omega_0 p}-1}{(j\omega_0 p)^2} \right) - \frac{16,5}{8}\cdot\frac{ (e^{-2j\omega_0 p}-e^{6j\omega_0 p})}{-j\omega_0 p } = \]

\[= \frac{-2,75}{8}\cdot\frac{(-6)j\omega_0 p\cdot e^{6j\omega_0 p }-1+e^{6j\omega_0 p}}{(j\omega_0 p)^2} + \frac{8,25}{8}\cdot \frac{-2j\omega_0 p\cdot e^{-2j\omega_0 p}-e^{-2j\omega_0 p}+1}{(j\omega_0 p)^2} +\]
\[+ \frac{16,5}{8}\cdot\frac{j\omega_0 p(e^{-2j\omega_0 p}-e^{6j\omega_0 p})}{(j\omega_0 p)^2 } = \]
\[ = \frac{-2,75}{8}\cdot\frac{-1+e^{6j\omega_0 p}}{(j\omega_0 p)^2} + \frac{8,25}{8}\cdot \frac{-e^{-2j\omega_0 p}+1}{(j\omega_0 p)^2} =\]
\[= \frac{2,75}{8}\cdot\frac{-1+e^{6j\omega_0 p}}{(\omega_0 p)^2} + \frac{8,25}{8}\cdot \frac{e^{-2j\omega_0 p}-1}{(\omega_0 p)^2} = \boxed{\frac{1}{8}\left(\frac{2,75e^{6j\omega_0 p}+8,25e^{-2j\omega_0 p}-11}{(\omega_0 p)^2}\right)}\]
Amire még szükségünk van az $\overline{U}^C_0$:
\[\overline{U}^C_0 = \frac{1}{8}\int_{-6}^{2} u(t) \; dt = \frac{1}{8}\int_{-6}^{0} (-2,75t-16,5) \; dt + \frac{1}{8}\int_{0}^{2} (8,25t-16,5) \; dt = \]
Ez pedig két háromszög (előjeles) területe, azaz:
\[= \frac{1}{8}\cdot\left(\frac{6\cdot (-16,5)}{2}+\frac{2\cdot (-16,5)}{2}\right) = -8,25\]

Mérnöki valós alak:
\[u(t) = U_0 + \sum_{p=1}^{\infty} U_p \cos(p\omega t + \varrho_p)\]
ahol:
\[U_p = 2|\overline{U}^C_p|\]
\[U_0 = \overline{U}^C_0\]
\[\varrho_p = \arc\{\overline{U}^C_p\}\]
Tehát:
\[u(t) = -8,25 + \sum_{p=1}^{\infty} 2 |\overline{U}^C_p| \cos(p\omega t + \arc\{\overline{U}^C_p\})\]

Az előzőekben kiszámolt $\overline{U}^C_p$-t $p=1,2,3$-ra kiszámolhatjuk és az alábbi táblázatba foglalhatjuk a kérdéses értékeket:

\begin{center}
\begin{tabular}{|c|c|c|c|}
\hline
$p$ & $\overline{U}^C_p$ & $2|\overline{U}^C_p|$ & $\arc\{\overline{U}^C_p\}$ \\\hline\hline
1 & $-2,2291 - 2,2291j$ & $6,3048$ & $-\frac{3\pi}{4}$ \\\hline
2 & $-1,1145$ & $2,2291$ & $\pi$ \\\hline
3 & $-0,2477 + 0,2477j$ & $0,7005$ & $\frac{3\pi}{4}$ \\\hline
\end{tabular}
\end{center}

Tehát akkor a Fourier polinom, ami legalább 3 nem nulla felharmonikust tartalmaz:
\[u(t) \approx \boxed{-8,25 + 6,3048 \cos\left(\frac{\pi}{4} t -\frac{3\pi}{4}\right) + 2,2291 \cos\left(\frac{\pi}{2} t + \pi\right) + 0,7005 \cos\left(\frac{3\pi}{4} t + \frac{3\pi}{4}\right)}\]

Most nézzük meg a DI jelet:
\[u[k] = 18-3,6k \quad \hbox{ha } 0\leqslant k \leqslant 5 \quad \hbox{illetve, } u[k+6] = u[k]\]

A komplex Fourier-sor $p$. együtthatója:
\[\overline{U}^C_p = \frac{1}{L}\sum_{k=0}^{L-1} u[k]\cdot e^{-jp\theta k}\]
Ahol $L=6$, illetve $\theta = \frac{2\pi}{L} = \frac{\pi}{3}$, tehát:
\[\overline{U}^C_p = \frac{1}{6}\sum_{k=0}^{5} u[k]\cdot e^{-jp\theta k}\]
Mérnöki valós alak, ha $L$ páros:
\[u[k] = U_0  + \sum_{p=1}^{\frac{L}{2}-1} 2|\overline{U}^C_p|\cos(p\theta k + \varrho_p) + |\overline{U}^C_{L/2}|(-1)^k\]
ahol $\varrho_p = \arc\{\overline{U}^C_p\}$
Számoljuk ki a komplex Fourier-sor együtthatókat $p=0, 1, 2, 3$-ra:
\[\overline{U}^C_0 = \frac{1}{6}\sum_{k=0}^{5} u[k] = \frac{1}{6}\sum_{k=0}^{5} \left(18-3,6k\right) = 9\]
\[\overline{U}^C_1 = \frac{1}{6}\sum_{k=0}^{5} u[k] e^{-j\theta k} = \frac{1}{6}\left(  18+14,4e^{-j\theta}+10,8e^{-2j\theta}+7,2e^{-3j\theta}+3,6e^{-4j\theta}\right) = \]
\[= 3+2,4e^{-j\frac{\pi}{3}}+1,8e^{-2j\frac{\pi}{3}}+1,2e^{-j\pi}+0,6e^{-4j\frac{\pi}{3}} = 1,8+2,4e^{-j\frac{\pi}{3}}+1,8e^{-2j\frac{\pi}{3}}+0,6e^{-4j\frac{\pi}{3}} = \boxed{3,6e^{-j\frac{\pi}{3}}}\]

\[\overline{U}^C_2 = \frac{1}{6}\sum_{k=0}^{5} u[k] e^{-2j\theta k} = \frac{1}{6}\left(  18+14,4e^{-2j\theta}+10,8e^{-4j\theta}+7,2e^{-6j\theta}+3,6e^{-8j\theta}\right) = \]
\[= 3+2,4e^{-2j\frac{\pi}{3}}+1,8e^{-4j\frac{\pi}{3}}+1,2e^{-2j\pi}+0,6e^{-8j\frac{\pi}{3}} = 4,2+2,4e^{-2j\frac{\pi}{3}}+1,8e^{-4j\frac{\pi}{3}}+0,6e^{-8j\frac{\pi}{3}} = \boxed{\frac{3,6}{\sqrt{3}}e^{-j\frac{\pi}{6}}}\]

\[\overline{U}^C_3 = \frac{1}{6}\sum_{k=0}^{5} u[k] e^{-3j\theta k} = \frac{1}{6}\left(  18+14,4e^{-3j\theta}+10,8e^{-6j\theta}+7,2e^{-9j\theta}+3,6e^{-12j\theta}\right) = \]
\[= 3+2,4e^{-j\pi}+1,8e^{-2j\pi}+1,2e^{-3j\pi}+0,6e^{-4j\pi} = 3-2,4+1,8-1,2+0,6 = \boxed{1,8}\]

Helyettesítsünk be a mérnöki valós alakba:
\[u[k] = U_0  + \sum_{p=1}^{\frac{L}{2}-1} |\overline{U}^C_p|\cos(p\theta k + \varrho_p) = \]
\[= 9 + |2\overline{U}^C_1|\cos(\theta k + \varrho_1)+|2\overline{U}^C_2|\cos(2\theta k + \varrho_2) = \]
\[= \boxed{9 + 7,2\cos\left(\frac{\pi}{3}k-\frac{\pi}{3}\right)+ \frac{7,2}{\sqrt{3}}\cos\left(\frac{2\pi}{3}k-\frac{\pi}{6}\right)+1,8(-1)^k}\]

\textbf{2 (c)}

Kezdjük az FI rendszerrel:

Használjuk a komplex számítási módot! Legyenek a polinom tagjainak megfelelő komplex számok $\overline{U}_0, \overline{U}_1, \overline{U}_2$ és $\overline{U}_3$ a következő módon:
\[\overline{U}_0 = -8,25\]
\[\overline{U}_1 = 6,3048\cdot e^{-j\frac{3\pi}{4}}\]
\[\overline{U}_2 = 2,2291\cdot e^{j\pi}\]
\[\overline{U}_3 = 0,7005\cdot e^{j\frac{3\pi}{4}}\]

Felhasználva, hogy:
\[H(j \omega) = \frac{0,45(j\omega)^2 +1,26j\omega-0,646}{(j\omega)^2+2,4j\omega+2,44} \]

A rendszernek a gerjesztés egyes komponenseire adott válaszainak megfelelő komplex számok:
\[\overline{Y}_0 = H(0)\cdot\overline{U}_0 = \frac{-0,646}{2,44}\cdot (-8,25) = 2,1842\]
\[\overline{Y}_1 = H\left(j\frac{\pi}{4}\right)\cdot\overline{U}_1 = (0,5162e^{1,5196j})(6,3048\cdot e^{-j\frac{3\pi}{4}}) = 3,2544\cdot e^{-0,8365j}\]
\[\overline{Y}_2 = H\left(j\frac{\pi}{2}\right)\cdot\overline{U}_2 = (0,7019e^{0,7185j})(2,2291\cdot e^{j\pi})= 1,5646\cdot e^{-2,4231j} \]
\[\overline{Y}_3 = H\left(j\frac{3\pi}{4}\right)\cdot\overline{U}_3 = (0,67e^{0,3110j})(0,7005\cdot e^{j\frac{3\pi}{4}})= 0,4693\cdot e^{2,6672j}\]

Ebből felírhatóak a gerjesztés egyes komponenseire adott válaszok:
\[y_0(t) = 2,1842\]
\[y_1(t) = 3,2544\cos\left(\frac{\pi}{4}t-0,8365\right)\]
\[y_2(t) = 1,5646\cos\left(\frac{\pi}{2}t-2,4231\right)\]
\[y_3(t) = 0,4693\cos\left(\frac{3\pi}{4}t+2,6672\right)\]
Így a rendszer válasza a megadott gerjesztésre:
\[y(t) \approx \boxed{2,1842 + 3,2544\cos\left(\frac{\pi}{4}t-0,8365\right) + }\]
\[\boxed{+ 1,5646\cos\left(\frac{\pi}{2}t-2,4231\right) + 0,4693\cos\left(\frac{3\pi}{4}t+2,6672\right)}\]

Nézzük a DI rendszert:

Hasonlóan az előzőnél, írjuk fel $\overline{U}_0, \overline{U}_1, \overline{U}_2$ és $\overline{U}_3$-mat:
\[\overline{U}_0 = 9\]
\[\overline{U}_1 = 7,2\cdot e^{-j\frac{\pi}{3}}\]
\[\overline{U}_2 = \frac{7,2}{\sqrt{3}}\cdot e^{-j\frac{\pi}{6}}\]
\[\overline{U}_3 = 1,8\]

Felhasználva, hogy:
\[H(e^{j \vartheta}) = \frac{1,3e^{2j\vartheta}+0,28e^{j\vartheta}+0,66}{e^{2j\vartheta}+ 0,2e^{j\vartheta}-0,24} \]
A rendszernek a gerjesztés egyes komponenseire adott válaszainak megfelelő komplex számok:
\[\overline{Y}_0 = H(e^{0j})\cdot\overline{U}_0 = \frac{7}{3}\cdot 9 = 21\]
\[\overline{Y}_1 = H(e^{j\frac{\pi}{3}})\cdot\overline{U}_1 = (1,1278e^{-0,6612j})(7,2\cdot e^{-j\frac{\pi}{3}}) = 8,1202\cdot e^{-1,7084j}\]
\[\overline{Y}_2 = H(e^{j\frac{2\pi}{3}})\cdot\overline{U}_2 = (0,82e^{0,7350j})\left(\frac{7,2}{\sqrt{3}}\cdot e^{-j\frac{\pi}{6}}\right)= 3,4087\cdot e^{0,2114j} \]
\[\overline{Y}_3 = H(je^{\pi})\cdot\overline{U}_3 = (3)(1,8)= 5,4\]
A gerjesztés egyes komponenseire adott válaszok:
\[y_0[k] = 21\]
\[y_1[k] = 8,1202\cos\left(\frac{\pi}{3}k-1,7084\right)\]
\[y_2[k] = 3,4087\cos\left(\frac{2\pi}{3}k+0,2114\right)\]
\[y_3[k] = 5,4\cdot(-1)^k\]
Így a rendszer válasza a megadott gerjesztésre:
\[\boxed{y[k] = 21 + 8,1202\cos\left(\frac{\pi}{3}k-1,7084\right) + 3,4087\cos\left(\frac{2\pi}{3}k+0,2114\right) + 5,4\cdot(-1)^k}\]

\textbf{2 (d)}

Kezdjük az FI rendszerrel:

\[U(j\omega) = \int_{-\infty}^{\infty} u(t)\cdot e^{-j\omega t} \; dt\]
\[u(t) = \left\{\begin{tabular}{ll} $-16,5-2,75t$ & \hbox{ha $-6\leqslant x\leqslant 0$} \\
$-16,5+8,25t$ & \hbox{ha $0< x\leqslant 2$} \\
0 & \hbox{különben}
\end{tabular}\right.\]
Így az integrál egyszerűsödik:
\[U(j\omega) = \int_{-6}^{0} (-16,5-2,75t)\cdot e^{-j\omega t} \; dt + \int_{0}^{2} (-16,5+8,25t)\cdot e^{-j\omega t} \; dt \]
Ehhez nézzük meg a következő integrált:
\[\int t\cdot e^{-j\omega t} \; dt = t\cdot \frac{e^{-j\omega t}}{-j\omega}-\int \frac{e^{-j\omega t}}{-j\omega} \; dt = t\cdot \frac{e^{-j\omega t}}{-j\omega} - \frac{e^{-j\omega t}}{-\omega^2} = \frac{e^{-j\omega t}}{\omega}\left(jt+\frac{1}{\omega}\right)\]
Akkor nézzük:
\[U(j\omega) = \int_{-6}^{0} (-16,5-2,75t)\cdot e^{-j\omega t} \; dt + \int_{0}^{2} (-16,5+8,25t)\cdot e^{-j\omega t} \; dt = \]
\[= \left[-16,5\cdot\frac{e^{-j\omega t}}{-j\omega}\right]_{-6}^{0} -2,75\left[\frac{e^{-j\omega t}}{\omega}\left(jt+\frac{1}{\omega}\right)\right]_{-6}^{0} + \left[-16,5\cdot\frac{e^{-j\omega t}}{-j\omega}\right]_{0}^{2} + 8,25\left[\frac{e^{-j\omega t}}{\omega}\left(jt+\frac{1}{\omega}\right)\right]_{0}^{2} = \]
\[= \left(16,5\cdot\frac{1}{j\omega}-16,5\cdot\frac{e^{6j\omega}}{j\omega}\right)
-2,75\left(\frac{1}{\omega^2} - \frac{e^{6j\omega}}{\omega}\left(-6j+\frac{1}{\omega}\right)\right) +
\left(16,5\cdot\frac{e^{-2j\omega}}{j\omega}-16,5\cdot\frac{1}{j\omega}\right) +\]
\[+ 8,25\left(\frac{e^{-2j\omega}}{\omega}\left(2j+\frac{1}{\omega}\right)-\frac{1}{\omega^2}\right) = \]
\[= -2,75\left(\frac{1-e^{6j\omega}}{\omega^2}\right) + 8,25\left(\frac{e^{-2j\omega}-1}{\omega^2}\right) = \boxed{\frac{2,75\cdot e^{6j\omega}+8,25\cdot e^{-2j\omega}-11}{\omega^2}}\]
A spektrum a következőképpen számolható:
\[|Y(j\omega)| = |U(j\omega)\cdot H(j\omega)|\]
Mivel
\[H(j\omega) = \frac{0,45(j\omega)^2 +1,26j\omega-0,646}{(j\omega)^2+2,4j\omega+2,44}\]
Ezért:
\[|Y(j\omega)| = \left|\frac{2,75\cdot e^{6j\omega}+8,25\cdot e^{-2j\omega}-11}{\omega^2}\cdot \frac{0,45(j\omega)^2 +1,26j\omega-0,646}{(j\omega)^2+2,4j\omega+2,44}\right|\]

Ábrázolva $\omega \in [0, 5]$ intervallumon:
\begin{center}
\psset{xunit=1.0cm,yunit=0.2cm,algebraic=true,dotstyle=o,dotsize=3pt 0,linewidth=0.8pt,arrowsize=3pt 2,arrowinset=0.25}
\begin{pspicture*}(-2.3,-3.91)(6.5,24.6)
\psaxes[labelFontSize=\scriptstyle,xAxis=true,yAxis=true,Dx=1,Dy=5,ticksize=-2pt 0,subticks=2]{->}(0,0)(-0.85,-3.91)(5.8,24.6)
\psplot[plotpoints=200]{0.001}{5.0}{sqrt(1.59*x^2+(-(0.45)*x^2-0.65)^2)*sqrt((2.75*SIN(6*x)-8.25*SIN(2*x))^2+(8.25*COS(2*x)+2.75*COS(6*x)-11)^2)/(x^2*sqrt((2.44-x^2)^2+5.76*x^2))}
\rput[tl](-1.8,24.14){$|Y(j\omega)|$}
\rput[tl](5.5,-1.7){$\omega$}
\end{pspicture*}
\end{center}

Folytassuk a DI rendszerrel:

\[U(e^{j\vartheta}) = \sum_{k=-\infty}^{\infty} u[k]\cdot e^{-j\vartheta k}\]
\[u[k] = \left\{\begin{tabular}{ll} $18-3,6k$ & \hbox{ha $0\leqslant x\leqslant 5$} \\
0 & \hbox{különben}
\end{tabular}\right.\]
Így a szumma leegyszerűsödik:
\[U(e^{j\vartheta}) = \sum_{k=0}^{5} (18-3,6k)\cdot e^{-j\vartheta k} = \]
\[= 18+14,4\cdot e^{-j\vartheta}+10,8\cdot e^{-2j\vartheta}+7,2\cdot e^{-3j\vartheta}+3,6\cdot e^{-4j\vartheta} \]
A spektrum a következőképpen számolható:
\[|Y(e^{j\vartheta})| = |U(e^{j\vartheta})\cdot H(e^{j\vartheta})|\]
Mivel
\[H(e^{j\vartheta}) = \frac{1,3e^{2j\vartheta}+0,28e^{j\vartheta}+0,66}{e^{2j\vartheta}+ 0,2e^{j\vartheta}-0,24}\]
Ezért:
\[|Y(e^{j\vartheta})| = \left|(18+14,4\cdot e^{-j\vartheta}+10,8\cdot e^{-2j\vartheta}+7,2\cdot e^{-3j\vartheta}+3,6\cdot e^{-4j\vartheta})\cdot \frac{1,3e^{2j\vartheta}+0,28e^{j\vartheta}+0,66}{e^{2j\vartheta}+ 0,2e^{j\vartheta}-0,24}\right| = \]
\[= \left|\frac{29,952+21,888\cdot e^{-j \vartheta}+23,76\cdot e^{j \vartheta}+13,824\cdot e^{-2j \vartheta}+23,4\cdot e^{2j \vartheta}+5,76\cdot e^{-3j \vartheta}+2,376e^{-4j \vartheta}}{e^{2j\vartheta}+ 0,2e^{j\vartheta}-0,24}\right|\]

Ábrázolva $\vartheta \in [0, \pi]$ intervallumon:
\begin{center}
\psset{xunit=2.0cm,yunit=0.03cm,algebraic=true,dotstyle=o,dotsize=3pt 0,linewidth=0.8pt,arrowsize=3pt 2,arrowinset=0.25}
\begin{pspicture*}(-0.9,-25.57)(3.59,147.84)
\psaxes[labelFontSize=\scriptstyle,xAxis=true,yAxis=true,Dx=\pstPI1,Dy=50,ticksize=-2pt 0,subticks=2,labels=y]{->}(0,0)(-0.78,-35.57)(3.59,147.84)
\uput[-90](!PI 0){$\pi$}
\rput[tl](-0.85,140.74){$|Y(e^{j\vartheta})|$}
\rput[tl](4.95,6.41){$\omega$}
\psplot[plotpoints=200]{0.0}{3.141592653589793}{sqrt((1.87*SIN(x)+9.58*SIN(2*x)-5.76*SIN(3*x)-2.38*SIN(4*x))^2+(45.65*COS(x)+37.22*COS(2*x)+5.76*COS(3*x)+2.38*COS(4*x)+29.95)^2)/sqrt((0.2*SIN(x)+SIN(2*x))^2+(0.2*COS(x)+COS(2*x)-0.24)^2)}
\rput[tl](3.4,-8.45){$\vartheta$}
\end{pspicture*}
\end{center}

\textbf{2 (e)}

Ábrázoljuk a spektrumot, és megkeressük a maximum helyet:

\begin{center}
\psset{xunit=2.0cm,yunit=0.3cm,algebraic=true,dotstyle=o,dotsize=3pt 0,linewidth=0.8pt,arrowsize=3pt 2,arrowinset=0.25}
\begin{pspicture*}(-1.5,-5.15)(5.6,23.33)
\psaxes[labelFontSize=\scriptstyle,xAxis=true,yAxis=true,Dx=1,Dy=5,ticksize=-2pt 0,subticks=2]{->}(0,0)(-2,-3)(5.6,23.33)
\psplot[plotpoints=200]{-2.001151667557683}{5.597292288673072}{sqrt(1.59*x^2+(-(0.45)*x^2-0.65)^2)*sqrt((2.75*SIN(6*x)-8.25*SIN(2*x))^2+(8.25*COS(2*x)+2.75*COS(6*x)-11)^2)/(x^2*sqrt((2.44-x^2)^2+5.76*x^2))}
\rput[tl](-0.66,22.97){$|Y(j\omega)|$}
\rput[tl](5.3,-0.79){$\omega$}
\psplot{-2}{5.6}{(-0.93-0*x)/-1}
\psline[linewidth=1.6pt](2.68,0.93)(2.68,0)
\psplot{-2}{5.6}{(-18.6-0*x)/-1}
\psdots[dotstyle=*](0.35,18.6)
\psdots[dotstyle=*](2.68,0.93)
\psdots[dotstyle=*](2.68,0)
\rput[bl](2.51,-1.7){$2,68$}
\psdots[dotstyle=*](0,18.6)
\rput[bl](0.04,18.97){$18,6$}
\rput[bl](0.04,1.4){$0,83$}
\end{pspicture*}
\end{center}

Ez $18,6$-nál van, így a kérdéses szávszélesség: $2,68$.

\end{document}